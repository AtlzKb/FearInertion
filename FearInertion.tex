%
% A simple LaTeX template for Books
%  (c) Aleksander Morgado <aleksander@es.gnu.org>
%  Released into public domain
%

\documentclass{book}
\usepackage[a4paper, top=3cm, bottom=3cm]{geometry}
\usepackage[T2A]{fontenc} 
\usepackage[utf8x]{inputenc}
\usepackage[english, russian]{babel}
\usepackage{setspace}
\usepackage{fancyhdr}
\usepackage{tocloft}
\usepackage{epigraph}


\begin{document}


\pagestyle{empty}
%\pagenumbering{}
% Set book title
\title{\textbf{Инерция страха. Социализм и тоталитаризм}}
% Include Author name and Copyright holder name
\author{Валентин Турчин}



% 1st page for the Title
%-------------------------------------------------------------------------------
\maketitle


% 2nd page, thanks message
%-------------------------------------------------------------------------------
\thispagestyle{empty}
\thanks{Thanks to Cicero and his \em{De finibus bonorum et malorum}}
\newpage



% General definitions for all Chapters
%-------------------------------------------------------------------------------

% Define Page style for all chapters
\pagestyle{fancy}
% Delete the current section for header and footer
\fancyhf{}
% Set custom header
\lhead[]{\thepage}
\rhead[\thepage]{}

% Set arabic (1,2,3\ldots) page numbering
\pagenumbering{arabic}

% Set double spacing for the text
\doublespacing



% Not enumerated chapter
%-------------------------------------------------------------------------------
\chapter*{Семь лет спустя}
Каждый раз, когда я берусь писать об общественных проблемах в нашей стране, я сталкиваюсь со следующим противоречием.

С одной стороны, я -- убежденный эволюционист и реформист, еще точнее (хотя это слово у нас мало принято) -- \textit{градуалист},  сторонник постепенных преобразований, проводимых параллельно с эволюцией общественного сознания. В этих воззрениях я не одинок: напротив, в Советском Союзе большинство образованных людей, насколько я могу судить, рассматривает общественные явления с таких же позиций. Хотя и говорят, что история учит только тому, что она никого ничему не учит, это, к счастью, не совсем так. Результат большевистской революции научил нас не верить пламенным призывам одним махом уничтожить правящий класс, сломать государственную машину и построить на ее обломках новое общество, справедливое и процветающее. Поэтому меньше всего хотел бы я становиться по отношению к существующему строю и правящему классу в ту позу безоговорочного отрицания, в которой находились в свое время большевики. Нам необходим критический, но конструктивный анализ ситуации. Задачу критиков я вижу не в том, чтобы противопоставить себя правящему слою как враждебную ему силу, а в том, чтобы нащупать путь, который позволил бы выйти из тупика и приступить к давно назревшим преобразованиям. Путь этот не может не быть в той или иной степени компромиссным, он не должен угрожать интересам правящего класса до такой степени, чтобы сделать его непримиримым врагом преобразований. Ясно, что критика, преследующая такие цели, должна быть до известной степени сдержанной. Кто стремится к компромиссу, не должен разрушать для него почву.

С другой же стороны, условия общественной жизни у нас в стране таковы, что когда просто называешь вещи их именами, то превращаешься, с точки зрения представителя правящего класса, в отъявленного экстремиста, с которым нет и не может быть никаких компромиссов. Вероятно, никогда в истории человечества не было такого постоянного, повсеместного и всем обществом принятого несоответствия между словами и действительностью, как в нашей стране в течение последних 50-ти лет. Когда человек начинает о белом говорить, что оно -- белое, а о черном -- черное, его за это наказывают, и он попадает в отщепенцы, диссиденты.

Осенью 1968 года я написал брошюру "Инерция страха", которая тогда получила  довольно значительное распространение в самиздате. Настоящее издание написано заново и яв­ляется, таким образом, новой работой, хотя и основанной на тех же идеях, что и первый вариант. Я хочу перечислить важнейшие из этих идей: позитивизм в области философии и социализм (но не марксизм) в общественных воззрениях; убеждение в ведущей роли мировоззрения в социальных движениях; основные демократические свободы и права личности как ключевое условие для нормального развития общества; градуализм в политике; усмотрение в интеллигенции той силы, которая в принципе может, а следовательно и должна, добиться демократических преобразований; чрезвычайно критическая оценка реально существующей интеллигенции; призыв к преодолению инерции страха, укоренившегося в сталинские времена и сковывающего общественную инициативу.

Однако в некотором отношении нынешний вариант существенно отличается от первоначального, что объясняется наличием между ними 
семилетнего интервала. 1968 год был высшей точкой эпохи "подписантства", когда вслед за именами А.~Синявского и Ю.~Даниэля 
прозвучали имена А.~Есенина‑Вольпина, А.~Сахарова, П.~Якира, А.~Гинзбурга, П.~Литвинова и других. Можно было надеяться, что в 
стране имеется какое‑то социально значимое меньшинство, готовое активно добиваться осуществления демократических перемен. Свою 
задачу я, как и многие самиздатовские авторы того времени, видел прежде всего в том, чтобы наметить идейную основу для выявления 
и объединения усилий этих людей. Действуя с позиций градуализма, я сделал кое‑какие терминологические уступки, подчеркнул точки 
пересечения своей системы взглядов с официальной советской идеологией и исключил из рассмотрения многие важные аспекты, по 
которым имеет место расхождение. Я считал, что остальное должно быть понятно по умолчанию, если есть желание понять.

Но оказалось, что базы для существенного расширения круга "подписантов" не было, как нет ее и по сей день. Для того чтобы произвести какие‑то перемены в обществе, должно существовать активное меньшинство, которое стремится к этим переменам. В процентном выражении оно может быть очень малым например, одна десятая процента от всего населения может оказаться чрезвычайно влиятельной, если речь идет о переменах, которые давно назрели и пассивно поддерживаются более широкими слоями (что и имеет место в данном случае). Но это должны быть люди, которые в самом деле стремятся к переменам, которые хоть чем‑то готовы рисковать или жертвовать для этой цели. У нас такой десятой доли процента нет, эти люди исчисляются буквально единицами. Распад культуры в советском обществе, тоталитарное уничтожение лично­сти зашли еще дальше, чем можно было думать в 1968 году. Власти разметали людей, выявивших себя в конце 60‑х годов: одни были посажены в тюрьмы или психиатрические больницы, других вынудили эмигрировать, большая часть замолчала, испуганная. Новых не нашлось или почти не нашлось. В 1970 году в совместном с А. Сахаровым и Р. Медведевым письме руководителям Советского Союза нами была предпринята еще одна попытка сформулировать тот минимум предложений, вокруг которого, казалось бы, должны были объединиться люди, понимающие необходимость постепенных демократических реформ. Никто не откликнулся.

Теперь ясно, что надо начинать работу с еще более ранней стадии. Речь идет не о том, чтобы выявлять и объединять сторонников демократизации, способных к социальному действию, а только еще о том, чтобы создавать какие‑то очаги мысли и чувства, которые, как можно надеяться, будут способствовать возникновению необходимого социально активного меньшинства.

В этой ситуации важнее всего называть вещи их именами и стремиться к целостной системе взглядов, не оставляя недоговоренностей. По‑видимому, единственно правильным является такой курс: совмещать бескомпромиссный анализ в плане фактов и принципов с осторожной градуалистской стратегией в плане выводов и действий. Быть градуалистом вовсе не значит руководствоваться полуправдой или не иметь определенных политических идеалов. Напротив, градуализм предполагает наличие какой‑то твердо признанной идеологии. Только на этой основе и можно идти на компромиссы и вообще заниматься политикой. (Впрочем, диссидентам пока что заниматься политикой не приходится. Власти просто не слушают нас. Нам не с кем идти на компромиссы, если бы мы этого и захотели.)

Я хочу заключить это краткое предисловие цитатой из Ганди, которой я всегда старался руководствоваться и которой я руководствовался при написании этой книги.

"Человек и его поступок -- вещи совершенно различные. Тогда как хороший поступок заслуживает одобрения, а дурной -- осуждения, человек, совершивший поступок, все равно хороший или дурной, всегда заслуживает уважения и сострадания, смотря по обстоятельствам. "Возненавидь грех, но не грешника", -- вот правило, которое, хотя его и довольно легко понять, редко осуществляется на практике. Этим и объясняется, почему яд ненависти растекается по всему миру."\cite{1}



% If the chapter ends in an odd page, you may want to skip having the page
%  number in the empty page
\newpage
\thispagestyle{empty}



% First enumerated chapter
%-------------------------------------------------------------------------------
\chapter{ТОТАЛИТАРИЗМ}

\section{Выход на стационарный режим}
Максим испытывал такое отчаяние, словно вдруг обнаружил, что его обитаемый остров населен на самом деле не людьми, а куклами\ldots Перед ним была огромная машина, слишком простая, чтобы эволюционировать, и слишком огромная, чтобы можно было надеяться разрушить ее небольшими силами. Не было силы в стране, которая могла бы освободить огромный народ, понятия не имеющий, что он не свободен\ldots Эта машина была неуязвима изнутри. Она была устойчива по отношению к любым малым возмущениям.
А. Стругацкий, Б. Стругацкий. \cite{2}


Сущность того, что происходит сейчас в Советском Союзе, может быть выражена следующим образом: тоталитарное общество стабилизируется, переходит в стационарную фазу развития, когда оно самовоспроизводится от поколения к поколению без существенных изменений. Времена Ленина и Сталина были героической эпохой нового общества, когда оно еще только создавалось, и перед его создателями стояла трудная задача: переделать сознание человека, превратить его в послушный винтик государственной машины. Эта задача потребовала для своего решения моря крови, миллионов человеческих жертв. Ко времени Хрущева она была уже, в общем, успешно решена. Оставалось только ухаживать за новым обществом, аккуратно выпалывая сорную траву и не замахиваясь на гран­диозные перестройки. Методы Никиты Сергеевича не подхо­дили для этой цели: они были слишком эксцентричны. Поэтому он и был замещен новыми, нынешними правителями, при которых бесцветность и безликость стали высшими государственными добродетелями. Теперь наша страна "уверенной поступью" идет к тому общественному порядку, который описывается в романах Замятина, Хаксли, Орвелла, Стругацких.

Под тоталитаризмом понимают тотальный контроль государства над всеми общественно важными  аспектами жизни граждан, включая их образ мышления. Тоталитарное государ­ство устанавливает единую для всех идеологическую систему и внедряет ее принудительным образом в головы граждан. Лица, открыто не разделяющие государственной идеологии, подвергаются наказаниям, которые, в зависимости от степени отклонения и других факторов, варьируются от блокировки продвижения по служебной лестнице до физического уничтожения. Основные права личности -- свобода ассоциаций, свобода получения и распространения информации и свобода обмена идеями -- ликвидируются. Борьба идей уступает место борьбе против  идей путем физического насилия. С точки зрения эволюционной теории тоталитаризм является извращением, дегенерацией, ибо более низкий уровень организации уродует и подавляет более высокий уровень. Тоталитарное общество теряет способность нормально развиваться и око­стеневает. Это тупик, волчья яма на пути эволюции.
Элементы тоталитаризма были свойственны многим циви­лизациям прошлого, и они приводили к тому, что общество застывало в своем 
развитии на многие столетия. В XX веке наука и технология дают неслыханно эффективные средства массовой манипуляции сознанием 
людей, поэтому опасность попасть в волчью яму и глубина этой ямы многократно возрастают. Сейчас на карте мира мы видим уже 
огромные пятна, пораженные тоталитаризмом; это словно участки омертвевшей ткани в живом организме. Современная цивилизация 
стремится к глобальности, по существу она уже глобальна. Если она станет тоталитарной, то откуда ждать излечения от болезни? 


\section{Будьте спокойны}

Есть несколько характерных отличий нашей эпохи от сталинской, которые свидетельствуют о переходе тоталитарного общества в стационарную фазу. Первое и самое важное из них таково. Во времена Сталина ни один человек не был уверен в своем завтрашнем дне: даже самый преданный сторонник режима (и даже на самом высшем уровне) мог попасть в мя­сорубку "архипелага ГУЛаг" и погибнуть. Теперь же вы мо­жете быть совершенно спокойны: если вы послушно выполня­ете все предписания властей и работаете на стабилизацию тота­литаризма, власти не только не тронут вас, но и постараются обеспечить то (довольно скудное) процветание, которое они могут создать. Это сравнение, разумеется, целиком в пользу нынешнего режима. Нельзя признать совершенным строй, который уничтожает своих сторонников. Сталинская мясорубка была нужна, чтобы внушить человеку Великий Ужас перед государством, чтобы перевоспитать его в новом, тоталитарном духе. И это делалось с размахом, с запасом. Шло экспериментирование, разрабатывались новые методы. При этом, естественно, нередко переступалась граница необходимого: происходили так называемые "перегибы". Боже, как было популяр­но это слово! Перегиб здесь, перегиб там\ldots Теперь это слово вышло из моды. Перегибов больше нет. Власти приобрели опыт, они научились бороться с идеями малой кровью, стараясь избежать чрезмерных репрессий. Сложился новый правящий класс, который отличает \textit{своих} от \textit{чужих},  и своих  никогда не трогают.

Зарубежные наблюдатели часто говорят о постепенном "смягчении",  "либерализации" политического режима в СССР и делают отсюда оптимистический вывод, что в конце концов советское общество "либерализуется" до того, что превратится в общество демократического западного типа. Эти выводы ни на чем не основаны. Напротив, все говорит о тенденции к увековечиванию тоталитарных порядков. Уровень насилия падает по мере того, как общество привыкает к этим порядкам, смиряется с ними. "Смягчение" режима по сравнению со сталинским периодом, если под этим понимать уменьшение числа жертв, действительно произошло, и весьма значительное. Можно говорить также о смягчении сталинского режима к 1952 году по сравнению с 1937‑м. Но все это является лишь следствием и свидетельством стабилизации тоталитаризма, Основные принципы, на которых зиждется новый строй, не меняются ни на йоту: полное бесправие личности и отсутствие элементарных гражданских свобод; бюрократическая систе­ма правления, при которой все решения обсуждаются и при­нимаются негласно; пресечение обмена информацией и идея­ми; массовая дезинформация населения средствами печати, радио и т.~д.; ложь и лицемерие, возведенное в норму общест­венной жизни; империалистическая внешняя политика. И те же тюрьмы для непокорных, разве что нет расстрелов. Впрочем, Юрий Галансков фактически убит в тюрьме. А сколько еще таких случаев, о которых мы ничего не знаем?


\section{Гибель полубогов}
\epigraph{Железный наш кулак сметает все преграды. Довольны Неизвестные Отцы!\ldots}{А. Стругацкий, Б. Стругацкий\cite{3}}


Другой характерной чертой перехода тоталитарного общества в стационарный режим является перенос центра тяжести пропаганды с поклонения конкретным людям -- героям, полубогам, которым мы обязаны нашей счастливой жизнью, на поклонение более абстрактным, но зато непрерывно воспроизводящимся понятиям: строй, партия, Центральный Комитет. Один американский журналист спросил меня как‑то: "А какие герои у советских детей? Кем их учат восхищаться в школе и кем они на самом деле восхищаются?" Оказалось, что я не могу толком ответить на этот вопрос. Я вдруг заметил, что у нас больше нет культа героев, который был характерен для времен моего детства. В тридцатые годы Валерий Чкалов был кумиром буквально каждого мальчишки в стране. Для нынешнего поколения с ним можно сравнить только Юрия Гагарина, но я уверен, что по глубине и искренности внушаемого им восхищения, а также по числу подражателей Чкалов намного опережает Гагарина. Да разве только Чкалов? А герои‑папанинцы? Я до сих пор помню эти четыре имени: Папанин, Кренкель, Федоров и Ширшов. А герои гражданской войны?

В тридцатые годы имена авиаконструкторов были известны всем, их популяризировали в качестве примера для подражания. А имя С.~П.~Королева -- руководителя нашей программы освоения космоса -- стало известно широкой публике только после его смерти. Сообщая о запусках спутников и вообще о продвижении космической программы, советские газеты упоминали таинственно о некоем "Главном Конструкторе" и о "Главном Теоретике" с больших букв. Формаль­но считалось, что это делается из соображений секретности. В действительности же зарубежным специалистам было прекрасно известно, что "Главный Конструктор" это С.~П.~Королев, а "Главный Теоретик" М.~В.~Келдыш. Но советской публике этого знать не полагалось. Пока советские лидеры были в глазах народа героями революции и гражданской войны, существование героев в других сферах деятельности не противоречило интересам системы. Однако на фоне безликих руководителей существование ярких фигур с большим автори­тетом таит в себе определенные опасности. В конце концов, С.~П.~Королев мог сказать любому члену Политбюро: "Я дал миру выход в космос. А ты кто такой?" Конечно, на самом деле он никогда так не сказал бы. Но уже возможность такого со­поставления вряд ли была бы приятна руководителям. Основным тезисом пропаганды, которая велась вокруг космической программы, было то, что успехи в космосе -- достижение советского строя, что это было возможно только в условиях социализма и только под руководством коммунистической партии и ее Центрального Комитета.

Когда в эпоху хрущевского либерализма я стал знакомиться с материалами по истории КПСС, я с удивлением узнал, что в 20‑е годы слово "вождь" часто, а быть может, и в основном, употреблялось во множественном числе: "вожди партии". Я родился в 1931 г., и я привык к тому, что вождь может быть только один: Великий и Мудрый Вождь всего прогрессивного человечества. Идея вождизма, возникшая и укоренившаяся в эпоху революции, сконцентрировалась ко времени моего детства и юности в одном человеке, стянулась в одну ослепительно яркую точку. Потом эта точка потухла. Вождей не стало, остались руководители.

Руководители стационарного тоталитарного государства представляются простому человеку единой, недифференцированной массой. Они произносят предельно стандартизованные, не отличимые друг от друга по стилю речи и никогда не выносят на обсуждение разногласий, которые между ними имеются. Быть может, среди них есть выдающиеся люди, быть может, и нет. Возможно, что они все одинаковые, возможно, что они все разные. Мы о них ничего не знаем и не должны знать -- по архитектуре нашей социальной системы. Мы должны только знать, что они являются средоточием и олицетворением "коллективной мудрости" партии, системы.



\section{Преодоление пережитков прошлого}

Происходит смена поколений, и дототалитарное время отодвигается в далекое прошлое. Еще 7‑8 лет назад мы были свидетелями коллективных протестов видных деятелей культуры, и в частности академиков, против реабилитации Сталина и сталинских методов. На официальном языке эти протесты с полным правом могут быть названы "пережитками капитализма". Протестовавшие были в большинстве людьми прежней формации, которые через ужасы сталинского времени пронесли веру в дототалитарные идеалы и возможность их осуществления. В то же время они занимали высокое положение, поэтому их коллективные выступления были серьезным общественным явлением, с которым нельзя было не считаться. С тех пор одни из них умерли, другие потеряли веру в то, что можно что‑нибудь сделать. Из числа первых "академиков‑подписантов" только А.Д.Сахаров продолжал идти по тому же пути. И вот в августе 1973 года мы увидели коллективное письмо совсем другого сорта, которое было напечатано во всех советских газетах, -- позорное письмо сорока академиков, осуждающее деятельность Сахарова.

Это письмо и последовавшая за ним клеветническая кампания против Сахарова открыли новую эру в истории советской науки. Сталин еще вынужден был прибегать к услугам "чужих" людей, если они были крупными специалистами своего дела. Человеку такого класса разрешали до известной степени оставаться самим собой, аппаратчики относились к нему по‑особому, как к некоей диковине или реликвии прошлого. Теперь все это кончилось. Наука полностью огосударствилась, она заняла то место, которое ей и полагается иметь в стационарном тоталитарном обществе. Новые академики -- люди тоталитарной психологии, прошедшие через частое сито государственного контроля. "Чужому" теперь в академики не попасть, у государства теперь больше чем достаточно "своих" кадров (хорошие они или плохие -- это другой вопрос). Весной 1968 года президент Академии наук СССР М.~В.~Келдыш сказал по поводу тех ученых, которые подписывали письма с протестом против политических репрессий и беззаконий: советская наука обойдется и без них.



\section{В целях дальнейшего совершенствования}

Необходимым условием стационарности является самовоспроизведение. Самовоспроизведение политической машины тоталитаризма было 
налажено еще Сталиным. Сейчас заканчивается налаживание самовоспроизведения тоталитаризма в культуре. Своеобразным измерителем 
этих процессов являются массовые кампании  против врагов (действительных или мнимых) тоталитаризма. Эти кампании для внешнего 
наблюдателя -- как землетрясение для геолога, по ним он может судить, что процессы формообразования в земной коре еще не 
закончены. Политические процессы 1937‑39 годов были последними крупными "землетрясениями"; с тех пор полити­ческие формы 
отвердели, и если что‑то и происходит, то лишь небольшие трещины и оползни -- главным образом, скрытые. Налаживание 
тоталитаризма в культуре отняло больше времени. Еще в последние годы Сталина мы видим массовые кампании против "менделистов", 
"космополитов" и т.~п. Затем кампании становятся все короче и уже по охвату. Возможно, что кампании против Солженицына и 
Сахарова в 1973г. -- последние в своем роде. Они выходят из моды, как и "перегибы".

Ибо контроль партийного аппарата над хозяйством и культурой захватил уже все уровни иерархии. Я имею в виду тот контроль, который не допускает проникновения на руководящие должности -- и вообще на сколько‑нибудь заметные места -- людей, способных бороться за свои убеждения и основные прав а личности, короче говоря, людей чужих с  точки зрения тоталитарного государства. Техника контроля достигла высокой степени совершенства и продолжает совершенствоваться. Время от времени партийные и правительственные органы издают постановления, которые обычно так и называются: "О мерах по дальнейшему совершенствованию\ldots" Вот, например, 9 ноября 1974 г. в "Правде" опубликовано изложение постановления ЦК КПСС и Совета Министров СССР "О мерах по дальнейшему совершенствованию аттестации научных и научно‑педагогических кадров". Здесь обращают на себя внимание два элемента. Во‑первых, Высшая аттестационная комиссия, присваивающая ученые степени и звания, переводится из ведения Министерства высшего образования в ведение Совета Министров. Следовательно, надзору за аттестацией научных работников придается увеличенное значение. Второй элемент виден из следующей цитаты:

"В практической деятельности Высшей аттестационной комиссии, советов высших учебных заведений и научных институтов рекомендовано установить незыблемое правило, чтобы к защите принимались только такие диссертационные работы, которые имеют научную и практическую ценность, а соискате­лями ученых степеней утверждались лица, положительно проявившие себя на научной, производственной и общественной работе".

Так что будь ты хоть Исаак Ньютон, а если не \textit{проявил себя положительно} на общественной работе, -- кандидатом наук тебе не бывать.

И не только кандидатом наук. Потенциальные нонконформисты вылавливаются с юных лет, им не дают получить приличное образование. За малейшее проявление инакомыслия студентов исключают из институтов, а затем презрительно именуют "недоучками". Недоучки! С каким вкусом газетные писаки и партийные работники -- сами люди полуобразованные -- произносят это слово! Недоучка Амальрик, недоучка Буковский\ldots

Однажды, когда я еще работал в солидном академическом институте, я организовал у нас выступление историка и философа Г.С. Померанца, известного самиздатовского автора, произведения которого ходили по всему Союзу. Г.С.Померанц -- очень интересный мыслитель, к тому же яркий и увлекательный эссеист. Он выступил на философско‑методологическом семинаре института. Зал был полон, доклад вызвал большой интерес. Однако выступление самиздатовского автора на институтском семинаре не прошло незамеченным. Кто‑то нажаловался в партбюро на "неправильную идеологическую линию" на семинаре, и я был призван к ответу. На заседании партбюро, куда я был приглашен, особенно суетился один малозаметный в институте человечек -- он не был, собственно говоря, научным работником, а что‑то где‑то около. Не был он и членом партбюро. Позже мне сказали, что он‑то как раз и настучал в партбюро по поводу семинара. Несколько раз человечек приступал ко мне с вопросом: а кто Померанц -- кандидат или доктор? В конце концов я был вынужден ответить, что Померанц -- не кандидат и не доктор, а просто работает библиографом в Фундаментальной библиотеке общественных наук.

Боже! Надо было видеть ту смесь негодования и презрения, которая выразилась на лице этого ничтожного стукача. "Как? -- воскликнул он. -- Даже не кандидат?"

Г.~С.~Померанц написал свою первую большую работу, которую хотел представить как диссертацию, незадолго до начала войны. Затем была война. Затем его посадили. Вскоре после выхода из лагеря Г.~С.~Померанц стал известным самиздатчиком, и ему просто не дали возможности защититься. Он написал новую диссертацию и представил ее в один институт, однако диссертацию вернули, не потрудившись даже подыскать приличного оправдания.

В XX веке уважение к ученым и к учености велико, велико уважение и к другим профессиональным достижениям. Но для неспециалиста (а в любой заданной области подавляющее большинство граждан -- неспециалисты) удостоверением профессионального уровня является признание государства. А государство знает, кого удостоверять, а кого -- нет.

Однако дело не только в удостоверении. Современное общество характеризуется столь высокой степенью интеграции, что почти никакое серьезное достижение невозможно в одиночку, без сотрудничества с какими‑то институционализированными коллективами людей. Пожалуй, только писатель и может работать один. Даже математик нуждается в наши дни в доступе к вычислительной машине. А можете ли вы вообразить физика‑любителя, который в свободное от основной работы время бежит к ускорителю, чтобы исследовать соударения элементарных частиц? Архитектор -- не архитектор, пока спроектированное им здание не построено, а кинорежиссер -- не режиссер, пока его фильм не вышел на экраны.

Рост профессионального уровня в современных условиях неизбежно требует одновременного повышения в служебной иерархии. Практически в любой сфере деятельности человек, желающий осуществить свои творческие замыслы и имеющий необходимые способности и опыт, должен руководить хотя бы небольшой группой людей. И здесь тоталитарное государство ставит его перед трудным выбором: или принять "причастие буйвола", выражаясь словами Г.Белля, или отказаться от своих планов и профессионального роста. Ибо, согласно основному принципу нашего государства, руководитель в любой сфере деятельности должен не только руководить работой подчиненных, но и \textit{воспитывать} их. Этот принцип бесконечно повторяется и подчеркивается в партийной теории и служит основой практической политики партии в хозяйстве и культуре.

Для воспитания подчиненных вовсе не требуется, чтобы начальник читал им лекции по марксистско‑ленинской теории. Нет, от него ожидается совсем другое. Воспитывать подчиненных -- значит подавать им пример угодного властям поведения. Как минимум, это включает гарантированное молчание по пово­ду тех вопросов, по которым велено молчать, и, конечно, беспрекословное исполнение "рекомендаций" партийных органов. Если вы выполняете это требование, вы можете подняться на первую ступеньку служебной лестницы. Это же условие необходимо для получения знаков отличия любого рода: премий, наград и т.~п.
Наступление тоталитаризма на культуру шло, начиная с Октябрьской революции, сверху вниз, то есть "чуждые элементы" оттеснялись на все более низкие уровни иерархии. Теперь этот процесс, по‑видимому, пришел к естественному завершению: стерилизована самая низшая ступенька лестницы, если, конечно, не считать тех лиц -- того большинства лиц, которые совершенно никем не руководят. Во всяком случае, по опыту в научных учреждениях могу сказать, что "чуждые элементы" еще могут занимать должности младших или старших научных сотрудников, но ни в коем случае -- заведовать лабораторией или сектором, или любой другой структурной единицей. Руководитель должен воспитывать своих подчиненных.

Так и осуществляется самовоспроизведение тоталитаризма в культуре. Руководители воспитывают себе подобных.
Дилемму -- совесть или работа, которую тоталитарное общество ставит перед человеком творческой профессии, каждый решает по‑своему. Большинство тех людей, которых называют порядочными, частично жертвуют работой, частично совестью. Они стараются свести к минимуму свое касательство к социальным проблемам, стараются ограничиться чисто профессиональными аспектами деятельности и чисто профессиональными контактами. Те же, кто не сохранил и капли порядочности, ничем не гнушаются для продвижения вверх. Иногда, например, они действуют в отвратительных инсценировках, где они якобы "свободно и откровенно" обмениваются мнениями с представителями Запада по вопросам политики и идеологии.


\section{Уровни лишения свободы}

\epigraph{Последнее открытие Государственной Науки:
центр фантазии -- жалкий мозговой узелок в области варолиева моста. Трехкратное прижигание этого узелка Х‑лучами -- и вы 
излечены от фантазии -- навсегда.
Вы -- совершенны, вы -- машиноравны, путь к стопроцентному счастью свободен. Спешите же все -- стар и млад -- спешите 
подвергнуться Великой Операции!}{Евг. Замятин, "Мы"}

Человека можно ограничить физически: например, приковать к галере или посадить за решетку. Это -- лишение свободы на самом 
низшем уровне. Человека можно ослепить, и тогда он, формально свободный, будет вынужден довериться поводырю. Это -- лишение 
свободы на более высоком, информационном уровне. Наконец, можно сохранить человеку все органы чувств и, следовательно, полную 
способность получать информацию о внешнем мире, но путем операции мозга или химических препаратов трансформировать его сознание, 
парализовать волю. Это -- лишение свободы на высшем уровне, которое неопытный наблюдатель не всегда и заметит. И он заведомо 
ничего не заметит, если раньше никогда не видел нормального, не оперированного человека.

Через соответствующие стадии проходит и тоталитаризм в своем наступлении на общество. Он движется снаружи внутрь, захватывая все 
более глубокие слои общественного бытия и уродуя все более высокие уровни организации живой материи.

В первые годы после захвата власти большевистский ре­жим опирался исключительно на насилие ("революционное"). В вопросах распространения информации проявлялось недо­пустимое, с точки зрения позднейших времен, легкомыслие. Издавались, например, воспоминания участников гражданской войны, сражавшихся против Красной Армии. Считалось, что "сознательный рабочий" отделит (с помощью предисловия) интересные исторические факты от злобных вымыслов врага. Члены партии имели довольно полную информацию о событиях на верхних уровнях иерархии, и уж, конечно, никому не приходило в голову ограничивать доступ к произведениям "буржуазной" философии. Напротив, считалось, что "врагов нужно знать".

В дальнейшем, однако, выяснилось, что врагов лучше не знать. Не следует также знать, что происходит в высших сферах, в местах заключения и во многих других местах. Так оно спокойнее. Было создано информационно закрытое общество, и его расцвет, его взлет приходился на последние годы жизни Сталина. Сотни тысяч осведомителей следили за каждым словом граждан. Все книги, имеющие хотя бы отдаленное отношение к политике, социологии или философии и написанные "с чуждых позиций", попали в спецхран. Контакты с иностранцами были сведены к минимуму и были возможны только под строгим контролем Государственной Безопасности. Само слово "иностранец" у простого человека вызывало страх; ассоциации, которые при этом возникали, были: шпионаж, органы, Лубянка.
Это была та стадия развития тоталитарного общества, когда основной упор делался на информационный уровень, а число физически уничтожаемых людей начинает уменьшаться. Конечно, сознание члена общества уже сильно трансформировано, но власти еще не очень этому верят. Поэтому они панически боятся информации, знания.  Цинизм, необходимый для стационарного тоталитаризма, еще не выработался окончательно, еще не вошел в плоть и кровь общественного сознания. Считалось, что люди не знают о  миллионах невинных жертв, о чудовищной мясорубке ГУЛага. Многие, действительно, не знали -- разумеется, потому, что не хотели, боялись знать. Существовало нечто вроде негласного соглашения между властью и гражданами: власти создают информационные барьеры, а граждане радуются, что они могут как бы "не знать".

На третьей, заключительной стадии тоталитаризма упор делается на поддержание тоталитарного сознания  членов об­щества. Эта стадия предполагает, что трансформация сознания закончена, воля к свободе полностью подавлена. При этом возникает возможность дальнейшего сокращения масштабов физического насилия и частичное (только частичное!) открытие информационных каналов -- к вящей радости "сытых, благожелательных иностранцев с блокнотами и шариковыми ручками в руках" (А.~Солженицын).

В 1956 году старые "сталинские соколы" возражали про­тив разоблачения преступлений Сталина, ибо они боялись, что если люди \textit{узнают и признают, что они узнали,}  то это нарушит равновесие и может привести к далеко идущим последствиям. Хрущев же, который делал личную карьеру на разоблачении Сталина, считал, что система достаточно прочна и ничего страшного не произойдет. И он, в общем, оказался прав. Люди \textit{узнали} -- и  ничего. Советский человек, выражаясь словами Шолохова, "выдюжил". Крупные неприятности произошли только в Восточной Европе, где люди не имели нашей выучки. Двенадцать лет от подавления Венгерского восстания 1956 года до вторжения в Чехословакию в 1968 году были, в сущности, переходным периодом к третьей стадии тоталитаризма. Их можно сравнить с двенадцатью годами с 1922 по 1934 гг., ко­гда совершался переход от голого насилия к информационно закрытому обществу.



\section{Принять или не принять?}

Переходные периоды отличаются от периодов застоя тем, что все‑таки что‑то происходит, что‑то меняется. При этом перед человеком встает проблема определения своего места в происходящих переменах. Вскоре после революции 1917 го­да русской интеллигенции стало ясно, что большевистский строй -- это не тот строй, о котором она мечтала и которого ждала от революции. И встала проблема: принять или не при­нять? Из тех, кто не эмигрировал во время гражданской вой­ны, большая часть приняла; остальные были репрессированы, небольшая часть сохранилась в виде "внутренней эмиграции". Но так как террор был страшный, осталось ощущение, что и политический режим, и образ мышления навязаны насильно. И я считаю, что можно без колебаний утверждать: так оно и было.

Реабилитация людей -- жертв сталинского террора -- потя­нула за собой частичную реабилитацию идей. Стали выходить книги, которые раньше были запрещены. В среде интеллиген­ции стали говорить о необходимости демократизации, глас­ности, более свободного обмена информацией. Рано или позд­но это должно было вызвать социальное движение. Так оно и случилось. Была найдена подходящая форма -- послание петиций высшим органам власти. Это, собственно говоря, была и есть единственная форма открытого политического движения, возможного в советских условиях. Людей, подписавших такие петиции, окрестили "подписантами".

В течение некоторого времени число "подписантов" увели­чивалось, и казалось, что движение может захватить широкие слои интеллигенции. Тогда власти стали принимать ответные меры. По сравнению со сталинским временем эти меры были смехотворны: обсуждение и осуждение на собраниях, отмена заграничных командировок, понижение в должности, иногда -- увольнение с работы. Для членов партии (были и они в числе подписантов) -- выговоры, а для упорствующих -- исключение из партии. Многим просто ничего не было, они только были взяты на заметку. Я, например, хоть и подписал несколько коллективных протестов, в то время вообще избежал непри­ятностей.

Однако и этих мер оказалось достаточно, чтобы остановить движение. Максимум "подписантства" приходился на фев­раль‑март 1968 года. К лету того же года, еще до вторжения в Чехословакию, стало уже ясно, что подписантство быстро идет на убыль. 21‑е августа только закрепило победу, одер­жанную тоталитаризмом. Подписанты оказались в меньшинст­ве -- в ничтожном меньшинстве; основная часть интеллиген­ции не поддержала их. Советский интеллигент снова был по­ставлен перед проблемой выбора -- и он снова сделал выбор в пользу тоталитаризма, однако на этот раз -- под несравненно меньшим давлением.

При Сталине рот был заткнут чудовищной, неумолимой силой. Идти хоть в чем‑то против означало почти верную и почти немедленную смерть. Активистам демократического Движения казалось, что теперь, когда можно что‑то делать для восстановления основных прав личности, люди должны ухватиться за эту возможность и движение -- хотя бы в среде интеллигенции -- должно разрастись лавинообразно. Это было заблуждение, оно обнаруживалось в процессе сбора под­писей, и у многих активистов опускались руки. Великого Стра­ха сталинских времен уже не было. Но работала инерция стра­ха. Не зря трудились основатели нового строя. Страх не про­шел бесследно. Он затаился в тайниках сознания, он изуродовал души, изменил представления о нравственных ценностях, о добре и зле.

Интеллигенция больше, чем любой другой слой общества, нуждается в основных демократических свободах и страдает от их отсутствия; они нужны ей профессионально ‑ для выполнения своей общественной функции. Поэтому интеллигенция и должна в первую очередь заботиться о соблюдении прав личности, добиваться их. Она отвечает за них перед обществом в целом. Бороться с предрассудками и обскурантизмом, добиваться интеллектуальной и духовной свободы ‑ такой же прямой долг интеллигента, как долг врача -- следить за здоровьем людей.

Я уже говорил о той ситуации выбора, в которой находится человек творческого труда в тоталитарном обществе, когда требования совести противоречат интересам работы. Однако эта ситуация не влечет с необходимостью ту пассивность интеллигенции, которую мы видим вокруг. Ситуация выбора и необходимость компромисса всегда присутствовали и будут присутствовать в любой сфере жизни; сама жизнь -- это непрерывный компромисс. И если бы интеллигенция обладала твердым желанием выполнять свой долг, идя, когда это необходимо, на компромиссы, то общественно‑политическая атмосфера в стране была бы совсем другой. Но такого желания нет. Почему? В следующем разделе я воспроизвожу свой ответ на этот вопрос, содержащийся в первом варианте "Инерции страха" (осень 1968 года).


\section{Философия коровы, догматический пессимизм и другие теории}

Страшным результатом сталинского террора было не только физическое уничтожение людей, но и дегуманизация оставшихся в живых, потеря ими человеческого облика. В той или иной степени этот процесс затронул каждого, а благодаря взаимодействию между поколениями повлиял и на молодежь, не заставшую сталинских времен. И самое печальное, что мы привыкли к этому, смирились с этим, нашу исковерканную психику мы принимаем за норму.

Когда бандит наводит дуло револьвера на безоружного че­ловека, тому на выбор предоставляются только две возмож­ности: подчиниться или умереть тут же на месте. Никто не может осудить того, кто подчинится в таких условиях. Но вот бандита нет. Не пора ли начать принимать человеческий облик?

Мы так привыкли к массовой, систематической лжи, что считаем ее не только вполне дозволенной, но даже как будто совершенно естественной и необходимой для поддержания общественного порядка. Мы считаем вполне нормальным гово­рить дома и с друзьями одно, а на людях -- совсем противопо­ложное, и мы учим этому своих детей. Нам нисколько не стыд­но проголосовать на собрании за решение, которое мы счита­ем неправильным, и тут же, выйдя из зала, поносить это реше­ние. Мы не считаем позором и предательством не выступить в защиту несправедливо обвиненного товарища. Порой совесть требует от нас сущего пустяка, но мы отказываем ей и в этом. Мы трусливы и беспринципны.

На какие только ухищрения мы не пускаемся, каких только жалких доводов не изобретаем, чтобы оправдать себя в своих собственных глазах и в глазах своих друзей! Многие представители научной интеллигенции укрываются за гениально простой отговоркой: это не мое дело; мое дело -- наука, все остальное меня не касается, и я буду делать что угодно, лишь бы мне не мешали; так я принесу максималь­ную пользу обществу. Это философия коровы, которая умеет только давать молоко и готова давать его кому угодно.

Рассуждая так, ученый претендует на несомненную исключительность своей профессии или своей личности. И действительно, такая 
претензия, когда она исходит от ученого, вы­глядит как будто более основательно, чем если бы она исхо­дила от представителей 
других профессий. Мы часто ставим науку на выделенное место, придавая ей высшую ценность, независимую от остальных ценностей, 
связанных более непо­средственно с общественной жизнью. На начальных стадиях развития науки и, может быть, еще в прошлом веке 
такую точку зрения можно было считать в какой‑то степени оправ­данной. Тогда наука развивалась более или менее автономно, ей 
нужно было еще накопить силы, чтобы стать важным общест­венным явлением. Веря, что в конечном счете наука преобра­зует общество к 
лучшему, копить силы было, возможно, такти­чески правильно. Но сейчас наука уже неразрывно связана со всей общественной жизнью; 
общие, глобальные проблемы науки неотделимы от проблем общества. Жертвовать решением общих проблем, чтобы несколько быстрее 
решить отдельную узкую проблему, -- можно ли защищать такую точку зрения, не занимаясь самообманом? За подобными взглядами 
обычно кроются чисто эгоистические соображения\ldots

Так обстоит дело, если рассуждать только об интересах науки. Но проблемы общественной жизни -- это не только проблемы науки, это проблемы счастья и горя, жизни и смерти многих людей. Неужели это менее важно, чем вопрос о том, будет ли, допустим, осуществлен данный эксперимент на год раньше или на год позже?

Наконец, само противопоставление заботы об общественных проблемах деятельности в своей узкой области -- не оправдано. Тот, кто желает выполнить свой общественный долг, всегда найдет для этого способы, не ударяющие фатально по его ра­боте. Философия коровы нужна тем, кто хочет уклониться от выполнения долга, откупившись от него молоком. Кстати, и молоко‑то у этих людей бывает чаще всего жидкое\ldots

А что сделаешь? ‑ говорит другой интеллигент. -- Кругом ложь и подлость. Высунешься -- стукнут, только и всего. Ни­чего не изменится, разве только в худшую сторону. Нет уж, лучше сидеть и помалкивать\ldots
Напрасно вы будете указывать ему на перемены, которые непрерывно происходят в нашей жизни, напрасно будете спра­шивать, как -- по его представлениям -- вообще происходит в мире прогресс, он будет либо мазать все черной краской, либо уходить от разговора. Его пессимизм -- догма, которую он вовсе не намерен подвергать сомнению и которая поэтому абсолютно неуязвима. Убеждение, что ничего сделать нельзя, необходимо ему для самооправдания. Когда побеждают силы разума и добра, он только пожимает плечами; но зато каж­дый раз, когда берет верх зло и невежество, он не упускает случая позлорадствовать:

--- Вот видите! Я же говорил! Я же предупреждал! И он радуется, что его не удалось "спровоцировать" на "не­обдуманный" поступок\ldots

Есть еще теория, которую можно назвать "Сама пойдет\ldots"

--- Ну что вы! --- говорит проповедник этой теории. -- Конеч­но, улучшение происходит. Только медленно, под действием объективных законов. Его нельзя ни остановить, ни ускорить. Надо просто ждать. Придет время, все будет хорошо\ldots

Как будто это улучшение происходит само собой, каким‑то мистическим образом, без всякого участия людей! Как будто этим улучшением мы не обязаны как раз тем людям, которые отдают ему свою энергию, здоровье, жизнь!

Да, можно просто ждать. Можно упасть в воду и, даже не барахтаясь, ждать пока тебя вытащат. Возможно, что в конце концов и вытащат, -- свет не без добрых людей. А возможно, что и не вытащат: не потому, что не захотят, а потому, что не смогут.

Есть еще много отговорок, помогающих интеллигенту уклоняться от поступков, которых требует совесть. Высокопоставленные говорят, что вот‑де хорошо <<простым людям>> -- им нечего терять. "Простые люди" говорят: хорошо высокопоставленным -- их не тронут. Один как раз заканчивает диссер­тацию, другой не хочет "подвести" начальника, третий боится сорвать заграничную командировку. Молодой считает, что он слишком молод, старый -- что он слишком стар.

Все эти оправдания не стоят ломаного гроша, они рассыпа­ются при первой же попытке серьезного и честного размышле­ния. Но интеллигент и не хочет размышлять серьезно и честно, он предпочитает сохранять эти декорации, прикрывающие его страх и глубокий душевный надлом. И он осторожно пробирается между ними, чтобы не задеть и не разрушить их нечаянно. А к тем, кто разрушает декорации, кто подает пример честного и мужественного поведения и ставит интеллигента перед нравственным выбором, он испытывает порой настоящую ненависть, ибо такие люди нарушают его покой. Он не только лжет и боится, он не хочет перестать лгать и бояться -- так он привык, так ему удобнее и спокойнее. Это -- глубочайшее нравственное падение.

Нравственность. Совесть. Честь. Странное у нас отношение к этим понятиям. Нельзя сказать, что мы отрицаем их вовсе, относя к буржуазным предрассудкам. Нет. Но мы считаем эти понятия какими‑то несерьезными, старомодными, "немарксистскими". Дескать, выплавка чугуна и стали -- это серьезный фактор, это -- базис,  это важно для общества. А всякая там нравственность -- это так, надстройка\ldots  Однако бывает, что с выплавкой чугуна и прочим "базисом" дело обстоит более или менее благополучно, а препятствием для нормального развития общества является именно массовое забвение некоторых элементарных нравственных принципов.
Именно так и обстоит дело у нас. Если заграничные тряп­ки для нас дороже чистой совести, если, боясь неприятностей по службе, мы способны предать товарища, -- мы недостойны называться людьми и не заслуживаем человеческой участи.

Мы вовсе не должны требовать друг от друга какого‑то героизма, какого‑то необыкновенного мужества. Но разве не мы сами являемся источником лжи и лицемерия! Разве честные люди не могли бы договориться между собой просто быть честными, если бы они этого действительно хотели? Разве мы делаем хотя бы то, что вполне в наших силах? Нет, мы предпочитаем находить отговорки, мы предпочитаем без остатка погружаться в мелочные эгоистические заботы и не думать о долге и совести, о жизни и смерти, мы предпочитаем голосовать за неправду и, возвратясь домой, спокойно лгать своим детям, что мы -- честные люди.


\section{Морально‑политическое единство}

\epigraph{Последний советский гражданин, свободный от цепей капитала, стоит головой выше любого зарубежного высокопоставленного чинуши, влачащего на плечах ярмо капиталистического рабства.}{И. Сталин\cite{4}}


Годы, последовавшие за 1968‑м, были годами поляризации. Выкристаллизовалась категория людей (численно крошечная), которые отказались принять основной принцип тоталитариз­ма: подавление прав личности; их окрестили пришедшим с Запада именем "диссиденты". Остальные вернулись в лоно тоталитарного большинства. Общество безмолвно и безучаст­но наблюдает, как власти расправляются с диссидентами.

Я помню, как осенью 1972 года я обращался к одному известному астрофизику Икс с просьбой помочь астроному Крониду Любарскому, над которым вскоре должен был состоять­ся суд. К.~А.~Любарский, автор нескольких десятков научных работ и бывший председатель Московского астрономическо­го общества, был арестован за участие в издании "Хроники те­кущих событий". От академика Икс я просил немногого: напи­сать для суда характеристику Любарского как ученого. Я знал по опыту других политических процессов, в частности по про­цессу математика Р.~И.~Пименова, что такая характеристика, подписанная видным ученым, академиком, способствовала бы смягчению приговора. Икс принадлежал к числу первых академиков ‑ подписантов, и это давало мне основания наде­яться на положительный результат моей миссии. Однако на­деждам не суждено было оправдаться. Времена переменились. Академик отказался написать характеристику.

Кронид Любарский был осужден на пять лет строгого ре­жима. Он -- тяжело больной человек, у него ампутирована большая часть желудка. Каждый день пребывания в тюрьме для него -- пытка.
Недавно произошел случай, незначительный сам по себе, но очень характерный для атмосферы последних лет. Профес­сор Ю.Ф. Орлов, физик, член‑корреспондент Армянской Академии наук, позвонил академику Игрек, тоже физику, чтобы  договориться о встрече и обсудить какие‑то вопросы. Увы,

Ю.~Ф.~Орлов известен не только как физик, но и как один из наиболее активных диссидентов. Едва Игрек понял, кто с ним говорит, и не успев еще услышать, о чем тот хочет разговаривать, он поспешил предупредить:

--- Только учтите, проблемы поступательного движения человечества меня не интересуют.

А ведь Игрек тоже подписал некогда письмо‑протест против применения сталинских методов. Однако нынче диссиденты не в моде. Ведь "все равно ничего не сделаешь". Серьезные люди этим не занимаются. Будет ли человечество двигаться вперед или назад -- академику это безразлично. Он заботится только о том, чтобы его не втянули в разные там выступления в защиту невинно заключенных и прочую чепуху.

По поводу отдельных случаев ‑‑ которых любой из диссидентов может привести великое множество -- всегда есть возможность возразить, что это все‑таки отдельные случаи, а не статистика. Но вот есть форма общественной активности, где статистика набирается автоматически, сама собой: это голосование на собраниях и заседаниях коллективных органов, таких как ученые советы научных институтов. Ленинградского литературоведа Эткинда уволили с работы в результате того, что КГБ довело до сведения руководства свое мнение о его неблагонадежности. Для увольнения человека, занимающего конкурсную должность в научном институте, необходимо решение ученого совета института, которое принимается путем тайного голосования. Такое решение и было принято, причем члены ученого совета высказались за увольнение Эткинда единогласно.  Иностранные корреспонденты в Москве, передавая сообщение об этом за границу, очень удивлялись: как это так, что из пятидесяти с чем‑то человек не нашлось ни одного, кото­рый при тайном голосовании  оказался бы против увольнения по политическим причинам? Как это может быть?

Может быть, господа, может быть. Не зря советская пропа­ганда кричит на весь мир о морально‑политическом единстве советского общества. Она, конечно, производит подтасовку, смещает смысл слов, но все же не врет напропалую. В своем пренебрежении к правам личности, отсутствии чувства собст­венного достоинства, в своем раболепии перед властью совет­ское общество едино: и морально, и политически.

Мне говорили (не знаю, насколько это верно), что члены ученого совета по какой‑то причине имели зуб на Эткинда, и это отчасти объясняет (хотя, разумеется, никак не оправды­вает) результат голосования. Но вот перед моими глазами другой случай. В одном научном институте работал физик А. ‑ человек большой доброты и безукоризненной честности, с ру­мяным круглым лицом ‑ живой символ русского доброду­шия и общительности. Я думаю, во всем институте не было и одного человека, который был бы настроен к нему недобро­желательно. И вот А. "провинился". Будучи членом КПСС, он написал письмо в высшие партийные инстанции, которое этим инстанциям не понравилось. Обратите внимание, А., в сущ­ности, лишь выполнял устав КПСС, согласно которому комму­нист должен сообщать вышестоящим инстанциям о тех дейст­виях нижестоящих инстанций, которые с его точки зрения являются неправильными. К тому же письмо, как и полагает­ся, было закрытое. Однако дело было в конце 1968 г., и в институте было то, что на партийном языке называется "слож­ная обстановка". Поэтому сверху спустили инструкцию: осу­дить вольнодумца. Ладно. В два счета провели партсобрание, осудили, вынесли строгий выговор. Потом собрали ученый со­вет. Сидят физики и думают: дело дрянь, обстановка сложная, надо А. наказать, а то как бы чего не вышло. Вносится предло­жение: перевести его с должности старшего научного сотруд­ника на должность младшего научного сотрудника. Это вле­чет понижение зарплаты более, чем на треть (а у А. жена и двое детей).
Лишь один человек счел предложение несправедливым -- заведующий лабораторией Б. Встает он и говорит: зачем же понижать в должности? Ведь работает он прекрасно. Никаких претензий у нас здесь нет. Вынесли выговор по партийной линии, и хватит.
Тут все зашумели, замахали на Б. руками. Дескать, как можно так рассуждать, надо думать об интересах института и т.~д. Перешли к тайному голосованию. Результат: единогласно за понижение в должности. Б. тоже голосовал за:  ведь если был бы один голос против, ясно было бы, что этот голос его. Впрочем, это его не спасло: через некоторое время Б. был переведен с должности заведующего лабораторией на должность старшего научного сотрудника.

Репрессии против людей творческих профессий -- ущемление или полное изгнание с работы -- проводятся, как правило, вполне законным, "демократическим" путем. Общество само оскопляет себя, и эта традиция самооскопления передается следующему поколению.
Из института, в котором я работал до увольнения летом 1974 г., был уволен незадолго до меня кандидат технических наук А.М. Горлов. Вся его вина состояла в том, что он был знаком с А. Солженицыным и однажды, приехав к нему на дачу в отсутствие хозяина, нарвался на сотрудников КГБ, которые не то делали там обыск, не то устроили засаду. Горлова как следует избили и пригрозили, что если он не будет молчать, то ему будет худо. Горлов, однако, молчать не стал, и дело получило огласку. Естественно, у него начались неприятности на работе, и в конце концов он был уволен путем тайного голосования. Горлов проработал в этом институте 15 лет.



\section{Личный опыт}

Мой личный опыт также содержит немало поучительного.

Когда в конце августа 1973 года началась клеветническая газетная кампания против академика Сахарова, я выступил с кратким заявлением в его поддержку. Через неделю я уже присутствовал на общем собрании сотрудников института, созванном с целью осудить мой недостойный поступок. Часть выступавших восклицала, что не может быть такой человек, как я, заведующим лабораторией и руководить людьми (в течение шести месяцев я, действительно, по недосмотру властей в процессе перехода из одного института в другой, был -- о чудо! -- заведующим лабораторией). Другие ораторы требовали вообще изгнать меня из института, утверждая, что мне не место в их здоровом коллективе. Одна пожилая женщина патетически воскликнула:

--- Я вот смотрю, к нему все студенты ходят. Разве мы можем, товарищи, доверить такому человеку воспитание наших детей?
Резолюция, осуждающая мое поведение, была принята единогласно. Из трехсот человек, присутствовавших на собрании, ни один не проголосовал против или хотя бы воздержался.

Вскоре после этих событий из издательства "Советская Россия", где готовилась к выходу моя книга "Феномен науки" (уже был начат набор), мне сообщили, что работа над книгой остановлена "из‑за нехватки бумаги". Рукопись другой моей книги "Программирование на языке РЕФАЛ", которая была сдана в издательство "Наука", находилась в это время у рецензента. Рецензент вернул рукопись в издательство, заявив, что он считает себя "морально не вправе" рецензировать книгу автора с таким политическим лицом.


\section{Морально‑политическое единство\ldots}


Осуждение моего выступления на собрании в институте производилось открытым голосованием. Но есть у меня опыт и тайного голосования.

В сентябре 1973 г., вскоре после собрания, была расформиро­вана моя лаборатория, но я был оставлен в институте в долж­ности старшего научного сотрудника. В течение последовавших месяцев я спокойно работал и надеялся, что мне удастся совместить свое диссидентство с продолжением профессиональ­ной деятельности. Не тут‑то было. Весной 1974 г. подошло время моего утверждения ученым советом института в занима­емой должности. Сначала вопрос о моем утверждении был от­ложен: вероятно, начальство не знало, как поступить, и запра­шивало инструкций. Затем было дано указание "треугольнику" отдела составить на меня характеристику.
В деловой части характеристики, написанной в отделе, были только хорошие слова по моему адресу. Но последний абзац характеристики гласил:

"В то же время, будучи близко связанным с академиком Сахаровым, В.Ф. Турчин сделал в сентябре 1973 г. заявление для представителей буржуазной прессы, в котором оправды­вал поведение Сахарова. Этот поступок В.Ф. Турчина был еди­нодушно осужден сотрудниками института".

В июле 1974 г. состоялось заседание ученого совета инсти­тута, на котором рассматривался, в частности, вопрос о моем утверждении в должности. Председательствовавший на совете заместитель директора не стал зачитывать деловую часть харак­теристики, а прочитал только заключительный абзац, начиная со слов "В то же время\ldots" Затем он выразил надежду, что члены ученого совета "сделают соответствующие выводы" из зачи­танного абзаца. Больше по этому поводу не было сказано ни слова. Моя работа не обсуждалась. Хотя за месяц до этого на заседании отдела было принято решение рекомендовать учено­му совету института утвердить меня в занимаемой должности, заведующий отделом, который присутствовал на заседании как член ученого совета, не счел необходимым встать и объя­вить об этом. Результат тайного голосования был таков: 5 за утверждение, 19 -- против. Так я вылетел из института.
Один из моих знакомых, когда я ему рассказал об этих со­бытиях, воскликнул:

--- Ого! Пять человек из двадцати четырех голосовали про­тив. Это -- дай‑ка мне линейку -- почти $21\%$. У вас на редкость порядочные люди в институте!

После увольнения я пытался устроиться в несколько науч­но‑исследовательских институтов, но безрезультатно. Схема была всегда одна и та же: заведующий лабораторией хотел меня взять, но когда вопрос поднимался на уровень партбюро и дирекции института, ответ неизменно был отрицательным. Иногда мне говорили: "Вот если бы вы дали обещание вести себя\ldots иначе, тогда еще можно было бы попытаться вас устро­ить". А в одном институте человек, который хотел меня взять, объявил мне с унынием в голосе, что не только директор, но и несколько человек, которые были заинтересованы во мне как в специалисте, сказали, что они, тем не менее, против моего приема. Они не хотят неприятностей.

Повествование об академиках я закончу следующей исто­рией. Один мой знакомый спросил академика Зет, не может ли он помочь мне 
устроиться на работу. С академиком Зет мы не только знакомы более пятнадцати лет, но даже имели сов­местные работы. Академик 
ответил кратко и ясно:

Нет. Эти люди идут против общества. Такие дела.


\section{Отщепенцы}
\epigraph{Я -- непризнанный брат, отщепенец в народной семье\ldots}{О.~Мандельштам}

У меня нет иллюзий: мой конфликт -- не только и, пожа­луй, даже не столько, конфликт с властями, сколько конфликт с обществом. Я хочу примерно того же и смотрю на вещи при­мерно так же, как люди круга, к которому я принадлежу. Это ‑ конфликт ценностей. Но именно система, иерархия цен­ностей -- что мы считаем более, а что менее важным -- опреде­ляет в конечном счете наши поступки; и от нее зависит, оказы­ваемся ли мы с большинством или попадаем в отщепенцы, диссиденты.

Не то чтобы я возражал против отщепенства как такового. Отщепенцы нужны каждой стране и человечеству в целом. Нужны люди, которые ведут себя не так, как большинство -- экспериментируют на себе (и -- увы! -- на своих близких). Без этого не было бы развития, движения вперед. Все новое бывает сначала в меньшинстве. В конце концов и доказательст­во теоремы зарождается в одной голове, прежде чем стать при­знанным фактом. Для меня диссидентство -- часть моей жиз­ненной задачи, как и научная работа.

Само по себе наличие отщепенцев -- вещь естественная. И яс­но, что люди, нарушающие общественные нормы поведения, не могут рассчитывать на легкую жизнь в своем обществе -- это, опять‑таки, естественно. Но неестественно и противоестествен­но другое: та линия, которая отделяет советских диссидентов от общества. Ведь для того, чтобы попасть в отщепенцы, доста­точно просто отказаться от лжи (хотя бы по умолчанию), до­статочно один раз заступиться за невинного человека, которо­го терзают на твоих глазах. А для того, чтобы попасть в полу диссиденты, в неблагонадежные, достаточно еще меньшего: живое слово, отказ от активного, систематического мракобе­сия. И это -- в XX веке, в Европе, после того как основные принципы гуманизма и права личности давно признаны, казалось бы, цивилизованным миром.

Гуманистов XVI века поддерживало, вероятно, чувство, что они -- первооткрыватели, прокладывающие дорогу к ново­му общественному порядку. Советские диссиденты всего лишь призывают помнить об уже открытых, ставших азбучными, истинах. Они всего лишь обороняются от наступающей тьмы.


\section{Все все знают}

Один американец, с которым мы обсуждали влияние технического прогресса на общественную жизнь, сказал мне:

--- У нас считают, что технический прогресс в Советском Союзе имеет важное значение для демократизации общества. Возьмем такой пример. Сейчас у вас в стране производится недостаточно автомашин. Их производство будет увеличиваться. Тогда любой человек сможет сесть на машину и проехаться по стране. Он тогда увидит, например, что во многих городах нет мяса и что вообще газеты пишут неправду.

Это соображение меня, признаться, рассмешило. Вовсе не обязательно иметь автомашину, чтобы знать, что газеты пишут неправду. У каждого есть глаза и уши, а также родные и знакомые в разных частях страны. И, в сущности, все все знают.

Знают, что мяса нет и что газеты врут. И что слова о свободе и демократии -- чистый вздор, а на самом деле начальство делает, что ему заблагорассудится. И что надо сидеть тихо, а то угодишь в лагерь. И что рабочие в Америке живут лучше, чем у нас профессора. И многое другое.

Все все знают. И признают, что они знают. В этом отличие тоталитаризма третьей стадии -- стадии сознания -- от тоталитаризма второй -- информационной стадии. Акцент делается теперь на принятии неизбежного, на необходимости режима. Подобно тому, как при Сталине существовало негласное соглашение между властью и гражданами, что граждане как бы "ничего не знают", теперь существует такое же соглашение, что граждане как бы "ничего не могут сделать", хотя и знают почти все. На официальном языке это соглашение именуется "комму­нистической сознательностью" советского народа.

Занятная это вещь -- официальная советская фразеология\ldots Нельзя сказать, что она использует слова в смысле прямо и открыто противоположном работа на.  Разрушать нечего -- все и так в развалинах. Нужны их истинному смыслу. Нет. Это скорее смещение смысла слов, которое происходит не прямо на поверхности, а на некоторой глубине. Сложные абстракт­ные понятия предполагают наличие некоторой лестницы, ие­рархии понятий, с помощью которой они декодируются, рас­шифровываются до уровня простых наблюдаемых объектов реальности. Где‑то посредине этой иерархии и происходит изме­нение смысла на противоположный, что приводит к смещению расшифровываемого понятия. Благодаря тому, что обращение смысла происходит не на поверхности, а в глубине, становит­ся возможным двоемыслие,  на которое указал впервые Орвелл и которое столь характерно для тоталитаризма. Ключе­вые для общественной жизни слова используются одновре­менно в двух смыслах: "теоретическом", то есть исходном, не смещенном, и <<практическом>> -- смещенном. Теоретиче­ский смысл несет положительную эмоциональную нагрузку, но ‑ увы! ‑ не имеет отношения к действительности. Зато в "практическом" смысле эти же слова правильно отражают реальность. Так и образуется гибрид, кентавр, которым успеш­но пользуется тоталитарный человек в своем мышлении.

Согласно теории, коммунистическая сознательность -- это принятие коммунистической идеологии в свободном обществе в результате борьбы идей и в обстановке свободного доступа к информации и к аргументам оппонентов. На деле же коммунистическая сознательность -- это сознание, вырабатываемое под давлением чудовищной машины пропаганды и насилия, когда каждую кроху информации приходится буквально вырывать зубами в постоянном страхе угодить за решетку. Однако я хочу подчеркнуть сейчас не различие между "теоретическим" и "практическим" смыслом коммунистической сознательности, а их сходство. Вспоминая жуткий сталинский террор и сравни­вая с ним современное положение вещей, мы все же можем сказать, что современный советский человек принимает тоталитаризм добровольно и сознательно. В этом‑то и трагизм положения.

Сейчас у властей уже нет того панического страха перед информацией, который был прежде. Конечно, пресечение обме­на информацией остается одной из важнейших задач -- это основа основ тоталитаризма, но теперь власти знают, что в условиях "коммунистической сознательности" масс просачивание отдельных капель информации не представляет собой серьезной угрозы. Действительную угрозу представляет то, что может повлиять на сознание людей, -- идеи. Поэтому свою первейшую задачу власти видят в поддержании идейного вакуума.

Года два назад в журнале "Вестник Российского христианского студенческого движения'' была напечатана статья, подписанная ХУ. В ней, в частности, анализировались причины уменьшения циркуляции самиздата в Советском Союзе по сравнению с концом 60‑х годов. Объяснение -- с моей точки зрения, совершенно правильное -- было таково. В 60‑х годах волна самиздата состояла главным образом из разоблачений,  касающихся как сталинского, так и современного периодов. Теперь эта волна кончилась: общество насытилось саморазоблачениями. нужны новые идеи, а идеи, в отличие от простой информации, требуют диалога. Самиздатские условия гораздо менее благоприятны для выработки идей, чем для разоблачений.

Сознание тоталитарного человека ‑ это прежде всего сознание опустошенное и развращенное. Задача борьбы с тоталитаризмом -- это задача не разрушения, а созидания. Это не борьба против,  это положительные идеалы. Нужна вера в их осуществимость. Нужна программа постепенной демократизации общественной жизни.

Строить, как известно, трудно, а разрушать -- легко. И еще легче мешать строительству. Этим‑то делом занимается пропагандистская 
и карательная машина государства. Огромная, мощная машина. Неудивительно, что мракобесы чувствуют себя так уверенно. Последние 
годы свидетельствуют о планомерном наступлении на культуру. Эпоха "хрущевского либе­рализма" вспоминается как какое‑то золотое 
время, когда еще иногда выходили книги?  С тех пор была проведена фундаментальная чистка всех учреждений, имеющих отношение к 
средствам массовой информации. Каждое живое слово рассматривается как потенциально опасное и вымарывается безжалостно. Казалось 
бы, какой смысл проявлять столь мелочную бдительность, если советский человек может по радио -- скажем, в пределах <<Немецкой 
волны>> -- услышать гораздо более опасные формулировки? А оказывается, в этом есть смысл. Ибо нестандартное слово, сказанное 
живым человеком и появившееся в печати, может явиться -- на вполне законной основе -- центром объединения инакомыслящих. Следить 
за тем, чтобы этого не случилось, поручено специального сорта людям, профессиональным мракобесам. И они свое дело знают, эти 
стратеги выжженной земли, мастера глубокого вакуума.


\section{Марксистско‑ленинская теория}


Что же все‑таки думает советский человек? Является ли официально исповедуемый марксизм‑ленинизм его действи­тельной идеологией? Или же это только идеология партийно‑государственной иерархии? Или же, наконец, и сама иерархия не верит в то, что проповедуется в миллионах печатных изда­ний и вещается по радио чуть ли не на всех языках мира?

Марксизм‑ленинизм именуется у нас передовой и единствен­но научной теорией  общественного развития. Каков бы ни был ответ на поставленные выше вопросы, одно можно сказать сразу же: теорией как средством предвидения и планирования марксизм‑ленинизм заведомо не является, и никто так к нему не относится, в том числе и партийные иерархи: не настолько они наивны.

Один мой знакомый, работавший в государственном аппа­рате на среднем уровне иерархии, рассказал такую историю. Он получил повышение в должности и вместе с повышением -- новый кабинет. Кабинет был отремонтирован, стены заново выкрашены, и, как полагается, надо было украсить их порт­ретами вождей. Мой знакомый зашел на склад -- и первое, что ему попалось на глаза, был портрет Маркса; он велел пове­сить его у себя в кабинете. На следующий день к нему зашел его начальник ~ человек, принадлежащий уже к весьма высо­кому уровню иерархии. Увидев портрет Маркса, он скривился:

--- Фу! Зачем ты этого еврея повесил? Ты бы сказал мне, я бы тебе Ленина дал.

Интересно в этой истории не то, что начальник настроен антисемитски (это‑то само собой), а то, что здесь явственно проглядывает пренебрежение к учению, созданному "этим евреем". Советский иерарх -- это прежде всего реалист, и как реалист он прекрасно знает, что практическая политика партии ни в какой связи с теорией Маркса не состоит. И его отношение к портретам определяется факторами чисто человеческими: Маркс -- еврей, чужой; Ленин -- наш, свой, основатель государства.

Любопытно, что иностранные наблюдатели, даже очень хорошо знакомые с жизнью в Советском Союзе, склонны переоценивать роль 
теоретических принципов или догм в определении конкретных, практических шагов советских руководителей. Недавно я прочитал одну 
статью Роберта Конквиста [5], автора книги <<Великий террор>> -- одного из первых фундаментальных исследований сталинской эпохи. 
В целом это очень интересная статья, содержащая совершенно правильный, с моей точки зрения, анализ взаимоотношений Советского 
Союза с Западом. Но его оценка роли теории мне представляется завышенной.  Р. Конквист пишет:

"Никто, я полагаю, не думает, что Брежнев декламирует "Тезисы о Фейербахе" каждый вечер перед тем, как отойти ко сну. Но 
все‑таки "марксистско‑ленинская" вера -- это единственное основание для него и для его режима, и не просто вера в частную 
политическую теорию, но вера в трансцендентальную, всепоглощающую важность этой политической теории. Как заметил Джордж Кэннан: 

"Дело не столько в конкретном содержании идеологии\ldots сколько в абсолютном значении, связываемом с нею". С этим нельзя не согласиться. Однако дальше мы читаем:

"Но мы можем, в действительности, документально засвидетельствовать -- и без большого труда -- привязанность советского руководства к конкретным догмам. Вторжение в Чехословакию было ярким проявлением доктринальной дисциплины. Другим поразительным примером является экстраординарный и явно в течение долгого времени обдумывавшийся совет, данный сирийским коммунистам в 1972 году и просочившийся через националистически настроенных членов мест­ного руководства. Было две отдельных серии совещаний с советскими политиками и теоретиками соответственно. И даже первая из этих групп, двое членов которой были идентифици­рованы как Суслов и Пономарев, сформулировала в чрезвы­чайно схоластических терминах вывод, что в соответствии с принципами марксизма нельзя признать существование "араб­ской нации". Или, если взять более важный вопрос, советская сельскохозяйственная система основывается исключительно на догме и является вследствие этого чрезвычайно неэффектив­ной".

С этим я уж никак не могу согласиться. Я охотно верю, что ответ сирийцам по поводу "арабской нации" долго обду­мывался и 
обсуждался. Но обсуждение шло, несомненно, в чи­сто политическом плане: отвечает ли интеграция арабов в дан­ный момент интересам 
Советского Союза. Пришли, очевидно, к выводу, что не отвечает. А затем поручили каким‑то работ­никам аппарата сформулировать этот 
вывод в "чрезвычайно схоластических терминах", подобрать необходимые цитаты и т.~д. В Чехословакии советские руководители 
стремились из­бежать заразительного примера -- опять‑таки с политической точки зрения. А колхозная система была создана Сталиным 
для решения весьма практической задачи: централизованного управления и выжимания соков из крестьянства. И система эта в своем 
социальном аспекте не новая: это то, что совет­ские марксисты называют "азиатским способом производства".

Марксизм‑ленинизм преподается во всех без исключения институтах, и отношение студентов к этой премудрости весь­ма показательно. Все знают, что не следует пытаться понять ее, а надо только произносить те слова, которые велено произ­носить. Иногда случается, что какой‑нибудь добросовестный новичок пытается отнестись к этой науке всерьез как к науке. Он обнаруживает в ней внутренние противоречия и противо­речия с действительностью и начинает задавать преподавате­лям вопросы, на которые те отвечают путано и невразумитель­но, а иногда и вовсе не отвечают. Для однокурсников это слу­жит развлечением на фоне скучных занятий по "общественным наукам". Однако развлечение обычно скоро кончается, так как "любопытный слоненок" обнаруживает, что его любозна­тельность отнюдь не способствует получению хороших отметок. Напротив, за ним устанавливается репутация \textit{идейно незрелого,} что может иметь весьма неприятные последствия. А чаще всего находится доброжелатель, который -- жертвуя развлечением -- объясняет товарищу, как надо относиться к марксистской теории\ldots


\section{Теория и действительность}

А как, собственно говоря, можно относиться к теории, если она находится в явном противоречии с действительностью?

Согласно теории, в промышленно развитых странах давно уже должна была произойти пролетарская революция, однако ничего такого не 
случилось и -- как всем уже ясно -- в обозримое время не предвидится.

Согласно теории, в капиталистическом обществе происхо­дит непрерывное обнищание -- относительное и абсолютное -- рабочего класса. 
В действительности же уровень жизни рабо­чих непрерывно растет, и он гораздо выше, чем в так называе­мых "социалистических" 
странах.

Согласно известному высказыванию Ленина, производительность труда -- это, в конечном счете, тот фактор, который опре­деляет 
прогрессивность общественно‑политического строя и обусловливает его победу. В действительности же производи­тельность труда у нас 
намного ниже, чем в передовых капита­листических странах. По сравнению с США у нас даже в про­мышленности производительность труда 
ниже по крайней мере в два‑три раза, а в сельском хозяйстве -- не менее чем в десять раз.

Согласно теории, немцы, стонущие под игом капитала в За­падной Германии, должны рваться в социалистическую Восточ­ную Германию. В 
действительности же миллионы немцев бежа­ли из Восточной Германии в Западную, и остановить это бегст­во удалось только с помощью 
пулеметов и колючей прово­локи.

Согласно теории, мы живем в самом свободном и демокра­тическом государстве на земном шаре. А в действительности? И говорить не 
хочется. Все все знают\ldots

Много раз мы были свидетелями того, как марксистско‑ленинская теория служила оправданием для совершенно противоположных выводов. 
Достаточно вспомнить, как Сталин открыл, что по мере продвижения к социализму классовая борьба не затухает, а обостряется! 
Ленинская идея прорыва цепи мирового капитализма в слабом звене с целью дальней­шего расширения революции превратилась в 
ленинскую же идею мирного сосуществования стран с различным общест­венным строем. А ленинская ставка на новое, сознательное 
отношение к труду превратилась в ленинский принцип мате­риальной заинтересованности.

Нет, глупы и наивны были бы люди, которые в самом деле искали бы ответы на конкретные вопросы в такой теории. Советский 
руководитель‑это кто угодно, но только не догма­тик, не доверчивый простак, который "тычет в книжку паль­чик", чтобы найти решение 
волнующих его проблем.

Однако было бы большой ошибкой думать, что огромные деньги, которые тратятся на внедрение в сознание каждого советского человека 
марксистско‑ленинской теории, тратятся впустую. И усиленная марксистско‑ленинская выучка, кото­рой подвергаются работники 
партийного аппарата, отнюдь не проходит для них бесследно. Вопрос о взаимоотношении тео­рии с действительностью отнюдь не так 
прост.

Прежде чем служить для предвидения событий в окружаю­щем нас мире, всякая теория дает нам понятийный аппарат, язык для описания 
действительности. Верны или не верны окажутся предсказания, но язык теории остается. И мы видим действительность через призму 
этого языка, этих понятий. Здесь я опять могу с полным согласием процитировать уже упоминавшуюся выше статью Роберта Конквиста:
"Марксистско‑ленинский язык, используемый правящей пар­тией, это не просто какая‑то формула. Это единственный спо­соб, с помощью 
которого руководители могут представлять себе явления, с которыми они имеют дело. "Каждый язык вырезает свой собственный сегмент 
действительности. И мы проносим этот язык через всю жизнь\ldots" Это замечание извест­ного языковеда (Джорджа Штейнера) 
несомненно приложимо к использованию в политике, начиная с самого рождения, опре­деленного политического диалекта. Очевидно, 
советские ру­ководители просто неспособны думать в каких‑либо других категориях".

К сожалению, не только руководители. Этот язык и способ мышления навязываются, начиная с самого рождения, каждому гражданину 
тоталитарного государства. В функциях, выполняемых марксистско‑ленинской теорией в советском государстве, можно выделить 
формальную и содержательную стороны. В XX веке государство не может обойтись вовсе без "теории": надо же что‑то говорить и 
писать, как‑то объяснять события гражданам. Единая и единственно разрешенная государственная идеология -- необходимый элемент 
тоталитаризма. Это символ веры. Его принятие без обсуждений и сомнений -- "причастие буйвола". В этом аспекте идеология служит в 
качестве армейской формы -- для отличения своих от чужих; содержание теории здесь роли не играет. Именно эта формальная роль 
теории отражена в замечании Дж. Кеннана, цитированном Р. Конквистом. Однако формальной функцией роль марксизма‑ленинизма не 
исчерпывается. Эта теория и по своему содержанию чрезвычайно подходит тоталитарному обществу, необходима ему. В частности, я 
хочу подчеркнуть значение основополагающего прин­ципа исторического материализма: "бытие определяет сознание". Это -- 
краеугольный камень, на котором зиждется тоталитарное сознание, теоретическое оправдание жизненного принципа: все равно ничего 
не сделаешь, плетью обуха не перешибешь. Позже мы уделим этому вопросу специальное внимание.

Казалось бы, явное противоречие между теорией и действительностью должно было бы дискредитировать теорию в глазах советского 
человека. Формальную функцию теории это противоречие, конечно, не нарушает. Но содержательную? Как можно совместить осознание 
противоречия теории и действительности с верой в теорию в целом? Не является ли это само по себе патологией?

В оправдание советского человека мы должны признать, что нет, не является. Ибо за исключением чисто математических теорий, 
никакие другие не являются полностью формализованными.  Теории содержат иерархию понятий и принципов, переход от высших уровней 
к низшим далеко не всегда осу­ществляется путем однозначных, строго формальных выводов или вычислений. Напротив, на пути от 
высших принципов к непосредственно наблюдаемым явлениям приходится делать дополнительные допущения, приближения и т.~п. 
Случаются и ошибки. Когда, например, физик обнаруживает противоречие между результатами своих экспериментов и теоретическими 
предсказаниями, он не спешит отвергнуть все здание теоре­тической физики. Сначала он будет проверять, не сделал ли он простой 
арифметической ошибки в вычислениях. Затем будет исследовать, учтены ли все факторы, влияющие на исход эксперимента. Потом будут 
поставлены под сомнение модель­ные упрощения, которыми почти наверняка пользовался фи­зик‑теоретик в своих расчетах, результаты 
экспериментов других ученых, которыми он тоже почти наверняка пользо­вался и т.~д.

Если так обстоит дело даже в физике, то чего же ожидать от наук об обществе? Тот факт, что противоречие с действи­тельностью на 
нижнем уровне понятийной иерархии не ведет к немедленному разрушению в сознании людей всей иерархии в целом, нисколько не 
удивителен.

\section{Идеологическая иерархия}

В советской идеологической системе, которая именуется в целом марксизмом‑ленинизмом, можно выделить следующие четыре уровня 
(описание содержаний уровней дается схематически, в расчете на знающего читателя).

\begin{enumerate}
 \item \textit{Уровень философии.}  Диалектический материализм. Материя первична, сознание вторично. Развитие как борьба 
 противоположностей. Исторический материализм. Общественное бытие определяет общественное сознание.
 \item \textit{Экономико‑социологический уровень.}  Учение о классах в обществе. Классовая борьба. Общественно‑экономические 
 формации. Неизбежность перехода от капитализма к социализму путем пролетарской революции (поправка последнего времени: 
 революция может быть "мирной"). Диктатура пролетариата.
 \item \textit{История КПСС и советского государства.} Необходимость партии нового, ленинского типа. Демократический централизм. 
 Конкретная история (разумеется, отлакированная до блеска и трансформированная с точки зрения интересов текущего момента). 
 Октябрьская революция 1917 г. как социалистиче­ская революция, предсказанная Марксом. Сбылась вековая мечта человечества.
 \item \textit{Текущая политика.}  Вооруженный единственно научной марксистско‑ленинской теорией, героический советский народ 
 совершает славные трудовые подвиги под мудрым руководством Коммунистической партии Советского Союза и ее ленинского (?!) 
 Центрального Комитета. Уверенной поступью\ldots и т.~д.
\end{enumerate}

Я совершенно убежден, что большая часть советских людей видит фальшь пропаганды четвертого уровня и относится к ней 
соответственно. Но я точно так же убежден, что уже третий уровень принимается в целом большинством населения. Услов­но эту 
психологию можно обозначить как проведение раздели­тельной черты между Лениным и Сталиным или между прин­ципами и их 
осуществлением. В принципах все, в общем, верно, но вот из‑за разных ошибок, "перегибов", плохих людей и т.~п. на практике 
получается не очень хорошо. Это психология массо­вого человека, которого не учат -- и даже мешают -- проводить самостоятельный 
анализ связи между принципами и действи­тельностью.

Люди, которым все же удалось выполнить этот анализ, кото­рые всерьез задумывались об истории своей страны и о том, что происходит 
вокруг, отвергают третий уровень официальной идеологической иерархии. Хотя эти люди и в меньшинстве, они, несомненно, 
исчисляются многими миллионами. Но из них лишь очень немногие решаются на переоценку принципов второго и первого уровней. В 
особенно выгодном положении находится принцип <<бытие определяет сознание>> -- святая свя­тых тоталитарного марксизма‑ленинизма. 
Я много раз убеж­дался, с какой цепкостью этот принцип держится в умах лю­дей, даже весьма образованных и думающих. Он подкупает 
своей кажущейся реалистичностью, "научностью". Противопо­ложная точка зрения кажется беспочвенным идеализмом, по­пыткой выдать 
желаемое за действительное. Кроме того, имеет место своеобразная экранировка  первого уровня вторым, на котором провозглашаются 
благородные цели создания справедливого общества, а возможность достижения этих целей -- и даже необходимость их конечного 
торжества -- как бы вы­водится из того же принципа исторического материализма.

Понятие экранировки вообще очень важно для понимания работы идеологической иерархии. Нижние уровни иерархии экранируют верхние 
уровни, ибо они отнимают у человека часть "энергии отрицания", если можно так выразиться. Очень трудно отрицать все. Человек, 
отвергающий под давлением фактов какие‑то концепции, принятые его окружением, обыч­но ощущает потребность доказать окружающим (и 
себе!), что он делает это именно под давлением фактов, а вовсе не упива­ется отрицанием ради отрицания. Поэтому ему хочется 
где‑то остановиться; его "энергия отрицания" исчерпывается по мере движения по ступеням лестницы от конкретных фактов ко все 
более абстрактным понятиям. Задача пропаганды четвер­того уровня -- отнять как можно больше энергии отрицания. Так получается, 
что ложь -- сколь это ни парадоксально -- не расшатывает "теорию", а укрепляет ее. Слова четвертого уров­ня, слова‑солдаты, 
бросаются в бой миллионами. Им никто не верит, они гибнут массами, не дойдя до цели, как будто впустую. Но за горами их трупиков 
укрываются более важные слова: слова‑офицеры и слова‑генералы. Именно ради этих последних, самых высокопоставленных слов и 
строится вся идеологическая иерархия. Внешне незаметно, но непрерывно и постоянно эти слова‑генералы и стоящие за ними представ­
ления воспитывают тоталитарного человека.

Вернемся к нашему сравнению марксистско‑ленинской идеологической иерархии с теоретической физикой. Мы отме­тили сходство в 
иерархической структуре и в непрямом и не­быстром пути, соединяющем принципы с наблюдаемыми фак­тами. Различие же состоит в целях, 
ради которых строится теория. Физика в самом деле создается ради того, чтобы пред­сказывать факты. Поэтому накопление 
противоречий между теорией и фактами в конце концов обязательно приводит к пе­рестройке теории на всех уровнях -- включая самые 
высшие. Идеологическая система тоталитарного государства строится ради самой себя, ради консервации своих основных принци­пов. 
Поэтому расхождение с действительностью и не может ее изменить. Таким образом мы видим здесь сочетание наихуд­ших свойств теорий 
вообще. С одной стороны, эта теория не обладает предсказательной силой, с другой стороны, она навя­зывает мертвый, не способный к 
развитию понятийный аппа­рат. Марксистско‑ленинская теория -- это кукла, манекен, занимающий то пространство, где должен быть 
живой человек. Это опилки, которыми забиваются головы людей, чтобы не осталось места для живой мысли.

Идеологическая картина советского общества будет непол­ной, если мы не упомянем о людях, которые отвергают марксистско‑ленинскую 
идеологию целиком, а именно: по принци­пу сверху вниз, а не снизу вверх. Я имею в виду людей религи­озных. Они образуют, так 
сказать, идеологические меньшинст­ва -- впрочем, довольно многочисленные. По официальным данным на 1974 г., в стране 
насчитывается в общей сложно­сти 32 миллиона верующих6. Все они в той или иной форме притесняются. Особенно жестоко преследуют 
сектантов, кото­рые препятствуют проникновению в свою среду информато­ров и других угодных государственным органам лиц. Борьба 
верующих за свои права вызывает сочувствие и поддержку всех тех, кто выступает в защиту основных прав личности. Мужество и 
упорство в убеждениях, которое проявляют мно­гие из них, является примером для основной массы населения. Наличие отстаивающих 
свои права идеологических, как и на­циональных, меньшинств -- серьезное социальное явление. Но в плане мировоззрения их влияние 
незначительно, и я не вижу -- вопреки высказываемой иногда точке зрения -- чтобы их влияние возрастало. Людям свойственно судить 
о других по себе, и при всех недостатках этого метода от него никуда не денешься. Разделяя многие идеи христианства, признавая 
их значение для современной цивилизации и испытывая глубокое восхищение перед личностью Христа, я в то же время не могу по­нять, 
как это можно в наше время принять христианство \textit{цели­ком} как веру и систему мышления. Едва отвергнув одну догма­тическую 
и устаревшую систему, принять другую, пусть более заслуженную, но еще более догматичную и устаревшую? Мне кажется, для этого 
надо совершить над собой какое‑то наси­лие, быть может, замаскированное и безотчетное. То же относится, конечно, и к другим 
традиционным религиозным системам.

Нет, единственная альтернатива фальшивой государственной идеологии -- это ее анализ с позиций критического научного 
мировоззрения и создание положительных идеалов с тех же позиций. Первая часть этой задачи сравнительно проста, вто­рая -- 
невероятно трудна; само выражение "положительные идеалы" представляется нам каким‑то устаревшим, ненауч­ным. Зато даже скромные 
результаты на этом пути имеют большой вес.



\section{Бытие определяет сознание?}

Кажущийся "трезвый реализм" основного принципа исто­рического материализма подкупает многих. Даже люди, высту­пающие против 
марксистского тоталитаризма, находят порой этот принцип наименее спорной частью марксизма[7]. Между тем, этот принцип и 
вытекающая из него концепция личности, общества и истории -- основа основ тоталитаризма.

Ленин пишет: "Общественное сознание \textit{отражает}  общест­венное бытие -- вот в чем состоит учение Маркса\ldots Сознание 
вообще отражает бытие -- это общее положение \textit{всего}  матери­ализма. Не видеть его прямой и \textit{неразрывной}  связи с 
положе­нием исторического материализма: общественное сознание \textit{отражает}  общественное бытие -- невозможно".[8]

В чем же состоит материализм? Согласно марксизму‑лени­низму, перед философами всех времен и народов всегда стоял и будет стоять 
вопрос: что первично -- материя или сознание? Если вы отвечаете: первична материя, то вы материалист и хороший человек, ибо этот 
ответ правильный и научный. Если вы отвечаете: первично сознание -- вы идеалист и плохой чело­век. Если вы отвечаете: не знаю, то 
вы агностик и тоже плохой человек.


Для философа, стоящего на позициях современного науч­ного мировоззрения, сам вопрос -- в той форме, в которой его задают 
марксисты‑ленинцы, бессмыслен. Это не значит, что его не следует задавать. Задавать бессмысленные вопросы надо. Но потом надо их 
анализировать и обнаруживать, в чем их смысл -- или бессмысленность. Величайшее достижение фило­софии XX века как раз и состоит в 
углубленном анализе абст­рактных философских понятий, которые ранее представлялись чем‑то первичным, само собой понятным. Это 
направление мышления, в числе основоположников которого мы видим физика Эрнста Маха, полностью отвергается 
марксизмом‑ленинизмом. В своей печально знаменитой книге "Материализм и эмпириокритицизм" Ленин предал анафеме Маха и его после­
дователей из числа социалистов. С тех пор нет худшего врага для ленинцев, чем махисты. Критическая философия не нужна ленинцу. 
Он должен глядеть в глаза начальству и отвечать без тревог и сомнений: материя первична, а сознание вторично, товарищ генерал.

Отвергая критический анализ понятий, марксистско‑ленинская философия остается на уровне наивного реализма пред­шествующих 
столетий. И больше ей ничего не нужно, ибо наив­ного реализма вполне достаточно для пропаганды; он даже наиболее удобная позиция 
для пропаганды. Философские поня­тия используются в неточном, приблизительном, нарочно сме­щенном смысле; они нагружаются 
эмоционально в соответст­вии с бытовыми ассоциациями и в таком виде отлично служат для воспитания "классового сознания" масс. 
Чего стоит, напри­мер, такой шедевр логики, который мы находим в советском учебнике по истории философии:

"Представители господствующих эксплуататорских классов, имеющие монополию на занятие умственной деятельностью, всегда стремились 
создать впечатление, будто физический труд, материальная производственная деятельность трудящихся масс является чем‑то 
второстепенным, подчиненном труду умствен­ному, играющему якобы главную роль в жизни общества. Подобные утверждения идеологов 
реакционных классов неми­нуемо вели к философскому идеализму, к попыткам обосно­вать первичность духовных явлений и вторичность 
явлений ма­териальных".[9]

Однако, если понимать под материализмом программу иссле­дования явлений, которые мы называем "духовными", через явления, которые 
мы называем "материальными", то я целиком за такой материализм, ибо в этом подходе -- сущность науки. Я могу также согласиться с 
марксистской формулировкой, что научный подход к явлениям духовной жизни (и в частности, к сознанию) -- это рассмотрение их как 
\textit{формы движения материи.}

В какой же связи находится так понимаемый материализм с основным принципом исторического материализма Маркса?
Ни в какой. Мы имеем перед собой две группы явлений: общественное бытие и общественное сознание. И те и другие связаны с 
человеком ‑ его телом и мозгом. И те и другие надо рассматривать как формы движения материи. И ниоткуда не следует, что явления 
одной группы определяют явления другой группы.

В приведенной выше цитате Ленин связывает исторический материализм с положением о том, что сознание \textit{отражает}  бытие. 
Сознание отражает бытие, это верно. Но для того, чтобы перей­ти отсюда к тезису: \textit{бытие определяет сознание,}  надо 
сделать гораздо более сильное допущение -- что сознание не только \textit{от­ражает}  бытие, то есть включает в себя какое‑то его 
отражение, но и \textit{является}  отражением, то есть исчерпывается им. Что это за концепция? Это концепция, согласно которой 
вся так назы­ваемая духовная жизнь человека есть просто совокупность отражений‑рефлексов, то есть прямых однозначных реакций на 
возбуждение нервных окончаний внешней средой. Стоит ли опровергать эту точку зрения? В открытую под ней не подпи­шется в наше 
время даже марксист‑ленинец. Он назовет такую точку зрения "вульгарно‑материалистической". Однако поти­хоньку она кладется в 
основу философии истории.

Понятие о рефлексе ввел Декарт. Но для него рефлексы лишь свидетельствовали об устройстве нашей телесной машины;кроме рефлексов 
у человека, по Декарту, есть еще и душа; Декарт высказывал даже определенные соображения по пово­ду того, какой именно участок 
мозга является органом души. Подчеркивая различие между рефлексами и душой, Декарт приводит такой пример. Допустим, что кто‑либо 
взмахнет рукой перед нашими глазами, собираясь как бы ударить нас. И хотя бы мы и знали, что он делает это только в шутку и 
далек от мысли причинить нам зло, нам все же трудно удержаться, чтобы не закрыть глаза. "Это показывает, -- пишет Декарт, -- что 
глаза закрываются отнюдь не при участии души, так как это происхо­дит помимо нашей воли\ldots это происходит от того устройства 
машины нашего тела, благодаря которому движение руки перед глазами возбуждает другое движение в нашем мозгу, и мозг направляет 
животные духи в мускулы, опускающие глазные веки".[10]

Если отвергнуть существование души как "идеализм и по­повщину" (любимое ругательство Ленина), то останутся одни рефлексы.

\section{Мышление в свете кибернетики}
\epigraph{Что такое душа? Человечек задумчивый\ldots}{Булат Окуджава}

Рене Декарт был ученый. Ученый строит модели малопонятных, неизученных явлений, опираясь на явления более изученные. Но когда он 
видит, что для какого‑то явления он не может построить сколько‑нибудь внушающей доверие модели, он так и говорит: стоп, об этом 
я ничего путного сказать не могу. И для описания таких явлений он употребляет те самые слова и понятия, которые были раньше и 
которые не претендуют на "научность". Например, душа. Прекрасное слово.

Со времен Декарта наука значительно продвинулась вперед в понимании мышления. И хотя наши модели мышления через какое‑то время 
будут вызывать у ученых улыбку, как вызы­вают у нас улыбку некоторые механические модели Декарта, то, что мы знаем уже сейчас, 
весьма существенно для разбираемого вопроса.

Еще во времена И.П. Павлова понятие о рефлексе было, по существу, единственным научным понятием, которое использовалось при 
описании поведения живых существ. Совре­менная кибернетика при описании поведения животных и чело­века опирается на несколько 
новых понятий. Прежде всего это понятия о цели, обратной связи и регулировании.

Когда ребенок подносит ко рту ложку супа, то в его мозгу не только фиксируется, "отражается" текущая ситуация -- положение 
ложки, но и идеальная, желаемая ситуация -- ложка во рту. Это -- цель.  Механизм, который обеспечивает достижение цели, в общих 
чертах таков. Реальная ситуация сравнивается с идеальной, и в результате вырабатываются нервные импульсы, которые сокращают 
мышцы таким образом, чтобы приблизить реальную ситуацию к идеальной. Измененная ситуация воспринимается с помощью органов чувств 
и снова сравнивается с идеальной. Таким образом, поток информации замыкается. Это -- обратная связь.  Непрерывная коррекция 
действий при наличии цели с помощью обратной связи называется регулированием. 

Какие материальные структуры в мозгу хранят цели? Как осуществляется сравнение цели с реальной ситуацией? Как фиксируется 
алгоритм выработки нервных импульсов в результате сравнения? Обо всем этом мы знаем очень мало, почти ничего. Но у нас нет 
сомнения, что все эти процессы осуществляются с помощью каких‑то материальных образований, и мы можем представить себе эти 
образования как некие програм­мы или планы  поведения11 по аналогии с программами, хра­нящимися в памяти вычислительной машины.
Еще одно важное понятие кибернетики, необходимое для описания поведения, это -- иерархия.  Цели и планы поведения образуют 
иерархию, в которой более сложные и общие цели требуют для своего осуществления постановки и осуществле­ния ряда более простых, 
вспомогательных целей; соответству­ющие планы поведения включают -- или вызывают -- более простые планы, подобно тому как 
программа для вычисли­тельной машины может вызывать несколько вспомогательных программ. Прежде чем съесть ложку супа, ребенок 
должен научиться держать ложку в руке, зачерпывать ею суп из тарел­ки, перемещать ложку таким образом, чтобы суп не выливался. 
Планы, осуществляющие эти действия, входят в качестве состав­ных элементов в план "съесть суп". Съедение супа в свою оче­редь 
является элементом более сложного плана -- у взрослого, во всяком случае. Прежде чем съесть суп, надо его сварить, а для этого 
надо иметь продукты, для чего необходимо иметь деньги. Чтобы иметь деньги, надо работать, а чтобы работать, надо сначала 
учиться, приобрести профессию. Так мы приходим к целям и планам самого высокого уровня.

При движении вверх по иерархии целей и планов цели становятся все более общими и долговременными. То же относится и к планам 
поведения. Для их выражения становятся необходимыми все более общие, абстрактные понятия. Параллельно с временным масштабом 
естественным образом растет и пространственный масштаб. Когда вы бежите к автобусу, стоящему на остановке, с единственной целью 
успеть вскочить в него, вы не думаете больше ни о чем и ни о ком. Это -- масштаб секунд и минут. Когда вы планируете свои 
действия в масштабе часов и дней, ваши интересы и планы неотделимы от инте­ресов и планов вашей семьи и других близких вам людей. 
Пла­ны в масштабе лет и десятилетий неотделимы от классовых и национальных интересов. Наконец, когда вы думаете в масш­табе 
поколений, вы должны думать о человечестве в целом.

Уже фиксация цели мозгом пробивает серьезную брешь в концепции "бытие определяет сознание". Птица, начинающая вить гнездо, 
руководствуется своим инстинктом; никакого реального "бытия" гнезда пока не существует. Однако еще важнее вопрос о происхождении 
целей и планов. Я буду раз­бирать этот вопрос в духе того подхода, на котором основана моя книга "Феномен науки" [12], а именно 
-‑ на основе понятия о "метасистемном переходе".


\section{Метасистемный переход}

Этот раздел покажется странным в общем контексте. Однако читателю придется смириться с этим. Язык и понятия кибернетики 
становятся частью общего образования, как арифметика, и я думаю, что это очень многообещающее явление. Так как мне все‑таки 
приходится быть очень кратким, я приношу извинения за проистекающую отсюда фрагментарность и кон­спективность. За связным и 
детальным изложением отсылаю к "Феномену науки".

Итак, представим себе некоторую систему $S$. И представим, что какое‑то число систем $S$ или систем типа $S$ соединены в еди­ное 
целое и снабжены вдобавок какой‑то управляющей ими системой. Образованную таким образом систему $S+$ назовем 
\textit{метасистемой} по отношению к системе $S$. Метасистема $S+$ содержит ряд систем $S$ в качестве своих подсистем и содержит 
также средства, позволяющие управлять  этими подсистемами -- в самом широком смысле: координировать их работу, модифицировать 
их, генерировать и т.~д. Переход от системы $S$ к метасистеме $S+$ мы называем \textit{метасистемным переходом.} 

Данное нами определение касается структурного аспекта. Очень часто мы не знаем детальной кибернетической структу­ры сложных 
систем, с которыми имеем дело, но можем наблю­дать за внешним проявлением деятельности этих систем, за их функционированием. Как 
будет выглядеть метасистемный переход в аспекте функциональном? Пусть система S обнару­живает некоторую деятельность $А$. Будем 
наблюдать за дея­тельностью метасистемы $S+$. Мы обнаружим новый тип дея­тельности -- $А+$, который состоит в управлении 
деятельно­стью $А$.

В "Феномене науки" я рассматриваю метасистемный пере­ход как элементарную единицу или квант  \textit{эволюции.} Все ка­чественные 
шаги на пути эволюции являются метасистемными переходами большего или меньшего масштаба. Уже сущест­вующие, сформировавшиеся 
образования могут совершенство­ваться в известных пределах путем мелких количественных изменений под действием естественного 
отбора. Но для ради­кальных сдвигов в структуре живого есть только один путь: метасистемный переход, при котором уже существующие 
образования используются природой как строительные блоки для нового сооружения. Естественный отбор, конечно, работает и здесь 
(природа всегда действует по методу проб и ошибок), но он теперь направлен на создание метасистемы $S+$ из подси­стем $S$. 
Образуется новый уровень иерархии по управлению, новый уровень организации живой материи. Или -- неживой! Ибо и человек, 
конструируя и усложняя создаваемые им вещи, пользуется тем же -- единственно существующим -- методом.


Примеры метасистемных переходов: возникновение много­клеточных животных из одноклеточных; образование нерв­ной системы для 
управления движениями; переход от безус­ловного  рефлекса к условному; производство орудий, а также производство орудий для 
производства орудий и т.~д.; образование общества из индивидуумов; разделение законо­дательной, исполнительной и судебной власти; 
станки с авто­матическим управлением; возникновение понятия о доказа­тельстве в математике; возникновение математической теории 
доказательства в математике (метаматематика).

Сравним условный рефлекс с безусловным. В первом случае ассоциации представлений врождены и фиксированы, они не зависят от 
индивидуального опыта животного. Во втором случае ассоциации управляемы,  они возникают и исчезают в результате воздействия 
внешней среды на данную особь. Животное приобретает способность ассоциирования,  оно становит­ся обучаемым. Это -- метасистемный 
переход.

Переход от высших животных к человеку это еще один -- и решающей важности -- метасистемный переход; возникает способность 
\textit{управления ассоциированием.}  У животного ассоциации возникают только под воздействием внешней среды, человек же 
способен сам, по своему произволу, создавать новые ассоциации. Эта способность проявляется как воображение, языкотворчество, 
планирование, припоминание, самообучение. Короче говоря, это то же, что мы обычно называем мышлением.  Эта же способность 
приводит и к низвержению абсолютной власти инстинкта.


\section{Человек свободен}

По мере увеличения способности к обучению инстинкты становятся более абстрактными, и все большая часть планов поведения 
усваивается живой особью путем подражания или вырабатывается самостоятельно путем проб и ошибок. Однако у животного инстинкт все 
же остается верховным судьей и распорядителем поведения. У каждой иерархии есть высший уровень. Цели и планы поведения животных 
на этом уровне инстинктивны и изменению не подлежат; они меняются только в процессе эволюции вида.

Человек -- первое живое существо, которое способно само себе ставить цели и разрабатывать планы, в том числе и самого высшего 
уровня. Конечно, основной объем планов поведения человек усваивает путем подражания старшим. Однако он мо­жет в процессе 
размышления отвергать их или модифициро­вать, или создавать заново. И новые планы поведения могут идти наперекор даже самым 
стойким, основным инстинктам: самосохранения и размножения. Благодаря способности управ­ления ассоциированием иерархия планов 
поведения оказалась незамкнутой, открытой для экспериментирования.

Человек свободен. Это единственное живое существо, кото­рое обладает абсолютной свободой поведения. Человек полу­чает эту 
способность вместе с генами, как другие создания по­лучают инстинкты. Как и всякая способность, она может остать­ся неразвитой. Ее 
можно подавить, как можно с помощью сильных средств подавить тот или иной инстинкт у животного. Но эта способность есть.
Появление человека открывает новую эру в эволюции жиз­ни. Инструментом эволюции всегда был и остается метод проб и ошибок. Но 
если раньше пробы и ошибки происходили на материале носителей наследственности -- генов и эволюция была лишь эволюцией 
биологической, то теперь возникла эво­люция человеческой культуры. Это процесс, при котором про­бы и ошибки происходят в 
воображении и в поступках людей. Человек способен к сознательному творчеству. Он может то, что раньше могла только природа в 
целом. Поэтому он и несет ответственность за жизнь на планете. А может быть -- и во вселенной.

Что такое сознание? Что такое воля? Как соотносится ощу­щаемая нами свобода воли, относящаяся к категории "духовных явлений", с 
физическими законами природы и с данными об устройстве мозга? На эти вопросы мы не можем пока дать вразумительных ответов. 
Поэтому не будем отрицать, что у человека есть душа.
Есть душа. И она неразрывно связана с самым удивительным
и важным аспектом наблюдаемой нами вокруг себя реальности -- с тем аспектом, который отражается понятиями: творчество, 
метасистемный переход, эволюция. 


\section{Перестановка верхних уровней}
Однако вернемся к основному тезису исторического материализма. Что такое общественное сознание? Самые общие социальные нормы 
поведения, выраженные в самых общих понятиях и определяющие, "что такое хорошо и что такое плохо". Иначе говоря, это планы, 
которые вместе с самыми общими планами биологического происхождения (инстинкты самосохранения и размножения) образуют высший 
уровень в иерархии планов поведения. Что такое общественное бытие? Сюда входит прежде всего материальная компонента цивили­зации 
-- те предметы, с которыми мы имеем дело в повседнев­ной жизни и в производстве. Но сюда входят также и опреде­ленные планы 
поведения: навыки обращения с предметами, отношения с людьми, в которые мы вступаем в процессе произ­водства (производственные 
отношения), формы участия в про­цессе распределения материальных благ. Короче, эти планы по­ведения можно назвать 
\textit{экономическими.}  Это тоже очень общие планы, и они, можно сказать, образуют второй уровень иерар­хии, непосредственно 
следуя за планами высшего, \textit{этического} уровня.

Каково взаимоотношение между этими уровнями? Можно ли сказать, что один из них определяет другой? Конечно, нель­зя. Отношение 
между ними -- это отношение \textit{управления,}  как и в любой иерархии. Генералы не "определяют" своих солдат: не 
воспроизводят их по своему образу и подобию, ни даже по своему желанию. Но они \textit{командуют} солдатами. Это опять‑таки не 
означает определения каждого действия, каждого движения солдата; командование ведется в определенных об­щих понятиях, а выбор 
средств в рамках задания остается на ус­мотрение солдата. Точно так же общебиологические и этические планы поведения "командуют" 
имеющимися в распоряжении экономическими планами: осуществляя выбор между ними, включая и выключая их, отчасти модифицируя и 
т.~п.

Что же утверждает по этому поводу марксизм? Что обще­ственное бытие определяет общественное сознание. В послед­нем советском 
учебнике по историческому материализму мы читаем: "К.Маркс и Ф.Энгельс показали полную несостоятельность идеалистических 
воззрений на общество, которые ранее господ­ствовали в социологии. Они научно доказали, что главной и определяющей стороной 
общественной жизни является не иде­альный, а материальный фактор, и прежде всего производство жизненных благ. Материальное 
производство, согласно учению исторического материализма, составляет основу и решающую силу общественного развития, определяющую 
в конечном счете все другие стороны общественной жизни. Сознание людей не только является первичным по отношению к материальной 
жи­зни общества, но наоборот зависит от нее, определяется ею."

Тремя страницами ниже находим следующее, весьма приме­чательное пояснение:

"Мысли, желания, настроения людей возникают и видоизме­няются на базе их практических жизненных потребностей и зависят в конечном 
счете от степени развития их материального бытия, от их экономического положения в обществе."

Таким образом, здесь происходит перестановка двух верхних уровней иерархии; утверждается, что те планы поведения, которые нам 
кажутся высшими, на самом деле являются производными от планов поведения -- и вообще от реальности -- экономи­ческого уровня. Где 
же "научное доказательство" этого чрезвычайно сильного утверждения, данное якобы Марксом и Энгельсом? В учебнике мы его, 
конечно, не находим, ибо такового не существует и никогда не существовало в природе. Все, что можно найти на эту тему у 
основоположников марксизма, это указание на несколько параллелей между развитием материального производства и идеями, 
господствующими в обществе.

Но эти параллели (существование которых бесспорно и не Марксом впервые отмечено) ничего не доказывают. Идеи и мате­риальное 
производство взаимозависимы и развиваются в тес­нейшей связи, это несомненно. Вопрос состоит в том, каков ха­рактер этой связи. 
Никто не отрицает, что действия генералов существенно зависят от того, а в чем‑то и определяются тем, ка­кие у них солдаты. Но 
значит ли это, что "в конечном счете" сол­даты командуют генералами? Что от искусства полководца не зависит исход битвы?

Для отношения управления характерно то, что нижний уро­вень не определяет, а лишь ограничивает возможности; \textit{определяет} 
то  как раз верхний, управляющий уровень: определяет выбор между различными возможностями. Имея взвод солдат, ни один, даже 
самый гениальный, полководец не сможет разбить армию. Только в этом тривиальном смысле и можно говорить, что бытие определяет 
сознание. Что же касается развития, то его пружины надо искать в области управляющего, а не управляемого.

В марксизме экономическая жизнь именуется \textit{базисом,} а институты, выражающие общественное сознание, -- 
\textit{надстройкой.} С этой терминологией вполне можно согласиться. Но почему отсюда следует, что базис имеет определяющее 
значение, а \textit{надстройка} -- нечто второстепенное, производное? Такой вывод можно сделать только, если ограничить свое 
мышление средневековыми категориями субстанции, количества и качества, первичности и вторичности и т.~п. Именно таков стиль 
мышления марксистов. Кибернетик и вообще современный человек, мыслящий в терминах систем и отношений, обнаружив базис  и 
\textit{надстройку}  в сложной системе, скорее сделает вывод об управляющей, определяющей роли надстройки.  В конце концов, и 
голова у человека -- \textit{надстройка}  над телом! Так что же существенно в отличии человека от животного: детали 
пищеварительного процесса или устройство головы?

Голодное брюхо к ученью глухо, -- справедливо замечает по­словица. Но это вовсе не значит, что, съев мозг умного человека, можно 
поумнеть самому и научиться тому, что умел делать по­койник. Основное положение исторического материализма на­поминает наивную 
веру дикаря в прямое воздействие съеденной пищи на образ мышления. Интересно отметить, что эта вера ос­новывается не на опыте, а 
на внешне правдоподобных "общих соображениях". Точно то же имеет место и в случае историче­ского материализма. Фактов нет. Есть 
только "общие сообра­жения".

Влияние экономики на политику и идеологию, столь милое сердцу марксиста, конечно, существует. И, конечно, оно бывает не только 
явным, но и неявным, скрытым. Прослеживать и обнаруживать его -- важная задача историка и социолога. Но воз­водить это в принцип 
общественного развития можно только догматически, вопреки очевидности.
Мы едим для того, чтобы жить, а не живем для того, чтобы есть. Мы знаем, что экономические факторы налагают на нас определенные 
ограничения. Но мы знаем, что у нас есть также и другие -- высшие -- цели. И мы умеем соотносить с ними экономические цели. "Все 
это иллюзии, -- говорит нам марксист. -- На самом деле все ваши высшие цели -- это ловко замаскированный интерес класса, к 
которому вы принадлежите."

О, волшебное сочетание слов "на самом деле"! Когда вам показывают на белое и говорят, что на самом деле  это черное, и только 
кажется вам белым, то возразить нечего. Можно только попытаться понять, почему  или зачем  вам это говорят. По отношению к 
Марксу это главным образом почему,  по отношению к современному марксистскому тоталитаризму -- это зачем. 


\section{Почему}
\epigraph{Он был непрактичен в мелочах, но практичен в великих делах. Совершенно беспомощный в тех случаях, когда приходилось 
справляться с собственным маленьким хозяйством, Маркс с несравненным талантом умел вербовать армию и руководить армией, которая 
должна совершить переворот в мире.}{Ф.~Меринг\cite{14}}

Основным побуждением, двигавшим Марксом, было отрица­ние современного ему капиталистического строя. И в самом деле, нельзя 
признать разумным и справедливым общество, в котором небольшая прослойка богатых людей почти бесконтроль­но распоряжается 
производством материальных благ, а миллионы бедных людей вынуждены продавать им себя как рабочую силу и влачить при этом 
полуголодное существование. Такое общество требует коренной перестройки. Средства производства, имеющие теперь четко выраженный 
общественный характер (и огромную стоимость), должны быть поставлены под эффективный контроль общества. Капитализм должен 
уступить место социализму.
Но Маркс не был первым социалистом. Он не был также первым революционером. И не он один понимал, что именно неимущие 
промышленные рабочие -- пролетариат, численность которо­го в то время непрерывно возрастала, представлял собой в тог­дашней Европе 
потенциальное взрывчатое вещество для революции. "Практичный в великих делах", Маркс был первым, кто поставил своей задачей ‑ в 
целях "вербовки армии", разумеется, -- \textit{доказать неизбежность}  победоносной пролетарской революции, сделать пролетариат 
\textit{избранным народом,}  призванным проложить путь в светлое будущее. В известном письме к И.~Вейдемейеру [15] Маркс пишет:

"Что касается меня, то мне не принадлежит ни та заслуга, что я открыл существование классов в современном обществе, ни та, что я 
открыл их борьбу между собою. Буржуазные историки задолго до меня изложили историческое развитие этой борьбы классов, а 
буржуазные экономисты -- экономическую анатомию классов. То, что я сделал нового, состояло в доказательстве следующего: 1) что 
существование классов связано лишь с определенными историческими фазами развития производства; 2) что классовая борьба 
необходимо ведет к диктатуре пролета­риата; 3) что эта диктатура сама составляет лишь переход к уничтожению всяких классов и 
обществу без классов."

Каким должно быть доказательство? Разумеется, \textit{научным;} иного не приняли бы в середине XIX века. Но что значит 
\textit{научно} доказать?  Это значит найти какой‑то \textit{объективный}  закон, т.е. закон, не зависящий от воли и сознания 
людей, подобно закону всемирного тяготения. Где искать такой закон? В экономике, на которую наука во времена Маркса уже начинала 
распространять свои методы. Так и возникает "исторический материализм", со­гласно которому материальное производство развивается 
по своим имманентным законам, а воля и сознание людей являются лишь их слепыми орудиями. И развитие производства "с неиз­
бежностью" приводит ‑ куда бы вы думали? -- к бесклассовому обществу, где все поровну, все справедливо, где человек челове­ку 
друг, товарищ и брат. Какая мудрость со стороны паровых машин, рычагов первого и второго рода и тому подобных желе­зок!

Концепция Маркса является, по существу, историческим \textit{детерминизмом,}  а не материализмом, причем детерминизм этот 
\textit{нигилистический,}  отрицающий ведущую роль духовной культуры в истории. Оба эти аспекта Маркс заимствовал из современной 
ему науки. Но это еще не делает марксизм научным мировоззре­нием, ибо он отвергает самое главное в науке -- критический научный 
метод. По своему стилю мышления марксизм ‑ учение религиозно‑догматического типа.

Когда человек основывает свою политическую деятельность на принципе "бытие определяет сознание", он должен непрерывно убеждать 
себя и других в справедливости этого принципа. Логика борьбы приводит к тому, что этот принцип постепенно преобразуется (на 
деле, если не на словах) к виду "экономика -- все, духовная культура ‑ ничто". И человек становится нигилистом -- врагом 
духовной культуры, а значит, и культуры вообще.



\section{Марксистский нигилизм}


Многих приверженцев марксизма привлекают в нем его пози­тивные аспекты: социалистические идеалы и решимость искать действенные 
методы для их осуществления. Однако нигилисти­ческий аспект марксизма -- это его важнейшая черта, определя­ющая судьбу 
марксистских политических течений. Именно эта черта была взята Лениным за основу, когда он переносил западное учение на 
восточную почву. Многое утерялось в процессе переноса, но марксистский нигилизм остался. Он дал чудовищные всходы на новой почве 
и привел ленинскую партию к массовому уничтожению людей, а затем и к самоуничтожению. Вопрос этот хорошо исследован русскими 
философами -- свидетелями воз­никновения и развития русского марксизма: Бердяевым, Фран­ком.

По марксистской теории считается, что в основе всего учения лежит диалектический материализм; затем из него выводится 
исторический материализм, а из этого последнего -- социальные и политические установки. Можно не сомневаться, что действительное 
движение в процессе становления марксизма шло в обратном направлении: от политической установки на пролетарскую революцию к 
общим принципам исторического материа­лизма, а отсюда -- к диалектическому материализму как фило­софии природы. Тезис -- материя 
первична, сознание вторично -- сам по себе не имеет точного смысла; если под этим понимать только то, что сознание появляется на 
определенном уровне развития материи (то есть всеобщей реальности), то это общеизвестный научный факт. Однако этот тезис 
включает в себя в обычной марксистской интерпретации нечто большее: утверждение, что сознание \textit{пассивно},  что оно 
\textit{исчерпывается} отражением действительности. Представление о творчестве и о связи этого понятия с сознанием и волей 
личности, которое столь характерно для современной европейской философии (беря начало, по‑видимому, от А.~Бергсона), полностью 
отсутствует в марксизме. Маркс писал, что для него процесс мысли есть только отражение процессов реальности, перенесенное в 
человеческую голову. Тезис "сознание вторично" имеет нигилистическую эмоциональную окраску: уменьшение значения таких понятий, 
как сознание и мысль, вплоть до полного их устранения как чего‑то несущественного, "вторичного". Посмотрите, как недвусмысленно 
(и с каким восторгом!) пишет об этом советский марксист Деборин:

"Историческая задача, стоявшая перед Марксом и Энгельсом, состояла в том, чтобы \textit{"поднять восстание против этого 
господства мысли"},  как говорится в проекте предисловия к "Немецкой идеологии". В этих нескольких словах действительно 
резюмируется тот огромный переворот в области философии, ис­торической науки и мировоззрения вообще, который совершен марксизмом. 
Марксизм \textit{поднял восстание против господства мысли, подчинив ее материальной действительности.}" [16]

Хороша философия, нечего сказать.

Если даже для мысли марксизм уготовил такую жалкую участь, то что ожидает доброту, терпимость, любовь?

Перестановка двух высших уровней в иерархии планов поведения, производимая марксизмом, уничтожает общечеловеческие ценности. Для 
революционера‑разрушителя эта операция -- сущая благодать. Человека как такового больше нет. Есть только представители различных 
классов. Классовый интерес -- наивысший интерес, а кто говорит, что он думает об общечеловеческих надклассовых  интересах, -- 
обманщик. Общество распадается. По отношению к "классовым врагам" все позволено -- вот вывод, к которому приходит марксизм и 
ради которого существует вся его философская часть.

С этой точки зрения интересно сравнить Маркса и Энгельса с Фейербахом. Единственная по‑настоящему ценная идея диалек­тического 
материализма ‑ перетолкование гегелевской диа­лектики в материалистическом духе -- принадлежит, как извест­но, Фейербаху. Такое же 
материалистическое, позитивистское толкование дает Фейербах и религии, в частности христианству. Читая "Сущность христианства", 
испытываешь порой удивле­ние, как современно это сочинение по подходу, методу, и как много высказанных в нем мыслей сохранило до 
сих пор свое значение.

Что же не понравилось Марксу и Энгельсу в Фейербахе? То, что в основе социальной философии Фейербаха лежит понятие о 
\textit{человеке}  и \textit{общечеловеческих} ценностях. Поэтому марксизм признает Фейербаха материалистом вообще, но объявляет 
иде­алистом в области истории. В письме Марксу от 19 августа 1846 г. Энгельс дает такую характеристику только что вышед­шей новой 
книге Фейербаха "Сущность религии": "Если от­влечься от нескольких тонких замечаний, то он, в общем, торчит целиком в старом 
болоте\ldots Опять все "сущность", "человек" и пр." Особенно возмущает Энгельса, когда в социологическом или философском 
контексте говорят о \textit{любви}.  В брошюре, по­священной Фейербаху, он пишет: "Мы не должны, однако, забы­вать, что именно за 
обе эти слабые стороны Фейербаха ухватился "истинный социализм", который, как зараза, распространялся с 1844 г. в среде 
"образованных" людей Германии и который научное исследование заменял беллетристической фразой, а на место освобождения 
пролетариата путем экономического пре­образования производства ставил освобождение человечества посредством <<любви>> -- словом, 
ударился в самую отвратитель­ную беллетристику и любвеобильную болтовню." [17] Это писа­лось в 1888 году. Вероятно, лет на 3040 
раньше Энгельс написал бы не "путем экономического преобразования производства", а "путем победоносной пролетарской революции" 
или что‑нибудь в этом роде. С годами он стал менее воинственным, но отвращение к слову "любовь" сохранилось в полной мере. Чтобы 
убедить читателей в неуместности этого слова, Энгельс несколько раз в своей брошюре прибавляет к слову любовь прилагательное 
"половая", совершая таким образом явную подтасовку, подлог при обсуждении учения Фейербаха. Он пи­шет: "И таким образом у 
Фейербаха, в конце концов, половая любовь становится одной из самых высших, если не самой выс­шей формой исповедания его новой 
религии".18 Между тем, на этой же странице приводится цитата из Фейербаха, в кото­рой говорится, что "сердце -- сущность 
религии". Сердце, а не половые органы.

Марксизм отличается от других социалистических учений тем, что он последовательно делает ставку на ненависть -- клас­совую 
ненависть. Кто отрицает ненависть, рассматривается марксизмом как враг -- такой же опасный, как и прямой классовый враг. Отсюда 
и желчь Энгельса по отношению к "люб­ви". Он снова и снова возвращается к этой теме:

"Но любовь! -- Да, любовь везде и всегда является у Фейер­баха чудотворцем, который должен выручать из всех трудно­стей 
практической жизни, -- и это в обществе, разделенном на классы с диаметрально противоположными интересами! Та­ким образом, из его 
философии улетучиваются последние остат­ки ее революционного характера и остается лишь старая песен­ка: любите друг друга, 
бросайтесь друг другу в объятия все, без различия пола и звания, -- всеобщее примирительное опьянение!" 19
Практичность Маркса и марксистов в великих делах по меньшей мере сомнительна. Отказ от общечеловеческих ценно­стей не про ходит 
безнаказанно. Сначала объявляются вне закона представители "враждебного класса". Но аппетит приходит во время еды. У Ленина мы 
читаем:

"Нельзя писать про товарищей по партии таким языком, ко­торый систематически сеет в рабочих ненависть, отвращение, презрение и т. 
п. к несогласномыслящим. \textit{Можно  и должно}  пи­сать именно таким языком про отколовшуюся организацию".
Но ведь "отколовшаяся организация" представляет тот же класс! Читаем дальше:

"\ldots Надо было возбудить в массе ненависть, отвращение, презрение к эти людям, которые \textit{перестали}  быть членами 
единой партии, которые стали политическими врагами, ставящими на­шей с.‑д. организации подножку в ее выборной кампании. По от­
ношению к \textit{таким}  политическим врагам я вел тогда -- и в случае повторения или развития раскола \textit{буду вести 
всегда} --  борьбу \textit{истребительную}".[20]

Систематически сеять ненависть, вести истребительную вой­ну -- это ключевые понятия ленинизма. Теперь этот арсенал при­меняется 
уже не к враждебному классу, а к соперничающей фракции в партии. Известно, к чему это привело в конечном счете. Волна ненависти 
и истребления прокатилась по всей стра­не, сметая выдуманные, не существующие реально границы между классами, партиями и 
фракциями. Ибо только граница, отделяющая человека от животных, существует реально, да и та не является абсолютной. А границ 
между классами нет вовсе.

У хищных животных, наделенных орудиями убийства, существует система инстинктов, которая не позволяет им убивать друг друга в 
массовом порядке. У человека таких инстинктов нет, а орудия убийства есть, и много более страшные. Но вместо инстинктов у 
человека есть культура, а в ней -- установления, которые выполняют ту же функцию. Пока культуры были племенными, эти 
установления относились только к соплеменникам; когда культура стала глобальной, они стали общечеловеческими. Когда эти 
установления разрушаются, человек превращается в нечто гораздо худшее, чем животное. Теоретическая желчь Энгельса по поводу 
"старой песенки: любите друг друга" оборачивается на практике Соловками и Магаданом.

\section{Зачем}

\epigraph{"Сколь жалко то общество, -- восклицает Маркс, -- которое не знает лучшего способа защиты, чем палач!" Но во времена 
Маркса палач по крайней мере еще не сделался философом\ldots}{А. Камю [21]}

Но если нигилизм по отношению к общечеловеческим ценностям духовной культуры, который выводится из принципа "бытие определяет 
сознание", оказывает разрушительное действие на общество, то зачем же нужен этот принцип тоталитарному обществу?
Ответ прост: разрушение духовной культуры опасно для общества, в котором духовная культура является основой стабильности или 
хотя бы ее существенным элементом. Для тоталитарного общества, основанного в конечном счете на страхе перед физическим насилием, 
разрушение духовной культуры необходимо для стабильности. Так, ядовитые вещества убивают органическую материю, но не вредят 
мертвой металлической конструкции: они лишь очищают ее от наростов. Поэтому одна и та же система идей, один и тот же язык и 
стиль мышления с успехом используется тоталитарным марксизмом как для разрушения общества до захвата власти, так и для его це­
ментирования после захвата власти. Но это цементирование -- насильственное, механическое скрепление частей, это заклю­чение в 
кандалы.

Здесь я должен сделать оговорку, которую, быть может, стоило бы сделать раньше. В каждом общественно‑политиче­ском течении, в 
частности в марксизме, обычно существует много различных слоев и прослоек. Есть марксисты, которые выступают за плюралистическую 
демократию. Есть марксисты, которые готовы принять позитивистскую философию природы. Есть марксисты, которые принимают Ленина, 
хотя и с некоторыми оговорками, но активно выступают против тоталитаризма. И все они -- как и правоверные адепты советского 
марксизма‑ленинизма -- ссылаются на Маркса и Энгельса, находят в их сочинениях выражение своих взглядов и считают себя их 
последователями. Я знаю, что некоторые из честных и мыслящих людей, считающих себя марксистами, будут обвинять (и на деле уже 
обвиняют) меня в том, что я искажаю концеп­цию Маркса, подхожу к ней односторонне и упрощаю ее. Мне будут говорить (и уже 
говорят), что то, с чем я воюю, -- это не "настоящий " марксизм, а его вульгаризованный советский вариант, далекий от взглядов 
основоположников.

Это в значительной мере так и есть. Я говорю действитель­но о советском официальном марксизме‑ленинизме и имею на то веские 
основания. Ибо именно этот марксизм и есть \textit{насто­ящий}  марксизм, под знаком которого живут миллионы людей, который 
преобразовал и продолжает (увы!) преобразовывать мир. Я не ставлю своей задачей подвергать всестороннему ана­лизу взгляды Маркса 
и Энгельса в их отношении к советскому марксизму. Конечно, Маркс и Энгельс не были проповедниками тоталитаризма. Напротив, к 
концу жизни они явственно увиде­ли угрозу тоталитарного ("казарменного") социализма и сдела­ли несколько предупреждений. Но это не 
меняет дела. Марк­сизм стал огромной мировой силой именно в тоталитарной фор­ме. Это невозможно отрицать и невозможно приписать 
случай­ности. В марксизме есть черты и элементы, которые я не толь­ко принимаю и приветствую, но в которых вижу единственную 
надежду на спасение человечества; я буду говорить об этом во второй части книги. Но не эти элементы специфичны  для марксизма, 
они щедро рассыпаны по всей европейской культуре 19‑го и 20‑го века. Специфичны для марксизма как раз те эле­менты, которые в 
своей совокупности привели -- и всегда будут приводить -- к тоталитаризму. Эти элементы просты и примитивны. Их невозможно 
вульгаризировать, потому что они и так до предела вульгарны. Все просто, как сказано у Михаила Булгакова.


После захвата власти марксистами принцип "бытие определяет сознание" меняет свой язык (выражаясь языком мате­матики) благодаря 
простому приему, которому нельзя отказать даже в некотором логическом изяществе. Новое общество объявляется бесклассовым или 
почти бесклассовым -- в том смысле, что если классы и остаются, то отныне противоречия между ними объявляются "не 
антагонистическими". Теперь вся та аргументация, которая работала на развал государства, начинает работать на укрепление 
государства. Гениально и просто. И действительно, нет в мире государства сильнее марк­систского. Правда, Маркс учил, что 
государство при социализ­ме отомрет, но марксисты об этом помнят почему‑то только до захвата власти, а после захвата власти сразу 
же забывают. Дик­татура пролетариата -- а на деле, конечно, диктатура партийной бюрократии -- которая считается в теории лишь 
переходным этапом, оказывается конечным этапом, целью преобразования. И это неудивительно, ибо, по правде говоря, марксистское 
уче­ние об отмирании государства -- чистый вздор, романтическое пустословие. Государство не может отмереть, и люди, пришед­шие к 
власти, это очень хорошо понимают. Согласно марксист­ской теории обобществление средств производства должно ре­шить все основные 
социальные проблемы. В действительности, конечно, ничего подобного не происходит. Новые властители встают перед теми же 
проблемами, что и прежние властители. Но они верят только в экономические преобразования, в необ­ходимость сплоченности, в 
железную дисциплину, в руководя­щую роль партии, в то, что бытие определяет сознание -- во что угодно, но только не в ведущую 
роль духовного начала, в необходимость терпимости и любви. Причем эта вера -- вопрос не только \textit{практики},  но и 
\textit{теории, принципа.}  Какое же государст­во, кроме тоталитарного, могут основать эти люди?



\section{Экономический фетишизм}

Исторический материализм. Бытие определяет сознание. Что это все означает в понимании советского человека?

Философские категории и формулировки нельзя приравнивать к утверждениям науки, которые могут быть проверены или использованы для 
предсказания событий, планирования действий и т.~п. Философия соединяет в себе черты науки и искусства. Как и наука, она 
пользуется абстрактными понятиями. Как и искусство, она использует сугубо неформализован­ный и, пожалуй, неформализованный язык; 
она работает на уров­не интуиции, создавая определенную атмосферу или стиль мышления, не осознаваемую отчетливо систему образов, 
предпочтений, оценок. На поведение человека эти образы и оценки оказывают самое решающее влияние -- более сильное, как правило, 
чем логические рассуждения и точный расчет. Говоря об об­разах и атмосфере философии, бессмысленно вступать в споры по поводу 
отдельных утверждений и контрутверждений: в философских текстах, как и в художественной литературе или в священных книгах разных 
народов, всегда можно найти достаточно материала, чтобы сделать желаемый формальный вывод или отвертеться от нежелаемого. 
Например, советские философы, долго и со вкусом доказывая, что бытие определяет сознание, тут же обязательно делают (вместе с 
основоположника­ми) оговорку об "относительной независимости сознания". Один Бог знает, что под этим подразумевается, но зачем 
делается оговорка, ясно: чтобы их нельзя было обвинить в полном зачеркивании, отрицании роли сознания. Однако эта оговорка 
ничего не меняет по существу, она не меняет общей атмосферы, общего стиля исторического материализма. Философская атмосфера -- 
нечто гораздо более определенное, чем формальные ответы на те или иные вопросы.

В тезисе "бытие определяет сознание" советский человек понимает бытие как некую внешнюю самодавлеющую силу. Прежде всего, это 
материальное производство, которое раз­вивается по своим собственным имманентным законам, имею­щим детерминистический характер. 
Детерминизм вообще чрезвычайно характерен для советского стиля мышления. Современная наука твердо пришла к выводу, что законы 
природы -- это, по существу, только \textit{запреты}.  Они не столько \textit{определяют} развитие событий, сколько 
\textit{запрещают}  некоторые варианты. Советского человека всячески оберегают от таких представле­ний, внушают, что они 
"идеалистичны и ненаучны". Советский человек живет в атмосфере детерминизма 19‑го века, он склонен к фатализму.

Я уже говорил и хочу подчеркнуть снова, что основной тезис исторического материализма ничего общего не имеет с философским 
материализмом. Он утверждает только, что материальные явления, связанные с производством жизненных благ, являются решающими для 
исторического развития, а материальные явления в мозгу человека, связанные с мышлением и духовной культурой, есть нечто 
производное и вторичное -- "отражение" первой группы явлений. Во всяком случае, имен­но так этот тезис воспринимается миллионами. 
Это не матери­ализм, а обскурантизм, примитивная философия потребителя. Она создает теоретическую базу для регресса, 
дегуманизации.

В учебнике исторического материализма мы читаем:

"Как система знаний о мире наука возникла необходимым образом из практики и для практических потребностей человечества. 
Современные же буржуазные идеологи рассматривают науку как чистое создание человеческого разума, считая, что все важнейшие 
события в истории науки происходили благодаря энтузиазму и жажде к знаниям отдельных одаренных людей. \textit{Однако при таком 
объяснении нельзя понять, откуда берутся энтузиазм, жажда к знаниям и интерес одаренных людей}". [22]  Простим советскому автору 
безбожное перевирание взглядов своих оппонентов -- это обычная практика: ему и в голову не приходит, что жажда к знаниям может 
быть имманентным свойством человеческой личности\ldots

Какой парадокс: Маркс в своей теории исходил из борьбы с товарным фетишизмом, обвиняя общество в том, что фети­шизм является 
частной  религией граждан; последователи Маркса возвели тот же экономический фетишизм в ранг обязательной 
\textit{государственной религии}. 


\section{Личность и общество}

Одно из любимых слов советской пропаганды и учебников марксизма -- это "народ". "Народ -- творец истории". "Народные массы -- 
решающая сила исторического развития". Но для советского человека <<народ>> -- такая же абстракция, такая же внешняя, безличная 
сила, как и "бытие". Чтобы установить связь между понятиями "я" и "народ", надо проанализировать, каким образом акт моей 
свободной воли может повлиять на народ в целом. Если анализировать эти информационные пути конкретно, то мы неизбежно приходим к 
понятиям об основных правах личности. Но это как раз то, чего советская идеология стремится ни в коем случае не допустить. 
Термин "народ" ис­пользуется для того, чтобы затушевать роль личностного нача­ла. А что я? -- говорит советский человек. -- Я -- 
как все.

Как трактует марксизм понятие о человеческой личности?

Раскроем "Философский словарь" последнего издания. [23] Эта книга, между прочим, имеет свою предысторию. В сороковых‑пятидесятых 
годах вышло несколько изданий "Краткого философского словаря", очень точно отражавших боевой дух той эпохи. "Краткий философский 
словарь" прославился тем, что в одно из его изданий вошел только что появившийся тер­мин "кибернетика", и объяснен он был 
примерно так: реак­ционная лженаука, созданная лакеями империализма для отвле­чения трудящихся от классовой борьбы. Эта 
формулировка стала притчей во языцах и попортила много крови советским философам. Пришлось признать ее ошибочной. В относительно 
либеральную эпоху конца 60‑х годов было решено составить новый словарь философских понятий. К работе были привлечены многие 
философы, слывшие вольнодумцами; для многих из них дорога в печать была в дальнейшем совершенно перекрыта, иные и эмигрировали. 
Новый словарь разительно отличается от старого. Исчезли бранные слова, стало видно же­лание дать хоть какую‑то информацию о 
"потусторонней" философии.

Так вот, раскрываем этот либеральный словарь на статье "Личность" и читаем:
"Человек, со своими социально обусловленными и индивидуально выраженными качествами: интеллектуальными, эмоциональными, 
волевыми. Научное понимание Личности опирается на марксистское определение сущности человека как со­вокупности общественных 
отношений. Отсюда вытекает, что свойства, присущие Личности, не могут быть врожденными, а в конечном счете определяются 
исторически данным строем общества\ldots"

Стоп, говорим мы и протираем глаза. Личности без врож­денных свойств -- это уже чересчур, даже для марксистов. Не может быть. 
Здесь что‑то не так.

И действительно, присмотревшись внимательнее, мы заме­чаем, что между началом расшифровки, которую мы читаем, и заголовком статьи 
("Личность") стоит цифра 1. Несколько ниже стоит цифра 2, и начинается вторая расшифровка:
"В психологии -- каждый отдельный человек с присущими ему индивидуальными особенностями характера, интеллекта, эмоциональной 
сферы. К психологическим свойствам Лич­ности относятся\ldots" и т.~д.

Вот оно, оказывается, в чем дело! Для марксизма сущест­вует два разных понятия Личности. В истории и социологии личность -- это 
совокупность общественных отношений. Лич­ности -- не живые люди со своими врожденными и благопри­обретенными свойствами, а ходячие 
аспекты общественных отношений, как бы их олицетворения. Эти призраки, эти зом­би, подчиняясь объективным (еще бы не объективным 
-- объек­ты ведь!) законам, определяют динамику развития общества. В психологии личность -- это как раз то, что обычно и 
понимает­ся под личностью. Но к истории, к социальным проблемам это понятие не имеет ни малейшего отношения. Оно касается только 
ситуаций вашей личной жизни: скажем, когда вы ссо­ритесь с женой или идете лечиться к психоневрологу.

Трудно выразить яснее сущность тоталитарной философии.

Роль выдающейся личности в истории марксизмом не отри­цается, но она сводится к выражению "объективной необхо­димости", причем 
какой‑либо вполне конкретной необходи­мости. Представление о том, что будущее открыто перед нами, что оно есть результат нашего 
свободного -- индивидуального и коллективного -- выбора, это представление совершенно чуждо советскому стилю мышления. Когда 
наставники общества смотрят назад и объясняют нам то, что произошло, они в каждом повороте событий усматривают объективную 
необхо­димость -- благо задним числом это сделать нетрудно. Когда они смотрят вперед, они заранее каждое будущее решение властей 
объявляют тоже объективной необходимостью. Твор­ческий акт, открытие, изобретение -- все это также оказывается формами 
объективной необходимости. Изобретатель, согласно марксизму, не изобретает, а "удовлетворяет объективную потребность в 
изобретении".

Как всякий детерминизм и фетишизм советский марксизм не лишен доли мистицизма. Слова "объективная необходимость" только по форме 
напоминают о науке, по своему содержанию это эквивалент Рока у греков и Божьей воли у христиан. И подобно тому как христианские 
монархи производили свою власть от Бога, вожди марксистского общества производят свою власть от объективной необходимости. 
Услужливые жрецы истолковывают объективную необходимость так, как им велено сегодня. Рядовой советский человек, как и человек 
средневековья, принимает прорицания жрецов с некоторой долей недоверия, но саму законность жречества, законность основных 
принципов и подхода он под сомнение не ставит. Исторический материализм подается ему как наука, а к науке он испытывает полное 
уважение и доверие; наука для него -- нечто бесспорное, как Бог для средневекового человека.

Параллель между стилем мышления советского человека и человека средневековья до Реформации, которую я настойчиво провожу, -- 
отнюдь не внешнее, а напротив, глубинное сходст­во при чрезвычайно различающихся внешних символах. Мы имеем здесь то же 
социальное мироощущение, ту же концепцию личности и общества, личности и истории.

Средневековый стиль мышления и средневековый стиль жизни поддерживают друг друга. Говорят, что если человеку настойчиво внушать, 
что он вор, то он в конце концов и впрямь становится вором. Когда человеку постоянно внушают, что бытие определяет сознание, то 
это бытие в конце концов действительно начинает определять его сознание. Его личностное начало выветривается, он становится 
рабом обстоятельств. А так как вокруг себя он видит таких же ущербных людей, как и он, живущих по тем же законам, что и он, то 
его вера в справедливость основного тезиса исторического материализма становится незыблемой -- с вытекающими отсюда 
последствиями для его поведения. Круг замыкается. И уже невозможно разобрать: ведет ли себя советский человек столь послушно по­
тому, что верит в истмат, или он верит в истмат для того, чтобы ему был удобнее вести себя послушно.

Ну а начальство?

Те, кто стоит на самом верху лестницы, которым никто не может ничего приказать -- свободны ли они, ощущают ли они себя творцами 
истории?

Вряд ли. Для того чтобы творить историю, мало занимать высокое положение; надо еще уметь творить.  Все, что мы знаем о наших 
вождях, заставляет думать, что они этого не умеют. А без этого умения, без глубокого понимания мира и проник­новения в будущее, 
люди остаются рабами обстоятельств. Они живут в сутолоке сиюминутных проблем, они только реаги­руют, делают только необходимое, 
чтобы сохранить свое поло­жение. Политика в этих условиях сводится к интриге. Движе­ние нашего общества не подчинено творческой 
воле человека, оно происходит само по себе, вслепую, стихийно.

Я говорил выше о разных марксистах. Я хочу теперь проци­тировать Жана Жореса:

"Маркс заявлял, что до сих пор человеческое общество управлялось только фатальными силами, слепым движением экономических сил; 
институты, идеи были не сознательным делом свободных людей, а отражением осознанной обществен­ной жизни в человеческом мозге. В 
этом смысле мы находим­ся только в предысторическом периоде. Человеческая история начинается по‑настоящему только тогда, когда 
человек, освободясь от тирании бессознательных сил, будет управлять самим производством по своему разуму и желанию".24
Такой подход вызывает у меня полное понимание. Я, правда, не могу согласиться с чрезмерно категорической оценкой всего прошлого 
человечества -- были в ней все же светлые прорывы влияния гения; в этих прорывах, быть может, и есть сущность истории. Но 
поставленную здесь цель я приветствую всей душой. Увы, мы сейчас дальше от этого идеала, чем когда бы то ни было.
При переходе на уровень политики марксистско‑ленинская философия естественным образом преломляется в полное отсутствие понятия о 
правах личности. Вот передо мною лежит книга "Основы политических знаний". Москва, 1974 год. Я листаю ее и читаю заголовки и 
подзаголовки. "Человеческое общество и его развитие\ldots Общественные формации\ldots Производство и его развитие\ldots Почему 
рабочему классу необходима партия\ldots Какая партия нужна рабочему классу\ldots Чего требует партия от коммуниста\ldots Формы и 
методы партийного руко­водства\ldots Экономика развитого социалистического общества\ldots Укрепление Советского государства\ldots 
Развитие социалистиче­ской демократии… "Стоп. Может быть, здесь будет что‑то о правах личности? Нет. "Советы депутатов 
трудящихся\ldots Задачи профсоюзов\ldots Марксизм‑ленинизм о воспитании нового человека\ldots Развитие сознания масс\ldots 
Социалистическое соревнование\ldots"

Понятие о правах личности не входит в основы политических знаний в стране победившего социализма.



\section{Тоталитаризм и экономика}

Уничтожая личность, тоталитарное общество лишает себя источника творчества. Жестко ограничивая обмен информацией и идеями, оно 
закрывает себе дорогу для нормального раз­вития и, в частности, для развития народного хозяйства. Ниже я привожу выдержки из 
нашего совместного с А.Д. Сахаровым и Р.А. Медведевым письма руководителям Советского Союза весной 1970 года.
В течение последнего десятилетия в народном хозяйстве нашей страны стали обнаруживаться угрожающие признаки разлада и застоя, 
причем корни этих трудностей восходят к более раннему периоду и имеют глубокий характер. Неуклонно снижаются темпы роста 
национального дохода. Возрастает разрыв между необходимым для нормального развития и реаль­ным вводом новых производственных 
мощностей. Налицо многочисленные факты ошибок в определении технической и экономической политики в промышленности и сельском 
хозяйстве, недопустимой волокиты при решении некоторых неотложных вопросов. Дефекты в системе планирования, учета и поощрения 
часто приводят к противоречию местных и ведомственных интересов с общенародными, общегосударственными. В результате резервы 
развития производства должным образом не выявляются и не используются, технический прогресс резко замедляется. В силу тех же 
причин нередко бесконтрольно и безнаказанно уничтожаются природные богатства страны: вырубаются леса, загрязняются водоемы, 
затопляются ценные сельскохозяйственные земли, происходит эрозия и засолонение почвы и т.~д. Общеизвестно хронически тяжелое 
положение в сельском хозяйстве, особенно в животноводстве. Реальные до­ходы населения в последние годы почти не растут; питание, 
медицинское обслуживание, бытовое обслуживание улучшаются очень медленно и территориально неравномерно. Растет чис­ло дефицитных 
товаров. Налицо некоторые явные признаки инфляции. Особенно тревожно для будущего страны замедление в развитии образования; 
фактически наши общие расходы на образование втрое меньше, чем в США, и растут медленнее. Трагически возрастает алкоголизм, и 
начинает заявлять о себе наркомания. Во многих районах страны систематически увеличивается преступность. В ряде мест растут 
симптомы явлений коррупции. В работе научных и научно‑технических организаций усиливается бюрократизм, ведомственность, 
формальное отношение к своим задачам, безынициативность.
Решающим, итоговым фактором сравнения экономических систем является, как известно, производительность труда. И здесь дело 
обстоит хуже всего. Производительность труда у нас по‑прежнему остается во много раз ниже, чем в передовых ка­питалистических 
странах, а рост ее резко замедлился\ldots
Сравнивая нашу экономику с экономикой США, мы видим, что наша экономика отстает не только в количественном, но и -- что самое 
печальное ‑ в качественном отношении. Чем новее и революционнее какой‑либо аспект экономики, тем больше здесь разрыв между США и 
нами. Мы опережаем Америку по добыче угля, отстаем по добыче нефти, безнадежно отстаем по химии и бесконечно отстаем по 
вычислительной технике. Пос­леднее особенно существенно, ибо внедрение вычислительных машин в народное хозяйство -- явление 
решающей важности, радикально меняющее облик системы производства и всей культуры. Это явление справедливо называют второй 
промышленной революцией. Между тем, мощность нашего парка вычислительных машин в сотни раз  меньше, чем в США, а что касается 
использования вычислительных машин в народном хозяйстве, то здесь разрыв так велик, что его невозможно даже измерить. Мы просто 
живем в другой эпохе.
Не лучше обстоит дело и в сферах научных и технических открытий. И здесь не видно возрастания нашей роли. Скорее наоборот. В 
конце пятидесятых годов наша страна была первой страной, запустившей спутник и пославшей человека в кос­мос. В конце шестидесятых 
годов мы потеряли лидерство, и первыми людьми, ступившими на луну, стали американцы. Этот факт является одним из внешних 
проявлений существен­ного и все возрастающего различия в ширине фронта научной и технологической работы у нас и в передовых 
странах Запада.
В двадцатые‑тридцатые годы капиталистический мир переживал период кризисов и депрессий. Мы в это время, исполь­зуя подъем 
национальной энергии, порожденной революцией, невиданными темпами создавали промышленность. Тогда был выброшен лозунг: догнать и 
перегнать Америку. И мы ее действительно догоняли в течение десятилетий. Затем положение изменилось. Началась Вторая 
промышленная революция, и те­перь, в начале семидесятых годов века, мы видим, что, так и не догнав Америку, мы отстаем от нее все 
больше и больше.
В чем дело? Почему мы не только не стали застрельщиками второй промышленной революции, но даже оказались неспособ­ными идти в 
этой революции вровень с передовыми капиталистическими странами? Неужели социалистический строй предоставляет худшие 
возможности, чем капиталистический, для развития производительных сил, и в экономическом соревновании между капитализмом и 
социализмом побеждает капитализм?
Конечно, нет. Источник наших трудностей -- не в социалистическом строе, а наоборот, в тех особенностях, в тех условиях нашей 
жизни, которые идут вразрез с социализмом, враждебны ему. Этот источник -- антидемократические традиции и нормы общественной 
жизни, сложившиеся в сталинский период и окончательно не ликвидированные и по сей день.   Внеэкономическое принуждение, 
ограничения на обмен информацией, ограничения интеллектуальной свободы и другие проявления антидемократических извращений 
социализма, имевшие место при Сталине, у нас принято рассматривать как некие издержки процесса индустриализации. Считается, что 
они не оказали серьезного влияния на экономику страны, хотя и имели тяже­лейшие последствия в политической и военной областях, 
для судьбы обширных слоев населения и целых национальностей. Мы оставляем в стороне вопрос, насколько эта точка зрения оправдана 
для ранних этапов развития социалистического народ­ного хозяйства -- снижение темпов промышленного развития в предвоенные годы 
скорее говорит об обратном. Но не подле­жит сомнению, что с началом Второй промышленной револю­ции эти явления стали решающим 
экономическим фактором, стали основным тормозом развития производительных сил страны. Вследствие увеличения объема и сложности 
экономи­ческих систем на первый план выдвинулись проблемы управ­ления и организации. Эти проблемы не могут быть решены одним или 
несколькими лицами, стоящими у власти и "знаю­щими все". Они требуют творческого участия миллионов людей на всех уровнях 
экономической системы. Они требуют широ­кого обмена информацией и идеями. В этом отличие современ­ной экономики от экономики, 
скажем, стран Древнего Востока.
Однако на пути обмена информацией и идеями мы сталки­ваемся в нашей стране с непреодолимыми трудностями. Прав­дивая информация о 
наших недостатках и отрицательных явле­ниях засекречивается на том основании, что она может быть "использована враждебной 
пропагандой". Обмен информацией с зарубежными странами ограничивается из боязни "проникно­вения враждебной идеологии". 
Теоретические и практические предложения, показавшиеся кому‑то слишком смелыми, пресе­каются в корне без всякого обсуждения, под 
влиянием стра­ха, что они могут "подорвать основы". Налицо явное недове­рие к творчески мыслящим, критическим, активным личнос­тям. 
В этой обстановке создаются условия для продвижения по служебной лестнице не тех, кто отличается высокими про­фессиональными 
качествами и принципиальностью, а тех, кто, на словах отличаясь преданностью делу партии, на деле отли­чается лишь преданностью 
своим узко личным интересам или пассивной исполнительностью.
Ограничение свободы информации приводит к тому, что не только затруднен контроль за руководителями, не только под­рывается 
инициатива масс, но и руководители промежуточного уровня лишены и прав, и информации и превращаются в пассив­ных исполнителей, 
чиновников. Руководители высших рангов получают слишком неполную, приглаженную информацию и тоже лишены возможности полностью 
использовать имеющие­ся у них полномочия\ldots
Какую бы конкретную проблему экономики мы ни взяли, мы очень скоро придем к выводу, что для ее удовлетворитель­ного решения 
необходимо научное решение таких общих, прин­ципиальных проблем социалистической экономики, как формы обратной связи в системе 
управления, ценообразование при отсутствии свободного рынка, общие принципы планирования и др. Сейчас у нас много говорится о 
необходимости научного подхода к проблемам организации и управления. Это, конеч­но, правильно. Только научный подход к этим 
проблемам позволит преодолеть возникшие трудности и реализовать те воз­можности, которые, в принципе, дает отсутствие 
капиталисти­ческой собственности. Но научный подход требует полноты ин­формации, непредвзятости мышления и свободы творчества. 
Пока эти условия не будут созданы (причем не для отдельных личностей, а для масс), разговоры о научном управлении оста­нутся 
пустым звуком.
Нашу экономику можно сравнить с движением транспорта через перекресток. Пока машин было мало, регулировщик лег­ко справлялся со 
своей задачей, и движение протекало нормаль­но. Но поток машин непрерывно возрастает, и вот возникает пробка. Что делать в такой 
ситуации? Можно штрафовать води­телей и менять регулировщиков, но это не спасет положения. Единственный выход -- расширить 
перекресток. Препятствия, мешающие развитию нашей экономики, лежат вне ее, в сфере общественно‑политической, и все меры, не 
устраняющие этих препятствий, обречены на неэффективность.
Пережитки сталинского периода отрицательно сказываются на экономике не только непосредственно, из‑за невозможности научного 
подхода к проблемам организации и управления, но в неменьшей степени и косвенно, через общее снижение твор­ческого потенциала 
представителей всех профессий. А ведь в условиях Второй промышленной революции именно творчес­кий труд становится все более и 
более важным для народного хозяйства.

В этой связи нельзя не сказать о проблеме взаимоотношений государства и интеллигенции. Свобода информации и творчест­ва 
необходима интеллигенции по природе ее деятельности, по ее социальной функции. Стремление интеллигенции к увели­чению этой 
свободы является законным и естественным. Госу­дарство же пресекает это стремление путем всевозможных ог­раничений, 
административного давления, увольнений с работы и даже судебных процессов. Это порождает разрыв, взаимное недоверие и глубокое 
взаимное непонимание, делающее труд­ным плодотворное сотрудничество между партийно‑государст­венным слоем и самыми активными, то 
есть наиболее ценными для общества слоями интеллигенции. В условиях современного индустриального общества, когда роль 
интеллигенции непрерывно возрастает, этот разрыв нельзя охарактеризовать иначе как самоубийственный.

\section{Наука и шахматы}

\epigraph{Да здравствует партия Ленина‑Сталина, открывшая миру Мичурина и создавшая в нашей стране все условия для расцвета передовой материалистической биологии!}{Т.~Д.~Лысенко, 1948~г.}

Детальный анализ влияния тоталитаризма на все аспекты жиз­ни не входит в мою задачу. Я хочу сделать лишь несколько штрихов.
Как оценить вклад Советского Союза в мировую науку? Можно попытаться сделать это, взяв в качестве критерия Нобелевские премии по 
естественным наукам. Ежегодно при­суждаются три Нобелевские премии за важнейшие открытия в области трех ведущих естественных 
наук: физики, химии и физиологии или медицины. Объективность присуждения этих премий признается во всем мире, они являются 
высшей меж­дународной наградой для ученого, и число Нобелевских премий, полученных гражданами какой‑либо страны, можно считать 
приблизительным выражением вклада страны в продвижение гра­ниц мировой науки (приложения и технология не учитываются).

Возьмем список Нобелевских премий за послевоенные тридцать лет: с 1945 по 1974 год включительно. Когда премия присуждается сразу 
двум или трем ученым, будем считать, что каждый получает половину или одну треть премии соответственно. Мы найдем, что за 
указанное время граждане двадцати стран получали Нобелевские премии. Если подсчитать полное число премий, полученных гражданами 
каждой страны, то первые шесть мест будут поделены между странами следующим образом :

1. США -- 42 2/3 премии
2. Англия -- 18 2/3
3. Германия -- 5 5/6
4,5. СССР, Швеция -- по 3 1/6
6. Франция --  2 1/2

Мы видим, что наш вклад не сравним с вкладом двух ведущих научных держав мира: США и Англии, которые вместе получили за 
последние 30 лет более 60 из 90 Нобелевских премий.

Чтобы как‑то оценить эффективность нашей системы в отно­шении научной продукции, вычислим для всех стран удельное число 
Нобелевских премий в расчете на каждые 10 млн. человек населения ("малая страна"). Получим следующую таблицу (стр.83).

Разумеется, порядок, в котором входят страны в эту табли­цу, нельзя считать точным отражением их научной эффективности из‑за 
большой относительной флуктуации нашего показателя в случае стран, граждане которых участвовали в получении премий один‑два 
раза. Однако в своей совокупности приведенные цифры позволяют сделать качественные выводы. Мы видим, что по удельному числу 
Нобелевских премий Советский Союз занимает двадцатое место из двадцати возможных. Мы опережаем только те страны, граждане 
которых за рассматриваемый период не получили ни одной Нобелевской премии. Если ограничиваться крупными странами с населением 
более 40 млн. человек, то это будут: Китай, Индия вместе с Пакистаном и Бангладеш, Индонезия, Бразилия, Нигерия и Мексика. Мы 
можем смело сказать, что наука развита у нас лучше, чем в этих странах.

Страна
Население в млн. чел. (на 1968 г.)
Полное число премий за 1945‑74 гг.
Удельное число премий
1. Швеция
7.9
3  1/6
4.0
2. Англия
55.3
18  2/3
3.4
3. Швейцария
6.1
1  5/6
3.0
4. Норвегия
3.8
5/6
2.19
5. США
200.0
42  2/3
2.133
6. Финляндия
4.7
1
2.128
7. Ирландия
2.9
1/2
1.55
8. Голландия
12.8
1 1/3
1.04
9. Германия (ФРГ + ГДР)
74.0
5  5/6
0.79
10. Чехословакия
14.4
1
0.69
11. Австралия
12.2
5/6
0.67
12. Аргентина
23.6
1  1/3
0.56
13. Португалия
9.5
1/2
0.53
14. Франция
50.3
2  1/2
0.50
15. Канада
20.8
1
0.48
16. Австрия
7.3
1/3
0.46
17. Бельгия
9.6
1/3
0.35
18. Италия
52.8
1  1/2
0.28
19. Япония
101.4
1  2/3
0.16
20.СССР
237.8
3  1/6
0.13

Люди, интеллектуально одаренные, лучше других видят фальшь и убогость нашей идеологической и политической системы. Уже одно 
только эстетическое чувство -- независимо от нравственных факторов -- мешает интеллектуально развитому человеку стать "своим" 
для системы; когда это удается, то только ценою самоискалечивания. Так что одаренные люди не только отталкиваются системой, но и 
сами стремятся от нее оттолкнуться. Это проявляется в том, что они ищут для себя как можно более "теоретического" поприща 
деятельности, где они могли бы быть в наименьшей степени связаны с систе­мой, достичь результатов в одиночку. В ту же сторону 
работают слабость нашей массовой технологии, организационная бестолковщина. Последствия этих факторов легко усмотреть в срав­
нительных успехах различных сторон советской культуры. На фоне мировых стандартов положение в математике у нас лучше, чем в 
физике, а в науке вообще -- лучше, чем в промыш­ленной технологии. Но лучше всего, в Советском Союзе быть шахматистом!

Человек, играющий в шахматы, делает ход сам,  на свой страх и риск; никто им не руководит и никто ему не мешает. Правила игры 
строго определены, постоянны и интернациональны --  это, кажется, единственный случай трогательного единства взглядов между 
представителями всех классов и партий. Нравственность партийна, искусство партийно, наука партийна и только шахматы беспартийны 
-- какая радость! И никому никогда -- даже в лучшие сталинские времена -- не приходило в голову разделить шахматные дебюты на 
материалистические и идеалистические. Наконец, государство любит и поддерживает шахматистов -- в целях завоевания престижа на 
международной арене. И вот вам результат: СССР -- неоспоримо сильнейшая шахматная держава.
Иногда мне кажется, что лучшие интеллектуальные силы страны брошены на шахматы.


\section{Искусство}

Одно литературное произведение вместе с контекстом его возникновения служит для меня символом советского искусства, каким его 
желает видеть партийное руководство. Этот символ -- карикатурен, но он не выдуман желчным сатириком, а взят из жизни.

Летом 1954 года я, свежеиспеченный выпускник университета, работал в городе Обнинске в 100 км  от Москвы, в том институте, 
стараниями которого в конце июня этого года была запущена первая атомная электростанция. 1‑го июля во всех центральных газетах 
была помещена информация Совета Министров СССР об этом событии. Была она помещена и в <<Литературной газете>> -- органе Союза 
советских писателей. На той же первой полосе газеты было помещено стихотворение Сер­гея Михалкова под названием "Бесценный 
вклад". С. Михалков и тогда уже был известным человеком, одним из авторов Гимна Советского Союза; сейчас он важный человек в 
Союзе советских писателей: председатель правления Московского отделения. Каждый читатель "Литературной газеты" имел возможность 
сравнить информацию Совета Министров с поэтическим шедевром Михалкова:

Совет Министров СССР:                               Сергей Михалков:
В настоящее время в Совет‑           Советские ученые
ском Союзе усилиями советских      Внесли в науку вклад,
ученых и инженеров успешно за‑      Пустив электростанцию
вершены работы по проектирова‑    В пять тысяч киловатт.
нию и строительству первой
промышленной электростанции
на атомной энергии полезной
мощностью 5 000 киловатт.
27 июня 1954 г. атомная элект‑       Во имя счастья Родины
ростанция была пущена в эксплу‑       Рабочих и крестьян
атацию и дала электрический ток       На дело благородное
для промышленности и сельского       Использован уран.
хозяйства прилежащих районов.
Впервые промышленная турбина         Стоит электростанция
работает не за счет сжигания угля         Могуча и сильна,
или других видов топлива, а за              На атомной энергии
счет атомной энергии -- расщеп‑          Работает она.
ления ядра атома урана.
Вводом в действие атомной эле‑           Стоит электростанция,
ктростанции сделан реальный шаг        Как явственный пример,
в деле мирного использования атом‑     Что мирным делом заняты
ной энергии,                                              У нас в СССР.

Все здесь проявилось: и задача, которую ставит себе поэт, и метод работы, и характер конечного продукта\ldots Не обошлось и без 
дефекта. Дело в том, что в информации Совета Министров был еще один -- пятый -- абзац, который почему‑то не нашел отражения в 
стихотворении Михалкова. Мне кажется невероятным, что Михалков мог допустить такой промах и не написать пятого куплета. 
Возможно, причина была техническая -- не хватило места на полосе или что‑нибудь в этом роде. Так или иначе, я тут же решил 
исправить ошибку и дописать необходимый куплет, который я и привожу вместе с последним абзацем информации Совета Министров:

Советскими учеными и инжене‑                  Советские ученые
рами ведутся работы по созданию                Готовят новый вклад:
промышленных электростанций на              Рассчитывают станции
атомной энергии мощностью 50‑                  В сто тысяч киловатт!
100 тыс. киловатт.

Мое завершение неоконченной Михалковым работы нашло полное одобрение моих товарищей -- тех самых, которые как раз и рассчитывали 
новые станции.

В уже цитированном мною учебнике по историческому материализму мы читаем:

"В буржуазном обществе вместо подлинно высокого искусства трудящимся навязывают продукты так называемой массовой культуры, 
примитивные по форме и антигуманистические по содержанию "творения" ремесленников от искусства. Лишь социалистическая революция 
открывает народу доступ к насто­ящему, высокому искусству, дает возможность каждому чело­веку, обладающему художественным вкусом, 
стать творцом искусства"25.


\section{Для послушных}

Счастлив ли советский человек?

При оценке субъективного мироощущения человека нашего общества в сравнении с людьми более открытых и свободных обществ очень 
важно учитывать фактор возраста. Как известно, онтогенез -- индивидуальное развитие особи -- до некоторой степени повторяет 
филогенез -- историю развития вида. Что‑то похожее имеет место и в культурном развитии. Особен­ности тоталитаризма относятся к 
высшему уровню социальной организации и культуры, поэтому до поры до времени, пока человек осваивает предыдущие слои культуры, 
он не чувствует непосредственно, впрямую оков тоталитаризма. Дети -- везде дети. И между детьми в разных странах больше общего, 
чем между детьми и взрослыми в одной стране. Взрослые могут видеть, как тоталитаризм калечит сознание детей с самого раннего 
возраста, как он воспитывает из них тоталитарных человеков. Но сами дети либо не чувствуют этого вовсе, либо воспринимают как 
нечто второстепенное. И правда, внутренний мир ребенка так динамичен и ярок, процесс познания действительности столь захватывает 
все его существо, что конкретные черты этой действительности отступают на задний план. То же относится и к юношескому периоду. 
Зрелище влюбленной пары или молодой матери с ребенком одинаково трогает при любом социальном строе. Любят везде одинаково; во 
всяком случае, это верно в смысле сравнения Советского Союза со странами Запада, ибо соответствующие слои культуры у нас 
примерно совпадают.

Но вот кончается период роста, и начинается период плодо­ношения. Субъективное чувство удовлетворенности зависит теперь от того, 
насколько человек осуществляет себя,  то есть в какой степени его вклад в культуру, в жизнь на планете соответствует тому, что 
он хочет и может дать. Потребность в самоосуществлении варьируется у людей в широких пределах. Есть люди, лишенные ее и вполне 
удовлетворяющиеся одной комфортабельностью жизни. Но это, я думаю, можно рассматривать как аномалию. Для нормального человека 
самоосуществление -- необходимое условие субъективной удовлетворенности, счастья.

И здесь тоталитарное общество ставит человеку свои жесткие пределы, и они тем теснее, чем ярче у человека выражены творческие 
импульсы. Обычно где‑то вблизи тридцати лет человек достигает этих пределов, упирается головой в этот потолок -- и превращается 
в старика, у которого все лучшее в жизни уже позади. В сущности, он умирает, ибо после тридцати лет жизнь без самоосуществления 
не имеет смысла. Продвиже­ние по службе, купленное ценой насилия над совестью, не дает удовлетворения; ощущение: не то, не то! ‑ 
остается. Эти‑то живые трупы, эти зомби, овеществленные общественные отно­шения, они‑то и голосуют за, горячо одобряют и 
единодушно поддерживают. А дома предаются своим тихим стариковским радостям.

Что можно сказать о быте, об условиях жизни?

Известно, что условия жизни рядового человека в Советском Союзе много хуже, чем в развитых странах Запада. В течение тридцати 
послевоенных лет они, в общем, непрерывно улучшались хотя и очень медленно. Быстрое улучшение жизни происходило только в течение 
нескольких лет после смерти Сталина (в последние сталинские годы уровень жизни в дерев­нях был ужасающе низок). За последние 
десять‑пятнадцать лет люди стали заметно лучше одеваться. С другой стороны, сохраняется острая нехватка некоторых из основных 
продовольственных товаров, в частности, мяса. Ассортимент продовольствия чрезвычайно узок, перед потребителями стоит вечная 
проблема "где достать?". Один француз, проживший в Москве год так резюмировал свои впечатления: "Советский человек идет в 
магазин так, как первобытный человек шел на охоту"\ldots А что бы он сказал, если бы жил не в Москве, а в провинции?

Недавно одна моя знакомая ездила в Мордовию на свидание с политзаключенным. В поселке, расположенном вблизи лагерной зоны, в тот 
день царило оживление: в магазин завезли треску, люди сообщали об этом друг другу, жены солдат и офицеров МВД, охраняющих 
заключенных, становились в очередь. Один офицер МВД говорил другому по телефону: "У нас тут треску привезли. Я на тебя тоже 
заказал".

Особенно плохое положение с овощами и фруктами, поэтому рацион питания советского человека совершенно неудовлетворителен с точки 
зрения современного диетолога. Овощи в государственных магазинах -- низкого качества, в продаже бывают в ничтожном ассортименте, 
да и то не всегда. Овощи на рынке очень дороги. Фрукты дороги и на рынке, и у государства. Зарплата моей жены (а у нее высшее 
образование и немалый стаж работы) целиком уйдет на то, чтобы покупать по государственным ценам два килограмма яблок в день26.

Так как у нас нет свободной печати, нет единственного действенного способа борьбы со злоупотреблениями властью и другими 
непорядками, затрагивающими интересы граждан. В газетах пишут о недостатках и злоупотреблениях, но лишь в том случае, когда они 
не связаны со слишком высоким начальством и не имеют общего значения; иначе это уже будет "очернение советской 
действительности". Данные о преступности, об эпидемиях, о стихийных бедствиях и т.~п. официально объявлены секретными. Еще 
больше данных, которые являются секретными неофициально. Все мы знаем, что медицинское обслуживание в нашей стране находится на 
ужасающе низком уровне. Врачи перегружены сверх всякой нормы, в поликлиниках -- очереди, в больницах невероятная теснота. Каждый 
может привести случаи, когда ошибки медиков приводили к смерти пациентов. Известно, что у нас практически невозможно достать 
современные эффективные лекарства, врачам запрещается даже выписывать больным дефицитные лекарства -- иначе от них посыпались бы 
жалобы на отсутствие этих лекарств в аптеках. Но эту проблему нельзя даже поставить открыто. И тем более нет средств потребовать 
от правительства, чтобы оно приняло какие‑то действенные меры. Что же касается медицинского обслуживания высшего 
партийно‑государственного слоя, то этим занимается специальное ведомство -- Четвертое Управление Министерства здравоохранения. И 
там, как всем известно, нет недостатка в современных лекарствах (их при­возят из‑за границы), а пациенты не стоят в очередях и не 
лежат в коридорах больниц.

Или возьмем проблему загрязнения среды обитания. Мне рассказали о таком факте. На территории Подольского завода цветных 
металлов, где расположены, в частности, жилые дома и общежития для рабочих, загрязнение воздуха в 80 раз превышает допустимую 
норму. Этот вопрос был поднят в ходе Всероссийского рейда по охране природы и обсуждался в местном райисполкоме в сентябре 1974 
года. Столь сильное загрязнение воздуха продолжается уже не первый год, и выступавший на заседании главный энергетик завода 
сообщил, что он не предвидит решения вопроса еще в течение двух‑трех лет. Ибо воздухоочистительные сооружения, во‑первых, очень 
дороги, а во‑вторых, их очень трудно достать.

Подольск расположен в 40 км от Москвы. Можно предположить, что на периферии подобные факты не менее редкое явление. Но как 
получить хотя бы оценочные цифры? Если какой‑то специалист в этой области и сумеет дать объективную, неприукрашенную оценку 
положения, то уж опубликовать ее не будет никакой надежды. Основная задача советской печати -- "воспитывать", а не 
информировать; информация допускает­ся лишь постольку, поскольку она не мешает "воспитанию".

Вот как освещает проблему загрязнения атмосферы кандидат географических наук К.А. Муравьев в брошюре, изданной обществом 
"Знание" для массового читателя:

"Проблема загрязнения атмосферы, как и всей окружающей человека среды, носит социальный характер. На первый взгляд может 
показаться, что загрязнение природной среды характерно для социалистических и капиталистических стран в равной степени, 
поскольку везде развита промышленность, и что загрязнение среды -- это неизбежное следствие технического прогресса. Но это 
только на первый взгляд. В действительности же главное различие состоит в отношении данного социального строя к природе, в 
способности системы к ликвидации загрязнения, к восстановлению жизнеспособной и благоприятной для человека среды.

Капиталистические корпорации отказываются нести ответственность за производимое ими загрязнение окружающей среды и любым 
способом пытаются скрыть свою вину. Они согласны на половинчатые меры и только при условии, если это не затронет их 
прибыли\ldots

Социализм устранил основные пороки капиталистического общества\ldots В социалистическом обществе сохранение природной среды 
является социальной необходимостью, поскольку это важно для общего блага и здоровья людей и отвечает инте­ресам всего общества. В 
социалистическом обществе также имеет место загрязнение среды, но в гораздо меньших масштабах, вызвано оно иными причинами, а 
именно тем, что ему досталась в наследство от капитализма уже в определенной степени загрязненная и нарушенная природная 
среда". [27]

Вот, оказывается, в чем беда -- в загрязнении среды, прои­зошедшем еще в царской России. Это очень оригинальное объяснение, 
особенно если принять во внимание следующие факты, которые сообщил мне один биолог:

В связи с загрязнением воды промышленными стоками резко сократилось количество рыбы в реках и озерах Советского Союза. В 1900 г. 
в Онежском озере вылавливалось 2000 центнеров лосося в год, сейчас -- 50 центнеров, то есть в 40 раз меньше. В Ладожском озере и 
окрестных реках уловы семги и лосося упали в десятки раз. Строительство заводов на берегах Байкала немедленно привело к 
загрязнению воды и падению уловов рыбы. По данным 1968 года, объем загрязняемой части озера приближался к 75 км3, что в 2500 раз 
превышает проектную цифру. С 1953 по 1967 г. улов омуля в Байкале уменьшился более чем в шесть раз. В Азовском море лишь за 
последние 20 лет улов рыбы уменьшился в 11 раз\ldots

Советское общество живет в неведении относительно самого себя. Есть старое русское выражение насчет того, что не следует, мол, 
"выносить сор из избы". Оно всегда поражало меня своей несуразностью: ведь совершенно очевидно, что если из избы не выносить 
сора, то в такой избе скоро станет не­возможно жить. Однако этот принцип и эта поговорка постоянно используются полуофициальной 
пропагандой. Аргумент таков: разоблачая наши недостатки, мы радуем наших врагов, даем им в руки оружие. И поговорка почему‑то 
работает, она настраивает людей против тех, кто критикует общество, обнажает его пороки. Она апеллирует к поверхностному, 
пошлому тщеславию, ради которого призывает пожертвовать элементарной чистоплотностью. Удивительно, что такая поговорка могла 
возникнуть, и еще более удивительно, что она продолжает существовать.

Человек в Советском Союзе живет в полной власти чудовищной бюрократической машины. Это относится и к так называемым "простым 
людям", и к тем, кто занимает высокое положение в бюрократической иерархии. Различие лишь в том, какие колесики люди вертят -- 
и, соответственно, какие колесики вертят ими.  Подчиненность бюрократической машине, принятие ее абсолютной власти над любой 
формой проявления жизни -- одинаково для всех, кроме отщепенцев; оно всех роднит и всех уравнивает. В настоящее время эта машина 
не так безумна, как раньше: она не уничтожает тех, кто ей подчиняется. Но она остается машиной -- вещью неодушевленной. Ее 
рычаги и колеса -- это бессмысленные, повторяемые из года в год формулировки, заменяющие нормальное логическое мышление; 
автоматическое, единогласное голосование; априорное признание правильным каждого решения вышестоящей инстанции; перекрытие 
информационных каналов и подтасовка фактов; тотальная слежка; устранение всякого проявле­ния индивидуальности; демагогия и 
массовое оболванивание. Тоталитарная государственная машина не поддается воздействию человеческой души и человеческого разума. 
Она подчиняется своим нечеловеческим законам, и тень этих законов ложится на каждое человеческое лицо.


\section{Для непослушных}

\epigraph{И ты раскаешься, бедный брат. Заблудший брат, ты будешь прощен. Под песнопения в свой квадрат Ты будешь бережно возвращен. \\ А если упорствовать станешь ты: \\
-- Не дамся!\ldotsПрежнему не бывать!\ldots-- Неслышно явятся из темноты Люди, умеющие убивать.}{Вл.~Лифщиц [28]}

При всей важности массовой обработки сознания и воздвижения информационных барьеров прямое физическое подавление инакомыслия 
продолжает оставаться центральным, необходимым элементом стационарного тоталитаризма. Это та ось, вокруг которой вращается 
маховое колесо тоталитарной государственной машины. Процент людей, к которому приходится применять насилие, невелик, но зато оно 
ложится на них всей тяжестью.

Просмотрим "Хронику текущих событий" за последние два года. В этом издании сообщается о тех политических репрессиях, которые 
становятся известными международной общественности. При чтении "Хроники" поражает жестокость наказаний за осуществление 
элементарного права человеческой личности -- права обмениваться информацией и идеями. Вот краткие сведения о некоторых (отнюдь 
не всех) судебных процессах.

К.~А.~Любарский, кандидат физико‑математических наук, астроном; октябрь 1972 г., г.~Ногинск, Московской обл. Обвиняется в 
размножении и распространении самиздата: "Технология власти" Авторханова, "Все течет" Гроссмана, "Хроника текущих событий". 
Приговор: 5 лет строгого режима.

С.~Глузман, врач‑психиатр; октябрь 1972 г., Киев. Обвинение в антисоветской агитации и пропаганде основывается исклю­чительно на 
показаниях свидетелей -- при обыске у него ничего изъято не было. Ему вменяется в вину "идеологическое разложение" сообвиняемой 
гр.~Л.~Середняк. Одна из свидетельниц, коллега Глузмана, сообщила, что на вопрос, почему он работает не в Киеве, а в Житомире, 
Глузман ответил: "Потому что я еврей". Это послужило для суда основанием обвинить его в сионистской пропаганде. Существует 
мнение, что действительной причиной осуждения Глузмана является подозрение КГБ, что он один из авторов документа, известного как 
"Заочная психиатрическая экспертиза по делу П.~Г.~Григоренко". Приговор: 7 лет лагерей строгого режима и 3 года ссылки.

Стефания Шабатура, художник‑прикладник. Ее работы упоминались в 6‑ом томе "Истории украинского искусства". В 1970 г. Шабатура 
вместе с группой львовских писателей и художников обратилась с просьбой присутствовать на суде над Валентином Морозом. Приговор: 
5 лет лагерей и 3 года ссылки. Детали об­винения неизвестны.

Супруги Калынец, Львов, 1972 г. Как и С. Шабатура, просили допустить их на процесс В. Мороза. Детали обвинения неиз­вестны. 
Приговор: 6 лет лагерей строгого режима и 3 года ссылки.

Г.~В.~Давыдов, инженер‑геолог, отец троих детей; Ленинград, июль 1973 г. Обвинение: изготовление и распространение самиздата. 
Приговор: 5 лет лагерей строгого режима и 2 года ссылки.

В.~В.~Петров, рабочий, по тому же делу. Приговор: 3 года лагерей строгого режима и 2 года ссылки.

И.~М.~Дзюба, литературный критик; Киев, март 1973 г. Единственный пункт обвинения -- написание и распространение ра­боты 
"Интернационализм или русификация". На суде Дзюба заявил, что работа не предназначалась для опубликования, а была сделана в виде 
письма на имя первого секретаря ЦК КПУ. Приговор: 5 лет лагерей строгого режима.

Е.~Сверстюк, Киев, апрель 1973 г. Обвинение: работы литературоведческого характера, опубликованные на Западе и в самиздате, 
"антисоветские" разговоры со знакомыми и сосе­дями. Приговор: 7 лет лагерей строгого режима и 5 лет ссылки.

Надежда Алексеевна Светличная, Киев, март, 1973 г. Инкриминируется хранение самиздата. Приговор: 4 года лагерей.
Иван Алексеевич Светличный, литературный критик; Киев, апрель 1973 г. В обвинительном заключении фигурировали пункты, 
предъявлявшиеся ему в 1965 году, когда он, просидев 8 месяцев в тюрьме, был выпущен за недоказанностью обвинения. Кроме того, 
ему инкриминировалось хранение неизданной художественной литературы на украинском языке, а также его собственные 
литературоведческие рукописи. Приговор: 7 лет лагерей строгого режима и 5 лет ссылки.

Леонид Плющ, математик; Киев, январь 1973 г. Инкриминируется: хранение нескольких экземпляров "Хроники текущих событий", "Украинского вестника" и др.; распространение некоторых из них среди знакомых. Написание 7 статей литературоведческого характера, содержание которых признано "антисоветским". Подписание открытых писем в ООН, "антисоветские разговоры". Л. Плющ был признан судом психически больным и направлен на принудительное лечение в психиатрическую больницу специального типа. Дело рассматривалось при закрытых дверях и в отсутствие обвиняемого. До настоящего времени (июль 1975 г.) Л. Плющ находится в закрытой психбольнице. В результате примененного к нему "лечения" его здоровье резко ухудшилось. Судьба Леонида Плюща привлекла внимание общественности во всем мире.

Ионас Лауцюс, завуч средней школы в Литве. Написал роман о жизни литовского народа после 1940 года. Роман был сдан в редакцию "Вага"; напечатан не был, так как был признан "антисоветским", "аморальным" и "антихудожественным". Тогда Лауцюс начал частями пересылать роман по почте своему брату, живущему в США. Арестован в июле, судим в декабре 1971 г. Приговор: 2 года лагерей.

А.~А.~Болонкин, авиационный инженер, доктор технических наук, автор около 40 научных работ. Москва, ноябрь 1973 г. Обвинение: изготовление и распространение самиздата. Приговор: 4 года лагерей и 2 года ссылки.

В.~Лисовой, Е.~Пронюк, И.~Семанюк. Киев, ноябрь 1973 г. При аресте Е. Пронюка (на улице) в его портфеле было обнаружено много машинописных экземпляров письма в ЦК и "видным людям Советского Союза": академикам, писателям, государственным деятелям и т.~п. Авторы письма -- В. Лисовой (кандидат философских наук, член КПСС) и Е. Пронюк -- сотрудники Института философии АН УССР. Они обращают внимание работников ЦК на ряд незаконных судебных про­цессов, прошедших на Украине в последнее время по политическим мотивам. Приговор: Лисовому -- 7 лет строгого режи­ма и 3 года ссылки, Пронюку -- 7 лет строгого режима и 5 лет ссылки, Семанюку -- 4 года строгого режима.

Полный и точный текст обвинительного заключения по политическому делу получить непросто. В тех случаях, когда это удается, бездоказательность, произвольность обвинения выступают особенно явственно в суконном, казенном языке судейских. Вот, например, обвинительное заключение в суде над тремя крымскими татарами, состоявшемся в г. Запорожье на Украине. Крымские татары, как известно, добиваются права жить на своей родной земле -- в Крыму, откуда они были выселены все до единого человека в 1944 году.

"Обвинительное заключение по уголовному делу по обвинению: Куртумерова Эскандера, Халикова Эвазера и Рамазанова Регата по ст. 187‑1 УК УССР.

Расследованием установлено:

Обвиняемые Куртумеров~Э., Халиков~Э. и Рамазанов после предупреждений органов государственной власти о нераспространении заведомо ложных измышлений, порочащих советский государственный и общественный строй, должных выводов не сделали и систематически продолжали свои преступные деяния.

Так, 4 марта 1973 г. в доме №~29 по ул.~Циолковского в г.~Мелитополе приняли активное участие в собрании молодежи, где в присутствии 25 человек, извращая национальную политику СССР, распространяли заведомо ложные измышления, порочащие советский государственный и общественный строй.

18 марта 1973 г. на втором собрании молодежи в доме №~36 по ул.~Цюрюпы в г.~Мелитополе, на котором присутствовало 20 человек, Куртумеров Э., Халиков и Рамазанов в своих выступлениях также клеветали на советскую действительность.

Кроме этого, обвиняемый Куртумеров изготовил письменные произведения "История", "Крым" и др., которые содер­жат ложные измышления, порочащие советский государственный и общественный строй, а также на брошюрах Т.И. Ойзермана "Марксистско‑ленинское понимание свободы", А. Кулагина "Поколение оптимистов" и на журнале "Вопросы истории" учинил явно клеветнические надписи.

Обвиняемый Халиков изготовил рукописные тексты: "Записку председателю Совета национальностей Верховного Совета СССР", "Протест", "Преступники торжествуют" и др., которые содержат клевету на советский государственный и общественный строй.

Обвиняемый Рамазанов также изготовил ряд рукописных текстов: "Людям доброй воли" и т.~п., адресованных в различные партийно‑советские органы, в которых клевещет на поли­тику КПСС, государственный и общественный строй.

Привлеченные и допрошенные в качестве обвиняемых, Куртумеров и Халиков виновными себя полностью не признали и пояснили, что они действительно принимали участие в собраниях молодежи, но на советскую действительность не клеветали".

Далее следует повторение тех же обвинений по отношению к каждому из трех в отдельности без какой‑либо конкретизации. Приговор: Куртумерову -- два года, Халикову и Рамазанову -- по два с половиной года лагерей.

Все обвинения по политическим делам очень похожи: "клеветнические" разговоры, изготовление и распространение "антисоветской" литературы; часто, как в только что приведенном случае, речь идет даже не о машинописных, а о рукописных документах: заявлениях, письмах и т.~п.

Листаем "Хронику" дальше. Весна 1974 года. Виктор Хаустов: 4 года лагерей и 2 года ссылки; Габриэль Суперфин:
5 лет лагерей и 2 года ссылки; Виктор Некипелов (г.~Владимир) : 2 года лагерей; Сергей Пирогов (г.~Архангельск): 2 го­да строгого режима; Валентина Пайлодзе (г.~Тбилиси, <<преступление>> -- письма религиозного содержания): полтора года лагерей; Фридрих Шнарр, советский немец, добивающийся выезда в ФРГ (г.~Джамбул, Казахстан) : 2 года. На допросы во время следствия Шнарра водили в наручниках. В камере след­ственного изолятора его сокамерники ежедневно, на протяжении трех месяцев, мучили и избивали его. Следователь знал об этом, но не пресекал бесчинства, угрожая Шнарру смертью.

"Хроника" №~33. Сообщается об условиях содержания политических заключенных в лагерях и тюрьмах. Заключенный В.П. Азерников, врач по специальности, в письме, которое ему удалось передать на волю, описывает условия в Мордовских лагерях строгого режима.
Заключенные живут в состоянии "скрытого голода". Калорийность пищи намного меньше той, что необходима по усло­виям тяжелого труда, которым занимаются заключенные\ldots В пище практически нет животных белков, витаминов. Нередки случаи пищевых отравлений.

Воздух в цехах густо насыщен древесной и абразивной пылью, парами ацетона и кислот. Это способствует развитию легочных заболеваний, силикозов. Лечение начинается лишь тогда, когда болезнь получает кризисное развитие, да и тогда оно сводится лишь к снятию симптомов. Хронические заболевания: желудочно‑кишечные, сердечно‑сосудистые, глазные болезни, грибковые заболевания, парадонтоз и т.~ п. не лечатся вовсе, хотя они имеют в лагере массовое распространение. Освобождение от работы можно получить лишь тогда, когда температура больного выше $37,4^{o}$.  Случаи освобождения при бестемпературных заболеваниях исключительно редки. Врач не может выйти за пределы так называемой "нормы" освобождений -- $17\%$ всех заключенных, даже во время эпидемии гриппа. Не во всех лагерях есть врачи, их заменяют фельдшеры, медсестры. Врачи‑специалисты посещают лагеря 1‑2 раза в год или реже. Врачи‑заключенные не могут помочь своим больным товарищам, это запрещено специальной инструкцией. Лагерные врачи имеют лишь простейшие лекарства, срок годности кото­рых иногда давно истек. Присылка медикаментов и витаминов с воли запрещена.
Грунтовая дорога между лагерями и больницей так плоха, а тюремные машины настолько неприспособленны для перевозки больных, что поездка может угрожать жизни больного. Известны случаи переломов конечностей и травм позвоночника в результате поездки. Для сердечников поездка по этой дороге вообще непереносима. Нередко люди, прибывающие в лагерь абсолютно психически здоровыми, к концу своих длительных сроков заболевают душевно.

2 ноября 1972~г. в Мордовских лагерях в возрасте 33 лет умер Юрий Тимофеевич Галансков [30]. Он был арестован в январе 1967~г. и приговорен к 7 годам строгого режима за свою деятельность в защиту гражданских прав. Язвенная болезнь в тяжелой форме, которой Галансков страдал еще до ареста, сильно утяжелила его жизнь в лагере. Родственники и друзья Галанскова, а также его солагерники, обращались к властям с ходатайствами об оказании Галанскову достаточно эффективной медицинской помощи. В частности, они просили о назначении ему диетического питания и о всестороннем обследова­нии в центральной больнице МВД, в Ленинграде. Эти ходатайства не были удовлетворены. Осенью 1972 г. Юрия Галанскова, в связи с ухудшением состояния, в очередной раз этапировали в больничную зону Дубровлага в поселок Барашево. После операции у него развился перитонит, и он умер.

В других лагерях положение не лучше. Вот один из эпизодов из жизни Пермских лагерей . Во время одной из медицинских комиссий ее председатель подполковник медицинской службы Т.~П.~Кузнецов заявил, что он приехал не затем, чтобы помиловать людей по болезни, а чтобы заставить их работать. Многим заключенным сняли инвалидность, которую те имели в течение многих лет. В частности, заключенный литовец Куркис страдал язвенной болезнью много лет и не работал. После решения комиссии о снятии инвалидности он был поставлен на тяжелую работу -- вспашку запретной зоны. В первый же день произошло прободение язвы. Начальник лагпункта -- 35 Пименов звонил Т.~П.~Кузнецову (хирургу этого лагеря); тот отказался выехать по вызову, сославшись на плохую погоду. Куркис умер.

Просочилось еще несколько писем от политзаключенных. К.~А.~	Любарский обращается к Всемирной федерации научных работников и к Конгрессу за свободу культуры. Научные работники, пишет он, люди в большинстве немолодые, занимаются в лагерях тяжелым и непривычным для них физическим трудом, который не оставляет ни сил, ни времени для интеллектуальной работы. Ученые дисквалифицируются. "Нас не только временно лишают свободы, -- говорит Любарский, -- нас навсегда лишают любимого дела, нашей профессии".

В письме группы политзаключенных Пермского лагеря ВС‑389/35 говорится о том, что при помощи максимально строгой изоляции власти стремятся скрыть правду о лагерной жизни людей, осужденных вопреки декларациям Конституции СССР о гражданских свободах. Цензурные правила таковы, что фактически позволяют задержать любое письмо. Уничтожение задержанных писем исключает возможность проверки обоснованности задержания. Власти не ставят перед собой провозглашенной, но непосильной для них, задачи переубеждения: их цель   сломить заключенных, заставить отречься от своих взглядов. Этой цели администрация стремится достичь придирками и наказаниями, вопреки закону подвергая заключенных физическим и моральным страданиям -- голоду, холоду, унижениям и т.~д. Тяжелый, иногда бессмысленный труд стал орудием наказания.

В других сообщениях мы читаем о судьбе борцов за гражданские права, получивших повторные длительные сроки заключения, в частности о Владимире Буковском и Валентине Морозе. Особенно тяжело положение В. Мороза. Украинец, преподаватель истории, он был первый раз арестован в 1965 г. за статьи, ходившие в самиздате, в которых был обнаружен "украинский национализм" и которые, следовательно, были признаны антисоветскими. Он был приговорен к четырем годам заключения. В 1969 году, отбыв срок, он вышел на свободу, а в 1970 г. был снова приговорен по статье 62~УК~УССР (аналог статьи 70~УК~РСФСР); на этот раз -- к девяти годам лишения свободы, из которых первые шесть лет -- в тюрьме, и пяти годам ссылки. Во Владимирской тюрьме сокамерниками В. Мо­роза оказались уголовные преступники; они изводили его, не давали спать ночью, а один из них порезал ему живот отточенным как нож черенком ложки, так что Мороза пришлось отпра­вить в больницу и наложить швы. После этого случая по просьбе самого Мороза и его жены его поместили в одиночку.

Отсидев больше трех лет в тюрьме, В. Мороз стал просить о переводе его в лагерь. По действующему законодательству, после половины назначенного судом срока тюремного режима он может быть заменен лагерем, при условии соблюдения заключенным правил режима. Валентин Мороз не имел нарушений, кроме одного: во время свидания с женой он говорил с ней по‑украински и отказался перейти на русскую речь, за что был лишен свидания. На этом основании ему отказали в переводе в лагерь. 1 июля 1974 г. В. Мороз объявил голодовку.

Жена В.~Мороза, Раиса Мороз, обратилась к мировой общественности с призывом спасти жизнь мужа. В течение последовавших месяцев она написала письма в Пен клуб, в Международный Красный Крест. Она много раз обращалась к администрации Владимирской тюрьмы за справками о здоровье мужа. Она обращалась в КГБ при СМ СССР, требуя свидания с мужем. В сентябре‑октябре Раису четырежды вызывали на бесе­ды в областное управление КГБ (г. Ивано‑Франковск) и уговаривали "перестать хлопотать о своем муже и заботиться лучше о себе". Ей угрожали увольнением с работы и возможностью расправы "со стороны каких‑нибудь хулиганов". Через день после разговора об этом камень, брошенный кем‑то, разбил окно в ее комнате и попал ей в лицо. Однако 5 ноября ей все же дали свидание с мужем. Обращаясь "ко всем добрым и гуманным людям на земле", Раиса Мороз пишет:

"5 ноября мой муж, политзаключенный Владимирской тюрьмы Валентин Мороз, получил свидание с семьей. Это был 128‑ой день голодовки\ldots

Валентин ужасающе худ (52 кг. при 175 см. роста). У него опухшее лицо и отеки под глазами. Он жалуется на боли в сердце. Но наибольшие мучения доставляет ему зонд, с помощью которого, начиная с 12‑го дня голодовки, осуществляется искусственное кормление. Этот зонд ранит стенки горла и пищевода. Когда его вытаскивают, он весь в крови, а боль, с самого начала сопровождавшая кормления, теперь не утихает, и в перерывах между ними Валентин почти постоянно находится в забытьи, но время от времени заставляет себя подняться на ноги, так как боится, что иначе они атрофируются\ldots

Сейчас, чтобы сохранить жизнь В. Мороза, его необходимо поместить в больницу и долго, тщательно лечить. Между тем, начальник тюрьмы утверждает, что, независимо от того, будет ли Валентин продолжать голодовку или окончит ее, он останется в тюрьме. Это равносильно смертному приговору\ldots

Неужели возможно в современном мире, чтобы человек, вся вина которого ‑ четыре журнальных статьи, признанных судом "антисоветскими", поплатился за это жизнью?".

Одно из самых бесчеловечных изобретений советской карательной системы -- помещение инакомыслящих в психиатрические больницы и принудительное "лечение" с помощью средств, подавляющих сознание и волю. Этот метод как бы символизирует роль тоталитаризма в историческом плане. Леонид Плющ в Днепропетровской специальной психиатрической больнице полу­чал по назначению врачей в больших дозах галоперидол в таблет­ках. На свидании с женой в октябре 1973~г. она увидела его та­ким: подавленное состояние, апатия, сонливость. В августе‑сен­тябре, до начала "лечения", он написал домой много писем, боль­ших и содержательных. Теперь же он почти перестал писать и да­же читать не может. Январь 1974 г.: состояние прежнее ‑ он поч­ти все время спит, читать и писать не может, на прогулку не хо­дит -- мерзнет. В феврале‑марте галоперидол заменили уколами инсулина с возрастающей дозировкой. Состоявшаяся в это вре­мя психиатрическая экспертиза сочла необходимым продолжать лечение. Члены комиссии с Плющем не беседовали. Лечащий врач Плюща, Л.~А.~Часовских, на вопрос жены, какие же симптомы заболевания свидетельствуют о необходимости продолжать лечение, ответила: "Его взгляды и убеждения"\ldots На дальнейшие вопросы о диагнозе и лечении она отвечать отказалась.
На свидании 4 марта 1974 г. Л.~И.~Плющ был неузнаваем. У него появилась сильная отечность, он с трудом передвигался, взгляд потерял свою обычную живость. Он сообщил, что врачи настаивают, чтобы он отрекся от своих взглядов и убеж­дений и обязательно в письменной форме. Это он сделать отказался.

На свидании 12~мая становится известно, что в апреле Л.~Плю­щу перестали давать какие бы то ни было препараты. Плющ объясняет это тем, что у него появились боли в брюшной поло­сти и врачи испугались. После отмены лекарств состояние его улучшилось: стали спадать отеки, прошли боли. Его перевели в другую палату, где меньше больных и тише. Он опять стал читать. Но затем ему снова стали делать уколы инсулина с возрастающей дозировкой. Появилась аллергическая сыпь, зуд, однако уколы не прекратили. После каждого укола Плю­ща на четыре часа привязывали к кровати.

В октябре 1974~г., когда инсулин уже был отменен, а но­вое лекарство еще не назначено, врачи предложили Л.~И.~Плю­щу написать заявление с осуждением своей "антисоветской деятельности" по типу заявлений Якира и Красина. Плющ кате­горически отказался это сделать:

-- Якир солгал. Вы хотите, чтобы я стал лжецом?

Вскоре ему назначили новый препарат -- трифтазин в таблетках, большими дозами. 15 ноября 1974~г. Плющ был помещен в "надзорную" палату, где вместе с ним находятся больше 20 агрессивных больных. Свет горит круглосуточно. Больных никуда не выводят: даже уборная находится в палате. Трифтазин ему стали вводить уколами. Уколы вызывают у него сон­ливость, инертность, судороги; свет мешает спать. Плющ не гуляет: не может или не разрешают, неизвестно. На очередном свидании с женой Плющ почти ничего не говорил, ни о чем не спрашивал, даже о детях (у них двое детей). Плющи обрати­лись к властям с просьбой разрешить выехать из СССР, но получили отказ.

\ldots Осень 1974 года. Суд над М.~Р.~Хейфецом, учителем русского языка и литературы в Ленинграде. Главный пункт обвинения -- написание им статьи "Иосиф Бродский и наше поколение", которую он показывал нескольким знакомым. Приговор: 4 года лагерей строгого режима и 2 года ссылки. Суд над Л.~А.~Ладыженским и Ф.~Я.~Коровиным (г.~Рига). Обвинение: распространение самиздата. Приговор: Ладыженскому 3 года строгого режима и 3 года ссылки, Коровину -- два года лагеря и два года ссылки.

В течение 1974 года в Армении проходила серия судебных процессов над армянами по обвинению в национализме и, в частности, в принадлежности к "Национальной объединенной партии Армении", ставящей своей целью отделение Армении от СССР. Всего было осуждено 14 человек, получивших разные сроки до семи лет строгого режима.

Винница, декабрь 1974~г. Врач М.~Ш.~Штерн (56 лет) приговорен к 8 годам лагерей усиленного режима по обвинению в получении взяток и мошенничестве. Люди, знакомые с делом, утверждают, что обвинение не было доказано в судебном разбирательстве. При обыске в квартире Штерна прокурор Кравченко в присутствии сына М.~Ш.~Штерна Виктора и его жены и понятых сказал: "Предъявление обвинения связано с желанием вашей семьи выехать в Израиль".

Вильнюс, декабрь 1974~г. Пять осужденных. Максимальный срок -- 8 лет строгого режима (Петрас Плумна).

Январь‑апрель 1974~г. -- серия судов в Казахстане над советскими немцами, добивающимися выезда в ФРГ. Абель, Тиссен, Вернер, Фертих -- по три года лишения свободы. Братья Валентин и Виктор Клинны -- 2 года. Август. Несколько нем­цев осуждено в Эстонии\ldots

Все эти события происходят на фоне множества обысков, допросов, задержаний и внесудебных репрессий (увольнение с работы и т.~п.). Готовятся новые судебные процессы. В октябре 1974~г. арестован В.~Осипов, в декабре -- С.~Ковалев. 18 апреля 1975~г. арестован А.~Твердохлебов, секретарь советской группы "Международной Амнистии". Одновременно было проведено еще три акции против членов этой группы: в Киеве был арестован, а через два дня отпущен с подпиской о невыезде писатель Микола Руденко, в Москве были проведены обыски у В.~Альбрехта и у меня. В течение 12 часов представители прокуратуры и КГБ хозяйничали в моей квартире. Список изъятых материалов (в их числе рукописные наброски, письма и т.~п.) включает 212 наименований. В течение июня меня допрашивали шесть раз: формально -- как свидетеля по делу Твердохлебова, по существу же, конечно, моя персона также была предметом расследования. Это видно хотя бы из вопросов, которые мне задавались. В их числе были такие: "Какое участие Вы принимали в помощи детям политзаключенных?", "Откуда у Вас информация о деле Светличного на Украине?" и т.~п. [32]


\chapter{СОЦИАЛИЗМ}

\section{Пессимизм и оптимизм}
\epigraph{Подлинный оптимизм не имеет ничего общего с какими‑либо снисходительными суждениями. Он состоит в стремлении к осознанному идеалу, который внушает нам глубокое и последовательное утверждение жизни и мира. Поскольку ориентированный таким образом дух здравомыслящ и беспощаден в оценке существующего, он при обычном рассмотрении предстает пессимизмом.}{Альберт Швейцер}

Картина советского общества, нарисованная в первой части книги, может внушить читателю мысль, что автор -- пессимист, склонный видеть вещи в мрачном свете. Но это далеко не так. Констатация неприятных истин обязательна и для пессимиста, и для оптимиста. Различие между ними в другом. Пессимист -- это тот, кто якобы "видит", что будущее не сулит ничего хорошего; поэтому он и не делает ничего для будущего: в его системе координат это бессмысленно. Оптимист ‑ это тот, кто не предсказывает будущее, а работает на него. Ибо будущее непредсказуемо. Оно зависит, в частности, от того, что мы о нем думаем и что мы для него делаем. Оптимизм -- гораздо менее самоуверенная позиция, чем пессимизм. Оптимист вовсе не утверждает, что Добро непременно победит; он только допускает эту возможность и отказывается верить в неизбежное торжество Зла.

В этом смысле я оптимист. Мой оптимизм покоится на двух устоях. Во‑первых, это концепция истории, которую можно назвать в противоположность историческому материализму марксистов -- "историческим идеализмом". Эта концепция имеет так же мало общего с философским идеализмом, как исторический материализм Маркса -- с философским матери­ализмом. Она утверждает только, что во взаимосвязи между общественным бытием и общественным сознанием именно со­знание является ведущим фактором, определяющим динамику развития общества в целом. А если так, то мы, люди, свободны в выборе своей судьбы, И если мы не хотим тоталитаризма, то у нас еще есть возможность его отвергнуть.

Стабилизация тоталитаризма, описанная в первой части, -- это то, что происходит сейчас на поверхности явлений, то, что мы наблюдаем непосредственно. Но кроме очевидных событий, всегда происходят и события под поверхностью. Борьба за гражданские права не остается без последствий. И не проходят бесследно бесчисленные частные разговоры, обсуждения и по­ступки, которые становятся известными лишь узкому кругу, но происходят по всей стране. Общественная жизнь всегда со­стоит из надземных и подземных течений, и в определенный момент времени подземные течения выходят на поверхность. Но до поры до времени они остаются скрытыми, не привлекаю­щими внимания. Идет процесс созревания, происходит медлен­ная трансформация взглядов на жизнь вообще и подходов к конкретным проблемам в частности.

В каких направлениях может пойти эта подземная трансфор­мация? Может ли она породить общественное сознание, способ­ное противостоять тоталитаризму и добиться социальных пере­мен?

Мы живем в век науки и промышленности. Научное мировоззрение является в наше время теоретической основой общественного сознания, а промышленное производство определяет условия и формы жизни, в которых общественное сознание развивается. Возникновение тоталитаризма связано, несомненно, с научно‑промышленной эрой, тоталитаризм -- порождение этой эры. Отсюда иногда делают вывод, что общественное сознание, которое своим краеугольным камнем имеет научное мировоззрение и метод, а не христианскую веру, как это было в Европе в дототалитарное время, приводит к тоталитаризму или, во всяком случае, неспособно вывести из него. Следовательно, единственная надежда на перемены -- возврат к хри­стианскому сознанию. Иначе неоткуда взять сил, духовного заряда для борьбы со злом. Такова точка зрения Солженицына, и она имеет некоторое (хотя и не широкое) распространение среди советской интеллигенции. Сборник "Из‑под глыб" напи­сан, в основном, с этих позиций.

Для меня возврат к христианскому сознанию невозможен, как невозможен он, по моему глубокому убеждению, и для подавляющего большинства наших современников. Так что, если бы я разделял точку зрения Солженицына, то я был бы глубоким пессимистом. Но я ее не разделяю. Я утверждаю, что идейные предпосылки тоталитаризма связаны с научными и околонаучными концепциями 19‑го, а не 20‑го века. Я утверждаю, что в наше время последовательное построение мировоз­зрения на базе научной картины мира и критического научного метода порождает жизнеутверждающее и антитоталитарное общественное сознание, способное к энергичному социальному действию. Это и есть второй устой, на котором покоится мой оптимизм.

Общественное сознание, о котором я говорю, я буду связывать со словом социализм.  Это слово употребляется в разных контекстах и в чрезвычайно разных значениях.

\section{Что такое социализм?}

Я начну обсуждение этого вопроса с нескольких цитат из статьи И.~Р.~Шафаревича "Социализм" в сборнике "Из‑под глыб". Шафаревич -- убежденный противник социализма, и я совершенно не согласен с его трактовкой причин и сущности этого явления. Но я нахожу знаменательным, что в то время как многие -- и, вероятно, большинство -- социалистов рассматривают социализм просто как политическое течение, противник социализма призывает видеть в нем всемирно‑историческое явление, далеко выходящее за сферу политики и имеющее основу в глубочайших пластах человеческой природы. И в этом я с ним целиком согласен.
Процитировав отрывок из статьи Ленина 'Три источника и три составные части марксизма", затем -- идеолога "африканского социализма" Дуду Тиама ("африканские общества всегда жили в рамках эмпирического, естественного социализма, который можно назвать инстинктивным") и идеолога "арабского социализма" эль Афгани ("Социализм является частью религии ислама\ldots"), Шафаревич пишет:

"Что же это за странное явление, о котором можно высказать столь различные суждения? Есть ли это совокупность течений, ничем друг с другом не связанных, но по какой‑то непо­нятной причине стремящихся называть себя одним именем? Или же под их внешней пестротой скрывается нечто общее?

По‑видимому, далеко еще не найдены ответы на самые основные и бросающиеся в глаза вопросы о социализме, а некоторые вопросы, как мы увидим позже в этой работе, даже и не поставлены. Подобная способность отталкивать от себя рациональное обсуждение сама является еще одним из загадочных свойств этого загадочного явления".

В качестве примеров древнего социализма Шафаревич называет Ур (22‑21~в. до н.~э.) и империю инков. Учение Платона и некоторые средневековые ереси он также относит к социалистическим учениям. Это наиболее расширительное толкование понятия о социализме, и с ним согласятся немногие. Однако я нахожу такое расширение не лишенным оснований. Сходст­во нашей страны или Китайской Народной Республики с импе­рией инков слишком глубоко и значительно, а ведь не только руководители этих стран, но и просвещенный Запад называет их социалистическими.

Шафаревич далее пишет:

"\ldots если социализм присущ почти всем историческим эпохам и цивилизациям, то его происхождение не может быть объяс­нено никакими причинами, связанными с особенностями кон­кретного периода или культуры: ни противоречием производи­тельных сил и производственных отношений при капитализме, ни свойствами психики африканских или арабских народов. По­пытки такого понимания безнадежно искажают перспективу, втискивая это грандиозное всемирно‑историческое явление в непригодные для него рамки частных экономических, истори­ческих и расовых категорий. Мы попробуем дальше подойти к этому вопросу, исходя из противоположной точки зрения: признав, что социализм является одной из основных и наибо­лее универсальных сил, действующих на протяжении всей исто­рии человечества". [3]

И еще:

"\ldots такая идеология как социализм, способная вдохновлять грандиозные народные движения, создавать своих святых и му­чеников, ‑‑ не может быть основана на фальши, должна быть проникнута глубоким внутренним единством". [4]
Я тоже так думаю. Расхождение состоит в том, что Шафаревич считает основными и первичными чертами социализма "упразднение частной собственности, уничтожение религии, раз­рушение семьи". Он считает, что конечная цель социализма ‑ стирание индивидуальности человека, установление всемирной шигалевщины и тоталитарного государства типа Хаксли‑Орвелла. Отсюда Шафаревич делает вывод, что торжество социализ­ма будет фактически смертью человечества и скорее всего при­ведет к его буквальной, физической гибели. А затем он делает следующий -- уже совершенно фантастический -- вывод, что именно инстинктивное стремление к смерти и является движу­щей силой социализма:

"Как это ни странно кажется сначала, но чем больше знако­мишься с социалистическим мировоззрением, тем яснее стано­вится, что здесь нет ни ошибки, ни аберрации: органическая связь социализма со смертью подсознательно или полусозна­тельно ощущается его последователями, но отнюдь их не отпу­гивает: наоборот, именно она создает притягательность социа­листических движений, является их движущей силой. Такой вывод, конечно, не может быть доказан при помощи логиче­ских дедукций, его можно проверить только сопоставлением с социалистической литературой, с психологией социалисти­ческих движений. Мы же здесь вынуждены ограничиться лишь несколькими разрозненными иллюстрациями". [5]

Иллюстрации Шафаревича не убеждают в справедливости его тезиса. Он обнаруживает у социалистов готовность пожертво­вать собой для достижения провозглашенной высшей цели, а порой и некоторое любование этой готовностью. Черта эта ‑вполне общечеловеческая и свойственна отнюдь не только од­ним социалистам, но почему‑то именно у социалистов Шафаревич расценивает ее как проявление инстинкта смерти и "пафос гибели". Приводя замечание Энгельса о неизбежности охлажде­ния Земли, Шафаревич и здесь усматривает никак не "плоды работы научного ума, вынужденного склониться перед истиной, как бы сурова она ни была", но проявление инстинкта смерти. В тот же котел идет и убеждение Мао, что "гибель половины на­селения Земного шара была бы не слишком дорогой ценой за победу социализма во всем мире". Затем ставка делается на тео­рию Фрейда, что инстинкт смерти является одним из двух основ­ных сил, определяющих психическую жизнь человека. А даль­ше логика такова:

"И социализм, захватывающий и подчиняющий себе миллио­ны людей в движении, идеальной целью которого является смерть человечества, -- конечно, не может быть понят, если не допустить, что те же идеи применимы и в области социальных явлений, то есть что среди основных сил, под действием кото­рых развертывается история, имеется стремление к самоунич­тожению, инстинкт смерти человечества". 

Я все‑таки думаю, что инстинкт смерти (если он и сущест­вует) здесь ни при чем. Разрушительные и тоталитарные аспек­ты социализма, которые мы наблюдаем в истории, представля­ются мне не ядром этого явления, а его оболочкой, шлейфом, который можно и обрубить, если понять его происхождение. Каково же ядро социализма? Я утверждаю, что оно имеет ту же природу, что все великие религии, давшие начало великим цивилизациям прошлого и настоящего. Социализм -- религия будущей глобальной цивилизации, той цивилизации, которая рождается сейчас в муках.

Уподобление социализма религии -- мысль сама по себе не новая. Их сходство слишком бросается в глаза, чтобы не обра­тить на него внимания. Но на протяжении более чем ста лет оно неизменно воспринималось со знаком минус: как социа­листами, так и людьми религиозными. Максимум, на что пошли христиане, это соединить христианскую идейную основу с  уме­ренно‑социалистической политической программой,  что лишь подчеркивает противопоставление социализма религии. Социа­листы приходили в ярость, когда им указывали на сходство социализма с религией. Я же провозглашаю это сходство со знаком плюс.


\section{Религиозное чувство}

Я уже говорил об иерархии целей в поведении высших живот­ных и человека. С точки зрения внешнего наблюдателя, достиже­ние животным различных целей, составляющих эту иерархию, необходимо для продолжения существования и размножения, то есть, в конечном счете, для продолжения существования вида. С субъективной точки зрения, процесс достижения целей связан с эмоциями.  Детали кибернетики эмоций нам не извест­ны. Мы знаем только, что невозможность достижения цели, которая каким‑то образом уже вписана в иерархию, в план, порождает отрицательную эмоцию, а достижение цели -- поло­жительную. При достижении различных целей, например, насыщении, удовлетворении полового инстинкта, победе над противником и т.~д. возникают различные эмоции. Тот факт, что эмоции вызываются целями различных уровней, а также разнообразие эмоций и некоторые данные о строении нервной системы заставляют сделать вывод, что эмоции -- весьма древнее и общее явление. Вероятно, не будет ошибкой сказать, что каждой цели соответствует своя эмоция, и разнообразие эмоций отражает разнообразие целей.

Вспомним теперь об отличии в механизме образования иерархии целей у человека и животных. Для животного высший уро­вень иерархии -- <<Высшая Цель>> --представляет собой совокуп­ность фундаментальных инстинктов. Этим инстинктам соот­ветствуют наиболее фундаментальные эмоции. У человека мы находим способность социально обусловленного конструиро­вания наивысшего уровня иерархии целей. Теперь зададим се­бе такой вопрос: если механизм эмоций заложен глубоко в структуре центральной нервной системы, то не следует ли ожи­дать, что в процессе метасистемного перехода, превратившего животное в человека, появление нового целевого аппарата должно было сопровождаться, в плане субъективном, появле­нием нового класса эмоций?
Я отвечаю на этот вопрос утвердительно и даю следующее определение: религиозным чувством я называю эмоцию, соот­ветствующую продвижению к Высшей цели.

В своей отрицательной форме эта эмоция проявляется как безотчетная тоска, когда ясно, что чего‑то очень важного не хва­тает, а чего ‑ непонятно. (Это одно из первых глубоко челове­ческих чувств.) В положительной форме эту эмоцию называют также ощущением присутствия Бога, мистическим чувством, просветлением, благодатью и другими именами. Но и в своей высшей положительной форме религиозное чувство не лишено привкуса тоски -- ибо в отличие от физического голода, духов­ный голод неутолим до конца: можно продвигаться к Высшей цели, но достичь ее невозможно по определению. Существо, достигшее своей Высшей Цели, должно умереть, как лосось пос­ле икрометания.

Попытаемся проанализировать понятия, связанные с рели­гиозным чувством. Первое из этих понятий -- понятие о смысле, то есть о смысле существования. Религиозность прежде всего понимается как наполненность существования смыслом, кото­рый делает осмысленным также и повседневные детали бытия. В свете данного мною определения религиозного чувства и взаимоотношения между высшими и низшими целями в иерар­хии этот аспект религиозности становится само собою понят­ным. Другой аспект религиозного чувства -- то, что его всегда описывают как нечто <<высшее>> --также является очевидным следствием из нашего определения.

Далее, с религиозным чувством тесно связаны понятия о бесконечности, вечности, бессмертии. <<Приобщение к вечности>> --термин, который используют для описания своего опы­та многие, если не все, мистики. Этот аспект связан с осозна­нием человеком своей смертности и со сверхличным  характе­ром Высшей Цели. Уже инстинкт продолжения рода, который мы находим у животных, ставит перед особью "сверхличные" (если можно так говорить о животных) цели. Но инстинкт жи­вотного не является осознанным. У человека выработка Выс­шей Цели происходит в сознании, в котором отражены как все уже поставленные ( в том числе инстинктивно обусловленные) цели, так и ограниченность во времени своего личного сущест­вования. Ясно, что такое сознание может быть удовлетворено только сверхличной Высшей Целью. Сознание своей смертно­сти-- один из ведущих факторов человекообразования. Оно тре­бует какой‑то формы приобщения к вечному, бессмертному.

И единственным ответом на это требование для смертного су­щества является формирование сверхличной Высшей Цели, выраженной в терминах вечного -- или, по крайней мере, столь далеко выходящего за пределы человеческой жизни, что наше воображение не отличает его от вечного.

Еще одним важным свойством религиозного чувства являет­ся то, что, подобно другим собственно человеческим эмоциям (как например, чувству прекрасного или чувству смешного, связь которых с процессом очеловечивания я анализирую в "Феномене науки"), оно обусловлено культурой,  а не только биологией. Биологически человек получает лишь возможность религиозного чувства, способность  к нему. Чтобы реализовать эту способность, религиозное чувство должно быть развито, подобно тому как развивают воображение, понимание красоты, чувство юмора. Не будучи развито, оно может сохраниться в виде смутных, неосознанных импульсов, не играющих большой роли в жизни.

Логическое понятие о Высшей Цели входит в религиозное чувство, но оно отнюдь не порождает его автоматически. Глубокая связь между миром идей и миром эмоций, конечно, существует; доказательством тому служит хотя бы волнение, испытываемое человеком при удачной мысли. Но связь эта не простая. Смешно было бы думать, что религиозное чувство можно навязать простым декретированием "Высшей Цели", выраженной в тех или иных словесных формулировках. Воспитание религиозного чувства -- обычно процесс длительный, требующий не только усвоения каких‑то логических понятий и идей, но и каких‑то образов,  а также личного примера и непосредственного воздействия на эмоциональную сферу. Поэтому вся культура общества в целом участвует в воспитании религиозного чувства, во всяком случае -- в том, что касается боль­шого числа, масс людей; отдельные люди всегда могут иметь непохожую на других внутреннюю жизнь. Совокупность способов воспитания религиозного чувства, включая сюда, конечно, и сами словесные формулировки Высшей Цели, образуют рели­гию  общества.

Реальность и сила религиозного чувства не вызывает сомне­ний: когда человек идет на лишения и трудности и даже жертвует своей жизнью ради каких‑то высших целей, то именно религиозное чувство поддерживает его и заменяет другие эмоции.


\section{К вопросу о терминах}

Когда мы используем старые термины в новом контексте -‑ и следовательно, модифицируем соответствующие понятия, -- мы берем на себя смелость утверждать, что какие‑то аспекты этих понятий "существенны", и мы их сохраняем, а какие‑то "несущественны", и мы можем обращаться с ними по своему усмотрению. Я беру на себя смелость утверждать, что в упот­реблении термина "религия" существенным является установ­ление сверхличной Высшей Цели (или, что то же самое, сверх­личных Высших Ценностей) и наличие системы воспитания соответствующих этой цели эмоций. Наличие этих признаков необходимо и достаточно, чтобы говорить о религии и рели­гиозном чувстве. Остальные признаки известных нам религий, в том числе и столь общие как вера в "потустороннее", выра­женная в той или иной форме, должны считаться, я полагаю, несущественными, вторичными. Ибо формирование Высшей Цели   это как раз и есть тот основной элемент религии, кото­рый придает ей историческую роль, приводя в движение массы людей. Поэтому я буду называть религией, без всяких кавычек, любую систему формирования Высшей Цели, независимо от то­го, в какой связи она находится с традиционными религиоз­ными учениями. Что касается этимологии слова "религия", то она тоже не мешает, а напротив, согласуется с расширитель­ным толкованием. Это слово производят от латинского корня, означающего "связывать"; речь идет о связывании низшего, земного, человеческого с высшим, вечным, божественным. Не­сколько осовременивая эту трактовку, мы можем переформу­лировать ее как связывание личного со сверхличной Высшей Целью.

Но почему все традиционные религии мира так упорно держатся за элемент потустороннего или, выражаясь более философским языком, трансцендентного? Потому что трансцендентные понятия дают им обоснование Высшей Цели, узаконивают ее как единственно "истинную". Они подводят человека к принятию Высшей Цели.

Одна из тенденций человеческой культуры, хорошо изучен­ная с позитивистских позиций, это тенденция к объективированию  явлений человеческой сферы, переосмыслению их как яв­лений вселенского масштаба, внешних по отношению к чело­веку. Я не знаю, кто впервые в ясной форме указал на эту тен­денцию, но у Фейербаха, в его "Сущности христианства", эта идея уже сформулирована и является основой всего сочинения. Объективирование появляется в истории культуры в разных обличиях. Объективирование связи между именем и значением порождает магию первобытных народов. Объективирование абстрактных понятий порождает теорию идей Платона, а затем и другие формы философского идеализма. Трансцендентный элемент религии является объективированием трансцендент­ности Высшей Цели. Действительно, Высшая Цель не может быть выведена ни из какой другой цели, ибо она -- высшая. Она не может быть найдена или открыта, потому что она -- цель. Она может быть только установлена волевым актом, "создана из ничего", из "запредельного". В объективирован­ном виде существование Высшей Цели преломляется как су­ществование "другого" мира, выходящего за пределы нашего ощущения и познания.

Понятие о Боге маскирует происхождение Высшей Цели. Центр тяжести перемещается на вопрос о существовании  Бога, а Высшая Цель объявляется одним из его аспектов, выводится из существования Бога. Вопрос воли  превращается таким обра­зом в вопрос знания  или веры  (принципиального различия между этими понятиями нет). Волевой в своей основе акт предстает как акт откровения, раскрытия сущего. Когда этот акт свершается и религиозное чувство становится для человека неоспоримым и самоочевидным фактом его внутренней жизни, оно принимается как неоспоримое свидетельство существова­ния Бога. Круг замыкается. 


\section{Человек Маркса и человек Достоевского}

Маркс в своей исторической теории рассматривал человека как существо экономическое.  Однако успех его теории, несмот­ря на ее полную беспомощность в предвидении событий, дока­зывает как раз обратную истину: человек не есть экономиче­ское существо. Марксизм приобрел огромное влияние потому и только потому, что оказался успешной формой выявления и сгущения религиозного чувства под видом научно‑политической теории. Ибо человек ‑- существо религиозное.

Не только марксизм, но и позитивизм и, наверное, вся куль­тура 19‑го века недооценивали важности религиозного элемен­та. "Любовь и голод правят миром" ‑ это был один из самых популярных афоризмов. Но любовь и голод мы находим уже в мире животных. Неужели различие между человеком и живот­ными столь незначительно, столь поверхностно, что те самые факторы, которые в мире животных являются важнейшими и определяющими его развитие, определяют развитие и в мире человека?

В рамках моего определения религии и религиозного чувст­ва утверждение, что человек ‑ существо религиозное, становит­ся почти тавтологией. Действительно, в иерархии целей и планов поведения религиозные планы образуют высший уровень. Зна­чит, они определяют наиболее общие и долговременные аспек­ты поведения. Но этот формальный ответ не слишком убеди­телен, ибо остается открытым вопрос: до какой степени, сколь интенсивно религиозный уровень влияет на жизнь каждого человека? И как он влияет на историческое развитие общест­ва в целом?

Выраженность религиозного элемента, интенсивность рели­гиозного чувства и роль его в судьбе человека -- вещь чрезвы­чайно индивидуальная и варьируется в широких пределах. Яр­кие образы людей с сильно -- очень сильно -- развитым религи­озным чувством рисует нам Достоевский. Человек Достоев­ского -- это человек религиозный по преимуществу.  Религиоз­ная субстанция ‑ проблемы высшей цели и смысла жизни ‑пронизывает его насквозь, определяет его мысли и поступки в каждой детали, в каждой мелочи. Человек Достоевского -- это человек, у которого религиозный этаж чудовищно разрос­ся и подавил собою все остальные этажи иерархии. У человека Достоевского нет тех "простых и естественных" человеческих чувств, которые привязывают нас к материнскому чреву био­логической, природной гармонией; он видит и чувствует через призму высших ценностей. И его инстинкты, сколь ни сильны они, преломляются в той же призме. В сущности, каждое его чувство -- религиозное, с присущей ему противоречивостью, ненасытностью, трагизмом. И человек Достоевского остается таковым в любви и ненависти, в добре и зле, в прекрасном и отвратительном. Великая заслуга Достоевского перед мировой культурой состоит в том, что он создал новый тип -- если угодно, новый биологический вид -- человека: человек религиозный, в его чистом, предельном случае. Этот новый образ, созданный Достоевским, оказал глубокое влияние на культуру 20‑го века.

Экономический человек Маркса -- это старый добрый чело­век 19‑го века. Не мудрствуя лукаво, он идет на рынок, чтобы обменять один сюртук на двадцать аршин холста. На вложен­ный им капитал он стремится во что бы то ни стало получить максимальную прибыль и т.~д. Он действует исключительно из соображений личной выгоды, из расчета, из стремления к удов­летворению своих потребностей.  Это слово является ключевым для понимания его сущности. Человек экономический может быть определен как существо, стремящееся к максимальному удовлетворению своих потребностей. (Не могу удержаться, что­бы не процитировать одну фразу из сочинения Энгельса, кото­рое я уже несколько раз использовал в первой части, а именно, из "Людвига Фейербаха": "\ldotsОн (человек) должен иметь обще­ние с внешним миром, средства для удовлетворения своих по­требностей: пищу, индивида другого пола, книги, развлече­ния\ldots"7 Трогательная простота, с которой Энгельс причисляет "индивида другого пола" к средствам удовлетворения потреб­ностей, была бы, наверное, невозможна в наше время.)

Экономический человек ‑ вполне законная абстракция. Его поведение относительно просто и более или менее пред­сказуемо; он может служить как модельный заменитель чело­века при решении отдельных частных задач, например, связанных с функционированием рынка, обращением капитала и т.~п. Но строить на этой основе всеобщую концепцию человеческой истории, да еще с претензиями на конкретные предсказания (о, безмерная самонадеянность 19‑го века!) -- это то же самое, как объяснять работу электромотора по аналогии с паровой машиной, не имея представления об электрическом токе и маг­нитном поле.

Человек Достоевского и человек Маркса -- два полюса чело­веческой природы. Человек Достоевского, с предельным доми­нированием религиозно‑этического уровня, с его каждосекундным мучительным присутствием -- это, конечно, не норма. Какую же роль играет этот уровень в жизни обыкновенных, нормальных людей? Каков механизм его влияния?



\section{Исторический идеализм}

\epigraph{\ldotsКаждая эпоха -- сознательно или подсознательно -- живет тем, что родилось в головах мыслителей, влияние которых она на себе испытывает.
Платон неправ, когда утверждает, что мыслители должны быть кормчими государства. Характер их господства над обществом иной -- более высокий, чем простое издание законов и распоряжений и осуществление официальной власти. Они -- офицеры генерального штаба, которые в уединении глубоко и всесторонне обдумывают предстоящие сражения. Те же, кто играет роль в общественной жизни, являются нижестоящими офицерами‑практикантами, воплощающими содержание директив генерального штаба в конкретные приказы частям и подразделениям: в такое‑то и такое время выступить, туда‑то и туда следовать, такой‑то и такой пункт занять. Кант и Гегель властвовали над умами миллионов людей, которые за всю свою жизнь не прочли ни одной строчки их сочинений и даже не подозревали, что повинуются им.}{Альберт Швейцер8}

В первой части книги я уже говорил о взаимоотношении между высшими и низшими уровнями в иерархии планов поведения. Рассмотрим теперь этот вопрос более детально.

Наблюдая за сложными кибернетическими системами, устроенными на основе многоуровневой иерархии по управлению, мы обнаруживаем вложенные друг в друга циклы,  то есть последовательности примерно одинаковых ситуаций, действий и т.~п. Больший цикл состоит из некоторого числа повторений меньшего цикла, а сам может входить в качестве элемента в еще больший цикл. Если, например, рабочий на заводе делает заклепки на какой‑то детали, то минимальным циклом здесь будет подъем и опускание молотка на заклепку. Некоторое число этих циклов, вместе с какими‑то одноразовыми действиями, образуют большой цикл -- постановку заклепки. Пусть на каждую деталь надо поставить двадцать заклепок -- это будет следующий цикл. Циклы, связанные с производством отдельных деталей и подразумевающие наличие многих рабочих и разнообразного оборудования, соединяются в цикл произ­водства одного изделия. Еще более крупномасштабный цикл -- производство и продажа партии изделий вместе с соответствую­щими элементами закупки сырья и т.~п. Наконец, самые крупные циклы связаны с запуском в производство новых изделий, принятием новой технической или экономической политики, сменой высшего руководства и т.~д.

Подобная структура функционирования свойственна всем организованным системам. Она отнюдь не случайна, а являет­ся следствием того факта, что организованные системы образуются (филогенетически, если не онтогенетически) путем последовательных метасистемных переходов -- интеграции уже существующих элементов с созданием нового уровня управления. Каждому структурному элементу соответствует определенная временная характеристика -- длительность рабочего цик­ла. Вновь созданный уровень управления не заменяет собой уже существующего аппарата управления в каждом из интегрируемых элементов, а лишь координирует их действия и, быть может, модифицирует их. Поэтому характерная единица действия высшего уровня включает в себя некоторую совокуп­ность единиц низшего уровня. Образуется цикл циклов. 

В поведении животных и человека мы можем выделить три четко различающиеся группы циклов:
\begin{enumerate}
 \item Циклы, связанные с передвижением. Для человека типич­ным циклом этой группы является один шаг. Характерное время -- порядка одной секунды.
 \item Циклы, связанные с пищеварением. У человека этот цикл занимает несколько часов, то есть примерно 20 000 двигатель­ных циклов.
 \item Циклы, связанные с половым размножением; один цикл здесь можно отождествить с периодом времени, необходимым для произведения потомства и доведением его до половой зрелости. Для человека это время можно принять равным 15 годам, что снова составляет примерно 20 000 циклов преды­дущего уровня.
\end{enumerate}

Это деление, конечно, не исчерпывает собою иерархической структуры поведения. Внутри каждого из указанных уровней есть своя тонкая структура, свои подуровни, циклы промежуточного масштаба. Но все эти циклы, имеющие общее биоло­гическое происхождение, имеют и одну общую черту: они не выходят за пределы жизни индивидуума. Только сверхличная Высшая Цель, устанавливаемая религией (в том широком смысле, в котором я употребляю здесь это слово), может со­здавать нормы поведения, действующие на протяжении многих поколений. Источником этих норм является культура общества, а не биологическая природа человека. Религиозные планы поведения обеспечивают связь между поколениями; только религиозные планы могут иметь протяженность во времени большую, чем продолжительность индивидуальной жизни. По отношению к биологическим планам поведения они находятся в том же самом положении, в котором вообще находятся верхние этажи иерархии по отношению к нижним. Они не отменяют, не подменяют и не противоречат (извращения не в счет) пла­нам поведения низших уровней. Они управляют ими -- по мере сил! Да, по мере сил, ибо управление невозможно без сбоев и не тождественно полной власти. Управление, повторяю, не есть подмена собой.

Пройдемся по заводу. Мы увидим рабочих, станки, транспор­теры. Но мы не сможем составить себе представления об экономической стороне работы завода: о спросе на его продукцию, о ее сравнительном качестве, себестоимости и т.~д. А ведь именно этот аспект определяет, в основном, судьбу завода как целого. То же относится и к жизни общества. Здесь мы постоянно "среди рабочих и станков". Мы постоянно путешествуем по циклам, вложенным в циклы. И самой близкой к нам, непос­редственно данной нам реальностью является реальность самых внутренних, то есть самых коротких циклов. Мы всегда нахо­димся во власти сил, которые движут нас по этим циклам. Это могущественные силы, и мы видим их действие всюду вокруг нас. Тогда мы говорим: любовь и голод правят миром. Но у движения по кругу есть одна особенность: совершив круг, мы возвращаемся в прежнюю точку. Все наши циклы имеют <<наибольшее общее кратное>> -- круг человеческой жизни. Человек рождается, испытывает голод и насыщение, любовь и нена­висть -- и умирает. Одного этого достаточно, чтобы в масштабе истории сделать решающими не те очевидные и мощные силы, которые двигают нами по кругам, а те, пусть с первого взгляда и не видные, которые действуют постоянно и приводят к несовпадению конечной и начальной точек, превращая круг -- по гегелевскому образу -- в спираль. Впрочем, это явление происходит и на меньших отрезках времени, сравнимых с про­должительностью нашей жизни. Мы постоянно ставим себе ка­кие‑то цели и достигаем их. Круг замыкается. Мы ставим новую цель, но цикл, который она породит, не будет в точности сов­падать с предыдущим циклом. Различие между этими циклами будет определяться внешними для них факторами, лежащими в следующем, более высоком, уровне иерархии.

Взгляд на общество как на многоуровневую кибернетиче­скую систему приводит к концепции, которую я назвал выше, в противоположность марксовой концепции, историческим идеализмом.  Она утверждает, что именно идеи  (в широком смысле слова), которые господствуют в обществе, определяют в конечном счете его историческое развитие, его судьбу. Во взаимосвязи образа жизни (включая систему производства) и образа мышления динамическим, революционным элемен­том является именно образ мышления, а не образ жизни. Для кибернетика это почти очевидное следствие самых общих зако­нов природы. Человек редко задумывается над тем, зачем он живет и как надо жить. И чаще всего, когда задумывается -- запутывается. Планы поведения высшего уровня, которые господствуют в его обществе, он принимает как нечто само собой разумеющееся и не склонен замечать их влияния на свою жизнь. Ему кажется, что "всякие там высшие цели" и рассуж­дения о смысле жизни -- это досужее философствование, не имеющее отношения к реальной жизни, к ее реальным пробле­мам. Но на больших отрезках времени планы поведения выс­шего уровня оказываются решающим фактором: они опреде­ляют трансформацию внутренних циклов с временем. Они не устраняют целей низшего уровня, но определяют методы до­стижения этих целей. Удовлетворить голод можно многими различными способами, и не  чувство голода определяет, ка­кой из этих способов будет избран. Человек Маркса, которо­му предложено выбрать между работой с зарплатой 150 руб. в месяц и работой с зарплатой 151 рубль, выберет вторую, не­зависимо ни от каких "высших соображений". В действительности же, как показывают исследования социологов, фактор престижности при выборе молодым человеком работы играет не меньшую, а как правило, большую роль, чем фактор зарплаты. А понятие о престижности различных видов деятель­ности -- это непосредственное отражение в массовом сознании Высшей Цели, которая осознанно или неосознанно присутствует в культуре общества.

В войнах, революциях, массовых движениях, индивидуаль­ных акциях, требующих напряжения душевных сил, -- во всех этих событиях, в которых как бы происходит сгущение истории, решающее значение приобретает религиозный элемент. Не расчет, а страсти движут в этих ситуациях людьми. Человек Маркса прячется под кровать, человек Достоевского выпрямляется во весь рост со своими солнечными и теневыми сторонами.

Маркс и Энгельс при построении своей теории опирались исключительно на историю Европы. Они видели параллелизм в развитии экономики и идеологии и объявили экономику, материальное производство, первичным, лидирующим фактором, а духовную культуру, идеологию -- вторичным. Этот вывод был произвольным, так как на основании единичного явле­ния нельзя делать заключений о причинно‑следственных связях. Культурный фон Западной Европы оставался для Маркса постоянным фактором, роль которого невозможно было выяснить, так как ему не было альтернативы. Со времен Маркса мы получили много свидетельств решающего влияния культурных факторов на экономические. В конце 19‑го века началось быстрое проникновение европейской технологии и идеологии во многие страны: независимые и колониальные. Сравним экономическое развитие Японии и Турции. Ни та, ни другая страна никогда не была колонией. Обе, казалось бы, имели оди­наковые возможности использования современных методов производства. Почему же мы видим такое резкое различие между этими странами по уровню и по темпу развития? Ясно, что именно в сфере культуры надо искать те отличия, которые позволили Японии встать на ее специфический "японский" путь развития и достичь поразительных экономических резуль­татов. Другое сопоставление не менее поучительно. В работах Макса Вебера показана глубокая связь между протестантиз­мом -- культурным фактором, который на первый взгляд ничего общего не имеет с экономикой, -- и развитием капита­лизма в Европе. Европейская колонизация Америки дает блес­тящую иллюстрацию этой связи. Здесь мы имеем почти чистый эксперимент.  На новый континент прибывают переселенцы из протестантских стран и из католических стран. Жизнь на­чинается заново, все находятся в более или менее равных усло­виях. Протестантская Америка -- США и Канада -- становятся ведущими промышленными державами мира. Католическая Латинская Америка до сих пор числится в слаборазвитых.


\section{Определение социализма}

Социализм -- это религия, провозглашающая Высшей Целью интеграцию человечества.

Давая это определение, я, как и в случае определения религии, беру на себя смелость утверждать, что в многочисленных учениях и течениях, называемых социалистическими, именно этот признак является "существенным". Я утверждаю, что именно он дает социалистическим течениям их силу, опираю­щуюся на религиозное чувство. В качестве альтернативы можно было бы предложить при определении социализма взять за основу стремление ко всеобщему равенству и справедли­вости. Однако эти черты социализма являются, как мне кажется, производными от идеи интеграции. Справедливость в социалистических учениях обычно понимается как равенство, а равенство служит средством для установления всеобщего братства. Человечество должно сплотиться в единую дружную семью -- такова мечта социалистов. Сама по себе идея равенст­ва вряд ли может вызвать религиозное чувство. Уничтожение собственности и семьи, которое проповедуется некоторыми социалистическими течениями и которое Шафаревич прини­мает за одну из первичных черт социализма, является произ­водной второго порядка, выведясь из идеи равенства.

Конкретное содержание понятия интеграции может быть различным в различных социалистических учениях, но, в отли­чие от традиционных религий, всем им свойственно "земное", посюсторонее истолкование этого понятия. Стремление приоб­щиться к чему‑то бесконечному, вечному -- органическая чер­та религиозного чувства. Социализм отвергает или по край­ней мере не удовлетворяется трансцендентной трактовкой ин­теграции. Понятие о Высшей Цели неотделимо для него от че­ловеческого общества и бессмысленно без него. Отсюда и слово "социализм". Как бы ни конкретизировалось понятие блага, для всех социалистов общественное благо -- высшая цель, ко­торую может иметь и должна иметь личность.

Вряд ли можно усомниться, что интеграция индивидуумов задевает какие‑то глубоко лежащие струны в душе человека, вызывает сильнейшие эмоции. Это хорошо известный в психо­логии факт. Эффект интеграции мы встречаем как со знаком плюс, так и со знаком минус. Со знаком минус он описывает­ся такими словами, как "эффект толпы", "массовый психоз". Со знаком минус мы оцениваем такие аспекты интеграции, как снятие личной ответственности, экстаз подчинения, отказ от бремени свободы. Упоение властью, с другой стороны, также имеет, по‑видимому, в своей основе явление интеграции. Опи­сывая эффект интеграции со знаком плюс, говорят о "чувст­ве локтя товарища", о "радости быть членом коллектива" и т.~д. Индивидуализированные личностные связи -- отношения дружбы и любви, основанные на тесном духовном общении, также относятся к эффектам интеграции, которым всегда при­писывается знак плюс.

Из этого перечня разнообразных аспектов интеграции можно сделать вывод, что и социалистические тенденции могут при­водить как к положительным, так и к отрицательным резуль­татам. Это мы и наблюдаем в действительности. Лишь привер­женность к идее интеграции безусловно роднит все социали­стические течения, являясь источником их энергии и проявляя наибольшую устойчивость по сравнению со всеми остальными чертами теории и практики. Особенно наглядно это видно на примере истории советского государства. Идеи свободы, ра­венства, демократии, которые содержались в исходном учении наряду с идеей социальной интеграции (стоявшей во главе угла: "Пролетарии всех стран, соединяйтесь!"), растерялись по доро­ге. Осталась одна интеграция. При данном мною пока опреде­лении социализма современная коммунистическая государст­венность должна быть причислена к одной из его форм ‑ тота­литарной форме. Тоталитаризм -- это варварская, уродливая форма социализма, когда интеграция индивидуумов достига­ется ценой такого насилия над ними, в результате которого они теряют свою человеческую сущность: сначала возможность, а затем и способность к творчеству, к полноценной духовной жизни.

Россия, Китай, Куба\ldots Мы видим, как социалистические ре­волюции неизменно приводили к тоталитарному строю. С дру­гой стороны, когда социалистические партии приходят к власти в североевропейских странах мирным парламентским путем, они не спешат называть свои страны социалистическими. Так что социализм, называющий себя социализмом, существует пока лишь в форме тоталитаризма. Это заставляет нас задать вопрос:

Возможен ли нетоталитарный социализм?

Современные тоталитарные системы порождены марксистской, а не какой‑либо другой формой социализма. Поэтому в поисках пути к нетоталитарному социализму мы снова возвращаемся к анализу марксизма, чтобы исследовать как причины его успеха, так и корни тоталитарно‑социалистического варварства.


\section{Научный социализм}

Прошло более ста лет с момента возникновения марксизма, а он и по сей день остается самой активной и, наверное, самой распространенной формой социализма ‑ даже исключая тота­литарно‑социалистические страны, где он навязывается насиль­но. Какие же черты марксизма обеспечили ему это положение? Общепризнанно, что ни одно предсказание Маркса не сбылось. Почему же так много людей упорно называют себя маркси­стами?

Ответ на второй вопрос заключается, как мне кажется, в том, что сами марксисты (хотя они в этом и не склонны признаваться) относятся к марксизму именно как к религиозному учению,  а не научной теории. Если все (пусть -- многие) предсказания теории оказываются ложными, то от теории не остается ничего. И в этой ситуации люди обычно меньше всего склонны держаться за имя создателя теории. Скорее, они стремятся откреститься от него, стремятся подчеркнуть, что теперь они создали или создают новую  теорию, которая, в отличие от старой. дает правильные предсказания. Напротив, когда речь идет о целевой установке, лежащей в основе религии, люди склонны связывать ее с именами первых, самых древних основателей. Древность религии идет ей на пользу. Она доказывает, что ее целевая установка свойственна природе человека, играет роль в истории человечества. Время освещает Высшую Цель, делает ее традиционной. Религиозная преемственность поколений несет в себе идею интеграции: интеграции во времени.

Причина огромного влияния Маркса на историю, очевидно, в том, что он осуществил чрезвычайно удачный для своего вре­мени синтез религиозного элемента с политическим и научным элементами.

Синтез социалистической религии и политического революционного движения осуществляется посредством понятия о пролетариате как о классе, исторической миссией которого является установление социализма путем революционного ниспровержения существующего порядка. Ленин пишет: "Направ­ление социализма к слиянию с рабочим движением есть главная заслуга К. Маркса и Фр. Энгельса: они создали такую револю­ционную теорию, которая объяснила необходимость этого слия­ния и поставила задачей социалистов организацию классовой борьбы пролетариата". Создатели, руководители и наиболее активные участники марксового социализма не были рабочими; не за свое собственное, личное, освобождение от уз капитала они боролись, не за кусок хлеба для своей семьи. Ими двигали силы религиозные, стремление к социалистическому иде­алу. И самым верным -- нет, единственно возможным -- способом достижения этого идеала они считали опору на гнев обездоленного человека, на его революционный потенциал. Задача социалистов‑интеллигентов, учил Ленин, есть привнесение революционной социалистической идеологии ("социалистической сознательности") в рабочие массы.

Синтез религии и науки был осуществлен Марксом посред­ством создания им понятия о научном социализме.  Осознание огромной роли науки и технического прогресса и влияния их на общество не было, конечно, чем‑то новым в эпоху Маркса. Идея прогресса -- любимое детище Эпохи Просвещения второй половины 18 века. Гердер, Кондорсе и другие упиваются идеей прогресса, она уже принимает у них религиозный характер. На рубеже 18‑го и 19‑го столетий происходит соединение философского рационализма с традиционными атрибутами религии, с религиозной эмоцией, начинается конструирование рационалистических религий. Якобинцы проповедуют культ Разума. Робеспьер выступает на празднестве в честь "Верховного существа". Огюст Конт (1798‑1857) проповедует позитивистскую религию человечества. Во французском социализме (домарксовом) развитие науки и промышленности -- один из столпов учения. У Сен‑Симона мы находим формулу: "основой свободы является промышленность". Более того, Сен‑Симон провозглашает, что к истории общества надо подходить с научных позиций и что такой подход приводит к выводу о неизбежности установления нового общественного порядка. Таким образом, не Маркса, а Сен‑Симона следовало бы считать основателем научного социализма. Однако случилось так, что именно концепция Маркса покорила многих современников как наиболее тесно связанная с научным мышлением, и термины "марксизм" и "научный социализм" стали отождествляться. Этому, конечно, способствовало и то, что Маркс и Энгельс постоянно подчеркивали научный характер своего вероучения, называли его не иначе, как научной теорией, научным открытием. В работе "Развитие социализма от утопии к науке" Энгельс пишет:

"Этими двумя великими открытиями -- материалистическим пониманием истории и разоблачением тайны капиталистиче­ского производства посредством прибавочной стоимости -- мы обязаны Марксу. Благодаря этим открытиям социализм стал наукой,  и теперь дело прежде всего в том, чтобы разработать ее дальше во всех ее частностях и взаимосвязях". Наконец, нельзя не признать, что Маркс и Энгельс сами были учеными, специалистами в области науки об обществе, и были неплохо знакомы с естественными науками. Сильные и слабые стороны их учения отражают дух науки того времени.

Что такое "научный социализм"? Если социализм -- это ре­лигия, то не является ли это сочетание слов противоречивым, бессмысленным?

Высшая Цель не может быть выведена ниоткуда. Даже из науки. Сколько бы мы ни занимались наукой, сколько бы мы ни изучали устройство мира, мы никогда не узнаем, к чему мы должны  стремиться. Но наши познания о мире, понятия, которые мы используем, язык, на котором мы говорим, -- все это имеет, тем не менее, самое непосредственное отношение к формированию и выражению Высшей Цели. Научный метод, научная картина мира не определяют однозначно религиозных ценностей, то есть религиозной Высшей Цели. Но они создают ограничения, делают одни варианты религии более приемлемыми, а другие -- менее приемлемыми; они дают способы выражения религиозных ценностей. Волевая компонента -- декларирование Высшей Цели -- является ядром религии. Но это ядро не висит в безвоздушном пространстве, оно неразрывно связано с культурой общества в целом. Если помнить об этом, то понятие о научном социализме представляется совершенно естественным. В эпоху науки социализм, чтобы иметь шансы на успех, должен быть научным.

Взглянув в свете этих соображений на научный социализм Маркса, мы сможем разделить его мнимое и подлинное значение и в противоречии между ними усмотреть одну из причин его деградации. Марксизм с самого начала выдавал себя за научную теорию. Он, действительно, содержал в себе элемент научной теории, которая, как и все теории в науке об обществе, была весьма упрощенной, приблизительной (можно даже сказать сомнительной) и очень быстро устарела. Но подлинное значение марксизма состояло в том, что он был религией, опи­рающейся на современную ему науку, религией, выражаемой и проповедуемой в научных терминах‑ Это был, действительно, научный социализм. И именно в этом качестве он произвел не­отразимое впечатление на современников и продолжает производить впечатление до сих пор. Ибо в 19‑ом веке возникла потребность заполнить вакуум, образовавшийся вследствие распада христианства. Эта потребность стала в 20‑ом веке еще более настоятельной. Порок марксизма не в том, что он есть форма религии (напротив, это его достоинство), а в том, что он, во‑первых, не хочет это открыто признать, а во‑вторых, базируется на безвозвратно пройденном этапе научной мысли.

\section{Век нынешний и век минувший}

Как я отмечал в первой части, логика становления марксизма -- это логика политического прагматизма, ставящего своей целью во что бы то ни стало и как можно скорее осуществить социальную революцию. Но содержание и успех марксизма -- нечто отличное от логики его становления. Политический прагматизм бывает свойственен лидерам движений, рядовые же участники вовлекаются в них не в результате хитрого расчета, а под действием эмоциональных и интеллектуальных факторов, имеющих более глубокую природу и часто не осознаваемых. Марксизм -- социализм 19‑го века. Марксистский нигилизм, идейную основу тоталитаризма, нельзя объяснить одним лишь политическим прагматизмом, он имеет корни в научном мировоззрении той эпохи.
Картина мира, которую рисовала Марксу и Энгельсу совре­менная им наука, была в общих чертах такова. В более или менее пустом трехмерном пространстве (то есть совершенно пустом, по одним теориям, и заполненном почти нематериальным эфиром, по другим теориям) движутся весьма малые частицы вещества. Эти частицы действуют друг на друга с силами, кото­рые однозначно определяются их взаимным расположением и скоростью. Силы взаимодействия, в свою очередь, однознач­но (по законам Ньютона) определяют дальнейшее движение частиц. Таким образом, каждое следующее состояние мира однозначно определено (детерминировано) его предыдущим состоянием. Детерминизм был характерной чертой научного мировоззрения того времени.

В этом мире нет места для свободы, она иллюзорна. Человек, который сам есть лишь совокупность частиц, не принадлежит самому себе, а движется, подобно заведенному часовому меха­низму. Вообще, вся духовная жизнь человека и человеческая личность -- иллюзорные понятия.

Конечно, философы всегда понимали относительность вся­кой картины мира. Тот факт, что все наши знания о мире -- продукт чувственного восприятия, был осью, вокруг которой вращалась вся европейская философия. Великий критик Иммануил Кант, предвосхищая науку 20‑го века, даже пространство, время и причинность -- святая святых механистического мировоззрения -- отказался признать "вещами в себе", а объявил формами нашего восприятия действительности. Но как раз философы‑материалисты в то время были наименее чувствительны к критической философии. Вместе с неискушенным "простым человеком" они стояли на позициях наивного реализма, который критическую философию просто отбрасывал, не зная, что с ней делать. По‑видимому, большинство естествоиспытателей также стояло на наивно‑реалистических позициях. И это было естественно. Наука в то время еще не нуждалась в критической философии. В некотором смысле для ученого было даже "правильнее" быть наивным реалистом, чем критиком: это активизировало работу в рамках механистической концепции, пока она себя не изжила.

Социализм 19‑го века возник на почве материалистической наивно‑реалистической философии. Маркс и Энгельс воспитывались как гегельянцы, то есть в духе предельно некритическом, антикритическом. Перейдя в материализм, они лишь сделали, как говорят в математике, некоторые переобозначения. Ленин был в такой степени политическим прагматиком, что не ясно, можно ли говорить о наличии у него взглядов как чего‑то .отличного от целей. А его целям, разумеется, соответствовал наивный реализм "простого человека"; со страстью и руганью защищал он его от подъема научно‑критической фило­софии в начале 20‑го века.

Итак, основатели социализма и первые поколения их после­дователей действовали на фоне механистической картины мира 19‑го века. Строения, которые они воздвигли, хорошо вписы­ваются в этот пейзаж. Неудивительно, что в них неуютно жить человеку. Основу общества, как и основу мира, составляют бездушные, абстрактные сущности: производительные силы, производственные отношения, классы. И они неумолимо транс­формируются под действием столь же бездушных и внеличных сил.

Даже самый рьяный сторонник научного мировоззрения (а я себя отношу к таковым) не может не признать, что в социа­лизме 19‑го века ровно столько человечности, сколько оста­лось в нем от христианского, то есть донаучного, мировоззре­ния. Для Маркса и Энгельса понятия о личности и свободе были чем‑то само собой разумеющимися. Они употребляли их довольно часто и не видели большой беды в том, что эти понятия не входят в основания  их системы. Вероятно, они этого просто не замечали. Но это так, и это очень важно. В луч­шем случае понятия личности и свободы в качестве наследства предыдущей эпохи могли быть пристегнуты к системе, умерив "социалистичность" социализма. Так и произошло в тех стра­нах, где это наследство прочно укоренилось в общественной традиции. А там, где такой традиции не было, марксистская си­стема взглядов действовала в чистом виде и привела к своему естественному результату: тоталитарному обществу.

В 20‑ом веке мировоззрение претерпело радикальное изме­нение: оно включило в себя критическую философию, сдела­ло ее своей непременной и неотъемлемой частью. Произошло это под давлением необходимости и не без сопротивления со стороны ученых, привыкших к наивно‑реалистическому подходу. Оказалось, что физические явления на атомном уровне не могут быть поняты в рамках наивного реализма, а требуют подхода с позиций критической философии. Таким образом, наряду с конкретными открытиями 20‑го века в области физики микромира, мы должны рассматривать неизбежность критиче­ской философии  так же, как научное открытие, --  величайшее научное открытие 20‑го века.

Принятие позиций критической философии означает, что смысл и задачу научных теорий видят не в том, чтобы раскрыть сущность вещей "как они есть, независимо от наблюдения", а лишь в том, чтобы упорядочить наши ощущения и частично предсказать их. Мы стали гораздо скромнее, бесконечно скром­нее, чем были ученые в 19‑ом веке. А мир стал гораздо таинст­веннее. Оказалось, что объект и субъект познания неразделимы, и из‑за этого при попытке зайти дальше некоторой черты в ис­следовании "вещей в себе" мы натыкаемся на различные невоз­можности и парадоксы.

Мы знаем теперь: представление, что мир "на самом деле" есть пространство, в котором по определенным траекториям движутся маленькие частицы, -- иллюзорно; оно приходит в противоречие с экспериментальными фактами. Картина мира, которую дает современная наука, не выходит за рамки наших ощущений. Метафизическое "на самом деле" отвергнуто окон­чательно и бесповоротно. Но этого мало. Вместе с "на самом деле" пал и его постоянный спутник ‑ детерминизм, убежде­ние, что "на самом деле" все явления имеют свои причины и однозначно определяются ими. И это не просто абстрактный результат принятия критической философии, а глубокая черта современной физической теории, многократно подтвержденная экспериментом. Мы знаем теперь, что есть явления, у которых нет и не может быть причины.

Критическая философия и крушение детерминизма лишают нас права объявить элементы нашего духовного опыта, и в частности свободу воли, "иллюзией", которой "на самом деле" не существует. Ибо это такой же факт бытия, как и данные пяти органов чувств. Крушение детерминизма не только оставляет в современной научной картине мира место для свободы личности, но, я сказал бы (хотя здесь, конечно, кончается наука и начинается философия), оставляет пустое  место, которое нельзя не заполнить.  В новом, таинственном мире науки 20‑го века понятия о личности и свободе органически входят в науч­ное мышление. Механистическая модель мира как часового механизма больше не застилает нам глаза.

Энгельс называет три научных открытия 19‑го века, которые имели фундаментальное значение для марксистского мировоззрения. Это закон сохранения энергии, клеточное строение живого вещества и эволюционная теория Дарвина. Я хочу назвать три открытия 20‑го века, имеющие фундаментальное значение для научного мировоззрения нашего времени. Это, впрочем, не открытия, а целые группы открытий.
\begin{enumerate}
 \item Новая физика: теория относительности и квантовая механика. Именно эти открытия заставили пересмотреть наши представления о пространстве, времени и причинности.
 \item Новая математика: математика и теория алгоритмов. Наиболее важные открытия здесь -- знаменитая теорема Геделя и существование алгоритмически неразрешимых проблем. Они нанесли еще одни удар по самоуверенности ученых, уже потрясенной открытиями новой физики. Новая математика изучает так называемые знаковые системы,  которые имеют двоякое значение. Во‑первых, их можно представить себе как идеализированные материальные устройства, например -- механи­ческие. Эти устройства сравнительно просты и могут находить­ся лишь в совершенно четко отличающихся друг от друга состояниях. Идеализация же состоит в том, что они могут работать сколь угодно долго без сноса и что в их распоряжении может быть сколь угодно много памяти,  например, ленты с квадрати­ками, в которых можно ставить и стирать точку. Во‑вторых, значение знаковых систем в том, что все наши теории (в том числе и теории о знаковых системах) являются знаковыми системами. И вот оказалось, что даже в этом идеализированном мире знаковых систем, где нет ошибок измерения, случайных сбоев и т.~п., -- даже в этом чистом и благородном мире сущест­вует уйма невозможностей, которые ограничивают нашу спо­собность предсказания. Уже такая несложная система как ариф­метика обладает тем свойством, что не существует конечного набора аксиом, из которого можно вывести все истинные арифметические утверждения. Другая невозможность: берем знако­вую систему, приводим ее в некоторое начальное состояние и запускаем в движение; не существует такого алгоритма (об­щего метода), который по начальному состоянию всегда опре­делил бы, остановится ли система в конце концов или нет.
 \item Новая научная дисциплина -- кибернетика, со своими поня­тиями и подходами, применимыми к объектам любой природы, в том числе к живым организмам и человеческому обществу. О ней у нас речь пойдет особо.
\end{enumerate}

К какой же концепции общества ведет современное научное мировоззрение? Каким может быть социализм 20‑го и 21‑го века? В чем он может совпадать и в чем отличаться от социализ­ма 19‑го века? И нужен ли вообще социализм? Какие есть аль­тернативы?

Обсуждение этих вопросов я построю следующим образом. Сначала мы рассмотрим, опираясь на исторический опыт социалистического движения, два общих вопроса, важных для религиозно‑этических учений: о роли трансцендентного и о взаимоотношении между знанием и волей. Затем, отталкиваясь опять‑таки от марксизма, я перейду к понятию эволюции как связующего звена между описанием действительности и Высшей Целью. Затем я сформулирую свое представление о научном социализме 20‑го века и перейду к обсуждению его некоторых конкретных черт в сопоставлении с социализмом 19‑го века.



\section{Трансцендентность и идолопоклонство}

Социализм отличается от традиционных религий отсутствием понятия о трансцендентном. Это религия, опирающаяся лишь на Человека и его Разум, а не на благоволение к нему неких "высших сил", выходящих за пределы нашего ощущения и понимания. Ее истоки -- в Эпохе Просвещения. Ученик Сен‑Симона Огюст Конт предпринял первую попытку системати­ческого построения "религии человечества" и общественного устройства, которые были бы основаны на достижениях науки и "позитивной" философии (от него и пошел термин позитивизм).  Контом руководили наилучшие побуждения -- указать человечеству путь к благоденствию и счастью, но построенная им система вызывает содрогание ужаса: это убогий и мрачный тоталитаризм.

Я думаю, что от начала 19‑го века ничего другого нельзя было и ожидать, и я уже говорил почему. Но есть и другая точка зрения, согласно которой порочен сам замысел религии человечества, а точнее -- религии без трансцендентности. Такова, в частности, точка зрения Альбера Камю. Камю -- глубокий и яркий мыслитель нашего времени. Он был одним из немногих, кто еще в 40‑х годах имел мужество, вопреки интеллектуальной моде, царившей на Западе, вникнуть в сущность тоталитарного социализма и восстать против него. Я приведу длинную выдержку из его "Мятежника":

"Конечные результаты, к которым пришел Конт, курьезным образом напоминают то, что было в конечном счете принято научным социализмом. В позитивизме мы совершенно явственно слышим отзвуки идеологической революции 19‑го столетия, одним из представителей которой является также Маркс и которая состояла в том, что Райский Сад и Откровение, которые традиция помещала ранее в начало истории, теперь стали помещать в конец ее. Позитивистская эра, которая должна была последовать за метафизической и теологической эрами, знаменовала собой, по замыслу Конта, наступление религии человечества. Анри Гуйе очень точно определяет направление мышления Конта, говоря, что он ставил целью открыть человека, лишенного всяких следов Бога. Первоначальная цель Конта, которая была заменить всюду абсолютное на относительное, очень скоро трансформировалась под давлением обстоятельств в обожествление относительного и в исповедание новой религии, одновременно универсальной и лишенной трансцендентности. В якобинском культе разума Конт видел предвосхищение позитивизма и рассматривал себя, имея на то полное право, как подлинного преемника революционеров 1789 года. Он продолжал эту революцию и расширял ее охват, ведя борьбу с трансцендентностью принципов и систематически конструируя религию биологического вида. Именно это означала его формула "устранить Бога во имя религии". Положив начало мании, которая с тех пор стала весьма модной, он хотел быть святым Павлом этой новой религии и заменить Римское католичество Парижским католичеством. Мы знаем, что он хотел видеть во всех соборах "статую обожествленного человечества в прежних алтарях Бога". Он рассчитал, что не позднее 1860 года в Соборе Парижской Богоматери будут проповедовать позитивизм. Этот расчет не так смешон, как кажется. Собор Парижской Богоматери, правда, еще сопротивляется, хотя уже и осажден: но рели­гия человечества весьма эффективно проповедовалась к концу 19‑го века, и Маркс, несмотря на то что он не читал Конта, был одним из ее пророков. Но только Маркс понял, что рели­гию, которая не имеет элементов трансцендентного, следует называть, как и положено, политикой. Если на то пошло, Конт тоже знал это или по крайней мере понимал, что его религия была, по существу, формой социального идолопоклонства, включающей политический реализм, отрицание индивидуальных прав и установление деспотизма. Общество, где ученые будут священниками, две тысячи банкиров и инженеров управ­ляют Европой с населением 120 млн. человек, где частная жизнь абсолютно отождествлена с общественной, где люди абсолютно послушны "в действии, мысли и чувстве" первосвященнику, -- такова утопия Конта, которая возвещает собою то, что можно назвать горизонтальными религиями нашего времени. Убежденный в просвещающей силе науки, Конт забыл о необходи­мости полиции. Другие будут более практичны: религия человечества будет основана на крови и страданиях человечества".11

Безапелляционное догматическое провозглашение существования чего бы то ни было неприемлемо для ученого. Современ­ный позитивизм, в отличие от его ранних форм, готов принять любые языковые конструкции, если они способствуют организации чувственного и духовного опыта. Если, например, четырехмерная модель мира объясняет экспериментальные факты, то мы можем говорить, что четвертое измерение существует без всяких кавычек, как мы говорим о существовании ньютоновской силы тяготения. И если бы в нашем распоряжении были какие‑то факты, которые было бы удобно описывать как продолжение существования души после распада тела, то мы говорили бы о бессмертии души. Но в традиционных религиях трансцендентные понятия вводятся догматически, и существование запредельного понимается без всякого анализа, чисто метафизически: что оно "на самом деле" существует и об этом становится известно благодаря "откровению". Уче­ный не может рассматривать так понятое "существование" иначе как чистый самообман, поэтому все эмоциональные ас­пекты трансцендентного для него теряются: они вянут и гибнут от сурового дыхания неверия. В интеллектуальном плане на этой основе также нельзя построить чего‑либо достойного уважения. Христианская теология поражает меня своей уны­лой бесплодностью. В свое время она, конечно, была важной формой интеллектуальной жизни, но сейчас ее нельзя рассматривать иначе как анахронизм. Я, правда, не могу причислить себя к знатокам теологии. Но я пытался познакомиться с ней, читал, в частности, одно элементарное введение в христианское мировоззрение, рассчитанное на новичков. Когда речь идет о Христе и нравственных принципах его учения, это живые человеческие слова. Когда начинается теология, читать стано­вится невозможно.

Тем не менее, я привел выдержку из Камю не столько для того, чтобы спорить с ним, сколько для того, чтобы согласиться в главном -- в противопоставлении трансцендентных религий различным видам идолопоклонства. Но я хочу осмыслить это противопоставление не в традиционных (для религий) ме­тафизических терминах, а с точки зрения операционалистской: нас будет интересовать вопрос, в каких именно наблюдаемых проявлениях  различаются идолопоклонство и трансцендент­ная религия. Поставив так вопрос, мы приходим к следующему определению: идолопоклонство связывает религиозное чувст­во с конкретными предметами,  трансцендентная религия -- с предельно абстрактными понятиями. 

В плане эволюционных возможностей различие между этими двумя видами религии чрезвычайно велико. Идолопоклонство неминуемо ведет к ограничению творческой свободы, к застою и окостенению общества. Трансцендентная религия по край­ней мере допускает  бесконечное развитие. Ибо содержание аб­страктных понятий меняется по мере эволюции культуры; це­ли, сформулированные в предельно абстрактных понятиях, трансцендентны, недостижимы, как и подобает Высшим Целям.

Идолопоклонство зовет человека к идолу, трансцендентная религия -- к Богу, то есть в смутно угадываемую бесконечность. Именно в абстрактности Высшей Цели, а не в наличии мета­физической трансцендентности состоит наблюдаемое разли­чие. Великие монотеистические религии стали возможны лишь после достижения некоторого уровня абстрактного мышления. Бог этих религий -- абстрактная идея.

В своем конкретном социальном воплощении никакие транс­цендентные религии не свободны полностью от идолопоклонст­ва (чего стоит один принцип непогрешимости Римского Папы). В процессе конкретизации абстрактных идей конкретные цели и предметы становятся их воплощением. Когда они засло­няют собой фундаментальные абстракции, можно говорить об идолопоклонстве. В таких случаях борьба против идоло­поклонства ‑ это борьба за возрождение и обновление подлин­ной трансцендентной религии. Реформация в Европе дает яркий тому пример. Идолопоклонство -- одно из варварств трансцен­дентной религии. Если мы находим его в христианстве, то не удивительно, что мы находим его и в социализме. Культ вождя, культ класса, культ расы -- различные формы социалистиче­ского варварства. Культ нации, который в 20‑ом веке стал всеоб­щим бедствием, -- также варварская форма социализма; совре­менный национализм силен и страшен тем, что имеет религиоз­ную окраску: здесь действует человек Достоевского, а не чело­век Маркса. Наконец, культ человечества в целом -- это тоже идолопоклонство и варварство. Никакие конкретные люди -- ни по отдельности, ни в своей совокупности -- не должны стоять на верхней ступеньке иерархии целей и планов. В лучшем слу­чае культ человечества ‑ это пошлое самолюбование и само­восхваление, нужное лишь тому, кто в сердце своем не ощуща­ет грандиозности "феномена Человека". Стыдно смотреть на тысячи плакатов: "Слава КПСС!", "Да здравствует Коммуни­стическая партия Советского Союза!", "Партия -- ум, честь и совесть нашей эпохи" и т.~п., которые развешиваются по всему Советскому Союзу по инициативе и под наблюдением этой са­мой партии. Было бы очень грустно, если бы эта манера само­восхваления (отражающая, в сущности, комплекс неполноцен­ности) распространилась на все человечество.

Определяя социализм как религию интеграции, я стоял на позиции описательной, а не нормативной. Я пытался лишь выя­вить ту общую основу, которая роднит различные социалисти­ческие течения и придает им жизненность. Но это не значит, что я вообще не намерен становиться на нормативные позиции, то есть высказывать свои суждения о том, каким должен быть социализм. Я намерен. Можно даже сказать, что я намерен проповедовать определенную религию или определенную фор­му религии. И эта религия будет тесно связана с наукой, неот­делима от нее, подобно тому как монотеистические религии неотделимы от языка, содержащего абстрактные понятия. Ирония Камю -- как по поводу моды на "Святых Павлов", так и по поводу <<просвещающей силы науки>> --меня не тро­гает‑ Я не вижу, почему проповедь новой религии должна рас­сматриваться как более претенциозное дело, чем провозглаше­ние новой научной теории. В конце концов, религий существу­ет столько же, сколько людей. Человек, пытаясь ответить на вопрос, зачем он живет, приходит к выводам, которые могут быть по меньшей мере столь же интересны для его собратьев, как выводы о движении планет или строении тела насекомых. Тот факт, что его выводы имеют нормативный характер, не сви­детельствует о нескромности: ведь таков характер предмета, с которым имеет дело религия. А наука и в самом деле обла­дает огромной просвещающей силой, хотя она никогда не пре­тендовала на то, чтобы заменить собою полицию.

В свете сказанного, я делаю нормативный вывод: религия должна основываться на предельно абстрактных формулиров­ках, связанных с Высшей Целью. Мое описательное определе­ние социализма опирается на никак не конкретизированное понятие интеграции человечества и, следовательно, удовлетво­ряет этому требованию. Оно очень широко и включает, в частности, варварские формы социализма. Я буду строить норматив путем сужения этого определения. Но сужение не есть полная конкретизация. Постоянная часть религии должна сохранить предельно абстрактную формулировку, выраженную в понятиях, взятых из науки и трансформирующихся по мере развития науки. Такую религию можно назвать трансцендентной. Она является естественным продолжением метафизически трансцендентных религий в обществе, принявшем критическое научное мировоззрение. Если воспользоваться метафорой Камю, такая религия не "горизонтальна", а "вертикальна", направле­на ввысь: ее идеалы, вызывающие религиозное чувство, лежат не в сфере непосредственно окружающих нас явлений, а в абстракциях высокого уровня. Замена метафизической транс­цендентности на подлежащую критическому анализу трансцендентную абстрактность -- неизбежный результат вторжения в область религии философии современного ученого.

\section{Знание и воля}

Я приведу еще одну выдержку из Камю. Сравнивая реальность сталинской империи с предсказаниями марксизма, он пишет:
"Как мог так называемый научный социализм до такой степени противоречить фактам? Ответ прост: он не был научным. Напротив, его поражение явилось результатом его весьма сомнительного метода, претендовавшего на то, чтобы быть одновременно детерминистическим и пророческим, диалектическим и догматическим\ldots Задачей исторического материализма может быть только критическое исследование современного общества; оставаясь в рамках научного метода, он может приводить лишь к некоторым предположениям об обществе будущего. Не потому ли важнейший труд Маркса называется "Капитал", а не "революция"? Маркс и марксисты ударились в предсказание будущего, в пророчества о неизбежной победе коммунизма в ущерб своим собственным постулатам и научному методу\ldots Марксизм не научен: в лучшем случае он обладает научными предрассудками. В нем ярко проявилось глубокое различие между научной логикой, этим плодотворным инструментом исследования, мысли и даже восстания, и исторической логикой, которую изобрела немецкая идеология путем отрицания всех принципов. Историческая логика -- это не та логика, которая может, в рамках своих собственных функций, выносить сужде­ние о мире. Претендуя на суждение о событиях, она в действи­тельности определяет их, выступая как явление воспитывающее и всепобеждающее\ldots Эта псевдологика в конце концов отождествляет себя с хитростью и стратегией, стремясь к созданию идеологической империи. Какую роль может играть наука в этой концепции? Ничто не лишено в такой степени тенденции к завоеванию, как рассудок. Историю не делают со щепетиль­ностью, требуемой наукой; можно даже сказать, что с того мо­мента, как мы решаем действовать с научной объективностью, мы обречены на безуспешность попыток делать историю. Рас­судок не проповедует, а если он проповедует, то это не рассу­док".12

Соглашаясь с Камю относительно противоречия между пророческой и научной компонентами марксизма, я решительно не согласен с ним по вопросу о роли рассудка и науки в истории. Камю, с моей точки зрения, неправильно проводит разделительную линию между элементами, подлежащими разделению. Немецкая историческая логика предстает у него безусловным злом, которое он противопоставляет научной логике. Действительными же элементами, глубоко отличными по своей природе и функции, которые надо выделить и противопоставить, являются элемент знания и элемент воли. Под научной логикой понимают первый элемент, взятый в его чистом виде. Научное исследование должно быть непредвзятым, бесстрастным: если исследователь ставит своей целью прийти к определенным выводам, то грош цена такому исследованию. Второй элемент -- чистое, ни на чем не основанное и никак не оправдываемое волеизъявление. Этот элемент можно назвать иррациональным, он выпадает из сферы действия рассудка, логического рассуждения.

Несколько слов о терминах "рассудок" и "разум". Франко‑английское слово raison‑reason, которое использует Камю в приведенном отрывке, трудно для перевода на русский язык. Я переводил его здесь то как "логика", то как "рассудок". В других случаях его переводят как "разум". В русском языке слово "разум", в отличие от слова "рассудок" явно включает в себя волевую компоненту. Если бы одну из фраз Камю в при­веденном выше отрывке я перевел так: "Ничто не лишено в такой степени тенденции к завоеванию, как разум", то вряд ли хотя бы один русский согласился с этим утверждением. Заменив "разум" на "рассудок" и понимая под рассудком чистое знание о мире, наличие в мозгу каких‑то моделей действительности без всякой примеси волевой компоненты, цели, я могу согласиться с этой фразой: она становится верной по определению.

Итак, мы будем разграничивать элементы знания и воли. Можем ли мы представить "историческую логику" немецкой идеологии (лучше всего выраженную, очевидно, Гегелем) как пол­ную противоположность научному методу? Это было бы несправедливо, ибо она содержит также и познавательный элемент. Но, конечно, и волевой элемент, элемент установки, оценки, она содержит в избытке. Этим и объясняется ее активная исто­рическая роль. Беда не в том, что немецкая идеология пыталась соединить волевой и познавательный элементы, а в том, что в ней Воля маскировалась под Знание. Из Гегеля маскировка Воли под Знание перекочевала в марксизм, я уже говорил об этом в первой части книги. Печальный результат этого трюка состоит в том, что элемент знания, науки оказывается в безнадежно трудном положении; потеряв свою беспристрастность, наука становится служанкой политики. Воля, замаскированная под Знание, -- это волк в овечьей шкуре, который пробирается в овчарню и уничтожает овец. Пророческо‑политический, то есть волевой, элемент в марксизме сожрал его научный элемент.

Но это вовсе не значит, что элементы воли и знания несовместимы в истории, исключают друг друга. Все мы знаем из своего жизненного опыта, что их удается, худо ли бедно, сов­мещать в личной жизни, так почему бы им не совмещаться и в истории? Не надо только маскировать одно под другое. После потрясения, вызванного в европейской культуре Первой мировой войной, стали модными иррационалистические и антирационалистические течения мысли. Все больше людей стали говорить о "разочаровании в силе разума", о "неспособности разума определять историю" и т.~п. Ужасы гитлеризма и сталинизма явились, как полагают, подтверждением этих взглядов, которым и Камю, в частности, отдает дань. В своих теоретических основах эти течения опираются на принципиальное различие между знанием и волей, на иррациональность волевого акта. Но антирационалисты почему‑то игнорируют тот факт, что тра­гические события 20‑го века -- результат не избытка разума, а его недостатка. Это не было столкновением между Академиями наук разных стран. Демагогия, приведшая к войнам, опиралась на психологию массового человека.  Именно массовая психология, а не борьба воинственной аристократии стала определять внутреннюю и внешнюю политику в странах Европы. А массовую психологию человека 20‑го века можно обвинить в чем угодно, но только не в чрезмерной роли рассудка и логики. Разум был оттеснен на задний план, и пока он не завоюет необходимых позиций (а мне кажется, что на это можно надеяться), рано судить о том, что он может и чего не может в истории.

Выдавая свою религию за научно обоснованную политику, Маркс, в сущности, совершал обман. (Конт был честнее: религию он называл религией.) Обман никогда не проходит безнаказанно: марксизм поплатился за него догматизацией. Хотя и неохотно, хотя и в неявной, непризнаваемой форме, марксизм вынужден был прибегнуть к тому самому способу соединения знания и воли, критика которого была лейтмотивом развития европейской науки. Этот способ -- догматическая вера, он лежит в основе христианской религии. Соединение знания и воли достигается путем слияния этих элементов в понятие о Боге. Конкретные аспекты понятия о Боге отражаются в ряде догматов и священных текстов, которые предлагается принять как истины высшей инстанции, не подлежащие сомнению и обсуждению. Так как догматы содержат понятия о Добре и Зле, они являются, фактически, целевыми установками, направляющими нашу волю. Ибо мы называем добром то, к чему надо стре­миться, а злом -- то, чего надо не допускать.
Для религии научной эпохи догматическая вера неприемлема. Соединение знания и воли необходимо, но они не должны сливаться в сплошную нечленораздельную массу. Они должны быть соединены в систему при четком различении между ними. И здесь, как и всюду, прогресс достигается путем метасистемного перехода! Элементом метасистемы в данном случае является наша способность постоянно разграничивать сферу знания и сферу воли.

Итак, мы можем сформулировать следующий важный вывод о различии между логической основой прежних религий и религии в эпоху критического мышления:

Вместо догматической веры в то, что "так надо, ибо это есть Добро", современная религия предлагает человеку совершить личный волевой акт установления Высшей Цели, сделать свобод­ный выбор и осознать свою свободу в этом выборе.

Движение к синтезу знания и воли имеет, кроме чисто философского плана еще и социальный план. Для действительного внедрения в общественную жизнь реального и в то же время совместного существования этих двух элементов необходимы гарантии для каждой личности свободного обмена информацией и идеями. Без признания обществом этого принципа, без его глубокого проникновения в сознание каждого человека, политическая воля неизбежно будет ослеплять самое себя и заводить в трагические тупики, подобные тому, в котором находится сейчас советское общество. Умение сохранять объективность в исследовании и в теории, независимо от целевых установок и эмоций, должно стать частью культуры, массовой культуры. До известной степени разделение между знанием и волей в обществе может выражаться в разделении функций между учеными и политическими деятелями. Но только до известной степени. Синтез эффективен лишь тогда, когда он со­вершается в мышлении каждого человека.

\section{Великая Эволюция}

В концепции Маркса есть привлекательные, с точки зрения ученого, черты. Главная из них не только не устарела к нашему времени, но, напротив, оказалась в самом центре внимания. Это -- представление об эволюции мироздания, его необрати­мом изменении, подчиняющемся какому‑то общему закону.

В плане методологическом и гносеологическом философия марксизма (диалектический материализм) -- типичный пример того, что Конт называет "метафизической философией", а именно, объективирование абстрактных понятий до некоторых сущностей. Критический позитивистский анализ, который лег в ос­нову современной научной философии, совершенно не коснулся диалектического материализма -- во всяком случае, в его советской форме. Но если не рассматривать диалектический ма­териализм как всеобъемлющую философскую систему, включающую в себя "единственно научную и прогрессивную", по утверждению советских учебников (а на деле антинаучную и реакционную), гносеологию, то он предстает в гораздо более привлекательном свете.

Прежде всего, мы видим, что в религиозной системе Маркса диалектический материализм занимает свое вполне законное и естественное место. Что нужно религии от науки? Общая картина мира, в которую можно было бы естественным образом вписать Высшую Цель. Этого и искали в современной им науке основоположники марксизма. Представление об эволюции мира под действием единых законов было общепринятым. Оно образует фундамент системы. Но одна лишь констатация эволюции -- это еще не слишком много. Чрезвычайно желательны какие‑то общие соображения, общие законы развития, которые подвели бы нас к Высшей Цели. Маркс находит их в диалектике Гегеля и -- по его собственному выражению -- <<переворачи­вает их с головы на ноги>> --объявляет законами природы, новыми законами, которых ранее наука не знала.

Произвол? Конечно! С точки зрения ученого‑ригориста, это просто неслыханная наглость. Новые законы природы -- это никакие не законы, а такая же метафизическая спекуляция, какой она была у Гегеля. Их нельзя выразить точным языком науки, их нельзя ни подтвердить, ни опровергнуть в эксперименте, они не могут служить для предсказания явлений. С точки зрения позитивизма Конта, все это должно быть отвергнуто как лженаука и лжефилософия.

Однако современный позитивизм сильно изменился по срав­нению с жестким позитивизмом Конта. Конт отвергал всякие понятия, не имеющие непосредственного выражения в сфере опыта, и всякую теорию, основанную на таких понятиях, если даже она прекрасно согласовывалась с наблюдаемыми данными и позволяла делать предсказания. В частности, он отвергал волновую теорию света из‑за содержащегося в ней понятия об эфире. Конт считал, что ученый имеет право принять лишь такую гипотезу, которая не только согласуется с опытом, но необходимо следует из него, то есть ее отрицание приводит к противоречию с опытом. Мы теперь знаем, что для таких требований нет никаких логических оснований, и если их принять, то придется отказаться от многих весьма мощных теорий. Мы рассматриваем научные теории -- наряду с произведениями философии, литературы и искусства -- как свободные творения нашего ра­зума, подчиненные единственной цели: служа в качестве моделей действительности, помочь нам ориентироваться в событиях и предвидеть их. Эти модели образуют целый спектр по степени их формализованности,  то есть оторванности от нашего субъек­тивного, не поддающегося внешнему выражению опыта. На од­ном конце спектра находятся полностью формализованные теории, например, арифметика. Когда мы применяем такие теории -- в данном случае, производим арифметические действия -- мы отвлекаемся от содержания ее понятий и действуем чисто формально, по совершенно строгим, однозначным и четким правилам, так что эти действия можно поручить и машине: арифмометру или компьютеру. Интерпретация формализованной теории, то есть установление связи между ее понятиями и явлениями действительности -- в данном случае, процесс пересчета предметов ~ также должна быть одинаково понимаема и производима всеми и, в принципе, быть доступной для машины. Формализованная теория, таким образом, как бы отделяется от создавшего ее мозга человека и превращается в автономную ("объективную") модель внешнего мира. На другом конце спектра моделей мы находим произведения литературы и ис­кусства, представляющие собой некоторые синтетические художественные образы, апеллирующие к нашему индивидуальному опыту. Эти образы могут оказывать мощное воздействие на восприятие действительности человеком и его поведение. Но это происходит в глубине сознания, на уровне интуиции, механизм воздействия не выявляется, здесь нельзя говорить о расчете и доказательстве.

Идеалом науки являются полностью формализованные теории. Но такие теории не возникают из ничего. Понятия, на которых они основываются, пред существу ют в неформализованном виде в языке и практической деятельности человека; часто они находят выражение в философии и литературе. Критический анализ понятий и экспериментирование приводят к со­зданию формализованных теорий. Поэтому весь спектр наших моделей действительности -- от романов Достоевского до учебников арифметики -- участвует в процессе эволюции человече­ского знания. Мы не можем разорвать этот спектр в какой‑то точке и отделить чистую науку от чистого искусства; точно так же невозможно отделить науку от философии и философию от искусства. Если мы сделаем это, мы закроем науке путь движения вперед -- к созданию новых теорий. Все модели имеют право на существование -- при таком условии, конечно, что мы помним о необходимости критического подхода к каждой из них.

Понятия и законы гегелевской диалектики в интерпретации Маркса можно назвать "научно‑художественными" образами мира, отражающими его развитие. Это вполне законное явле­ние. Они нужны религии, они образуют связующее звено между формализованным, техническим элементом научной картины мира и теми -- как правило не известными нам и не осознавае­мыми нами -- факторами, которые побуждают нас принять Выс­шую Цель, почувствовать ее своей. Если эти образы и произ­вольны, то лишь в том же смысле, в котором произвольна и всякая научная теория. А задачу организации нашего опыта они, безусловно, выполняют: мы усматриваем в них какую‑то истинность, какое‑то соответствие с наблюдаемыми явлениями. Воздействие гегелевской диалектики на умы было вызвано, очевидно, тем, что она схватывала в какой‑то форме, хотя и до­вольно неопределенной, важные черты развития и люди это интуитивно чувствовали. А идея развития, эволюции, имеет тес­ную связь с идеей о цели. Представление об эволюции создает анизотропность времени, неэквивалентность прошлого и будущего и тем самым подводит к формированию цели. Идея эволюции -- связующее звено между наукой и религией. Именно поэтому, надо полагать, гегелевская система оказала такое могучее воздействие на умы, вызвала восторг, явно не лишенный религиозного элемента. Маркс направил этот религиозный восторг в область политики. Немецкий идеализм создал художественный образ мира, такого мира, который в самой своей основе и сущности содержит идею эволюции, почти отождествляется с ней.

Описывая эволюцию в терминах метасистемных переходов, объявляя метасистемный переход универсальным квантом эволюции, я отдаю себе отчет в том, что иду по стопам Гегеля и Маркса. Мы имеем здесь дело с типичным примером постепен­ного движения понятий в сторону уточнения и формализации. Мое определение метасистемного перехода (см. Часть 1) уже полностью удовлетворяет стандартам научного определения13. Но когда я использую это понятие для описания явлений мыш­ления, языка и истории, я перемещаюсь в спектре понятий в сторону меньшей формализации, вторгаясь в область научно‑философских, а иногда, быть может, и научно‑художественных образов.

Основной закон диалектики Гегеля -- триада: тезис, антитезис, синтез -- тесно связан с понятием о метасистемном переходе. Гегелевскую триаду можно рассматривать как частный случай или как упрощенное описание метасистемного перехода. Когда два элемента, которые находятся между собой в отношении противоречия -- тезис и антитезис, -- мы объединяем в метасистеме, то получаем их синтез. В этом синтезе противоречия между конституентами не исчезают, но -- выражаясь гегельянско‑марксистским языком -- снимаются, благодаря наличию общего механизма управления. Это -- частный случай метасистемного перехода. В общем случае происходит интег­рация многих подсистем, и вовсе не обязательно находящихся в отношении прямой противоположности. Но и в этом случае диалектик может использовать свой язык, объявив наличие множества  интегрируемых подсистем отрицанием единичности, которое затем снова сменяется единой системой -- метасистемой. Это -- отрицание отрицания. Появление нового качества в процессе развития всегда происходит путем метасистемного перехода. Образованию метасистемы, как правило, предшествует количественное накопление подсистем. Следовательно, метасистемный переход можно рассматривать как переход количества в качество.


\section{Эволюция, социальная интеграция и свобода}

В "Феномене науки", рассматривая эволюцию в терминах метасистемных переходов, я пришел к некоторым выводам относительно перспектив эволюции человеческого общества, роли науки в этом процессе и религиозно‑этической концепции, которая естественным образом включается в научное ми­ровоззрение. Здесь я приведу эти выводы в конспективном виде.

\paragraph{1.} Структура биосферы указывает на три крупномасштабных, поистине революционных, метасистемных перехода: интеграция макромолекул с образованием клетки, интеграция клеток с образованием многоклеточного организма, социальная интеграция -- образование человеческого общества. Первые две революции завершены, последняя только еще начинается.

\paragraph{2.} Из многоклеточных организмов лишь человек представляет собой такой элемент, интеграция которых является революционным метасистемным переходом, создающим качественно новый уровень организации материи. Сообщества животных можно рассматривать как первые (безуспешные) попытки осуществить эту революцию. Отличие человека -- в способности создания языка, которая сама является результатом определенного метасистемного перехода в структуре мозга человека по сравнению с высшими животными. Язык выполняет две функции: общения между людьми и моделирования действительности. На уровне социальной интеграции это в точности те же две функции, которые на уровне интеграции клеток в многоклеточный организм выполняет нервная система. Язык является как бы продолжением мозга человека; в материале языка создаются такие модели действительности (знаковые системы), которых не было в мозгу индивидуума. Далее, язык -- это общее, единое продолжение мозгов всех членов общества. Это коллективная модель действительности, над совершенствованием которой трудятся все члены общества и которая хранит опыт предыдущих поколений.

\paragraph{3.} Можно рассматривать общество как единое "сверхсущество". Его <<тело>> --это тела всех людей плюс предметы, созданные людьми: одежда, жилища, машины, книги и т.~д. Его <<физиология>> --это физиология всех людей плюс культура  общества, то есть определенный способ управления предметной компонентой общественного тела и образом мышления людей. Появление человека и социальная интеграция означают возникновение нового механизма усложнения организации материи, нового механизма Эволюции Вселенной. До человека развитие и совершенствование высшего уровня организации -- устройства мозга -- происходило в результате борьбы за существование и естественного отбора. Это медленный процесс, требующий смены многих поколений. В человеческом обществе развитие языка и культуры является результатом творческих усилий всех его членов. Отбор вариантов, необходимый для усложнения организации материи по методу проб и ошибок, происходит теперь в значительной степени в голове человека, он становится неотделимым от волевого акта человеческой личности. Человек становится средоточием Космического Творчества. Темп эволюции многократно возрастает.

\paragraph{4.} Эволюция культуры человеческого общества происходит путем метасистемных переходов в ее структуре. Современная наука является наивысшей достигнутой в настоящее время стадией в последовательности метасистемных переходов. Более того, овладев принципом метасистемного перехода, наука стала саморазвивающейся системой. Если культура -- это физиология, а язык -- нервная система социального "сверх‑существа", то наука -- его мозг. Наука представляет собой высший уровень иерархии в организации живой материи, она -- верхушечная почка растущего дерева, активная точка Великой Эволю­ции. В этом значение космического феномена науки как части феномена человека.

\paragraph{5.} Наука есть организация знания человека, но не его воли. Она отвечает на вопросы "что есть", "что будет, если", "что надо, чтобы". Но просто на вопрос, "что надо делать?", без всяких если и чтобы, она ответить принципиально не может. Проблема Высшей Цели остается вне науки, решение этой проблемы необ­ходимо требует волевого акта, оно является в конечном счете результатом свободного выбора.
Однако это вовсе не значит, что наука не влияет на то, как люди решают проблему Высшей Цели. Наука дает определенное представление о мире и о возможной роли человека в этом мире. В зависимости от того, каковы эти представления, те или иные волевые акты могут становиться более или менее ве­роятными. Еще важнее, что наука дает нам язык для описания -- а следовательно, в какой‑то степени и восприятия -- действительности. В частности, любые цели, которые мы можем поста­вить, должны быть выражены на этом языке.

\paragraph{6.} Представление об эволюции -- центральная идея современной науки. Эволюция привела к возникновению на Земле разумной жизни. И хотя нам известна ничтожная малость сферы влияния человека в Космосе, мы все же имеем право рассматривать человечество как явление космического значения. Опыт исследования развивающихся систем показывает, что новое качество появляется сперва в небольшом объеме, но благодаря заключенному в нем потенциалу захватывает со временем максимум жизненного пространства и создает плацдарм для нового, еще более высокого, уровня организации. Перспектива "гоминизации Вселенной" представляется нам реальностью. Никто не может утверждать, что это обязательно произойдет: человечество может и погибнуть или остановиться в развитии -- но никто не может отрицать существования этой возможности.

Волевому акту человеческой личности наука 20‑го века отводит гораздо более почетное место в процессе эволюции, чем наука 19‑го века. Эволюция не есть детерминированный процесс. Законы природы не определяют однозначно каждое следующее состояние мира, они лишь накладывают ограничения, оставляя огромное количество неопределенности. Эволюция есть непрерывное и повсеместное снятие этой неопределенности. Соударение элементарных частиц и принятие решения человеком -- примеры снятия неопределенности. Второй из этих актов характеризуется несравненно большим пространственно‑временным масштабом, и по мере "гоминизации Вселенной" масштаб этот будет возрастать. Этот акт знаком нам изнутри как проявление нашей свободной воли. Волевой акт -- это нить в ткани Великой Эволюции. В большой и сильно связанной системе, каковой является человечество, каждое отдельное действие, малое само по себе в общем масштабе, может повести благодаря наличию триггерных механизмов к чрезвычайно крупным переменам. Творческие достижения отдельных людей оказывают со временем влияние на всю историю человечества; влияние решений, принимаемых государственными деятелями, очевидно каждому. Однако не менее важную роль играют причинно‑следственные цепочки из малозаметных событий, приводящих к большим событиям. Можно разрушить всю ткань, разорвав одну из нитей. Значение каждого волевого акта личности может быть огромно.

\paragraph{7.} Никто не может действовать вопреки законам природы. Этические учения, противоречащие общему направлению эволюции, то есть ставящие цели, несовместимые с ним, не могут привести своих последователей к конструктивному вкладу в Эволюцию, а это значит, что в конечном счете они будут вычеркнуты из памяти мира. Таково свойство развития: то, что соответствует его общему, абстрактному, плану, увековечивается в структуре развивающегося мира; то что ему противоречит, преодолевается и гибнет.

По отношению к социальной интеграции отсюда следует такой вывод. Если человечество будет ставить себе цели, несовместимые с социальной интеграцией или как‑либо ограничивающие этот процесс, то оно придет к эволюционному тупику -- дальнейшее творческое развитие, порождение новых качеств станет невозможным. Тогда оно рано или поздно погибнет, и задачу одухотворения Вселенной возьмут на себя другие ветви Великой Эволюции. В развивающемся мире покоя нет; все, что не развивается, -- гибнет.

\paragraph{8.} Эволюционный тупик ожидает человечество не только на пути отказа от интеграции, но и на пути такой интеграции, когда в жертву объединению людей приносится их творческая свобода. Процесс социальной интеграции никогда еще не протекал так бурно и так явно, как сейчас. Современная наука и техника сделали каждого человека находящимся в сфере влияния каждого другого. Современная культура глобальна. Современные государства -- это огромные механизмы, имеющие тенденцию все более жестко регламентировать поведение каждого гражданина, навязывать ему извне его потребности, вкусы, мнения. Поэтому сейчас мы лучше, чем когда бы то ни было, видим основное противоречие социальной интеграции: противоречие между необходимостью включить человека в систему, в непрерывно консолидирующееся целое, и необходимостью сохранить его как свободную творческую личность. И здесь возникает фундаментальнейшая проблема: как разрешить это противоречие? Как совместить движение по пути интеграции с движением по пути свободы? От того, сколь успешно будет разрешена эта проблема, зависит будущее человечества -- зависит, есть ли у человечества будущее.

У нас нет оснований считать априори, что противоречие между интеграцией и свободой неразрешимо. Личное, творческое начало является сущностью человека, основным двигателем Эволюции в эпоху разума. Если оно будет подавлено социальной интеграцией, то движение остановится. С другой стороны, и социальная интеграция необходима. Без нее невозможно дальнейшее развитие культуры, увеличение власти человеческого духа над природой; в ней -- сущность нового уровня организации материи. Почему же мы должны полагать, что социальная интеграция и свобода личности несовместимы? Ведь осуществлялась же интеграция успешно на других уровнях организации! Когда клетки объединяются в многоклеточный организм, то свои биологические функции -- обмен веществ и размножение путем деления -- они продолжают выполнять. Новое качество -- жизнь организма -- появляется не вопреки биологическим функ­циям отдельных клеток, а напротив, благодаря им. Творческий акт свободной воли -- это "биологическая функция" человека. Следовательно, в интегрированном обществе она должна сохраниться в качестве неприкосновенной основы, а новые ка­чества должны появляться через нее и благодаря ей.

\paragraph{9.} Вряд ли кто‑нибудь возьмется предсказать сейчас, как далеко пойдет и какие формы примет интеграция индивидуумов. Несомненно, что в будущем станет возможным прямой обмен информацией между нервными системами отдельных людей, их физическая интеграция. Вероятно, физическая интеграция породит качественно новые, высшие формы надличностного сознания, и это будет процесс, который можно опи­сать как слияние душ отдельных людей в единой Высшей Душе. Всечеловеческая Высшая Душа будет в принципе бессмертна, как бессмертно в принципе человечество. О соотношении между индивидуальным сознанием и всечеловеческим сознанием можно только гадать, но представляется вполне возможным, что физическая интеграция индивидуумов явится разрешением извечного противоречия между человеческим разумом и смертью.

\paragraph{10.} Пока это противоречие не разрешено, единственной формой бессмертия, о которой мы можем всерьез говорить, является бессмертие нашего вклада в Великую Эволюцию Вселенной. Воля к так понятому бессмертию всегда была движущей силой в творчестве людей, стоящих в своем мировоззрении на научных позициях.

Осознание своей смертности -- одна из исходных точек очеловечивания. Мысль о неизбежности смерти создает для разумного существа мучительную ситуацию, из которой оно ищет выхода. Протест против смерти, против распада своей личности общ всем людям. Он является тем источником, из которого в конечном счете все религиозные учения черпают необходимую им волевую компоненту. Это -- воля к бессмертию.

Христианство исходит из безусловной веры в бессмертие души. При этом протест против смерти используется как сила, побуждающая человека принять такое учение, -- ведь оно с самого начала обещает бессмертие. Под мощным воздействием критического разума представления о бессмертии души и о загробной жизни, некогда совершенно конкретные и ясные, становятся все более абстрактными и бледными, теряют свою эмоциональную убедительность. Соответственно, старые религиозные системы теряют свое влияние. Если говорить о человеке, воспитанном на идеях и образах современной науки, то воля к бессмертию может привести его лишь к одной цели -- внести свой личный вклад в Великую Эволюцию, увековечить свою личность во всех последующих актах мировой драмы. Этот вклад, чтобы быть вечным, должен быть конструктивным. Такова Высшая Цель, которую ставит себе человек. Ее конкре­тизация зависит от состояния нашего знания -- от наших пред­ставлений об Эволюции. Замечательной чертой этой целевой установки является органический синтез личного и всеобщего. Она порождает чувство ответственности за общее дело, за процесс Эволюции. Ибо все, что угрожает этому процессу, в частности тупой эгоизм обывателя, угрожает и личному вкладу каждого человека, грозит сделать его жизнь бессмысленной.

Во всех этих рассуждениях мы должны помнить, что различие между знанием и волей сохраняется, оно неустранимо. Если человек не может или не хочет совершить необходимого волевого акта, то никакая наука, никакая логика не заставит его принять Высшую Цель, ощутить свою ответственность за Вели­кую Эволюцию. Обывателя, твердо решившего жить рабом об­стоятельств и довольствоваться своим убогим частным идеа­лом, не возродит ничто, и он бесследно сойдет со сцены. Кто не хочет бессмертия, тот и не получит его. Подобно тому как животное, лишенное инстинкта размножения, не выполнит сво­ей животной функции, так и человек, лишенный воли к бес­смертию, не выполнит своей человеческой функции, изменит своему человеческому предначертанию.


\section{От традиционных религий к социализму}

С учетом всего вышесказанного, я называю социализмом -- тем социализмом, который я принимаю и готов проповеды­вать, -- религиозно‑этическое учение, провозглашающее Высшей Целью человека неограниченную социальную интеграцию при обеспечении и развитии творческой свободы личности. Это -- научный социализм, ибо, во‑первых, его идея и его эмоциональ­ная убедительность зависят от "научно‑философских" и "науч­но‑художественных" образов, и во‑вторых, инструментом ин­теграции провозглашается наука.

Такое понимание социализма, по‑видимому, ближе всего к "этическому социализму" немецкой социал‑демократии; если бы мне потребовалось обозначить свои взгляды каким‑то из уже существующих терминов, то я, вероятно, остановился бы на этом словосочетании. Но я считаю чрезвычайно важным под­черкнуть религиозный, а не только чисто этический характер социализма. Религия включает в себя этику, но не исчерпывает­ся ею. Она содержит некое представление о сущем -- представ­ление, которое не просто подводит нас к целевым установкам и нормам поведения, но и порождает определенные эмоции -- религиозное чувство. Под этикой мы традиционно понимаем нечто бесстрастное, выведенное из соображений целесообраз­ности или постулированное в качестве априорного логического принципа, как аксиомы в математике. Религия апеллирует к чувству и воспитывает его. А непосредственной силой, движу­щей массами людей, являются чувства, эмоции. Один из наибо­лее действенных способов влияния разума на историю -- через воспитание чувств. История социализма недвусмысленно указы­вает на ту роль, которую в этом движении играют эмоции. Про­тивники социализма видят в нем лишь игру разрушительных чувств. Но без детального анализа существа явления не следует судить о нем по начальным шагам. И совсем недопустимо про­сто игнорировать часть движения, как это делает в своей статье Шафаревич, игнорируя реформистские социал‑демократические течения. Обличая социализм, он обличает лишь его варварскую форму. Можно даже сказать, что из варварского социализма он берет только варварство, а социалистический элемент игно­рирует. Шафаревич стоит на христианских позициях. Но ведь и христианство знает свою варварскую форму. Одни из первых христиан растерзали Ипатию. Аргументы Шафаревича против социализма -- это аргументы римского патриция против хри­стианства.

Если отбросить всевозможные упрощения и вульгаризации, то мы найдем, что чувство, заключенное в представлении о со­циальной интеграции, созидательно. Оно задевает какие‑то глу­бокие струны в душе человека. Идея бессмертия неотделима от идеи интеграции личностей (если, конечно, исключить наив­ные представления о такой загробной жизни, которая мало чем отличается от жизни в нашем мире). В традиционных религиях идея интеграции предстает в спиритуалистическом облачении, как единение в Боге.   Спиритуализм, как и научный подход, влечет за собой определенный способ организации нашего духов­ного и чувственного опыта. Было бы неправильно просто отбра­сывать все, что сделано человечеством в прямой и тесной связи со спиритуалистическим подходом. Но вопрос в том, какой подход положить в основу  мировоззрения, и в частности трактовки интеграции. Приняв за основу спиритуализм, мы закры­ваем себе дорогу к движению вперед. Максимум, на что мы можем здесь рассчитывать -- это дать каждому человеку воз­можность определенных переживаний, быть может, весьма экзо­тических. Это не слишком много, особенно если учесть, что в создании экзотических переживаний со спиритуализмом небе­зуспешно конкурируют даже грубые химические методы воз­действия на мозг, не говоря уж о потенциально возможных биологических способах. Приняв за основу критический науч­ный метод и материалистический подход к интеграции, мы открываем перед собой фантастические (и, надо признаться, жутковатые) перспективы.

Переход от спиритуализма к научному методу -- это метасистемный переход, ибо последний включает в себя критиче­ский анализ всякого словоупотребления и всяких форм орга­низации опыта, в том числе и спиритуалистических. Следова­тельно, этот переход -- шаг по пути Эволюции, и он необратим. Социализм является прямым наследником великих религий прошлого, это единственная великая религия, возможная в век науки.

Спиритуалистическая трансцендентность прежних религий ус­тупает место в научном социализме сочетанию предельной аб­страктности в формулировке Высшей Цели и конкретности в производных целях на всех стадиях движения. Что нового вносит эта замена в религиозное чувство? Не действует ли она на него разрушительно?

Мне думается, что нет. Ведь природа человека остается неиз­менной. Если есть в ней нечто, вызывающее религиозное чувст­во, то оно никуда не денется. Преобразованию подвергается лишь концептуальная, вербальная сторона духовной жизни, причем, как я уже указывал, основные понятия, лежащие на границе с религиозно‑эмоциональной сферой, -- такие, как таинст­венность мироздания, смысл жизни, интеграция личностей, бес­смертие, трансцендентность, -- не выходят из игры, но перео­смысляются.  Это переосмысление является, в сущности, осовремениванием, приведением пограничного концептуального слоя в соответствие с реальностями нашего языка и мышления в научно‑промышленную эру. Поэтому из общих соображений кажется скорее более вероятным, что это преобразование должно содействовать воспитанию религиозного чувства, а не подавлению его.

Размышления о смысле жизни в контексте идей об эволюции, творчестве и первозданной свободы личного волевого акта оказывают огромное влияние на психику человека, а следо­вательно и на его жизнь. О себе я совершенно точно знаю, что именно идеи. выраженные мною в "Феномене науки" и в данной книге, изменили мою структуру ценностей и сделали диссиден­том. Моя профессия (физика плюс кибернетика) играла в раз­витии этих идей определяющую роль. Между прочим, среди советских диссидентов непропорционально большая доля при­надлежит к ученым, а среди ученых ‑- к физикам. Это, по‑видимому, является результатом действия нескольких различных факторов, но я полагаю, что важнейшим из них является представление о Великой Эволюции, которое в качестве неотъемлемой составной части входит в мышление ученого, особенно физика, биолога, кибернетика. Иногда осознанно, иногда неосоз­нанно это представление служит основой для построения собственных нравственных критериев, не совпадающих с теми, которые навязываются циничным тоталитарным обществом.


\section{Относительность общих принципов}

Итак, основным отличием социализма 20‑го века от социа­лизма 19‑го века является то, что он гораздо более органично включает в себя идею свободы личности, -- она входит не только в формулировку Высшей Цели, но и в представление об устрой­стве мира и механизме эволюции. Тоталитарный социализм со­ветского типа можно, с этой точки зрения, считать не только вар­варской формой социализма, но и не социализмом вовсе. Теперь перейдем к другим чертам современного научного мышления, накладывающим печать на социально‑политическое мышление. Я хочу отметить две таких черты.

Есть одно отличие в стиле мышления современного ученого по сравнению с учеными предыдущих эпох (включая 19 век), на которое не часто обращают внимание, но которое имеет, как мне кажется, весьма ощутимые социальные последствия. Оно касается отношения ученого к содержанию своей работы. Раньше ученые мыслили исключительно в терминах открытий. 

Считалось, что законы природы существуют как нечто вполне законченное; задача ученого лишь в том, чтобы обнаружить их, раскрыть глаза своим современникам на их существование. Теперь же, главным образом благодаря новой физике, психо­логия ученого заметно изменилась. Осознание того, что наука является лишь определенным способом организации чувственного опыта, оказалось необходимым физикам для их работы, для того чтобы понимать новые физические теории. Но прини­мая это представление, мы начинаем сознавать себя не столько открывателями, сколько творцами: создателями новых моде­лей действительности, которые оказываются более или менее пригодными для целей организации опыта. Элемент открытия не исчезает, конечно, совсем, но он скорее принимает характер удачного изобретения.  В научных дисциплинах, более близких к непосредственному чувственному опыту, чем новая физика, этот фактор не играет такой большой роли, однако его влияние распространилось на философию и методологию науки в целом. Слова модель  и моделирование  принадлежат теперь к числу наиболее употребляемых в научной литературе. Ученый пред­почитает говорить, что он построил модель (или теорию),  а не открыл закон.  Термин "открытие" употребляется только по от­ношению к непосредственно наблюдаемым феноменам. Относи­тельность всех общих принципов и законов, их инструменталь­ный, служебный характер -- эта мысль прочно укоренилась в науке.

Какое отношение это имеет к общественным проблемам и к политике?

Образ мышления ученых шаг за шагом (правда, очень мед­ленно: в масштабе поколений) распространяется на все общест­во. Непрофессионал не может усвоить технических деталей, но постепенно и часто неосознанно улавливает дух перемен, новый философски‑методологический стиль. Особенно это относится к тем политикам, которые претендуют на то, чтобы идти в ногу со временем, и следовательно, призывают в союзники науку. Если в эпоху открытия законов для политика было естественно "открывать глаза" своим современникам на якобы найден­ные им объективные законы развития общества, как это сделал Маркс, то теперь это стало непопулярным. Чтобы иметь успех в обществе, где умы находятся под влиянием научного мировоззрения 20‑го века, политик должен выражаться примерно таким языком:

"Есть несколько конкурирующих моделей, которые описы­вают проблемы нашего общества и, соответственно, указывают пути их решения. Мне наибольшее доверие внушает такая‑то модель. Давайте основывать принятие решений на этой модели. Я полагаю, что через такое‑то время мы увидим, хороша эта модель или плоха и сделаем соответствующие выводы."

Психология открытия законов порождает в политике психологию "все или ничего". Ибо закон или существует, или нет; он является либо великой истиной, либо печальным заблуждением. В сознании масс, к которому апеллируют политики, это преломляется так: общество устроено либо "правильно", то есть в соответствии с законами природы, либо "неправильно", то есть в противоречии с ними. Во втором случае общество должно быть безжалостно разрушено -- "до основания, а затем" перестроено заново. Эта черта прежней науки роднит ее с догматическими религиями. Ибо догмат тоже или целиком истинен, или целиком ложен. Методологически европейская наука с самого своего возникновения стояла на позициях скептицизма и критицизма: многократная экспериментальная проверка и признание относительности всякой высказанной истины. Но онтологически -- в представлении о сущем, о том, как дело обстоит "на самом деле", -- она была столь же догматична, как и религия. Методология науки была профессиональным делом ученых, а онтология лежала в фундаменте мировоззрения и пе­редавалась широким слоям людей.

Современная наука во всех своих аспектах стремится сделать человека эволюционистом и градуалистом в политике. Чем шире распространяется ее влияние, тем устойчивее становятся эти черты в обществе. Но это не означает отказа от революционности в мыслях и делах, если под революционностью понимать глубину преобразований. История науки 20‑го века дает нам яркий пример революции и учит искать смелых и неожиданных решений. Эволюционизм и градуализм -- вопрос метода. Идет процесс все более глубокого проникновения в сознание людей критического, научного метода. В частности, проникновение этого метода в философию изгнало из нее мышление в терминах "все или ничего" по отношению к общим законам и принципам, которое было родственно классическому религиозному догматизму и служило питательной средой фанатизма.

\section{Кибернетическое мышление}

Вторая черта современного научного мышления, на которое я хочу обратить внимание и вокруг которой буду вести изло­жение до конца этой части книги, совсем недавнего происхождения, и она не успела еще проникнуть достаточно глубоко в общественное сознание. Однако процесс проникновения происходит, и я придаю ему чрезвычайно большое, можно даже сказать -- решающее -- значение для судьбы научного социа­лизма. Эту черту называют кибернетическим  (или иногда системным) мышлением. 

Кибернетика в моде, ее достижения общеизвестны и оказы­вают преобразующее влияние на производство и быт. Но какое отношение это имеет к образу мышления и общественно‑поли­тическим воззрениям большинства людей, которые не являются кибернетиками? Не имеет ли в виду автор косвенное влияние через технический прогресс и изменение образа жизни?

Нет, я имею в виду прямое концептуальное влияние. В своих оценках и целях мы исходим из окружающей нас реальности, а эту реальность описываем с помощью некоторой системы по­нятий и терминов. Всякая система описания реальности непол­на, она подчеркивает одни аспекты и затушевывает другие. Поэтому она активно влияет на нашу систему оценок и целей, хотя это влияние не бросается в глаза: находясь постоянно в некоторой системе отсчета, мы склонны не замечать ее, отож­дествлять наше описание с "объективной действительностью". Наиболее общие и часто употребляемые понятия формируют то, что можно назвать "фоновой концепцией реальности". Одно­значной связи между фоновой концепцией реальности и систе­мой оценок и целей нет, но влияние имеет место: различные фоновые концепции подталкивают к различным оценкам и це­лям. Поэтому с изменением доминирующей фоновой концепцией реальности меняется и доминирующая система оценок.

В европейской культуре до 20‑го века фоновая концепция реальности была основана на понятиях механики и химии. Ядро этой концепции таково: мир есть совокупность атомов, обладающих определенными качествами.  Понятия о системе, структуре, организации являются в этой концепции онтологически вторичными, производными. Напротив, кибернетическая фоновая концепция реальности (которая, вероятно, будет до­минирующей фоновой концепцией 21‑го века) выдвигает эти понятия на передний план. Кибернетик мыслит не в терминах качеств  элементов, а в терминах отношений между ними. Механико‑химическое мышление видит в организации вещей проявление их "сущности". Кибернетическое мышление объяв­ляет "сущностью" вещей их организацию.
Механико‑химическая концепция реальности естественным образом подводит к индивидуалистической концепции общест­ва. Общество -- это совокупность людей‑атомов. Человеческая личность обладает по своей природе качествами фундаменталь­ности, абсолютности и неделимости. Она обладает также дру­гими качествами и способностью самосовершенствования -- развития "хороших" качеств и подавления "дурных". Общест­во необходимо постольку, поскольку оно дает возможность человеческой личности проявлять свои качества и совершенст­вовать их. Сверх этого никакой ценностью общество не обла­дает, оно есть лишь форма "мирного сосуществования" инди­видуумов.

В кибернетической концепции реальности человеческая лич­ность -- один из уровней единой космической организации. Во‑первых, человек рассматривается не как атом, а как систе­ма, имеющая сложную иерархическую структуру. Во‑вторых, человек рассматривается как подсистема объемлющей систе­мы человеческого общества. Кибернетическая фоновая кон­цепция реальности подводит к социалистической концепции общества. Смысл и ценность существования человека при любой конкретизации этих понятий должны выводиться из смысла существования общества. Общественное благо -- нечто большее, чем сумма индивидуальных благ.

\section{Неизбежность интеграции}

Индивидуализм может быть разным -- в зависимости от то­го, какие качества индивидуума считаются "хорошими", а ка­кие "плохими". Здесь мы соприкасаемся со сферой религии. Индивидуалистическая концепция общества хорошо уживает­ся с метафизически трансцендентной религией, она дополняет­ся такой религией до некоторой комбинации, на которой может быть основано более или менее жизнеспособное общественное устройство. Понятие о Боге образует тот высший уровень, ко­торый объединяет индивидуумов и способствует стабильности общества. Из идеи Бога выводятся основные нравственные принципы, и они, конечно, таковы, что делают возможным совместное существование людей. Именно так обстояло дело в Европе, пока христианская религия была основой духовной жизни общества. Единение во Христе -- сублимированная форма социальной интеграции, проповедь любви к ближнему -- ее по­сюсторонний инструмент. Спиритуалистическая трансцендент­ность христианства до известной степени преодолевает ограни­ченность индивидуализма, а точнее, служит формой, в которой проявляется присущая человеку тяга к интеграции. В то же время представление о божественной искре, заложенное в ду­шу каждого человека, возводит человеческую личность на не­досягаемую для прагматико‑политических соображений высоту (в теории, во всяком случае). Оно служит идеологическим фундаментом свободы личности.

По этим причинам Эпоха Возрождения -- возрождения антич­ного индивидуализма в рамках христианской религии -- при­вела к бурному расцвету культуры в Европе. Она осуществила тот синтез интеграции и свободы, при котором только и воз­можна конструктивная эволюция. Механико‑химическая фоно­вая концепция природы занимала в этом синтезе важное и ес­тественно определенное место.

Упадок христианской религии разрушает созданный многи­ми поколениями европейцев синтез интеграции и свободы. Раз­рушение происходит не сразу, в течение какого‑то времени традиционные понятия и нормы поведения (в тех секциях об­щества, где они существовали!) передаются следующему поколению. Но почва из‑под них выбита. Рано или поздно они разру­шаются.

Если интеграция без свободы ведет к окостенению общест­ва, к церковному или партийному мракобесию, то свобода без интеграции ведет к развалу общества. На чисто индивидуали­стической основе общество построить нельзя. Существует не­сколько мудрых принципов, которые в рамках индивидуали­стической концепции стремятся сделать жизнь и общество снос­ной для человека. Главный из них: поступай с другим так, как ты хотел бы, чтобы поступали с тобой. Такие принципы апел­лируют к рассудку и дальновидности членов общества, и нель­зя сказать, что они апеллируют впустую: их влияние на общест­во не равно нулю и возрастает с развитием культуры. Тем не менее их недостаточно. Общество требует от человека опре­деленных жертв, особенно в кризисные моменты. Единственным оправданием жертв при индивидуалистическом подходе будет расчет -- в духе теории "разумного эгоизма" Чернышевского -- что они в конце концов окупятся тем, что принесут некоторое благо самому себе или, на худой конец, близким людям. Но об­щество требует жертв (малых -- постоянно, но иногда и боль­ших) , которые не окупаются в жизни индивидуума или близ­ких ему людей. Теория разумного эгоизма может быть моди­фицирована путем расширения круга "близких людей" до об­щества в целом. Но в безрелигиозной механической концеп­ции общества отсутствует связующее звено между индивиду­альными благами незнакомых мне членов общества и моей лич­ностью, моими эмоциями. "Общественное благо" оказывается искусственной абстрактной конструкцией, которая не выдер­живает испытания трудностями и распадается на составные части. Наконец, роковым для индивидуалистического общест­ва оказывается вопрос: кто и в какой степени должен прино­сить жертву? Если жертва является не проявлением устремлен­ности к религиозной Высшей Цели, органически вошедшей в сознание и в сферу эмоций человека, а результатом расчета, то важное значение приобретает распределение жертв "поров­ну". Но распределить жертвы "поровну" невозможно. Попыт­ки измерения жертв и возникающая при этом склока порож­дают такое общество, ради которого уже никому не хочется ничем жертвовать.

Факт способности человека к социальной интеграции делает интеграцию неизбежной. Те идеологии и те сообщества людей, которые не обеспечивают уровня интеграции, достижимого при других условиях, должны будут сойти со сцены. Социаль­ную интеграцию можно сравнить с процессом кристаллизации. Существует механизм, который объединяет атомы в кристалл -- определенную упорядоченную структуру, более прочную, чем другие структуры. Рано или поздно менее прочные структуры распадутся на части, которые либо перегруппируются в новый кристалл, либо будут поглощены уже существующими крис­таллами. Проблема, которая стоит перед каждой страной и перед человечеством в целом, не в том, будет ли происходить про­цесс социальной интеграции или нет, а в том, как  он будет про­исходить. Приведет ли он к уродливому тоталитаризму, обеспечивающему социальную интеграцию насилием, страхом и духов­ной кастрацией человека, или к подлинному социализму, дви­жущей силой которого является свободная человеческая лич­ность.


\section{Механическая и кибернетическая интеграция}

При химико‑механическом подходе к обществу человече­ские существа, подобно атомам Дальтона, в некотором смысле элементарны.  Они равны ‑ отсюда эгалитаризм, они взаимно непроницаемы -- отсюда индивидуализм. Конечно, здесь нет прямой логической связи (люди все‑таки не атомы), но есть распространение, расширение определенной концептуальной схе­мы, взятой из науки. Создаваемые на основе этой схемы науч­но‑художественные образы дают определенное направление мышлению, хотя и не всегда проявляются в виде четких ло­гических умозаключений.

Как выглядит в механистической модели социальная интег­рация?

В общих чертах -- как превращение газа в жидкость и твер­дое тело. Вспомним, как описывается этот процесс в школь­ном курсе физики. Когда расстояние между центрами атомов больше суммы их радиусов, атомы притягиваются друг к другу; когда они сближаются на меньшее расстояние, они начи­нают сильно отталкиваться. Сила притяжения между атомами быстро убывает с расстоянием, поэтому в разреженном газе атомы почти не взаимодействуют -- они лишь изредка сталки­ваются друг с другом. По мере увеличения плотности вещест­ва атомы оказываются в сфере притяжения друг друга, и, если температура не слишком высока, вещество переходит в кон­денсированное состояние, в котором сила притяжения объе­диняет атомы в одно тело без давления извне. В конденсиро­ванном состоянии атомы упакованы, грубо говоря, вплотную друг к другу, поэтому приложение даже сильного внешнего давления приводит ‑ в отличие от случая газа -- лишь к незна­чительному уменьшению объема.

При кибернетическом подходе к социальной интеграции ин­тегрируемая единица ‑ человек ‑ предстает, прежде всего, как сложная система,  характеризующаяся многоуровневой иерархией по управлению. Социальная интеграция есть метасистемный переход, при котором образуется новый уровень иерархии, управляющий высшим уровнем организации интег­рируемых подсистем. Управление, как я уже не раз подчерки­вал, не есть взятие на себя функций управляемого объекта, оно состоит в том, что эти объекты ставятся в определенные условия, вследствие чего происходит координация их деятель­ности. В процессе управления и осуществляется интеграция. Таким образом, кибернетическая интеграция имеет своим пер­вым и основным объектом высший управляющий уровень ин­тегрируемых подсистем.

Мы видим, что механическая и кибернетическая модели ин­теграции имеют в некотором смысле противоположную на­правленность. Механическая интеграция направлена снаружи внутрь. В результате сближения атомов друг с другом они на­чинают взаимодействовать своими внешними частями. Когда взаимодействие распространяется на внутренние части атомов, возникает сильное отталкивание. Социальная интеграция на механический лад -- это насильственное объединение людей в коммуны, колхозы и т.~п. в уверенности, что, будучи прижаты друг к другу, они естественным образом соединятся в идеаль­ную общественную структуру, подобно тому как атомы углерода, подвергнутые сильному давлению, образуют алмаз.

Кибернетическая интеграция общества, напротив, направ­лена изнутри наружу. Она начинается в сфере духовной куль­туры, точнее она начинается вместе с  формированием духов­ной культуры, общества, которая и есть не что иное, как си­стема координации и интеграции сознания индивидуумов. Про­явление интеграции на низших уровнях -- в сфере производст­ва, быта, физических действий людей ‑ должно явиться следст­вием, автоматическим результатом интеграции на уровне со­знания. Только такая интеграция и имеет шансы на продолже­ние развития, на конструктивную эволюцию.

Противопоставляя механическую и кибернетическую моде­ли социальной интеграции, мы возобновляем древний спор о том, как приступить к установлению на Земле всеобщего братст­ва: разделить все имущество поровну и сойтись в коммуны или воспитывать себя и других в духе любви, терпимости и про­щения? Первый путь связан в популярном сознании с понятием социализма, второй -- с христианством. Я пытался по мере сил показать, что, когда мы анализируем понятие о социализме в терминах, подсказываемых современной наукой, то подлинно социалистическим предстает только второй путь, первый же представляется социалистическим варварством. Учение Христа -- путь к социализму. Но конечно, одной лишь проповеди хри­стианских нравственных принципов недостаточно. Это лишь на­чало -- самое, самое начало пути. Христианство с его отделением духовного от телесного и заботой только о духовном ограничи­вало идею интеграции интеграцией душ. Оно правильно, с кибернетической точки зрения, указывало исходную точку интег­рации, но не дало перспективы, взгляда на процесс в целом. Принцип, выразившийся в знаменитом изречении "отдай Кесарево Кесарю", играл до поры до времени благотворную, может быть, даже спасительную роль. Это была уступка, кость, брошен­ная душой телу, чтобы сохранить какую‑то степень независимости. Трансцендентный спиритуализм христианства, отделе­ние духовного от телесного уберегло Западную Европу в про­цессе интеграции от всепожирающего тоталитаризма. Но в наше время мировоззренческие основы христианства стали препятствием для дальнейшего движения. Социалист должен рассматривать христианина как своего предшественника. Он испытывает к нему чувство благодарности и многое берет от него. Но он берет лишь то, что считает нужным, и на свой лад переосмысливает и переопределяет взятое.


\section{Громящий кулак}

\epigraph{Партия -- \\
рука миллионопалая, \\
сжатая \\
в один \\
громящий кулак.}{В.~Маяковский [16]}

Если выбирать краткую формулировку для обозначения того, в чем состоит варварство марксистско‑ленинской формы социализма, то можно сказать просто: механическая социаль­ная интеграция. Основные негативные черты марксизма‑лени­низма отражены в ней: экономический детерминизм, прини­жение роли духовной культуры, ставка на "революционное" насилие. Эти черты -- результат поспешного и неквалифици­рованного использования некоторых аспектов науки 19‑го ве­ка; они напоминают о френологе, который пытался изменить психические наклонности пациента, выравнивая шишки в одних местах черепа и образуя шишки в других местах. Механиче­ская интеграция не может привести к другому обществу, кроме тоталитарного, это насильственное сдавливание, спрессовыва­ние людей.

При кибернетическом подходе к интеграции социализм ‑ это прежде всего явление культуры.  Развиваясь и укрепляясь в сознании людей через посредство философии, науки, литера­туры, искусства, социалистическое сознание создает предпосыл­ки для выработки новых форм отношений между людьми, для перестройки общественных и государственных учреждений и системы производства.

Попытка осуществлять интеграцию организованных систем, начиная не с верхнего, а с нижнего или среднего уровня иерар­хии управления, в частности с экономического уровня в случае социальной интеграции, ‑ это кибернетический абсурд, уродство. Это как если бы мы вздумали объединить десять ра­бочих в одного "большого рабочего", связав их руки вместе в одну "большую руку", а ноги ‑ в одну "большую ногу". Другой образ -- весьма поучительный -- создан Маяковским в стихах, которые дети в Советском Союзе заучивают наизусть в школе и которые я поместил в эпиграфе. Рука человека с ее пальцами, свободными от взаимного давления и управляемыми из общего центра -- мозга, являет собой пример кибернетиче­ской интеграции. Это рука, которая может делать операцию на сердце и играть на фортепьяно. Кулак ‑ это символ механи­ческой, тоталитарной интеграции. Он способен лишь громить, разрушать, уничтожать. Тоталитаризм -- это миллионопалый кулак, занесенный над человечеством.


\section{Структурно‑функциональный параллелизм}

Противоречие между социальной интеграцией и свободой личности в рамках общего позитивистского подхода к пробле­мам общества нашло отражение в следующем отрывке из статьи Герберта Спенсера, в которой он формулирует свои расхож­дения с Огюстом Контом.
"По мнению Конта, самое совершенное общество есть такое, в котором управление  достигло своего высшего развития; в котором отдельные функции подчинены в значительно большей степени, чем теперь, общественной регламентации; в котором иерархия, крепко сложенная и снабженная признанной властью, заправляет всем; в котором индивидуальная жизнь должна быть подчинена по большей части жизни социальной.

По моему мнению, напротив, идеалом, к которому мы идем, является общество, в котором управление  будет доведено до минимума, а свобода достигнет наивозможной широты; в ко­тором человеческая природа будет путем социальной дисцип­лины так приспособлена к гражданской жизни, что всякое внешнее давление будет бесполезно и каждый будет господи­ном сам себе; в котором гражданин не будет допускать ника­кого посягательства на свою свободу, кроме разве того пося­гательства, которое необходимо для обеспечения равной сво­боды для других; в котором самопроизвольная кооперация, развившая нашу промышленную систему и продолжающая раз­вивать ее с быстротою все возрастающей, поведет к присвоению себе почти всех социальных функций и оставит в качестве цели правительственной деятельности былого времени только обя­занность блюсти за свободой и обеспечивать эту самопроизволь­ную кооперацию; в котором развитие индивидуальной жизни не будет ведать себе иных пределов, кроме наложенных на не­го социальной жизнью, и в котором социальная жизнь будет преследовать только одну цель -- обеспечение свободного раз­вития индивидуальной жизни".

Спенсер противопоставляет здесь свой английский либера­лизм французскому этатизму Конта. Однако противополож­ность между этими двумя идеалами общества не так абсолютна, как это, по‑видимому, представляется Спенсеру. Совместимость либерализма и этатизма зависит от того, что мы понимаем под иерархией и управлением или вернее, какие конкретные формы они принимают.

Начнем со спенсеровского конца. Самопроизвольная коопе­рация граждан, за которую выступает Спенсер, нисколько не противоречит идее интеграции; более того, самопроизвольная кооперация не избавляет нас от необходимости  иерархии. Ибо в обществе, насчитывающем миллионы членов, самопроизволь­ных объединений граждан (любого характера: производствен­ного, культурного, бытового) будет так много, что предста­вители этих объединений должны будут в свою очередь всту­пать в какие‑то объединения друг с другом, иначе общество просто развалится на части. Однако и этих объединений -- объе­динений "представителей первого уровня" будет, вероятно, слишком много, так что им придется образовать еще один уро­вень иерархии, и т.~д., пока не сложится многоуровневая иерар­хическая система представителей, на вершине которой находит­ся обозримое число людей, способных эффективно контакти­ровать друг с другом. Иерархия в той или иной форме необ­ходима для организации больших систем. Это -- закон кибер­нетики, и от него никуда не денешься. Что же касается свободы, то и Спенсер признает, что она не может быть абсолютной. Он вынужден говорить о социальной дисциплине, о пределах, наложенных на индивидуальную жизнь социальной жизнью, и т.~п.

Посмотрим теперь на дело с контовского конца. Развитие общества есть развитие иерархической системы управления. Но что такое управление? Управление подсистемой не есть лишение ее всякой степени автономности. В области общест­венных отношений управление вовсе не сводится к подчинению и принуждению -- это лишь самый грубый, несовершенный вид управления, можно сказать, механический вид.  Если представ­лять себе общество в виде гигантского часового механизма, как это делал О. Конт, то иерархия зубчатых колесиков будет и в самом деле полностью определять движение каждой детали, передавая ей энергию пружины. Но саморазвивающиеся ки­бернетические системы устроены иначе. Они строятся снизу вверх, путем последовательных метасистемных переходов, при которых каждый новый уровень не отменяет функций пре­дыдущего уровня, а дополнительно создает новые функции, новый вид деятельности. В хорошо сконструированной кибер­нетической системе имеет место структурно‑функциональный параллелизм:  различным структурным уровням организации соответствуют качественно различные функции, виды деятель­ности. Им соответствуют также различные понятия, в которых мы описываем функционирование уровня, так что можно гово­рить о "структурно‑функционально‑концептуальном" паралле­лизме.

Рассмотрим такой пример. На нижнем уровне организации компьютера мы находим электротехнические элементы: ферритовые сердечники, транзисторы, сопротивления и т.~п. Их функционирование описывается в терминах напряжения и си­лы тока, длительности импульсов и т.~д. Из этих элементов строятся основные узлы компьютера. Возьмем арифметическое устройство. Способ его построения и его функционирова­ние требуют для своего описания совсем других понятий: пред­ставление числа в двоичной системе, четыре действия арифметики, округление и т.~д.

Последовательность действий, выполняемых арифметическим устройством, определяется программой, хранящейся в памяти компьютера. Это еще один уровень в структуре нашей системы, и для описания функционирования этого уровня мы использу­ем еще один набор понятий: передача управления, цикл, распределение памяти, отладка программы. Связь между уровнями такова: самые крупные структурно‑функциональные единицы предыдущего уровня становятся самыми мелкими единицами следующего уровня. Программист мыслит в терминах арифметических операций, но он -- за редкими исключением -- не заботится о том, как числа представлены в машине и как производится округление. И уж совсем никогда не будет он думать о той физике, на которой в конечном счете основана работа компьютера.

В системе, которая включает в себя лишь собственно компьютер ("железки"), программа -- высший уровень управления. Но на самом деле компьютер выполняет определенную служебную функцию, являясь подсистемой более обширной системы, включающей в себя человека. Программа для компьютера может быть, например, программой расчета ядерного реактора. В таком случае она будет результатом деятельности одного или нескольких людей в области физики ядерных реакторов -- деятельности, которая характеризуется своими спе­цифическими понятиями. Это еще один уровень иерархии. На­конец, мы можем выделить еще один уровень -- систему обу­чения. Специалисты в области физики ядерных реакторов и про­граммирования не возникают сами собой -- их обучают этому, и процесс обучения также имеет свои специфические черты и понятия, помимо тех черт и понятий, которые относятся собст­венно к делу. Понятие управления в том широком смысле, в котором я его употребляю, включает в себя, в частности, и обучение.

Рассмотрим теперь несколько примеров структурно‑функ­ционального параллелизма в области социальной интеграции. Возьмем политику -- управление поведением людей. Простей­ший способ управления -- это непосредственное командование, подчинение. Часть людей выделяется из основной массы и обра­зует уровень начальников. Начальники распоряжаются своими подчиненными точно так же, как подчиненные распоряжаются частями своего тела, -- на основе волевого акта (я рассматри­ваю предельный случай командования, который имеет место, скажем, между рабом и надсмотрщиком). Таким образом, здесь нет структурно‑функционального параллелизма: один и тот же способ управления переносится с уровня отдельного че­ловека на уровень группы людей, подчиненных одному началь­нику. Этот способ организации "антикибернетичен". Напротив, общество, где управляет закон,  а не люди,  дает нам пример структурно‑функционального параллелизма. Здесь управление осуществляется в конечном счете теми людьми, которые тол­куют закон и следят за его исполнением. Функционирование этого уровня определяется не понятием воли,  а понятиями права  и обязанности.  Этот способ управления много гибче и совер­шеннее, чем командование. Он, конечно, отнюдь не является абсолютом, универсальным способом решения всех проблем управления в обществе. Тем не мен ее, правление закона -- одно из величайших изобретений человечества, большой шаг по пу­ти сочетания интеграции и свободы.

Выделив функцию наблюдения за исполнением законов из всех прочих функций (которые можно назвать исполнением законов), мы разделили общественную деятельность на два крупных класса, образующих два уровня иерархии. В рамках каждого класса есть своя тонкая структура, своя иерархия. Возьмем любую административную систему, например, си­стему государственной исполнительной власти. Здесь иерархия образована отношением командования. Начальник имеет над подчиненным власть, и управление в значительной мере сво­дится к изданию приказов и распоряжений. В отличие от рабо­владельческого общества, эти приказы касаются лишь сферы служебных обязанностей, да и в этой сфере они ограничены рамками закона. Тем не менее, это отношение командования, и отсюда проистекает ряд трудностей и недостатков, которые обычно обозначаются словом "бюрократизм". Основной способ борьбы с бюрократизмом -- это строгое соблюдение структур­но‑функционального параллелизма. На каждом уровне иерар­хии должен быть найден свой специфический способ функцио­нирования, который позволил бы эффективно управлять преды­дущим уровнем, не подменяя его. Организация управления ‑ задача чрезвычайно трудная и творческая. Она требует огром­ной концептуальной работы: нужно построить иерархию поня­тий и параллельную ей структурную иерархию так, чтобы на каждом уровне структуры можно было в терминах соответствующих понятий описать управляющие функции уровня, ука­зать его задачи. Если это не сделано должным образом, то неиз­бежна бюрократическая неразбериха, при которой каждый на­чальник то превышает полномочия, то стремится переложить ответственность на вышестоящую инстанцию, а продвижение бумаг вверх и вниз по иерархии в значительной степени опре­деляется случайными факторами.

Однако, помимо чисто делового аспекта функционирования иерархией, есть еще один аспект: человек, как известно, сущест­во несовершенное и способное к злоупотреблению предостав­ленной ему властью. Для пресечения злоупотреблений многие организации вводят дополнительный уровень в своей структу­ре, который не отдает руководящих распоряжений, а только контролирует деятельность других уровней. Такое решение, в общем, вполне в духе принципа структурно‑функционально­го параллелизма, и оно приносит свои плоды. Но как показы­вает практика, этого недостаточно. Контролирующий орган здесь слишком близок ‑ и структурно, и функционально -‑ к контролируемым органам, они фактически образуют единую систему, находят общий язык и приемлемые для обеих сто­рон способы злоупотреблений. Чтобы выйти за пределы этого порочного круга, западное общество изобрело новый метод: свободу печати. Печать и руководимое ею общественное мнение -- это метасистема  по отношению ко всем уровням всех иерархий, как исполнительных, так и судебных. Поэтому и функция ее радикально отличается от функций всех уровней. Журналисты и писатели не отдают никаких приказаний, они также не имеют никаких обязательств контролировать кого бы то ни было. Их стимулы и методы ‑ все совершенно другое. Однако они всюду проникают и все контролируют. У них нет прямой административной власти поощрять или наказывать кого бы то ни было, однако они поощряют и наказывают ‑ че­рез общественное мнение или путем предоставления информа­ции судебным органам.

В сфере производства материальных благ интеграция произ­водителей через посредство рынка  дает нам яркий пример струк­турно‑функционального параллелизма. В отличие от непосред­ственного обмена материалами и продуктами в процессе производства (на нижних уровнях структуры), обмен через рынок между независимыми производственными единицами порож­дает качественно новую функцию -- функцию купли‑продажи, которая характеризуется такими понятиями, как спрос, предло­жение, цена, себестоимость и т.~д. Товарно‑денежные отношения через посредство рынка оказались изумительным интегратором, позволившим в рамках капиталистического общества обеспе­чить неслыханные темпы развития производства -- в количест­венном и качественном отношении. Маркс справедливо под­черкивал необходимость капитализма как фазы развития об­щества. Однако выход из этой фазы и переход к следующей фазе -- социализму -- Маркс связывал с отменой товарно‑де­нежных отношений. Социалисты и коммунисты -- традицион­ные враги рынка. Почему?

Наверное, просто потому, что деньги и рынок -- неотъемле­мая черта капитализма, и борьба с капитализмом кажется неот­делимой от борьбы с рынком. Но каковы бы ни были ее причи­ны, враждебность к рынку и стремление уничтожить рынок путем государственных декретов -- типичное социалистическое варварство, результат механистического мышления и полного непонимания основных черт эволюционного процесса. Отменять рыночные отношения в пользу дорыночных отношений, то есть отношений прямого обмена в соответствии с разными планами, графиками, расписаниями и т.~п., -- это в точности то же самое, что отменять правовые отношения в пользу прямого командо­вания. У командиров это может создать -- на первых порах -- ощущение могущества и упоение властью, но это -- не путь в светлое будущее. Напротив, это регресс, это антикибернетично. Прогресс, конструктивная эволюция, достигается путем со­здания новых уровней иерархии по управлению. Это значит, что рыночные отношения надо не отменять, а управлять ими. Примером такого управления является подход Кейнса к госу­дарственному регулированию экономики с помощью финансо­вой политики. Сюда же надо отнести участие государства в дея­тельности рынка в качестве одного из производителей и потре­бителей и государственное регулирование цен; обе формы уп­равления -- при условии, что они не нарушают основного прин­ципа действия рынка. Этот подход кибернетичен,  он доказал, между прочим, свою эффективность. Функции, выполняемые государством, стоят в этом случае над  рынком, а не вместо рынка, и это качественно новые функции, требующие для своего описания новых понятий, как и положено по принципу струк­турно‑функционального параллелизма. Напротив, национализа­ция промышленности, как она произошла в Советском Союзе, означает, что государство становится вместо  рынка, оно стано­вится собственником всей системы производства и управляет им как одним гигантским предприятием. На всех структурных уровнях управления методы и функции примерно одинаковы, структурно‑функциональный параллелизм отсутствует или выра­жен в слабой степени. По существу, это механическая, а не ки­бернетическая система, она неповоротлива и неприспособленна к развитию, она так же тоталитарна, как государственная поли­тическая система. На начальных стадиях своего развития, пока иерархия по управлению еще не слишком растянута, такая си­стема может показывать хорошие результаты благодаря своей способности мобилизовать природные и человеческие ресурсы (второе, между прочим, называется эксплуатацией),  однако со временем показатели неизбежно должны падать. Это мы и на­блюдаем в действительности. Попытки улучшить функциони­рование советской экономики с помощью реформ представляют собой попытки восстановления утраченного структурно‑функ­ционального параллелизма. Сначала была резко увеличена роль денежных отношений с помощью повсеместного введения "хозрасчета". В шестидесятых годах была сделана робкая (и потому, видимо, неудачная) попытка повысить уровень экономической свободы, приблизив его к рыночному уровню. Все это лишний раз доказывает справедливость законов кибернетики.



\section{Точка соединения индивидуумов}

Итак, разрешение противоречия между интеграцией и свободой осуществляется при кибернетическом подходе к интеграции с позиций структурно‑функционального параллелизма. Как и всякое диалектическое противоречие, оно не имеет окон­чательного разрешения, а разрешается лишь в динамике -- в динамике эволюции общественных структур и функций. При этом связи между людьми становятся все теснее; люди становятся все более необходимыми друг другу и научаются все лучше понимать друг друга. В то же время "точка соединения индивидуумов", если можно так выразиться, непрерывно перемещается вверх по иерархии управления. Точка соединения индивидуумов -- это то обязательное, что ограничивает свободу человека в обществе, то, что он должен принять, чтобы общество не распалось. Это обязательное не становится с историческим прогрессом менее обязательным, но оно становится более абстрактным  и поэтому оставляет людям больше свободы. Прогресс в области общественных отношений состоит в выработке таких понятий и таких методов управления, которые позволяют создать это абстрактное и связать его с конкретными действиями индивидуумов. В современном западном обществе мы уже отчетливо различаем трехуровневую систему качественно различающихся функций в области политики (исполни­тельная власть, правовые учреждения, свободная пресса) и эко­номики (производство, рынок, государственное управление рынком). Немалый и нелегкий путь прошло человечество, пока выработало соответствующие понятия и методы.



\section{Что же дальше?}

Бросим взгляд на человечество и его историю с более уда­ленной точки. В первобытном состоянии интеграция индивиду­умов происходит примерно по тем же образцам, что и в живот­ном мире, она еще не является объектом мысли, объектом со­знательных усилий человека. Начало цивилизации -- это начало сознательной, целенаправленной социальной интеграции. Можно выделить три уровня, три механизма, с помощью которых осуществляется интеграция, иначе говоря, три точки (строго говоря, области расположения точки) соединения индивидуумов. Это:
\begin{enumerate}
 \item прямое физическое принуждение,
 \item экономическая необходимость, то есть необходимость добывания жизнеобеспечения, и
 \item свободная интеграция, направляемая духовной культурой общества. 
\end{enumerate}

Общественное устройство, где основным типом интеграции является первый способ, называется рабством;  при втором способе мы имеем дело с капитализмом;  общественное устройство, в основе которого лежит свободная интеграция на уровне духовной культуры, я называю социализмом  (как это часто бывает, одно и то же слово используется для обозначения как общественного течения, так и его идеала).



\section{Капитализм}

Эта классификация включает в рабство как рабовладельческий, так и феодальный строй. Прямое физическое принуждение работать на хозяина является основой общественного устройства в обоих случаях. В конце концов, не так уж важно, может ли хозяин убить раба или "только" запороть его до полусмерти.

Капитализм предполагает правовое государство, в котором все люди равны перед законом и никто не может заставить другого человека работать на себя путем физического принуждения. Рабочий и капиталист вступают в добровольное трудовое соглашение. И хотя со стороны рабочего эта добровольность весьма относительна, ибо он вынужден продавать свою рабо­чую силу, чтобы существовать, вряд ли надо доказывать, какой большой шаг по направлению к свободе представляет собой капитализм по сравнению с рабством.
Капитализм -- это власть капитала. Что это значит? Капитал -- это средства производства. Следовательно, капитализм -- это власть средств производства. Капиталистическое общество -- такое общество, где главной целью является производство ма­териальных благ. Это не исключает, конечно, возможности от­дельным людям и группам людей иметь свои личные и группо­вые цели, далеко выходящие за рамки производства. Но осно­вой социальной интеграции  продолжает оставаться производст­во, оно определяет способ людей жить вместе, мотивы их объе­динения, важнейшие социальные учреждения.

Капитализм становится возможным на определенной ста­дии развития производительных сил. Обозначим вместе с Марк­сом вклад средств производства (или постоянной части капи­тала) в один производственный цикл через $С$, а вклад рабочей силы (переменный капитал) -- через $V$ . Относительный вес членов $С$ и $V$ в сумме $С + V$, вкладываемой в производство, меняется по мере развития производительных сил, а именно, доля $С$ растет, а доля $V$ падает. Пока $С$ много меньше $V$, рабо­чая сила является основой производства; прямой и простейший способ организации рабочей силы -- физическое принуждение -- создает рабство. На некоторой стадии развития производства вклад $С$ становится решающим. Кроме того, коэффициент расширения производства -- отношение суммарного продукта, который мы обозначим через $Р$, к затратам $С + V$ сильно уве­личивается. В результате средства производства начинают раз­виваться как бы сами собой; труд человека, изображаемый членом V , остается хотя и необходимым, но не определяющим фактором: нанять рабочего оказывается легче, чем раздобыть средства производства, капитал. Капитал приобретает свойст­во расти экспоненциально, а рабочий становится "придатком машины".

Классический капитализм -- это такое общество, где отно­шения людей исчерпываются  их экономическими отношениями, осуществляемыми через посредство рынка (в частности, конеч­но, рынка рабочей силы). В этом обществе нет ни монополь­ных объединений производителей, ни профессиональных объе­динений рабочих. Классический капитализм был проанализи­рован Марксом, который пришел, в частности, к выводу о мак­симизации средней нормы прибыли в этих условиях Иначе го­воря, классический капитализм -- это такой строй, который при данных средствах производства обеспечивает максималь­но возможный рост производства. Это свойство капитализма является результатом конкуренции на рынке, при которой каж­дый, кто хочет уцелеть, обязан включиться в общую гонку. Доля конечного продукта Р, идущая на потребление, в част­ности -- на оплату рабочей силы, минимизируется; доля, вкла­дываемая снова в производство, максимизируется. Всякая воз­можность увеличения производительности труда изыскивает­ся и используется.

Классический капитализм -- это идеализация, предельный случай, который в чистом виде не существует и никогда не су­ществовал. Но в определенный период времени европейский капитализм был близок к нему. Затем под влиянием как эко­номических, так и внеэкономических факторов он претерпел существенную эволюцию. В частности, благодаря коллективной борьбе рабочих за повышение зарплаты уровень потребления давно превысил тот минимум, который необходим для макси­мального роста производства, и продолжает расти. Но и сейчас государства Западной Европы и Северной Америки нельзя наз­вать иначе, как капиталистическими: важнейшие черты общест­венного устройства несомненно определяются капиталистиче­ской системой производства и потребления материальных благ.



\section{Переход к социализму}

Капиталистическое общество подвергалось и подвергается интенсивной критике, в значительной мере вполне справедли­вой. Но если вдуматься в эту критику, то мы увидим, что она является критикой недостаточности капитализма,  а не обличе­нием активного зла. Во всяком случае, так обстоит дело с ра­зумной, обоснованной критикой; она не столько указывает на наличие чего‑то вредного, сколько на отсутствие чего‑то полезного, даже необходимого. Власть денег? Но деньги -- то есть средства жизнеобеспечения и производства -- не могут не иметь власти, ибо они необходимы: плохо не то, что деньги имеют власть, плохо, когда власть имеют только  деньги. По­гоня за прибылью? Но само по себе это прекрасно -- это стрем­ление к максимальному росту производства -- цель, которую так называемые социалистические страны кладут в основу го­сударственной политики. Плохо, если погоня за прибылью яв­ляется единственным  стимулом и регулятором развития эко­номики. Концентрация большой экономической власти в руках крупных собственников? Но в большой и сильно связанной экономической системе кто‑то должен обладать большой эко­номической властью, подобно тому как в большом государст­ве кто‑то неизбежно обладает большой политической властью. Плохо, если отсутствуют формы общественного контроля  над этой властью.

Переход от капитализма к социализму -- это метасистемный переход, он предполагает не разрушение предыдущего этажа системы, а постройку следующего этажа. (Нужно, впрочем, заметить, что это отнюдь не исключает какую‑то перестройку предыдущего этажа.) Революционеры, как Маркс и Энгельс, призывали разрушить капитализм до основания и построить социализм на его обломках. Конечно, под обломками не пони­мали обломков машин и домов. Разрушать средства произ­водства не предполагалось -- напротив, они должны были обра­зовать материальный фундамент будущего общества. До осно­вания  предполагалось разрушить лишь общественные отно­шения. Здесь просвечивает "материалистический" (читай: меха­нистический) интеллектуальный фон 19‑го столетия, согласно которому лишь вещи имеют подлинное бытие, обладают инер­цией и требуют энергии для их создания, а отношения -- нечто воздушное и производное. Кибернетик 20‑го века знает, что система отношений -- это реальность, которая требует для своего создания огромной творческой работы. Для кибернетика раз­рушение рынка или "буржуазного" права -- такое же варварст­во, как разрушение машин и железных дорог.



\section{Частная собственность на средства производства и трудовая теория стоимости}

Чтобы обосновать необходимость разрушения чего‑то, надо усмотреть в нем некое активное зло, коренной принципиаль­ный порок. Такой порок Маркс усмотрел в частной собствен­ности на средства производства, которая по, его утверждению, вступает в непримиримое противоречие с общественным харак­тером производства. Теоретическим обоснованием этой кон­цепции является трудовая теория стоимости, принадлежащая Адаму Смиту, но развитая Марксом и ставшая неотъемлемой частью марксизма . По Марксу, истинная стоимость товара, которую он отличает как от "потребительной стоимости", так и от рыночной цены, есть количество "общественно необходи­мого труда", затрачиваемого на изготовление этого товара. Парадоксально, что эта теория возникла и получила призна­ние в момент начала расцвета капитализма, то есть как раз тогда, когда она стала ошибочной, утратила право на сущест­вование! Ибо продукт производственного цикла пропорционален, грубо говоря, сумме $С + V$. В докапиталистическую эпоху член $С$ был мал, и поэтому можно считать, что произ­водственная стоимость пропорциональна затраченной рабочей силе V : продукт производится работой. Но в эпоху капитализ­ма именно член $С$ является решающим: машины производят продукцию! Зачем же понадобилась Марксу трудовая теория стоимости? Чтобы доказать, что капиталист грабит  рабочего. Если признать, что полный продукт пропорционален $С + V$, то есть в стоимостном выражении
\[
	Р = k(С + V),
\]

где $k$ -- коэффициент воспроизводства, то и прибавочный продукт, или прибавочная стоимость, будет пропорциональ­на $С + V$:
\[
	m=Р‑(С+V)=(k‑1)(C+V).
\]

Но тогда получается, что владелец капитала С имеет право по крайней мере на пропорциональную часть прибавочной стои­мости. Чтобы избежать этого вывода, Маркс и проводит черту между "потребительной стоимостью" и якобы "истинной" стоимостью, определяемой количеством затраченного рабочего времени. Затем он постулирует, что стоимость, содержащаяся в $С$, просто переходит один к одному в конечный продукт, а прибавочная стоимость m  пропорциональна вложенному труду $V$; коэффициент пропорциональности $m/V$ носит назва­ние нормы прибавочной стоимости. Это, конечно, чисто мета­физический постулат, не имеющий никакого реального смысла. Именно "потребительная стоимость" товара, его материальная форма, является той стоимостью, ради которой он производит­ся и которая участвует в процессе ценообразования на рынке. Прибавочная стоимость в этом смысле отражает свойство раз­вивающихся систем увеличивать со временем свою массу и про­изводить новые материальные формы. Рабочее же время, заклю­ченное в товаре, остается невидимым, когда товар поступает на рынок; оно влияет на цену лишь косвенно, через свое влия­ние на предложение. И уж конечно, нет никаких разумных ос­нований помножать рабочее время на норму прибавочной сто­имости: время, в отличие от материи, не обладает свойством самовоспроизведения. При попытке вычислить норму приба­вочной стоимости мы должны вводить в рассмотрение конечный продукт, который пропорционален не $V$, а $С + V$, из‑за чего возникает множество противоречий и нелепостей. Эти не­лепости становятся особенно очевидны, когда речь заходит о труде организатора, об изобретении новых машин, об автоматических линиях и т.~п.

В логике Маркса заслуживает внимание то, что он борется с капиталистической собственностью, опираясь на понятие соб­ственности, апеллируя к собственническому инстинкту, а отнюдь не пытаясь подняться над ним. Стать выше собственности -- это значит увидеть и объяснять другим, что собственность есть просто форма управления предметной компонентой цивилиза­ции, которая, как и всякая форма управления, может транс­формироваться постепенно. Принять такой подход -- значит стать на путь реформ: прогрессивный подоходный налог, высо­кий налог на наследство, ограничение на право распоряжаться крупной собственностью и т.~п. Но нет худшего зла для рево­люционера, чем реформы, и нет худшего ругательства, чем ре­формизм. Марксисты доказывают, что капиталист грабит  рабо­чего, то есть отнимает его собственность;  что прибавочная сто­имость, которая в капиталистическом обществе считается при­надлежащей капиталисту, на самом деле  принадлежит рабочему. Это метафизическое "на самом деле" сохраняет мистику собст­венности, опирается на нее. Вытекающий отсюда лозунг обрат­ного грабежа -- грабить награбленное или экспроприировать экспроприаторов ‑ встречает у различных слоев населения поддержку, обратно пропорциональную культуре.


\section{Эволюция государства}

Представление о движении "точки соединения индивиду­умов" вверх имеет нечто общее с учением Маркса об отмира­нии государства. И если бы я решил из политических сообра­жений называть себя марксистом, то я мог бы сказать, что это движение и есть как раз отмирание государства по Марксу, только конкретизированное в новых кибернетических терми­нах. О том времени, которое наступит после социалистической революции, Энгельс пишет:

"Вмешательство государственной власти в общественные отношения становится тогда в одной области за другой излиш­ним и само собой засыпает. На место управления лицами стано­вится управление вещами и руководство производственными процессами. Государство не "отменяется", оно отмирает ". 

Действительно, если в соответствии с марксистским учением понимать государство как исключительно и преимущественно орган насилия одних групп людей над другими, то государство должно отмереть и будет отмирать. Однако именно эта трак­товка понятия государства и вызывает у меня протест. Это -- вульгарное, демагогическое упрощение. Государство принято понимать как совокупность инструментов социальной интег­рации. Обойтись без этих инструментов общество не может. Сколь они совершенны -- другой вопрос, но объявлять инстру­менты интеграции до социалистической революции безуслов­ным злом и лишь после революции -- добром, это демагогия.
Характер государства зависит от материальной и духовной культуры общества и трансформируется вместе с трансфор­мацией культуры. Пока в обществе нет идеи права и пока не признано, что эта идея распространяется на каждое человече­ское существо, государство не может быть иным, кроме рабовладельческо ‑ феодального. Пока общество не выработало идей о том, как организовать интеграцию на основе, отличной от производства и потребления, и пока не созданы для этого необ­ходимые материальные предпосылки, государство остается капиталистическим. Эволюция государства происходит по ме­ре того, как общество в поисках и борьбе открывает новые способы интеграции, обеспечивающие индивидууму большую степень свободы. Эта эволюция, как всякая эволюция, есть усложнение,  а не упрощение и тем более не отмирание. В част­ности, самые низкоуровневые методы управления -- физическое насилие, а порой и убийство остаются в резерве государства. Прогресс состоит в том, что масштаб применения таких мер сокращается, но можно ли их избежать полностью -- этот вопрос пока остается открытым.

Отрицание государства перешло в марксизм из анархизма. Один из первых влиятельных анархистов в истории европей­ской мысли, Вильям Годвин, утверждает, что "всякое правительство есть зло: оно равносильно нашему отречению от соб­ственного суждения и совести".20 Годвин не называл себя анархистом; анархию, под которой он понимал ничем не сдер­живаемый разгул страстей, он считал переходным периодом к установлению нового порядка: "Разумеется, это ужасное испытание для народа -- дать волю своим страстям, пока со­знание последствий не придаст новых сил рассудку; но это ис­пытание тем более действенно, чем более оно ужасно".21
В основе представлений Годвина о новом порядке лежит противопоставление государства и общества. "Общество есть благо, государство в лучшем случае -- только необходимое зло", -- говорит он.   Это противопоставление проходит крас­ной нитью через все анархистские и многие социалистические учения. В сущности, оно довольно бессодержательно, так как противопоставляемые понятия очень общи, очень близки и не всеми понимаются одинаково. В основном, оно служит лишь для оценочных суждений: плохо устроенное общество называют государством, хорошо устроенное государство называют об­ществом.

В либеральном и социалистическом направлениях мысли 19‑го века управление людьми  безоговорочно считалось злом. Идеальный общественный порядок представлялся не сложной многоуровневой системой отношений, которую надо постоян­но поддерживать с помощью специальных учреждений (что есть управление людьми), а как нечто фундаментально простое, как нечто, что будет происходить само собой,  лишь бы только не было принуждения, "управления". Это представление, харак­терное для химико‑механической фоновой концепции реаль­ности, роднит анархистов Годвина и Кропоткина с социалиста­ми Марксом и Лениным (последним, впрочем, лишь до при­хода к власти!) и с либералом Спенсером. Оно было свойственно всем прогрессистам 19‑го века.
Кропоткин писал:

" Как только государство не может более навязывать свое­го союза, он возникает сам собою, в согласии с естественными потребностями. Уничтожьте государство, и на его развалинах возникнет вольная федерация, действительно единая, недели­мая, но свободная и в силу этой свободы все растущая в солидарности".[23]
Происходить это будет следующим образом:

"Народ примет временные меры, чтобы обеспечить себя пи­щей, платьем, жилищем. Народ завладеет сначала хлебными ам­барами, бойнями, складами съестных припасов. Гражданки и граждане добровольно сделают опись того, что находится в каждом магазине, в каждом амбаре: миллионы экземпляров точных списков всех товаров будут розданы всем с указанием мест, где они собраны, а также способов распределения. На­род возьмет полной горстью все, что имеется в избытке, и по­делит на строгие доли все, что должно быть размерено, пре­доставляя самую легкую пищу больным и слабым. Потреблен­ные съестные припасы будут возмещаться привозом из дере­вень, причем для крестьян следует производить полезные для них вещи и обмениваться ими; кроме того, городские жители начнут обрабатывать барские парки и окружающие луга".


\section{Разделение труда}

Нельзя читать это без улыбки. Марксисты тоже посмеивают­ся над Кропоткиным, над его наивностью и нереалистичностью предсказаний. Они утверждают, что только Маркс поставил учение об отмирании государства на "реалистическую" и "науч­ную" основу, связав его с уничтожением классов. Однако на деле учение об уничтожении классов не дает ничего нового по сравнению с картинами, рисуемыми прямодушными утописта­ми и анархистами. Если принимать его всерьез, то отсутствие классов возможно только в обществе, где нет разделения тру­да, о чем неоднократно писали основоположники марксизма. Но уничтожение разделения труда -- это меньше, чем утопия, это нелепица. В "Немецкой идеологии" мы читаем:

"Дело в том, что, как только появляется разделение труда, каждый приобретает свой определенный, исключительный круг деятельности, который ему навязывается и из которого он не может выйти: он -- охотник, рыбак или пастух, или же критический критик и должен оставаться таковым, если не хочет лишиться средств к жизни, -- тогда как в коммунисти­ческом обществе, где никто не ограничен исключительным кругом деятельности, а каждый может совершенствоваться в любой отрасли, общество регулирует все производство и именно поэтому создает для меня возможность делать сегод­ня одно, а завтра -- другое, утром охотиться, после полудня ловить рыбу, вечером заниматься скотоводством, после ужина предаваться критике -- как моей душе угодно ‑ не делая ме­ня в силу этого охотником, рыбаком, пастухом или крити­ком".25

Здесь, как и почти во всех своих конкретных примерах, Маркс ориентируется не на будущее, а на прошлое. Примеры, иллюстрирующие трудовую теорию стоимости, не выглядят абсурдными потому, что они берутся из той сферы, где роль машин и изобретательства еще не велика, а не из сферы круп­ного машинного производства, которому как раз и суждено было в ближайшем будущем определить лицо капитализма. Точно так же, в приведенном выше примере фигурируют арха­ические занятия охотника, рыболова, скотовода и комиче­ское занятие "критического критика". Если мы заменим эти занятия на более современные, то получим картину "коммуни­стического человека", который сегодня разрабатывает новый тип компьютеров, завтра на головокружительной высоте сва­ривает стальные балки, утром оперирует больного язвой же­лудка, после полудня читает лекции по квантовой механике, вечером выступает в оперном театре, а после ужина переводит с древнегреческого. Вряд ли эту картину нужно комментиро­вать.

Иногда приходится слышать, что с ростом производитель­ности труда, когда производство жизнеобеспечения, а следо­вательно и обязательный труд, будет занимать ничтожную часть времени, создадутся условия для бесклассового, бесструктур­ного общества, в котором не будет разделения труда и роль каждого индивидуума в обществе будет одинакова. Но это не­верно. Разделение труда не связано с тем, каковы цель и ре­зультат труда, а связано лишь со сложным и коллективным характером труда. Мы видим это на примере научной работы. Не производство предметов потребления является ее целью, а между тем разделение труда в науке существует и все увели­чивается. Интеграция и специализация неразделимы: это две стороны одного и того же движения. Так было при образова­нии многоклеточных организмов, так есть и будет при объе­динении людей в общество. До тех пор, пока общество будет существовать как целое, до тех пор будет существовать и раз­деление труда, и иерархическая система управления.

Есть две разновидности эгалитаризма: эгалитаризм права и эгалитаризм доли (участия). Первая разновидность утверж­дает, что все люди имеют от рождения равное право на общест­венное достояние и на участие во всех сторонах общественной жизни. Такой эгалитаризм полностью оправдан с точки зрения эволюционизма. Ибо любые формы неравенства, накладывае­мые на человека по соображениям, не связанным с его кон­кретной личностью, сокращают творческий потенциал общест­ва. Отрицательное отношение ко всем привилегиям , получае­мым по наследству, в частности к наследованию капитала, -- всегда было и будет характерной чертой социализма. Но из эгалитаризма права вовсе не следует, что реальное участие каждого человека в общественной жизни и его доля в распре­делении результатов общественной деятельности должны быть одинаковыми, как этого требует вторая разновидность эгали­таризма. Разумеется, общество должно стремиться обеспечить наилучшие условия существования и наиболее полно удов­летворить всех своих членов, но эгалитаризм доли как принцип не имеет никакого разумного оправдания. Это принцип дейст­вует на общество разрушающе, и провести его в жизнь можно только, разрушив общество.



\section{Социальная интеграция в СССР и в западных странах}

Если собственность понимать не как какую‑то мистическую связь между человеком и вещью, а как высший уровень в ие­рархии управления вещью, то собственность будет существо­вать до тех пор, пока будет существовать разделение труда, и это, между прочим, подчеркивал не кто иной, как Маркс. На­ционализация есть не уничтожение собственности, а передача ее в другие руки. Это не отрицается и советской официальной идеологий: считается, что частная собственность  на средства производства заменена у нас на общественную.  Но общественной собственностью можно считать только такие вещи, кото­рыми каждый член общества распоряжается по своему усмот­рению: например, воздух. При наличии иерархии управления вещами они являются собственностью организации, осущест­вляющей управление, точнее -- собственностью высшего уров­ня иерархии в этой организации. С экономической точки зре­ния, советский общественный строй является государственным капитализмом. Государство в лице его верховных властите­лей является собственником огромной системы промышлен­ного и сельскохозяйственного производства. Фактически, это единственный собственник в стране, кроме него лишь кре­стьяне обладают крохотными приусадебными участками (кол­хозная собственность -- не в счет; она формально считается общинной, но на деле принадлежит государству). С населе­нием государство‑производитель вступает в рыночные отно­шения: оно покупает рабочую силу и продает продукты произ­водства. Оно стремится к максимальному расширению произ­водства и вообще ведет себя как заправский капиталист. Так как оно обладает почти абсолютной монополией (конкурируют с ним лишь продавцы на рынке и лишь в сфере продуктов пи­тания) и абсолютной монополией на рынке рабочей силы (толь­ко оно может нанимать людей на работу), государство само устанавливает цены на товары и заработную плату. В США су­ществуют специальные антитрестовские законы, чтобы предот­вратить захват рынка монополиями и навязывание потребите­лю завышенной цены; существует право на забастовки, чтобы предотвратить навязывание заниженной заработной платы. В Советском Союзе государство‑капиталист находится в таких условиях, о которых западные капиталисты могут только мечтать.

(Впрочем, тепличные условия оказывают развращающее действие. Избавленное от необходимости конкурировать на рынке, государство‑производитель и его каждая структурная компонента лишаются необходимости энергично использовать все возможности для повышения экономической эффективности. В первую очередь это относится к внедрению новой техники и повышению качества продукции -- эти вечные заботы руководителей советской промышленности. Кроме того, государство может позволить себе роскошь продвигать людей вверх не по деловым качествам, а по "преданности делу партии".)

Экономическая жизнь не исчерпывает собою общественную жизнь, и экономический уклад не равнозначен общественному строю. Государственный капитализм -- существенная черта советского образа жизни, но основным инструментом социальной интеграции в СССР является коммунистическая партия, то есть явление внеэкономическое. Партийная, а не производственная иерархия играет ведущую роль. При всей важности экономического развития единство партии и единство вокруг партии играют еще более важную -- решающую -- роль. И ког­да под вопрос ставится это единство, руководители, не заду­мываясь, жертву ют экономикой.

Далее, власть партийной иерархии осуществляется не путем голого, прямого принуждения, осознаваемого всеми как тако­вое (что имеет место в рабовладельческом обществе), а путем внедрения через пропагандистскую машину нужного руко­водству образа мышления. Формально советское общество является правовым и даже до известной степени демократи­ческим (выборы в Советы депутатов, выборы в партийные органы), и лишь благодаря особому способу мышления совет­ского человека эти формальные установления не выливаются в действительную свободу и демократию. Таким образом, ме­ханизм социальной интеграции не исчерпывается ни физиче­ским насилием, ни экономическим принуждением, а существен­но опирается на сферу культуры. Идеология коммунистиче­ской партии лишена всяких следов метафизической трансцен­дентности, интеграция понимается в наиболее земном и пря­мом из всех возможных смыслов: морально политическое единство, осуществление воли партии и т.~д. Все эти признаки дают основание считать советскую систему формой социализ­ма -- тоталитарной, то есть уродливой формой, но все же фор­мой социализма.

В буржуазно‑демократическом государстве органы власти формируются также внеэкономическими образованиями -- политическими партиями, и часто правительство является, как и в Советском Союзе, однопартийным. Однако взаимоотношения между системой производства и политическими партия­ми здесь радикально отличаются от того, что мы имеем в СССР, они, можно сказать, противоположны. Идеи, вокруг которых граждане в капиталистических странах объединяются в партии, являются их личным делом; эти идеи, а вместе с ними и группи­ровка по партиям могут претерпевать быстрые изменения. Основой общественной стабильности является частная собст­венность на средства производства, которая признается подав­ляющим большинством населения как краеугольный камень государства и охраняется законом. В СССР, напротив, именно партийная структура и партийная идеология являются основой стабильности, а управление производством может претерпе­вать резкие изменения -- как в смысле своей структуры , так и состава лиц, его осуществляющих. Руководящая роль комму­нистической партии закрепляется Конституцией СССР.

Структура и власть партийной иерархии и навязывание граж­данам партийной идеологии осуществляются в Советском Союзе не только внеидеологическими, но и внеэкономическими мето­дами, поэтому, по сравнению с буржуазно‑демократическим обществом, где свободной борьбе идей мешают лишь экономи­ческие факторы, советская система является шагом назад, а точнее -- назад и в сторону ‑ в эволюционный тупик. Она сбли­жается с древними общественными системами, где государст­во и религия поддерживали себя и друг друга ничем не огра­ниченным насилием. Эти соображения, в противовес высказан­ным выше, дают веские основания, чтобы не называть совет­ский строй социалистическим. Трудное дело -- терминология. Вряд ли Маркс или Энгельс согласились бы признать социали­стическим общество, столь лишенное элементарных граждан­ских прав личности, как советское. В теории социализм всегда связывался с формулой "интеграция плюс свобода". Но на де­ле социалистические движения вращались главным образом вокруг идеи интеграции -- до такой степени, что обобществле­ние средств производства стало считаться чуть ли не сущностью, определением социализма. Именно в идее интеграции была и есть специфика социалистов. Что же касается свободы, то уже довольно давно в Европе все -- за свободу (тоже в теории, разумеется). Поэтому естественно связывать понятие социализма с социальной интеграцией вообще, что я и сделал в нача­ле этой части книги. Это имеет и то основание, что весь мир ‑и социалисты, и не социалисты -- называет Советский Союз социалистической страной. Но тогда, как называть тот социа­лизм, который изображается формулой "интеграция плюс сво­бода" и который для многих, в том числе и для меня, продол­жает оставаться общественным идеалом? Невозможно постоян­но носить за собой прилагательное "нетоталитарный" (это да­же как‑то унизительно: нельзя же, например, требовать от че­ловека, чтобы он постоянно повторял, что он не вор) или та­кие претенциозные словечки, как "истинный", "подлинный" и т.~п. Поэтому остается только называть его просто социализ­мом. А тоталитарный социализм мы можем с полным правом называть просто тоталитаризмом. Но нельзя забывать, что тоталитаризм содержит в себе социалистический элемент, и именно в этом его сила. Тоталитаризм -- не нелепость, не игра случая, а нечто гораздо более опасное -- извращение. Опасность тоталитаризма в том, что он похож на социализм.  Вещества, совершенно чуждые живому организму, непохожие ни на одно из веществ, участвующих в физиологической активности, как правило, выводятся из организма, не причиняя ему большого вреда. Но когда вещество похоже на какое‑то важное для фи­зиологии соединение, но все же не тождественно ему, оно "об­манывает" организм и зачастую оказывается сильнейшим ядом. Так обстоит дело и с тоталитаризмом в мировом общественном "сверхсуществе".

Из того, что конечная цель социальной интеграции теряется в тумане будущего, следует, что "подлинный" (да простит мне читатель употребление этого эпитета в некоторых случаях) социализм представляет собой бесконечный процесс, а не какую‑то определенную черту общественного устройства, которую можно осуществить и объявить раз и навсегда, что "социализм построен". То или иное общество может в большей или мень­шей степени считаться социалистическим. Вс'е же мы можем провести черту, условно отделяющую капитализм от социализ­ма, подобно тому как мы можем отличать человека образо­ванного от необразованного, хотя движение по пути образо­вания непрерывно и бесконечно. Преобразования, проведенные за последние десятилетия в западных капиталистических стра­нах, -- государственное регулирование экономики, высокие по­доходные налоги, непрерывный рост социального обеспечения и т.~п. -- несомненно представляют собой крупные достижения на пути к социализму. Но до черты, отделяющей капитализм от социализма, еще, я полагаю, далеко. Ибо все эти реформы за­трагивают лишь сферу производства и распределения матери­альных благ. А социализм ‑ это явление культуры, это религия и способ жить в соответствии с этой религией. Если предста­вить себе, что завтра в какой‑то стране будет создана автомати­ческая система производства, которая будет сама, без всякого участия людей  производить для них все материальные блага и распределять их между людьми строго поровну (пусть даже "по потребности", пусть даже с десятикратным избытком!), то общество от этого не станет социалистическим. Напротив, можно подозревать, что оно начнет вырождаться и разложится. Или же станет тоталитарным. Ибо других механизмов внеэконо­мической интеграции, кроме тоталитарного, современное чело­вечество еще не знает. Парламентская демократия в капитали­стическом государстве поддерживает известную долю единства, вследствие необходимости сохранить экономику от развала. Что будет, если эта необходимость отпадет?..



\section{Валентные и массовые связи}
[26]

Связи между людьми в обществе -- это всегда какое‑то воздействие, влияние людей друг на друга. Движение к социализму есть смещение центра тяжести этих влияний и воздействий вверх -- от насилия, через экономическую необходимость, в сферу духовной культуры, а в самой этой сфере -- от простого подражания и обмена информацией на эмпирическом уровне ко все более тесному интеллектуальному и эмоциональному контакту, который является необходимым условием прогресса культуры.

Тесный интеллектуальный и эмоциональный контакт между людьми требует усилий и времени. Число связей такого типа, в которые может одновременно вступать человек, ограничено. Я буду называть эти связи валентными --  по аналогии со связями между атомами, которые образуются вследствие обмена электронами и число которых поэтому ограничено. Валентные связи (сокращенно: $\nu$ ‑  связи) между людьми -- связи личные, индивидуализированные. Это отношения, предполагающие тес­ный духовный контакт, взаимопонимание и взаимное доверие, обмен самыми сложными и, быть может, неясными мыслями и чувствами. Любовь, дружба, творческое содружество -- при­меры таких связей. Это самый человеческий, быть может, даже единственно человеческий в полном смысле слова способ обще­ния, при котором собственно человеческое раскрывается в чело­веке во всей полноте и без которого оно раскрыться не может. Валентная связь -- существенно двусторонняя связь. Вместе с тем, один из членов такой связи может быть ведущим, а другой -- ведомым, что иллюстрируется, например, отношениями между родителями и детьми.

Максимальное и оптимальное число всех валентных связей, в которые может вступать человек (так сказать, его "валент­ность"), не есть строго определенная величина. Тем не менее, я буду обозначать ее через $\nu$,  помня, что она определена, как го­ворят физики, с точностью до порядка. А именно, $\nu = 10$. Трудно представить себе человека, у которого число друзей исчисляется сотнями. С другой стороны, и человек, который спо­собен дружить не более, чем с одним человеком, представляет скорее исключение, чем правило.

Связи, которые не являются валентными, я назову массовы­ми связями  (сокращенно: т ‑  связями). Число таких связей у одного человека может быть очень велико, а благодаря изоб­ретению средств массовой коммуникации оно становится принципиально неограниченным. Отношения, в которые чело­век вступает с прохожими на улице или со случайными попут­чиками в автобусе, -- пример m‑связи. Президент Соединенных Штатов Америки находится в т ‑  связи со всеми гражданами, которые смотрят по телевизору его выступление. Священник находится в массовой связи со всеми прихожанами, которые слушают его проповеди и время от времени приходят испове­доваться, а милиционер ‑ с жителями своего участка, которых он иногда отвозит в вытрезвитель или арестовывает. Поверх­ностное, "шапочное" знакомство -- тоже массовая связь. Поэтому я вынужден вводить новое словосочетание "валентная связь", а не просто говорить о "личной" связи. Когда пользуют­ся выражением личная связь,  хотят противопоставить ее об­щественной связи,  то есть связи, формализованной в каких‑то общественных структурах, но ничего не говорят о характере взаимоотношений между людьми.

Человек обладает врожденной потребностью в валентных связях: она досталась ему в наследство от общественной ор­ганизации его животных предков, которая основывалась на индивидуализированных связях между особями. Этот способ организации свойственен не только предкам человека, но и мно­гим другим позвоночным, как мы узнаем из замечательной книги одного из крупнейших современных специалистов по по­ведению животных Конрада Лоренца. [27] Это наиболее сложный способ социальной организации в животном мире. Кроме организации, основанной на индивидуализированных связях, Конрад Лоренц описывает еще три типа сообщества животных.

\paragraph{1. Самый примитивный тип общества -- это анонимная стая или стадо.}  Мы находим его, например, у сельди, у леммингов, у многих копытных. Животные в стаде испытывают друг к другу взаимное притяжение, распознавая представителей своего вида по общим для них признакам и не делая никаких разли­чий между отдельными индивидуумами. Сила притяжения со стороны стада, действующая на одно животное или неболь­шую группу, "чудовищно велика и возрастает с размером стада, возможно -- в геометрической прогрессии".[28]

\paragraph{2. Второй тип общества обладает, в отличие от стада, неко­торой структурой.} Здесь силы связывают конкретную пару особей, а именно самца и самку, производящих и воспитываю­щих общее потомство. Но силы эти определяются исключитель­но распределением особей по территории; сила связывает не особь $X$  с особью $Y$, а особь, живущую в месте $А$, с особью другого пола, живущую в том же месте. Поэтому, когда по каким‑либо причинам территориальное распределение меняет­ся, связи безболезненно разрываются и заменяются новыми. Так живут, например, ящерицы и аисты. В книге Лоренца описы­вается следующий эпизод из семейной жизни аистов, свидете­лем которого был проф. Э.~Шюц. Весной аисты‑самцы прилетают раньше самок. Одной весной аист, живший на крыше дома профессора Шюца, прилетел, как обычно, на старое место. Спустя некоторое время, появилась самка, но не хозяйка гнезда, а другая (хозяева были окольцованы). Самец приветствовал ее всеми знаками внимания, причитающимися супруге, после чего они стали возиться с гнездом. Через два дня прилетела прежняя хозяйка гнезда и вступила в отчаянную территориальную схватку с новой хозяйкой. Самец наблюдал за битвой равнодушно, не делая попыток помочь какой‑либо из сторон. Победила старая хозяйка, после чего самец принялся как ни в чем не бывало продолжать ремонт гнезда. Такого рода связи можно назвать "производственными": особи выступают здесь как исполнители определенной функции и поэтому являются взаимозаменяемыми.

\paragraph{3. Третий способ общественной организации -- жить клана­ми  или большими семьями.} Из млекопитающих так живут, например, крысы. К этому же типу относится жизнь обществен­ных насекомых, как например, пчел. Число особей в клане слишком велико, чтобы его члены могли распознавать друг друга индивидуально. В большинстве случаев принадлежность к клану распознается по запаху. Кланы ведут друг с другом свирепые войны. Животных, попавших в чужой клан, обычно убивают.

\paragraph{4. Четвертый и наивысший тип общества -- это общество, основанное на индивидуализированных связях.} Каждая такая связь представляет собой самостоятельное явление: она связывает конкретную особь $Х$  с конкретной особью $Y$  (хотя, конечно, механизм возникновения связи един). Особи $Х$  и $Y$  могут быть как разных полов, так и одного пола. Лоренц называет такие отношения между животными дружбой -- без всяких кавычек, ибо смысл и механизм этих явлений у человека таков же, как и у высших животных; совокупность особей, связан­ных между собой попарными отношениями дружбы, он называет группой.  Зачатки индивидуализированных связей встречаются у некоторых рыб; шире они распространены среди птиц (галки, гуси) и еще шире -- среди млекопитающих (классический пример -- волки). Лоренц подробно и с большой теплотой описывает отношения дружбы между дикими серыми гусями.

Валентные связи между людьми не исчерпываются, конеч­но, теми отношениями дружбы, которые доступны и животным, но эти отношения входят как важная биологическая компонен­та. Культура вносит новые элементы в валентные связи, при­дает им дополнительные измерения. Для кибернетика здесь ва­жен следующий факт: когда мозг животного развивается в до­статочной степени, чтобы уверенно отличать друг от друга много различных особей своего вида, возникает и становится все бо­лее распространенным новый тип общественной организации -- основанный на индивидуализированных связях. Индивидуали­зированные связи, подобно валентным связям между атомами, дают возможность строить сколь угодно сложные структуры, не сводящиеся к простой куче, стаду. Эти структуры могут быть одновременно сильно связанными и подвижными, способ­ными к трансформации. В этом -- отличие индивидуализирован­ных V‑связей  от массовых и смысл их появления в процессе эволюции жизни.


\section{Модель пересекающихся $V$‑иерархий}

Президент США избирается путем всеобщего прямого голо­сования. Его выступления может смотреть и слушать каждый гражданин, и принятые им решения непосредственно влияют на каждого. Это -- массовая связь; на связях такого рода осно­вано современное демократическое государство, объединяющее миллионы граждан. Валентные связи ограничены сверху числом $n$.  Можно ли на основе валентных связей построить общест­венную систему, представляющую из себя единое целое и вклю­чающую миллионы индивидуумов? Можно, но необходимым условием является наличие в такой системе четкой многоуров­невой иерархии.

Я предлагаю читателю рассмотреть следующую схему или модель принятия решений и управления обществом, основан­ную на $V$‑связях.

В жизни человека и человеческого общества можно выделить несколько важных аспектов: производственный, бытовой, научно‑исследовательский, политический и др. Каждому аспекту соответствует иерархическая система валентных связей, с по­мощью которой общество принимает решения, связанные с этим аспектом и проводит принятые решения в жизнь. Будем назы­вать их $V$‑иерархиями. Общее число выбранных аспектов, а сле­довательно  $V$‑иерархий, не должно быть велико: вероятно, в пределах от 2 до 8.

Образование каждой $V$‑иерархии происходит следующим образом. Вначале все граждане объединяются в -- и, следова­тельно, разбиваются на -- группы по $V$  человек. Здесь $V$  су­щественно меньше v и  заключено, вероятно, в пределах от 4 до 8. Предполагается, что в группе устанавливаются -- в резуль­тате активных контактов -- валентные связи между каждой па­рой членов. Образование группы должно происходить на строго добровольных началах; процесс этот, конечно, не простой, и займет он немало времени. Когда образование групп закончено, каждая группа -- будем называть эти группы группами основ­ного или нулевого уровня -- выделяет одного представителя. Это будут представители первого уровня. Они объединяются в группы первого уровня, опять‑таки основанные на V‑  связях между членами. Представитель первого уровня продолжает оставаться членом своей группы основного уровня и участво­вать в ее жизни. Таким образом, он будет участвовать в жизни двух групп одновременно. Группы первого уровня выбирают представителей второго уровня и т.~д., пока на вершине пира­миды не оказывается один представитель.

Иерархия не остается фиксированной раз и навсегда. В любой момент (или раз в течение определенного времени -‑ этот, как и многие другие вопросы, я оставляю без конкретизации) группа $i$‑го уровня может отозвать своего представителя на $i+1$‑ом уровне и заменить его другим (но отозвать своего представителя, поднявшегося на $i+2$‑ой уровень, без согласия соответствующей группы $i+1$‑ого уровня она не может). Так как принять решение, что человек чего‑то недостоин, требует гораздо меньше времени, чем принять решение, что он достоин, представитель высокого уровня, совершивший единодушно порицаемое деяние, может в мгновение ока скатиться до само­го низа иерархии. Далее, контакты между членами разных групп (которые не запрещаются и значит будут иметь место) могут приводить к изменению состава групп, что также может влиять на движение вверх и вниз по иерархии. Массовые связи тоже, разумеется, сохраняются и будут играть роль: человек, написавший хорошо читаемую книгу, воздействует на все об­щество в целом (и надо думать, на свое продвижение по одной из $V$‑иерархий).

Каждая группа представителей принимает решения, относя­щиеся ко всем представителям и рядовым членам групп, участ­вовавшим в выборе этой группы. Деятельность $V$‑иерархии, и в частности взаимоотношения между уровнями, должны быть основаны на принципе структурно‑функционального паралле­лизма. Я не могу конкретизировать эту мысль, так как строю чисто абстрактную схему, но я допускаю, например, что в политической $V$‑иерархии (или одной из политических иерархий, ибо их может быть несколько, в соответствии с различными ас­пектами политической жизни) применение физического наси­лия может разрешаться только со стороны членов группы, к которой принадлежит данный индивидуум, включая представи­теля.

Сделаем несколько несложных математических расчетов, чтобы убедиться, что наша схема не вступает в противоречие с реальными цифрами, характеризующими сегодняшнего чело­века и человеческое общество.

Обозначим через $N$  число членов общества. Высота иерархии, которую обозначим через $h$,  будет связана с $N$ и $V$  соотноше­нием
\[
	V^h=N
\]
или 
\[
	h = \log N / \log V. 
\]
Положим $V=5$, $N=250$ млн. Тогда $h=12$.  Допустим, что образование группы на каждом уровне будет занимать полго­да (с учетом времени, необходимого для установления долго­временных валентных связей). Тогда формирование всей $V$‑ие­рархии будет выполнено за 6 лет.

Займемся теперь вопросом о полном числе валентных связей, в которые должен будет вступать член нашего гипотетического общества. Член каждой иерархии вступает в $V-1$  связей со свои­ми товарищами по группе основного уровня. Если его избирают представителем первого уровня, то число его связей будет $2(V‑1)$.  Если он поднимается до второго уровня, -‑ $3(V‑1)$  и т.~д.

Глава иерархии должен иметь $h(V‑1)$  связей (столько же, сколько на уровне $h ‑ 1$, ибо ему уже не с кем контактировать по горизонтали). В нашем примере $h (V ‑ 1) = 48$.

Итак, чем выше положение гражданина в некоторой $V$‑иерар­хии, тем больше он должен иметь валентных связей. Это, во‑первых, предъявляет к высокопоставленным представителям более высокие требования, а во‑вторых, препятствует продви­жению по другим V‑  иерархиям, что является здоровой чертой нашей схемы. Я называю эту схему моделью пересекающихся V‑  иерархий, потому что каждый член общества участвует од­новременно в нескольких системах отношений -- иерархиях, причем никакого параллелизма в структуре этих систем не предполагается; напротив, будет иметь место противоположная тенденция: лица, имеющие больше контактов и обязанностей в одной иерархии, будут иметь их меньше в других иерархиях. Таким образом, будет выравниваться число валентных связей у различных лиц.

Нетрудно подсчитать, что среднее число связей у одного гражданина в одной $V$‑иерархии есть $V$ (в предположении, что $N$ много больше единицы).[29] Оно лишь немного превышает мини­мальное число связей, потому что число представителей убывает с высотой в геометрической прогрессии. Обозначим через r  число иерархий. При $r= 30$  от среднего гражданина потребуется 15 валент­ных связей. Разумно было бы организовать столько $V$‑иерар­хий, чтобы среднее полное число связей было равно максималь­ному числу валентных связей, в которое может вступать средний человек; мы обозначили выше это число через $n$.  Таким обра­зом, мы получаем следующее соотношение между величинами $r$, $V$ и $n$: 
\[
	rV=n.
\]
Как мы видим, это соотношение удовлетворяется при вполне правдоподобных значениях всех величин.

Сравнение двух схем принятия решений

Модель пересекающихся $V$‑иерархий -- это не проект устройст­ва общества, а лишь схема принятия решений, и отчасти -- в той степени, в которой принятие и выполнение решений неотделимы -- их выполнения. Моя цель -- противопоставить эту мо­дель другой схеме принятия решений, которая сейчас повсеместно используется и которую многие считают единственно допустимой в демократическом обществе: всеобщему прямому и равному голосованию. Какую бы из этих двух схем мы ни приняли (или их комбинацию, или третью схему), мы еще останемся очень далеки от определения действительного общественного строя. Договориться решать все проблемы голосованием -- это отнюдь не значит решить все проблемы. То же относится и к нашей мо­дели.

Принятие решений путем создания $V$‑иерархии -- более слож­ная и длительная процедура, чем прямое голосование. Зато со­зданная V‑  иерархия дает постоянно действующий механизм, ко­торый не просто принимает решения, а вырабатывает  их. Хорошо голосовать, когда твердо знаешь, за что голосовать, то есть когда твое решение уже готово. А если решения еще нет? Если его еще надо выработать? Каждому знакома ситуация, когда сложный вопрос приходится решать на большом собрании: возникают бесконечные, бестолковые споры, которые мешают ухватить суть дела, и в результате сколько‑нибудь разумное решение най­ти не удается. Только в небольшой группе людей может проис­ходить работа, необходимая для решения сложных задач. А что­бы эта работа была по‑настоящему эффективной и творческой, надо чтобы члены группы были связаны валентными связями. Это условие особенно важно при решении общественных проб­лем, которые в большинстве неформализуемы и тесно связаны с личными интересами и с эмоциями.

Голосование возникло, вероятно, как замена сражения, и от­печаток этого происхождения оно будет нести на себе вечно. Голосование -- инструмент борьбы, валентная иерархия -- инст­румент работы. В подчинении меньшинства большинству нет ни­какой справедливости, это лишь простейший способ избежать кровопролития. Принуждение остается принуждением, даже ес­ли оно является результатом голосования. По мере того, как в обществе западного типа принуждение рассматривается все более нежелательным, растут и трудности во взаимоотношении большинства с разного рода меньшинствами, прежде всего, национальными. Голосование не решает никаких проблем. Решение конфликтной проблемы требует работы: анализа ее причин и по­строения метасистемы, объединяющей конфликтующие системы и снимающей конфликт. Процесс создания $V$‑иерархии заставля­ет людей выполнить эту работу, подсказывает пути к решению проблемы.

С точки зрения формальной организации, в модели пересека­ющихся иерархий нет ничего нового; это кооперация или самоор­ганизация граждан на добровольных началах с вытекающим от­сюда самоуправлением. Я усматриваю нечто новое в соединении этой организационной формы с условием, чтобы все связи в ней были валентными, откуда вытекает ограничение на число лиц, выбирающих одного представителя. Валентные связи и в совре­менной общественной организации играют положительную роль:
если бы не личные связи, основанные на дружбе, взаимопонима­нии и доверии, то, наверное, ни одна большая организация не мог­ла бы эффективно работать. Рост культуры общества -- это преж­де всего улучшение условий для возникновения валентных свя­зей и увеличение их роли. Но не они до сих пор образуют струк­турный костяк государства. Единственная структурная единица, основанная на валентных связях, -- это семья, и она, вне всяких сомнений, показала свою важность для общества. Да и то, с фор­мальной точки зрения, семья основана на договорных отношениях между мужем и женой и родственных отношениях между осталь­ными парами индивидуумов; лишь тот факт, что эти отношения, как правило, являются следствием (в первом случае) или причи­ной (во втором случае) валентных связей, дает семье ее силу. Когда от семьи остается одна форма, она становится не благодея­нием, а тяжелым бременем. Это отделение формального  (или официального:  между прочим, русское "официальный" перево­дится на английский как "формальный") от личного  настолько вошло в плоть и кровь, что многие полагают, будто это концеп­туальное противоречие, антиномия. В действительности же ника­кого непреодолимого противоречия здесь нет: просто организа­ционная форма должна быть достаточно гибкой, чтобы следо­вать за личными отношениями. (Мое <<просто>> --это, конечно, грамматическая связка, а не оценка сложности задачи. В дейст­вительности разработка таких организационных форм -- задача чрезвычайно сложная; хочу снова подчеркнуть, что модель пересекающихся $V$‑иерархий -- лишь абстрактный набросок, этюд, направление поиска.)



\section{В защиту иерархии}

На пути к подлинному соединению общественного и личного в современных демократических странах есть, как мне кажется, одно препятствие. Я называю его квазидемократическим преду­беждением против иерархии. Считается, что хороший демократ должен быть врагом всякой иерархии. Понятие иерархии отож­дествляется с принуждением, подавлением.

Исторически это вполне понятно. С древних времен иерар­хии создавались именно с этой целью ‑ с целью управления си­лой. Борьба за свободу и демократию принимала форму борьбы против иерархии власти. Но объяснение предрассудка не есть его оправдание. Человек 20‑го века должен понять ту простую мысль, что не наличие иерархии (то есть, говоря языком математики, отношения частичного порядка между людьми) является злом, а те принципы, которые человек использует при управлении че­ловеком. Толпа, лишенная всякой иерархии, способна на самые худшие виды принуждения и насилия над личностью.

Понятие иерархии отнюдь не включает в себя угнетения или по­давления. Для кибернетика слово "иерархия" несет положитель­ную эмоциональную нагрузку. Он всюду пытается обнаружить или установить иерархию. Ибо иерархия ‑ это организация, это структура. Большие системы не могут быть организованы иначе, как иерархически. Антитеза иерархии -- это хаос, беспорядочная куча, толпа, а не свобода. Даже когда вы ищете в словаре нужное вам слово, вы пользуетесь плодами применения иерархического принципа: первая буква слова рассматривается как старшая по отношению ко второй, а вторая -- по отношению к третьей и т.~д.; буква А старше, чем Б, а Б старше, чем В и т.~д. Здесь ни одна буква не "подавляет" и не "угнетает" другую, но если бы не ие­рархия, вам каждый раз пришлось бы просматривать все слова в словаре подряд!

Так как величина $V$  лежит в пределах первого десятка, высо­та $V$‑иерархии в обществе с миллионами членов получается до­вольно значительной. Тут и вступают в игру квазидемократические предрассудки. Человек хочет "сам", а не через посредство иерархической цепочки, участвовать в управлении государством. Но ему все‑таки приходится кому‑то передоверять свои полно­мочия и создавать какую‑то иерархию управления. Считая иерар­хию злом, он стремится сделать ее пониже и выступает за пря­мые выборы на все уровни иерархии, чтобы иметь хотя бы то утешение, что он сам, лично,  передает свои полномочия. Так и возникают иерархии, основанные на массовых связях (т ‑  иерархии), которые совмещают в себе недостатки как иерархии, так и толпы, человеческого стада.

"Личная" передача полномочий в прямых выборах с тысячами и миллионами участников и, следовательно, как бы "личное уча­стие в управлении-- это, конечно, иллюзия. Такой способ управле­ния ведет именно к ликвидации личного участия, личного влия­ния. Как и всякое бесструктурное, массовое явление, этот спо­соб порождает специфический вид детерминизма, основанный на законе больших чисел. Реакция масс избирателей на тот или иной ход в предвыборной кампании может предсказываться с большой степенью достоверности, так как личность каждо­го избирателя теряет значение. Реальное влияние на события личность может оказать только через посредство какой‑либо структуры, иерархии: например, партийной иерархии или иерар­хии средств массовой коммуникации. И вот, тот факт, что связи в этих иерархиях суть т ‑  связи, а не V‑  связи, создает явление, известное как отчуждение  личности от общества. Ибо связь -- слишком слабая, неполноценная, безличная связь. Современное демократическое общество еще слишком напоминает аноним­ное стадо.  То, что ему требуется, ‑ это не борьба с иерархиями, а, напротив, внедрение и усовершенствование их.

Интересно, что высота иерархии в современных системах уп­равления государством все‑таки оказывается большой, так что для распространения влияния снизу доверху приходится пройти большое число $m$‑связей! Спрашивается, чем же такая система лучше, чем система, целиком построенная на $V$‑связях? Инте­ресно также, что отношение числа лиц на определенном уровне к числу лиц на следующем уровне бывает особенно велико на нижних уровнях и уменьшается при движении вверх по иерар­хии. Это создает возможность для перерастания т ‑  связей в V‑  связи на верхних уровнях. Появляется корпоративная соли­дарность верхушки, разрыв же между верхушкой и нижним уровнем не уменьшается. Это, конечно, отнюдь не способствует преодолению отчуждения.

Механизм $V$‑иерархии обеспечивает каждому гражданину наилучшие возможности влияния на общество в целом, то есть управления обществом. Во‑первых, потому, что всякая новая, творческая мысль требует обсуждения в обстановке валентных отношений: иначе ей не получить признания. Валентная иерархия служит фильтром для выделения стоящих идей. Во‑вторых, $V$‑иерархия -- наиболее справедливый  способ до­ступа к влиянию. Действительно, в обществе, где валентные связи не формализованы, а формальные -- не валентны, реальный путь к влиянию лежит через цепочки неформальных валентных связей (цепочки "знакомств"). Эти связи существуют как бы в подполье, и очень часто они образуют замкнутые фигуры, объе­диняя некий круг лиц, доступ в который со стороны невозможен или чрезвычайно затруднен. Иерархия же связывает каждого гражданина с вершиной и с любым другим гражданином.

Отсутствие замкнутых фигур в иерархии имеет значение в свя­зи с еще одним важным аспектом жизни общества: преступно­сти. Наличие кругов, изолированных (по крайней мере в смыс­ле валентных связей) от остальной части общества, в том числе от правящих и вообще престижных кругов -- постоянный источ­ник преступности. Быть может, единственный способ существен­но снизить преступность в многомиллионной стране с разнород­ными социальными, культурными и этническими элементами -- это организовать всеохватывающую V‑  иерархию.

Чтобы покончить с моделью пересекающихся $V$‑иерархий, скажем несколько слов о движении информации в направлении сверху вниз, то есть об управлении в узком смысле слова. Ясно, что $V$‑иерархия позволяет создать высокую степень единства. Валентная связь -- сильная и гибкая связь, позволяющая соче­тать интеграцию и свободу. Но осуществление этой возможности зависит не только от выбора той или иной организационной схемы, но от культуры общества в целом. Прежде всего оно зависит от того слоя культуры, который определяет высшие цели индивидуума. Общество, которое не ставит своей целью социальную интеграцию, не будет и двигаться к социализму. И единства в нем не будет.


\section{Последний раз о терминах}

Мое понимание социализма, изложенное выше, во многих отношениях необычно. Возможно, что какие‑то люди, выступаю­щие за социализм, найдут мое понимание этого термина совер­шенно для них неприемлемым и объявят, что "это не социализм". Возможно, что другие люди, которые, напротив, в какой‑то ме­ре разделяют мои взгляды, тоже предпочтут не называть эти взгляды социалистическими. Наконец, многие читатели сочтут неправильным называть социализм формой религии.

Все эти терминологические проблемы меня мало волнуют. Нет ни необходимости, ни возможности решить их директивным методом: они обычно решаются сами собой, в процессе много­сторонних обсуждений. Должен сказать, что я столь безапелля­ционно заявил в начале этой части книги, что социализм есть религия, с целью, можно сказать, литературной: мне хотелось сделать упор на глубокое единство этих явлений культуры. Но если делать окончательный выбор слов, то, возможно, будет лучше говорить, что социализм есть преемник религии, а не ее новая форма.

Связь реального, исторического социализма с принципом социальной интеграции является для меня несомненным фактом. Читая и думая о социализме, а также живя в стране, которую все называют социалистической, я испытывал попеременные импульсы притяжения и отталкивания. В конце концов, я при­шел к следующему выводу: то, что меня притягивает в социализ­ме, выражается формулой "социальная интеграция плюс свобо­да", а отталкивает меня экономический детерминизм, механи­стический подход к интеграции, ставка на борьбу и разрушение вместо работы и творчества. Как назвать концепцию, основан­ную на этой формуле, и общественный строй, воплощающий ее в жизнь? Социализм ‑ наиболее естественное и правильное на­звание. Я говорил выше о сходстве этой концепции с "этическим социализмом"; учитывая важную, даже решающую, роль, которую играет в ней кибернетика, ее можно было бы назвать "кибер­нетическим" социализмом. Термины "эволюционизм" и "интеграционизм" также могут быть использованы для обозначения этого подхода.

Некоторые читатели будут, возможно, разочарованы неконк­ретностью изложенных здесь представлений о социализме. Дейст­вительно, я рассматривал здесь проблемы общества с позиций философии, то есть мысля в наиболее общих понятиях. Но этот уровень мышления необходим и играет незаменимую роль в раз­витии общества, порождая конкретные исследования, теории и проекты и давая основу для оценок различных явлений. Не каж­дый гражданин должен быть специалистом в социологии или экономике, но политическая философия -- дело каждого: со­знает он это или нет, в его мышлении всегда присутствуют какие‑то чрезвычайно общие понятия, опираясь на которые он воспри­нимает общеизвестные явления и выносит свои суждения. Уяс­нение и обсуждение этих понятий есть, следовательно, задача большой важности.

Философские понятия и достижения синтезируют в себе поня­тия и достижения из всех областей человеческой деятельности. Каждый подходит к ним со своей стороны. Я подхожу к ним как физик и кибернетик.



\chapter{ТОТАЛИТАРИЗМ ИЛИ СОЦИАЛИЗМ?}

\section{Борьба за идеи и борьба за власть}

Если в процессе наступления на общество тоталитаризм продвигается снаружи внутрь -- от захвата власти через огра­ничение информационного обмена к трансформации мышле­ния и воли, то освобождение от тоталитаризма должно прохо­дить в обратном направлении. Началом должны быть сдвиги в мышлении людей, в общественном сознании. Они приведут к увеличению обмена информацией, более свободному выра­жению идей и большей гласности общественной жизни. Это поз­волит демократизировать управление и эффективно бороться со злоупотреблениями властью. Таков единственно возможный путь. Альтернативой ему является либо загнивание, либо разрушительный взрыв, катаклизм, наподобие революции 1917 года. Катаклизм наверняка принесет с собой неисчислимые жертвы, а поможет ли он построить лучшее общество, весьма сомнительно. Скорее всего, он снова отбросит нас назад.

В общественных конфликтах мы можем усмотреть два ас­пекта: борьбу за идеи и борьбу за власть. Чаще всего оба этих вида деятельности рассматриваются не как цель в себе, а как средство для достижения другой цели, например, личного обогащения или роста всеобщего благосостояния. Но не менее важно, что и борьба за власть, и борьба за идеи являются формами самовыражения и самоутверждения. Борьба за власть -- гораздо более древняя, дочеловеческая, форма самоутверждения. У многих животных доминирование в группе обеспечивает пер­вое место в питании и спаривании и, следовательно, закрепле­ние своего генетического кода в потомстве. Борьба за идеи -- специфически человеческое явление, это утверждение своей личности на социальном уровне. Борясь за признание своих идей, человек борется за увековечивание своей личности в об­разе жизни потомков.

Борьба за идеи и борьба за власть тесно переплетаются в жиз­ни общества. Борясь за власть, человек делает ставку на те или иные идеи. Борясь за идеи, человек часто опирается на власть или вступает с ней в противоречие. Но переплетение этих форм не означает, что проведение различия между ними невозмож­но или несущественно. Борьба за власть и борьба за идеи отно­сятся к разным уровням вселенской иерархии по управлению:

вторая -- наиболее высокому, чем первая. С точки зрения эво­люции системы,чрезвычайно важно, чтобы соблюдалось опреде­ленное соотношение между этими формами противоречий, а именно: борьба за власть должна занимать подчиненное поло­жение по сравнению с борьбой за идеи, она должна быть лишь ее неизбежным следствием. Борьба за власть, ставшая само­целью, приводит к анархии или к тирании, но никак не к кон­структивной эволюции общества. Если мы хотим утвердить в жизни некие идеи, то из всех мыслимых путей надо выбирать тот, который в наименьшей степени связан с изменением струк­туры власти. Иначе произойдет подмена: борьба за власть по­глотит борьбу за идеи.

Тоталитарный марксизм в своей теории и практике сливает идеи и власть в единое целое. Власть используется для насаж­дения идей, в борьбе за идеи видят борьбу за власть. Это со­здает безвыходный порочный круг, закрывает путь к эволю­ции. Чтобы сдвинуться с мертвой точки, мы должны прежде всего научиться разделять борьбу за идеи и борьбу за власть.


\section{Однопартийная (она же беспартийная) система}

Критиков первого варианта "Инерции страха" больше всего возмутил тот факт, что я высказался против борьбы за много­партийную систему в условиях Советского Союза. Я предло­жил отделить вопрос о политических свободах от вопроса о борьбе за политическую власть между партиями, представля­ющими интересы больших социальных групп (классов). Я предложил рассматривать коммунистическую партию в пер­спективе как интеллектуальный и духовный интегратор общества, действующий в условиях широких гражданских и полити­ческих свобод, то есть принять в теории то, что нынешний ре­жим  выдает как якобы уже осуществленное на практике. Многопартийность отнюдь не гарантирует наличия демократи­ческих свобод, а наличие свобод вовсе не обязательно порож­дает многопартийную политическую систему: можно попы­таться найти другие пути разрешения социальных противоре­чий, при которых эти противоречия не доводятся до высшего государственного уровня и не усиливаются искусственно борь­бой групповых интересов политиков.

Хотя моя брошюра в целом была принята оппозиционно настроенными кругами интеллигенции очень хорошо, несо­гласие со мной в этом пункте было едва ли не всеобщим. Ти­пична была реакция одного довольно известного кибернетика. Я не был знаком с ним лично, но мне рассказали, что, когда его спросили, что он думает об "Инерции страха" Турчина, он сказал: "Кошмар. Он там за коммунизм, за однопартийную систему\ldots". Пикантная деталь: в отличие от меня, этот кибер­нетик -- член КПСС.

Замечательно, что даже официальный партийный рецензент моей рукописи не одобрил моих идей относительно однопар­тийной системы. Я имел наглость послать "Инерцию страха" в журнал "Коммунист", понимая, конечно, что она не будет напечатана, но желая подчеркнуть открытый и конструктив­ный характер работы и свою готовность идти на обсуждение. Обсуждение, действительно, состоялось: один из сотрудников журнала (заведующий отделом, кажется) прочитал мою ру­копись и объяснил мне, что она не может быть опубликована, так как в ней не чувствуется "классового подхода" (то есть готовности служить интересам правящего класса и ставить их выше истины ‑ довольно точная формулировка). Относитель­но одно‑и многопартийности он сказал: "А почему, собствен­но говоря, должна быть только одна партия? В стране может быть и несколько партий, одинаково преданных делу комму­низма. В Польше, например, несколько партий".

Теперь, после того как французские коммунисты, вслед за итальянскими, решительно заявили о своей приверженности к многопартийной системе, моя позиция кажется многим моим друзьям анахронизмом, не оправданным даже с тактической точки зрения. Но я по‑прежнему стою на этой позиции и отнюдь не считаю ее анахронизмом; мне кажется, что противодействие, которое она встречает, связано с тем, что она скорее опережает время, чем отстает от него. Я намерен теперь изложить свою позицию более обстоятельно, чем это было сделано в 1968 году.

Прежде всего, как полагается, о терминах. Однопартийная система, говорили мне многие критики, это бессмыслица по са­мой своей сути, по определению. <<Партия>> --значит часть,  по­этому о партиях можно говорить только тогда, когда их не­сколько.

Верно, термин неточный. Исторически он возникает из такой ситуации, когда одна из партий приходит к власти и запрещает все остальные. После того как существование нескольких пар­тий уходит в историческое прошлое, "партия" перестает быть партией и превращается в единую всеобъемлющую политиче­скую систему или сеть. Современная советская система не одно­партийная, она беспартийная.  У нас нет политических партий, есть лишь единая политическая сеть, не отделимая от государ­ства. Читатель, конечно, не удивится, если я заявлю, что не наме­рен оправдывать запрещение пришедшей к власти партией всех остальных партий. Но я полагаю, что будущее человечества -- в беспартийной системе с единой политической сетью. А если так, то от человека, стоящего на позициях реформизма и градуализма, естественно ожидать такой установки: не пытаться вырвать власть путем вооруженной борьбы или борьбы за го­лоса избирателей, а стремиться к трансформации общественно­го сознания, увеличению свободы и гуманизации общества в целом и его политической сети в частности. Это та установка, которая была у чешских коммунистов‑реформаторов в 1968 го­ду и получила название "социализма с человеческим лицом".

Рассмотрим сначала перспективу. Обычный первый аргумент за многопартийную систему тот, что только при наличии не­скольких конкурирующих партий, которые потенциально мо­гут вытеснить правящую партию (или коалицию), можно со­хранить политические свободы и обеспечить эффективный контроль за властью. Однопартийная система, говорят мои кри­тики, неизбежно приводит к злоупотреблению властью. Любая организация, называй ее хоть партия, хоть политическая сеть, если она является полным хозяином политической власти и не имеет конкурентов, обречена на загнивание и коррупцию, она не в силах будет сопротивляться разлагающему влиянию власти. Независимо от расстановки сил различных социальных групп, независимо от того, кто кого представляет, нужно, что­бы было хотя бы две независимые политические партии. До­статочно беглого взгляда на политические системы в различных странах, чтобы убедиться, что там, где мы видим многопартий­ную (в частности, двухпартийную) систему, мы видим и поли­тические свободы, а там, где принята однопартийная система, свирепствует тирания партийно‑государственного аппарата.

Корреляция между тиранией и многопартийной системой несомненно существует, но это ничего не доказывает, ибо при­чинно‑следственная связь здесь такова: отсутствие свободы ведет к правлению одной партии, а не правление одной партии ведет к отсутствию свободы. Общество, в котором не укоре­нилась идея политических свобод, не может сохранить много­партийную систему, если даже такая система по прихоти исто­рии и будет однажды установлена. Именно таково было разви­тие событий в большинстве стран Третьего мира, приобретших независимость после Второй мировой войны. Это лишний раз доказывает решающую роль культуры общества (по сравнению с конкретно‑историческими, экономическими, географическими и т.~п. факторами) в определении его политической системы. Если общество неспособно сохранить политические свободы, то не так уж важно, какую форму примет тирания: будет ли это абсолютная монархия или откровенная диктатура военных, или однопартийная система с комедией выборов, или такая многопартийная система, как в Польше, на которую любезно указал мне рецензент из журнала "Коммунист". Говоря о пер­спективе, мы должны предположить, что идея уважения основ­ных прав личности уже прочно вошла в культуру общества, и никто не намерен покушаться на демократический образ прав­ления -- хотя бы из‑за бесперспективности подобных попыток. Можно ли утверждать, что в этих условиях политическая система должна быть или неизбежно будет много партийной?

Контроль над имеющими власть или стремящимися к власти необходим, но основной фактор, обеспечивающий эффективный контроль, это не разделение на партии, а широкая гласность, активное участие масс в политике и нетерпимость общества к нарушению этических принципов. Говорят, что при наличии нескольких партий всегда есть люди, лично заинтересованные в том, чтобы обнаружить ошибку или злонамеренность в дей­ствиях руководящих политиков. Но в беспартийной системе таких людей будет не меньше, а больше! Если, например, в стра­не две примерно одинаковые по силе партии, то половина поли­тических деятелей, то есть те, которые принадлежат к правящей партии, не очень‑то заинтересованы в критике руководства. Если же партии отсутствуют, то каждый политик наживает личный капитал на разоблачении ошибок другого. Можно срав­нить беспартийную политическую систему с системой научных учреждений. Когда ученый делает открытие или доказывает новую теорему, его коллеги, сгорая от нетерпения, бросаются проверять и перепроверять его сообщение в надежде опроверг­нуть или уточнить его. То же явление будет наблюдаться и в политике, если она будет устроена по образцу и по опыту науки. Деление на партии -- обоюдоострое оружие. С одной стороны, оно помогает разоблачить обман и преодолеть сопротивление укоренившихся мнений, с другой стороны, создает предпосыл­ки для нового обмана в угоду групповым интересам и для но­вого консерватизма. Какой из этих двух эффектов сильнее? Скорее второй, чем первый. Чтобы разоблачить обман, в усло­виях свободы достаточно и одного человека; чтобы создать обман, необходим сговор. Или, выразимся точнее: способность к разоблачению обмана и заблуждений возрастает с увеличе­нием числа людей в группе медленнее, чем способность к обма­ну и коллективному самообману.

Деление на партии в науке считается прямо‑таки неприличным, ученые оскорбляются, когда им говорят, что они руководству­ются партийными интересами. И уж, конечно, никому не приходит в голову разделить все научные учреждения -- в целях борьбы с обманом, коррупцией, застоем, загниванием и т.~д. -- на две или три не контактирующие между собою части. Совокупность всех образовательных и научных учреждений пред­ставляет собой единое целое, единую систему, и это отнюдь не приводит к застою и коррупции, к подавлению личной ини­циативы и свободы творчества, к ликвидации плюрализма и насильственному введению единообразия. Почему единая поли­тическая сеть не может функционировать так же успешно? В обществе, где важнейшие этические и демократические прин­ципы так же прочно вошли в жизнь, как вошли в жизнь ученых основные методологические принципы науки, где покушение на политические свободы столь же немыслимо (и кажется столь же абсурдным), как покушение на свободу научного ис­следования, деление всей совокупности людей, которые занима­ются политикой, на несколько частей покажется искусственным и никому не нужным.



\section{Политика и наука}

Но можно ли сравнивать политику с наукой? Более обычным является их противопоставление: наука есть стремление к единой и общей для всех людей истине, политика борьба за власть, за личные и групповые интересы.

Это противопоставление имеет основания, но не исчерпывает вопроса. Политика есть искусство социальной интеграции, и основное противоречие социальной интеграции -- противоречие между личным и общественным -- отражается в двух дополнительных аспектах политики: борьбе за личные или групповые интересы и стремлении к общей пользе, общей цели. При обсуждении второго аспекта мы отвлечемся от определения понятия общей пользы или цели (что в конечном счете одно и то же). В какой‑то степени в обществе всегда существует единодушие на этот счет, а в той степени, в которой его нет, -- это вопрос другого слоя культуры, вопрос религии, а не политики. В политике вопрос стоит так: при заданной цели или заданном способе исчисления общественной пользы, как надо организовать общество (и прежде всего систему производства), чтобы добиться скорейшего достижения цели или максимума общественной пользы? Задача эта -- научная по своей сути. Она ничем не отличается от задачи поиска истины, а точнее -- наилучшего приближения к истине, которую ставит наука. Поэтому об указанных двух аспектах политики мы будем говорить как о борьбе интересов  и стремлении к истине.  Это почти то же самое противопоставление, что борьба за власть и борьба за идеи, только объективизированное. Борьба за власть -- наиболее прямое и непосредственное выражение борьбы групповых ин­тересов; борьба за идеи -- выражение стремления к истине.

Чем же определяется сравнительная важность двух аспектов политики? Уровнем развития общества, степенью его интегрированности. В слаборазвитом обществе отсутствует представ­ление об общей цели не связанных между собой непосредствен­но групп людей, не говоря уж об общей цели человечества. Общие интересы ограничиваются интересами небольших групп тесно связанных между собой людей. Кроме того, в условиях примитивного производства каждый человек или группа может больше получить для себя, вырывая кусок у другого, чем тру­дясь для общего дела. (Лучше всего это видно на том предель­ном случае, когда производства нет вовсе, и люди, подобно животным, просто соревнуются между собой за дары природы.) Политика в таком обществе сводится к борьбе интересов, это "война всех против всех" Гоббса. По мере социальной инте­грации и усложнения производства борьба всех против всех становится все более невыгодной для всех вместе и каждого в отдельности. В современных промышленных странах революция стала экономически невыгодной --  даже для со­циальных низов. И не только революция, но и некоторые из более мягких способов борьбы за групповые интересы, такие, как забастовки. Я не знаю, являются ли в настоящее время массовые забастовки экономически выгодными для рабочих в передовых странах: с одной стороны, они дают увеличение заработной платы, с другой стороны, приводят к возрастанию инфляции. Возможно, экономисты знают ответ на этот вопрос; пример Англии во всяком случае заставляет задуматься. Но если сейчас забастовки и выгодны, то можно не сомневаться, что настанет день, когда они станут невыгодными.

Партийная система отражает подход к политике как к борь­бе интересов, она учит граждан видеть в политике прежде всего борьбу интересов. Беспартийная система учит видеть в политике прежде всего общее дело, стремление к истине. В обстанов­ке войны всех против всех многопартийная система -- неиз­бежное и законное следствие демократических свобод. Но в процесс движения общества к социализму партийная система должна уступить место беспартийной. Для этого вовсе не нуж­но, чтобы борьба интересов исчезла из политики вовсе -- этого, конечно, не случится никогда. Нужно только, чтобы стремле­ние к истине было осознано обществом как более важный аспект. Образцом опять‑таки является наука. Наука как абстрактное понятие олицетворяет чистое стремление к истине. Наука как реальное общественное явление представляет собой, как и политика, тесное переплетение стремления к истине и борьбы интересов. Реальная наука -- это система, характеризующаяся определенной структурой, определенными иерархиями престижа и власти. Ученым свойственны те же слабости и пороки, что и остальным людям. В частности, им свойственно, обманывая себя и других, выдавать личные и групповые интересы за стремление к истине. И вообще, поскольку мы не можем влезть со скальпелем в мозг человека и проанатомировать его побуждения, мы далеко не всегда можем сказать с уверенностью, определяется ли принятая им линия стремле­нием к истине или личным интересом. И все же наука остает­ся беспартийной.

Если единая политическая сеть будет построена по образцу научно‑образовательной системы, она будет служить основным инструментом социальной интеграции, поставляющим руково­дящие кадры для законодательной, исполнительной и судебной власти, а также, вероятно, для верхушки производственной иерархии, ибо крупные производственные проблемы не отде­лимы от социальных. Разделение между политической сетью и указанными иерархиями власти, как и разделение между этими видами власти, не должно быть, конечно, ни в коем случае лик­видировано: не раньше, чем будет найдено более совершенное (и, очевидно, более сложное) решение проблемы управления в духе принципа структурно‑функционального параллелизма.

Партийная система в политике постепенно становится ана­хронизмом. Во времена Французской революции деление об­щества на три сословия естественным образом вело к разбиению представителей общества на три партии. В современном демократическом обществе деление на сословия отсутствует: все граждане обладают равными правами. Четкое деление на классы в марксистском смысле также отсутствует: людей, которых нельзя без натяжки отнести к определенному классу, стало больше, чем "классических" рабочих или капиталистов. Размытие границ между классами усиливается. Это не озна­чает, конечно, ликвидацию разделения труда. Напротив, разде­ление труда непрерывно углубляется, и именно поэтому увеличивается разнообразие социальных ролей. Мы имеем теперь перед собой непрерывный спектр социальных ролей, и разби­ение его на несколько категорий всегда будет условным и на­тянутым.

В этих условиях естественно строить единую политическую сеть, в которой был бы представлен весь спектр социальных ролей; фиксированное разделение на несколько партий, неиз­менное в течение десятилетий, представляет собой искусствен­ное явление, сохраняющееся как пережиток. Возникает ситу­ация, когда партии существуют ради партий, ради самих себя. Они перестают быть органической частью одного из сословий или классов и превращаются в самодовлеющие профессиональ­ные организации, для которых важно, не кто  за них голосует, а сколько  за них голосует. Целью партии становится не пред­ставление чьих‑то интересов и тем более не стремление к исти­не, а борьба за власть.

Разумеется, из политики никогда не удастся изгнать ни борь­бу за власть, ни своекорыстный личный интерес. Еще в мень­шей степени, чем в случае ученого, способны мы безошибочно разграничивать в случае политика стремление к власти ради истины и стремление к власти ради власти. Человек -- не маши­на, а если и машина, то не нами сконструированная. Но поли­тическая партия -- это машина. Зачем же нам конструировать такие машины, основной целью которых является стремление к власти? Не будет ли политическая система лучше, если будут допустимы лишь такие организации, которые не ставят целью борьбу за власть? Так, в сущности, обстоит дело в науке. Борь­ба мнений и разделение на партии происходят в науке вокруг конкретных проблем: партии возникают и распадаются по мере разрешения одних проблем и появления других.  Так будет и в политической сети при беспартийной системе. Ситуацию, ког­да партии существуют не в связи с конкретными проблемами, а как самостоятельные сущности, ученый рассматривает как извращение. Между тем, в многопартийной политической си­стеме это норма.

В политике, в отличие от науки, идеальной целью является не только стремление к истине, то есть к некоторым образом определенному общественному благу, но и соблюдение инте­ресов каждого гражданина, как он их сам в данный момент понимает, вне зависимости от общих идей. Это не подлежит сомнению. Но действительно ли для выражения этих интересов необходима организация? Лучший ли это способ? Ведь едва возникнув, такая организация приобретает свои собственные цели. В обществе высокой культуры, в обществе с огромным разнообразием социальных ролей и типов не является ли доста­точным для соблюдения индивидуальных интересов свободное участие каждого индивидуума в политике, в частности, участие в различных референдумах и голосованиях? Я могу предста­вить себе синтез в политической системе социализма двух схем принятия решений, о которых я говорил выше: борьба интере­сов будет осуществляться путем всеобщего равного голосова­ния, а согласование интересов -- с помощью пересекающихся валентных иерархий.

Политические партии, борющиеся между собой за власть, приобретают черты военных организаций: появляется необхо­димость в партийной дисциплине, в секретности и т.~д. В единой политической сети все это становится излишним, так как она нс противостоит как целое никакой другой организации. Она может позволить себе предельную открытость и либерализм, ей нет необходимости ни ограничивать появление различных группировок и фракций, ни даже регистрировать  их каким‑либо образом (снова напрашивается сравнение с научной систе­мой). Можно сказать, что беспартийная система в условиях широких политических свобод это не система без партий и не система с одной партией, а система с неопределенным числом партий. Здесь мы встречаемся с парадоксом, который можно назвать системной относительностью.  Каждое множество можно рассматривать как одну систему, то есть единицу. Каждая единица представляет из себя некую систему, следовательно, состоит из множества подсистем. Когда мы сравниваем одно­партийную и многопартийную системы, имея в виду самый об­щий смысл этих понятий, то различие между ними становится условным, относительным. В многопартийной системе полити­ческие партии не являются изолированными: каждая учиты­вает политическую программу конкурентов, не говоря уже о коалициях, политике в парламенте и т.~п. Поэтому можно ска­зать, что в своей совокупности они образуют единую полити­ческую систему, или сеть. С другой стороны, единая политиче­ская партия, или сеть, в условиях широких демократических свобод будет на деле состоять из различных групп, фракций, партий. Реальное различие между этими понятиями состоит в том, какая концепция политической жизни в них подразу­мевается, какую они содержат установку. Многопартийная система означает установку на борьбу за власть, беспартий­ная -- на борьбу за идеи.

\section{Революция или реформация?}

\epigraph{Стремясь к защите прав людей, мы должны выступать, по моему убеждению, в первую очередь как защитники невинных жертв существующих в разных странах режимов, без требования сокрушения и тотального осуждения этих режимов. Нужны реформы, а не революции.}{А.~Д.~Сахаров [1]}

В демократических странах с многопартийной системой пе­реход к беспартийной системе не должен, конечно, совершать­ся путем запрещения политических партий и тем более путем запрещения всех партий, кроме одной. Это было бы грубым нарушением демократических свобод, и можно с большой вероятностью предсказать, что за ним последовали бы новые нарушения, вплоть до установления тирании правящей бюрократии. Заверения западных коммунистов в признании однопартийной' системы -- это заверения в признании демократических свобод, и их надо приветствовать. Переход к беспартийной системе должен произойти под воздействием сил, лежащих в сфере культуры, а не в сфере политического управления. Вероятно, движение общественного сознания в сторону социализма повлечет все более отрицательное отношение к межпартийной борьбе в ее традиционных, зачастую весьма непривлекательных, формах, оправдываемых поговоркой: на войне, как на войне. В этих условиях можно представить, например, такие два пути развития событий.

\begin{enumerate}
 \item Будет образована государственная надпартийная систе­ма -- зачаток и основа будущей политической сети, обеспечи­вающая возможность всем гражданам, независимо от их партий­ной принадлежности и их взглядов, проявлять политическую активность. При наличии такой системы все большее число по­литиков предпочтет оставаться беспартийными, и будут созда­ваться какие‑то новые формы объединения и размежевания между ними. Быть членом партии станет немодно, и партии, подобно марксовому государству, не будут отменены, а отомрут. 
 \item Одна из партий (образовавшаяся скорее всего в резуль­тате коалиции нескольких партий) будет столь плюралисти­ческой и обеспечит своим членам столь благоприятные условия для деятельности, что будет привлекать и удерживать всех сколько‑нибудь влиятельных политиков. Такая система будет формально многопартийной, фактически    однопартийной, а по существу -- беспартийной, ибо партия превратится в еди­ную политическую сеть, обслуживающую все слои общества. "Исторический компромисс", предложенный в Италии комму­нистами, мог бы стать, возможно, началом этого пути.
\end{enumerate}

Большевистская партия, захватив власть, вскоре запретила все остальные партии; таково происхождение нашей однопар­тийной системы. Не удивительно, что понятие о демократиза­ции обычно связывается с восстановлением многопартийной системы, если не в качестве первоочередной задачи, то в ка­честве конечной цели. А.~Д.~Сахаров в своей последней работе "О стране и мире" в число необходимых реформ включает введение многопартийной системы. Р.~А.~Медведев, стоящий на марксистских позициях, также высказывается за многопартий­ную систему. Он пишет:

"Я надеюсь на усиление демократических движений различ­ных оттенков. Не исключаю при этом и возможности появле­ния на нашей политической арене новой социалистической пар­тии, отличной от нынешних социал‑демократических и от нынеш­них коммунистических партий. Такая новая социалистическая партия могла бы образовать лояльную и легальную оппозицию существующему руководству и тем самым косвенно способ­ствовать обновлению и оздоровлению КПСС. Не являясь преем­ником старых русских социалистических партий, такая новая социалистическая партия могла бы положить в основу своей идеологии лишь те положения Маркса, Энгельса и Ленина, ко­торые выдержали испытание временем. Такая партия, не бу­дучи связана характерным  для нашей официальной науки догматизмом, могла бы свободно развивать теорию научного социализма и научного коммунизма в соответствии с требова­ниями современной эпохи и с учетом пройденного нашей стра­ной пути. Свободная от ответственности за преступления про­шедших десятилетий, такая партия могла бы более объектив­но оценить как прошлое, так и настоящее нашего общества и разработать социалистические и демократические альтернативы его развития". [2]

Я стою на другой позиции в вопросе о многопартийности. Для того чтобы свободно развивать теорию научного социализ­ма, объективно оценивать прошлое и настоящее нашего общест­ва и разрабатывать социалистические и демократические альтер­нативы его развития, вовсе не нужна политическая партия. Это скорее задачи научного института или авторского коллек­тива, объединенного вокруг печатного органа. Политическая партия есть организация, стремящаяся к власти. Замалчивать этот факт означало бы только увеличивать общую сумму не­домолвок, и без того огромную.

В условиях Советского Союза требование многопартийной системы -- это путь революции. Не только вооруженный захват власти ведет к революции; крутая перемена структуры вла­сти -- это тоже революция, и она вряд ли обойдется без массо­вого насилия, особенно в условиях многонационального госу­дарства, как Советский Союз. Если вообразить, что в СССР вдруг с завтрашнего дня вводятся основные демократические свободы и обычная для демократических стран система сво­бодной конкуренции между политическими партиями, то КПСС, такая как она есть сегодня, скорее всего не удержит власти. Это ясно всем: и руководству КПСС, и сторонним наблюдате­лям; это и является причиной того политического тупика, в котором мы находимся. Демократия отождествляется со сво­бодными выборами в условиях многопартийной системы, что в свою очередь практически отождествляется с потерей влас­ти правящей бюрократической иерархией. Отсюда панический страх бюрократии перед всякой формой демократизации, перед всяким обменом идеями и информацией и ее отчаянное сопро­тивление минимальным реформам. Требование элементарных прав личности рассматривается как призыв к свержению вла­сти, к революции.

Существует ли нереволюционный путь демократизации? Я думаю, что существует. Путь реформ означает, с моей точки зрения, прежде всего четкое разграничение борьбы за власть и борьбы за идеи и отказ от борьбы за власть в пользу более успешной борьбы за идеи. Это означает отказ от требования многопартийной системы на ближайший обозримый период ‑тот период, который потребуется, чтобы укоренились основные права личности. А так как в перспективе целесообразность многопартийной системы по меньшей мере сомнительна, то мно­гопартийность надо вообще снять с повестки дня. Идеалом ре­формации является постепенное превращение КПСС под дей­ствием эволюции общественного сознания и норм поведения в единую политическую сеть подлинно социалистического госу­дарства. Демократические свободы, которые должны иметь советские граждане, должны включать, конечно, и свободу ассоциаций. Но и в самом демократическом обществе не вся­кая ассоциация допустима; не разрешается, например, ассо­циация с целью воровства или вооруженного захвата власти. Я думаю, что было бы вполне логично исключить возможность образования ассоциаций с целью "захвата" власти путем голо­сования. (Я поставил "захват" в кавычки потому, что не соби­раюсь преуменьшать разницы между вооруженным захватом власти в буквальном смысле слова и приходом к власти в ре­зультате голосования. Но в эпоху манипуляции сознанием масс путем пропаганды нельзя забывать также и о сходстве этих двух методов; вспомним хотя бы о механике прихода к власти Гитлера.)

Никто не может предсказать, осуществится ли этот идеальный вариант демократизации. Возможно, что и не осуществится. Но важно, что такая возможность есть, и ее осуществление зависит от нас самих.

\section{Опыт старой России}


Отрицание возможностей реформ вместо попыток их осу­ществления всегда в большей или меньшей степени догматич­но, и целью его является либо моральное оправдание полной пассивности, либо, напротив, пропаганда революционного пути как единственно возможного. Вторая позиция -- это позиция большевиков в старой России. Чем хуже, тем лучше ‑ было их лозунгом. Своими злейшими врагами они считали не реакционеров, не тупых чиновников, которые блокировали реформы, а именно реформаторов, старавшихся, не разрушая системы, улучшить положение бедных слоев населения, устранить наиболее вопиющие безобразия и направить страну по пути непрерывного прогресса. Все эти усилия большевики объявляли "обманом трудящихся", имеющим целью ослабить революционные настроения.

В течение долгого времени после революции большевистская концепция политических перемен оставалась единственно допустимой и универсально применимой. Советский человек с юных лет воспитывался в убеждении, что:

\begin{enumerate}
 \item Власть в обществе всегда принадлежит некоторому клас­су и служит интересам этого класса.
 \item Никакие серьезные реформы невозможны без смены правящего класса.
 \item Так как правящий класс ни за что не отдаст власть без боя, осуществить серьезные перемены в обществе можно толь­ко путем вооруженного захвата власти, революции.
\end{enumerate}

Нежелание рабочих в западном мире устраивать революции привело постепенно к изменению программ западных компар­тий в духе прихода к власти конституционным парламентским путем. В результате и советская пропаганда была вынуждена признать этот способ прихода к власти принципиально возмож­ным. Но в остальном, то есть в главном, концепция не измени­лась. К чему же ведет эта концепция применительно к советско­му обществу?

Прежде всего -‑ к дискуссии о том, какому классу "принад­лежит власть" в СССР. Если согласиться с официальной доктри­ной, что власть у нас принадлежит рабочим и крестьянам, то перемены возможны лишь в смысле "конфликта лучшего с просто хорошим". Многие западные левые стоят на этой точке зрения. Советский Союз ‑ государство рабочих и крестьян, и это главное. Поэтому он должен служить объектом восхищения и образцом для подражания. Разные частности и мелкие недо­статки, такие как миллионы замученных в сталинское время или перевоспитание инакомыслящих с помощью инъекций нейролептиков, неприятны, но не могут изменить главного.

Те, кто не желают смириться с "мелкими недостатками" и  пытаются объяснить их с марксистских позиций, говорят вместе с Милованом Джиласом о <<новом классе>> --партийной бюрократии, которой реально принадлежит власть. Отсюда следует, что единственная надежда на перемены -- это вырвать власть у КПСС, а так как парламентских форм борьбы у нас не существует, то остается лишь создание нелегальной организации с целью свержения власти. Это и было бы большевистской позицией, перенесенной в современные условия. Однако каждому ясно, что шансов на успех на этом пути мало. Поэтому на практике большевистская позиция ведет к тому, что человек просто разводит руками и ничего не делает. Марксизм в марксистском государстве еще раз оказывается инструментом не движения, а застоя.

Если я перенесу свою позицию в дореволюционную Россию, то попаду, по‑видимому, в конституционные монархисты ‑ не от большой любви к новой "династии", а потому что шансы на прогресс в нашей стране я вижу в демократизации общества при сохранении основной структуры власти. Я отрицаю больше­визм не только как принципиальный противник насилия, но и потому, что, как видно из предыдущих частей книги, я отрицаю всю идеологию марксизма‑ленинизма, на которой основана большевистская политическая программа. Я не верю, в частно­сти, что проповедь ненависти к правящему классу, будь то поме­щики, капиталисты или партбюрократия, может привести к че­му‑нибудь, кроме бессмысленного разрушения. Реальный про­гресс в обществе -- это перемена идей, а не власти.

Параллель между реформизмом в советской России и ре­формизмом в царской России имеет под собой глубокое осно­вание. Пройдя через чудовищно кровавую революцию и граж­данскую войну, мы вернулись к структуре власти и обществен­ному сознанию, которые больше напоминают традиционную Россию, чем революцию. Мы вновь оказались поставленными перед теми же проблемами, которые стояли в России 19‑го ве­ка. Та же всесильная предельно централизованная бюрокра­тия, тот же произвол власти, то же пренебрежение к основным правам личности как со стороны управляющих, так и со сторо­ны управляемых, та же нетерпимость к идеям, не одобряемым высшими инстанциями. Психология советского бюрократа -- это психология бюрократа царской России, что же касается его мировоззрения, то сменены лишь декорации, а по сущест­ву оно изменилось мало. В его основе по‑прежнему лежит три­ада: православие, самодержавие, народность; только правосла­вие теперь называется марксизмом‑ленинизмом, самодержа­вие превратилось в руководящую роль партии, а вместо "на­родность" говорят "морально‑политическое единство совет­ского народа". Я убежден, что когда советский чиновник смот­рит кинофильм о революционерах в царской России, он внут­ренне находится на стороне царских чиновников, он отожде­ствляет себя с ними, а не с большевиками. И в самом деле, что общего между ним и этими демагогами, которые плетут сети заговора, чтобы подорвать государство, созданное тысячелет­ней русской историей, разрушить порядок и ввергнуть страну в кровавый хаос?


\section{Чему же учит нас исторический опыт России?}

Для движения по пути реформ необходимы два условия.

Во‑первых, должно существовать серьезное общественное давление на власть в пользу реформ. Общество, которое раболепствует перед властью, порождает, с одной стороны, тиранию, а с другой стороны -- разрушительный экстремизм, большевизм. В России всегда не хватало сочетания твердости с уме­ренностью. Пока мы не научимся этому, мы будем бросаться из одной крайности в другую, вместо того чтобы неуклонно продвигаться вперед.

Во‑вторых, необходимо, чтобы власть перестала бояться реформ и научилась их во время проводить. Хорошо известно, что такая политика не ослабляет власть, а укрепляет ее. В советских курсах обществоведения любят, ссылаясь на кого‑то из основоположников (не то Маркса, не то Энгельса, не помню), приводить английскую буржуазию в качестве примера использования власти, не исключающего политических реформ. Но в своей собственной политике советские руководители подра­жают почему‑то не английской буржуазии, а худшим образцам из русской истории. Стабилизация тоталитаризма -- путь к ка­тастрофе. Окостенение и загнивание не может продолжаться вечно; рано или поздно, под влиянием какой‑то внешней или внутренней причины должно произойти разрушение такого об­щества, и это будет ужасно. Слепой страх перед движением мысли, сопротивление политической и экономической либера­лизации ведут в пропасть. Роковым образом коммунистиче­ская власть в России повторяет ошибки династии Романовых.

Впрочем, исправлять ошибки Романовых можно различны­ми способами. Один диссидент сказал кагебисту: "При царе и то было больше свободы, чем сейчас!". На что тот возразил: "Вот и доигрались до революции!" Ответ не лишен логики. Он лишний раз показывает, кто есть кто в новой России при сравнении со старой и отражает, надо думать, точку зрения части партаппарата, и во всяком случае его верхушки.

Что ж, мы уже почти вышли на режим стационарного самовоспроизводящегося тоталитаризма, и возможность его сохранения в течение поколений не исключена. Но кагебист не учитывает двух обстоятельств. Во‑первых, стационарный тоталитаризм возможен только при условии абсолютной  стационарности -- полного постоянства форм и норм жизни в качественном и количественном отношении. Ибо любые изменения, даже количественные, потребуют в конце концов каких‑то новых решений, какого‑то творчества, на которое тоталитарное общество не способно. Этого, к сожалению, не понимают партаппаратчики, вследствие отсутствия у них необходимой куль­туры, философского кругозора. Они, по‑видимому, искренне полагают, что общество, наложившее запрет на свободную мысль, может до бесконечности "удовлетворять непрерывно растущие потребности". Во‑вторых, не весь мир еще, к счастью, тоталитарен, он не остановился еще в своем развитии. Живя в этом мире, мы не можем его игнорировать, и это накладывает определенные ограничения на тех, кто стремится к вечному мраку.

Не то погубило Романовых, что они дали "слишком много" свободы. Свободы было хотя и больше, чем сейчас, но ‑ ска­жем прямо ‑ не так уж и много. В других странах было не­сравненно свободнее, и ‑ ничего. Погубила Романовых неспо­собность вовремя  проводить необходимые реформы, погуби­ло отчуждение между государством и передовой частью обще­ства. Возникло специфически русское явление ‑ интеллиген­ция, образованный слой общества, находящийся в конфликте с государством. Не только государство было виновно в этом конфликте. Авторы сборника "Вехи" высказали много спра­ведливых упреков в адрес интеллигенции, они провидчески указали на те черты российской интеллигенции, которые в ко­нечном счете привели к большевистскому террору. Но все же основная, изначальная вина лежит без всякого сомнения на царской власти. И вот теперь мы видим, что советское государство идет по тому же самоубийственному пути, который привел к гибели царское государство (ах, если бы только государство!). Отказывая своим гражданам в элементарных политических правах и свободах, оно углубляет и увековечивает конфликт между властью и культурой, при котором общество не может нормально развиваться. Неужели все‑таки исторический опыт ничему не учит Россию?



\section{Инерция страха}

Оба условия постепенной демократизации, давление снизу и способность к реформам наверху, не выполняются у нас, в сущности, из‑за страха, а точнее, из‑за инерции страха,  вошедшего в нашу жизнь при Сталине. Страх, который парализует общество это страх сталинских жертв, страх, испытываемый властью, -- страх самого Сталина. Пришедший к власти в результате невиданного в истории террора, Сталин подозревал каждого в тайном вынашивании планов возмездия, в каждом видел скрытого врага. Очевидно, этот элемент и до сих пор сохраняется в высшем руководстве. Жестокие и бессмысленные репрессии против инакомыслящих (которые вовсе не стремятся к вытаскиванию руководителей из их кресел) свидетельствуют о наличии этого элемента и в то же время регенерируют, подкрепляют его. Образуется порочный круг. Чтобы разорвать его, нужен хотя бы какой‑то минимум доверия между властью и обществом, чтобы разграничить борьбу за идеи от борьбы за власть. Но при той пелене страха и лжи, которая нас окутывает, даже достижение этого минимума -- труднейшая задача. Власть настолько боится реальных проблем, которые стоят перед страной, что даже не хочет назвать их по имени; она предпочитает отрицать очевидные факты. Это политика страуса, который прячет голову в песок от страха.



\section{Дискуссия с большевиком}

Разумеется, процесс демократизации не может не повлечь каких‑то перемещений в партийно‑государственной иерархии. Люди, решительно неспособные к работе в меняющихся условиях, должны будут сойти с политической сцены. Но если проводить реформы умело и постепенно, то они не будут угрожать основной массе правящего слоя. Человек -- существо обучаемое, способное менять стиль жизни и работы при изменении условий. Почему мы должны думать, что советский партработник в этом отношении радикально (чуть ли не биологически) отличается от остальных людей? Мастодонты, конечно, должны будут постепенно вымереть, но обществу это пойдет только на пользу.

Тут я слышу голос современного большевика:

"Все это идеализм и иллюзии. Классовый интерес партийной верхушки состоит в том, чтобы ничего не менять ни на йоту. Они выросли в определенных политических условиях и привыкли, приспособились к ним. Они вполне довольны жизнью.

Зачем им демократизация, которая нарушит их покой, выну­дит как‑то выкручиваться в новых условиях, доказывать свою правоту или другие достоинства на широких собраниях, риско­вать провалом на свободных выборах? Заставить их пойти на демократизацию, это все равно что заставить волка кушать капусту, это противоречит их природе, их классовому интересу".

Внешне правдоподобное, это возражение грешит тем, что вы­дает часть истины за всю. Указанный в нем эффект несомнен­но имеет место, смешно было бы его отрицать. Желание спо­койной жизни правящим слоем препятствует демократизации. Но чтобы сделать из этого эффекта решительный вывод о не­возможности демократизации, надо его дополнить еще несколь­кими положениями. Во‑первых, надо предположить, что клас­совый интерес правящего слоя исчерпывается спокойной жиз­нью, так что никакие другие устремления ему как классу не свойственны. Но это неверно даже в рамках чисто марксистско­го подхода. Классовый интерес -- это интерес, порожденный функцией  данного класса в обществе. По Марксу, капиталист стремится к наживе не потому, что он жаден как личность, а потому, что такова его роль в обществе, в системе произ­водства, и если он будет вести себя иначе, то разорится и пере­станет быть капиталистом. Функция партийно‑государственно­го аппарата -- управлять страной. Он эту функцию и выполняет, однако далеко не наилучшим образом: тяжело опираясь на страх и широкий диапазон наказаний и не обеспечивая необ­ходимых условий для развития народного хозяйства и куль­туры. Более того, можно с уверенностью сказать, что если он не сменит стиля управления, то это приведет либо к полному окостенению с неизбежным разрушением от внешних причин, либо к революционному взрыву изнутри. Ни то, ни другое не соответствует классовым интересам правящего слоя. В его интересах была бы именно постепенная демократизация с вы­свобождением творческих сил народа, но при сохранении свое­го руководящего положения и власти. Трудности этого пути, в частности риск, что правящий слой не сможет сдержать про­цесс демократизации в определенных рамках, удерживают пра­вящий класс от шагов в направлении демократизации. Но какое отношение это имеет к социальным функциям  правящего слоя? Неспособность найти приемлемое решение в сложной ситуации, а именно, найти путь демократизации с уверенным сохранением своей власти, никак не может быть выведена из со­циальной функции или социальных интересов правящего слоя. Если люди по лени, трусости или глупости не стремятся к тому, что было бы для них идеальным, то так и надо говорить, а не выдавать человеческие недостатки за социальный классовый интерес. Но действительно ли эти недостатки у представите­лей правящего класса так велики, что полностью, в любых условиях, блокируют возможность демократизации? Откуда это известно? Только из того, что они до сих пор этого не сде­лали? Но ведь состав каждого социального слоя непрерывно обновляется, и каждый процесс имеет свое начало.

Второе положение, неявно содержащееся в "большевист­ской" точке зрения, состоит в том, что классы в обществе разделены как бы непроницаемыми стенками, обладают каж­дый своей культурой и моралью и борются между собой, вы­ступая каждый как единое целое за свои классовые интересы. Это -- марксистская вульгаризация реальной общественной жизни. Культура общества едина и оказывает огромное вли­яние на все классы, на всех членов общества. Классы не моно­литы, и борьба между классами отнюдь не единственный и не всегда самый важный фактор, определяющий развитие обще­ства. Деление и объединение людей по их психологическим качествам, по мировоззрению, по таким признакам, как честность, доброта и т.~п., не менее важно, чем деление на классы, а оно проходит через все социальные слои.

Марксисты обычно стремятся представить социальные сдвиги исключительно результатом борьбы угнетенных классов против правящих, игнорируя те изменения, которые происходят в правящих классах вследствие эволюции культуры. Между тем, эти изменения по меньшей мере столь же важны, как и прямое силовое сопротивление угнетенных, и если культура не эволюционирует, то, несмотря на периодические восстания угнетенных, их участь может не улучшаться на протяжении, столетий, что мы видим на примере ряда стран Востока. История европейской цивилизации определенно указывает на решающую роль эволюции культуры, а не силового фактора. Усо­вершенствование оружия и транспорта дало физическую воз­можность небольшой части общества держать в полном по­виновении всех остальных; технически это стало легче, чем в странах Древнего Востока. Тем не менее, европейская циви­лизация, если не считать некоторых отклонений, идет по пути непрерывного уменьшения уровня насилия управляющих над управляемыми. Тоталитаризм в восточных странах со всей наглядностью показывает, как можно повелевать людьми с помощью западной технологии при отсутствии западной культурной традиции.

Согласно марксизму, изменение общественных отношений является следствием развития материальной культуры: при определенном уровне производительных сил оказывается выгоднее иметь свободного арендатора, чем раба, и т.~д. Что этот эффект имеет место, столь же несомненно, как и то, что он не является решающим. Более важный и прямой эффект мы наблюдаем непосредственно вокруг себя, если только не закрываем глаза, чтобы изобрести наукообразное "материалистическое" объяснение. Просто эволюция общественного сознания (под действием сил, которые не выводятся из материального производства, несмотря на все усилия марксистов) приводит к тому, что и цели, и методы правящего класса меняются, и он уже не хочет и не может поступать в соответствии со ста­рыми рецептами. Возьмем англичан в Индии. Они представля­ли собой военно‑бюрократический правящий класс, и никаки­ми ухищрениями невозможно доказать, что в их "классовых интересах" было уйти из Индии. Если бы они были полны реши­мости остаться любой ценой и применяли бы в 20‑м веке столь же или еще более жестокие методы подавления, чем в 19‑м, получила бы Индия независимость? Я не хочу преуменьшить значения борьбы индийцев за независимость, но решающими ас­пектами этой борьбы были идейный и моральный. Если бы играли роль только экономические и военные факторы, то англичане не ушли бы из Индии. Физическая возможность у них была.

Необходимые условия демократических реформ -- в нашем образе мышления. Общественное сознание в своем существе едино, оно пронизывает все слои общества. Нельзя сваливать в большевистском духе всю вину на "новый класс". В стране, где ученые с мировым именем, выслуживаясь перед властью, способны поливать грязью своего единственно честного и мужественного коллегу, чего ожидать от партийных и государственных чиновников?

Нет, никто не убедит меня в том, что существуют какие‑то "объективные" причины, по которым невозможна постепенная демократизация. Все это лишь способы оправдания бездействия. Встанем мы или нет на путь, открывающий перспективы на достойное человека будущее, зависит только от нас самих. И если не встанем, то никаких оправданий этому не будет.



\section{Движение за права человека}

Советский человек, воспитанный в духе принципа "экономика -- базис, идеология и политика -- надстройка", склонен требовать от каждого, кто решается высказываться на обще­ственно‑политические темы, прежде всего конкретного проекта экономических преобразований (желательно с длинными столбцами цифр и диаграммами). Иной подход считается "несерьезным": ведь политика есть отражение экономических интересов; как же могут сограждане поддерживать вас, если они не знают, какие у вас конкретные планы в области экономики? А если у вас вообще нет таких планов, так о чем же говорить?

На самом деле в наших нынешних условиях именно этот подход -- с экономического конца -- является совершенно несерьезным. Из изложенной мною социальной философии следует необходимость либерализации экономики, повышения роли частной инициативы, перестройки экономической системы в духе принципа структурно‑функционального параллелизма. Но я не собираюсь никак конкретизировать эти общие принципы. Экономика -- огромная, сложнейшая система с мно­жеством запутанных косвенных связей, и сколько‑нибудь ответственный подход к модификации этой системы требует ее детального изучения. Дело не только в том, что я не экономист по специальности: даже очень знающий экономист не мог бы, я думаю, дать в одиночку обоснованный и достаточно конкретный план экономических реформ. Необходима работа большого коллектива в обстановке свободных обсуждений и свободного обмена информацией. Необходима свобода экспериментирования в широких масштабах. Эти условия -- политические, и пока они не выполнены, экономические проблемы не только не разрешимы, к ним даже невозможно найти конкретного подхода. Камнем преткновения у нас является политика, а не экономика.

То же относится к конкретным вопросам законодательства, административного управления, партийной жизни и т.~п. Серьез­но обсуждать их и искать решений можно только при условии соблюдения элементарнейших, самых основных гражданских и политических прав личности. Проблема основных прав лич­ности стоит особняком, это начало всех начал.

Под основными правами личности понимают обычно следу­ющие права, выраженные в четырех статьях Всеобщей декла­рации прав человека ООН (это понимание принято, в частности, организацией "Международная Амнистия"):

\textbf{Статья 5.}  Никто не должен подвергаться пыткам или жестоким, бесчеловечным или унижающим его достоинство обращению или наказанию.

\textbf{Статья 9.}  Никто не может быть подвергнут произвольному аресту, задержанию или изгнанию.

\textbf{Статья 18.}  Каждый человек имеет право на свободу мысли, совести и религии; это право включает свободу менять свою религию или убеждения и свободу исповедовать свою религию или убеждения как единолично, так и сообща с другими, публичным или частным порядком в учении, богослужении и выполнении религиозных и ритуальных обрядов.

\textbf{Статья 19.} Каждый человек имеет право на свободу убежде­ний и на свободное выражение их; это право включает свободу беспрепятственно придерживаться своих убеждений и свободу искать, получать и распространять информацию и идеи любыми средствами и независимо от государственных границ.

Сознание необходимости отстаивать основные права челове­ка породило в Советском Союзе движение, которое к насто­ящему времени имеет примерно десятилетнюю историю. В нем участвуют люди самых различных политических и философ­ских воззрений, оно не является политической партией, не имеет формального членства, организации, руководства и т.~п. Оно не имеет даже определенного названия: раньше его обычно назы­вали "Демократическим движением", теперь чаще называют "Движением за права человека". Тем не менее, оно реально су­ществует и представляет собой, несмотря на свой ничтожный в масштабе страны численный состав, серьезное общественное явление. Его участники ("диссиденты" или "инакомыслящие") связаны общим неприятием тоталитаризма, общими выступле­ниями против конкретных нарушений прав человека, общим интересом к неподцензурным литературным произведениям. Печатным органом Движения можно считать "Хронику текущих событий".

Так как у Движения за права человека в СССР нет определен­ной организационной структуры, то нельзя говорить и о нали­чии у него определенной программы. Существуют лишь общие принципы, разделяемые большинством участников Движения. В рамках этих принципов каждый диссидент высказывает свои собственные соображения и предложения.

Я хочу предложить следующий примерный план демократи­ческих реформ, который намеренно выражен не в терминах конкретных законодательных актов, постановлений прави­тельства и т.~п., а в общих терминах, и является поэтому ско­рее схемой мероприятий, чем их конкретным планом. Конкре­тизация этой схемы выходит за рамки моих задач как автора настоящей книги, она имеет смысл лишь на политическом уровне, то есть в процессе коллективных обсуждений и вы­ступлений (обсуждения желательны с участием представите­лей власти). Кроме того, она может меняться по ходу дела.

\paragraph{1.} Прекратить судебные и психиатрические преследования за обмен информацией и идеями, за критику общественной системы и власти, за проповедь религиозных убеждений и за изъявление желания покинуть страну. Обеспечить гласность всех открытых судебных заседаний, то есть возможность при­сутствовать на них любого советского или иностранного граж­данина, изъявившего на то желание заранее, а также возмож­ность делать фотоснимки и звукозапись.

\paragraph{2.} Объявить амнистию всем политическим заключенным, то есть узникам совести  в определении "Международной Амнистии" Это определение таково: люди, подвергаемые какому‑либо физическому принуждению либо из‑за своих религиоз­ных или политических убеждений, либо по какой‑либо другой причине, касающейся их убеждений, либо из‑за своего этниче­ского происхождения, цвета кожи или языка, при условии, что эти люди сами не применяли насилия и не призывали дру­гих к применению насилия. К этой категории людей относят­ся прежде всего все лица, осужденные по статьям 70 и 190‑1 УК~РСФСР или аналогичным статьям других союзных респуб­лик.

\paragraph{3.} Отменить предварительную цензуру печати и других средств массовой коммуникации.

\paragraph{4.} Разрешить свободный обмен людьми и информацией со всеми странами мира. А именно:
\begin{itemize}
 \item разрешить свободный выезд за границу и возвращение в страну;
 \item разрешить свободную продажу зарубежных газет, журна­лов и книг;
 \item прекратить произвольное задержание писем и других ма­териалов, посылаемых по почте за границу и из‑за границы;
 \item прекратить глушение радиопередач;
 \item прекратить отключение телефонов за не понравившиеся властям разговоры.
\end{itemize}

Во всех указанных видах обмена акты запрещения или изъя­тия могут предприниматься только по решению открытого су­да.
(Осуществить преобразования, указанные в этих четырех пунктах, можно было бы очень быстро (в течение месяца) без всякого риска для системы и власти. Это, собственно гово­ря, то, что советское правительство обязалось сделать на сове­щании в Хельсинки, и выполнение этого обещания сильно повы­сило бы престиж СССР за рубежом и открыло бы новые благо­приятные перспективы в отношениях со странами Запада. Сле­дующие два пункта потребуют больше времени и некоторой осторожности.)

\paragraph{5.} Вместо существующей у нас комедии "выборов" из одно­го кандидата, ввести выдвижение нескольких кандидатур на каждое место при выборах во все государственные и партий­ные органы.

\paragraph{6.} Разрешить все ассоциации граждан, не пропагандирующие насилия и не имеющие статуса политической партии, и предоста­вить им возможность нормального функционирования. В частно­сти, разрешить ассоциациям иметь свои, независимые от прави­тельства и партии, печатные органы и средства размножения. Свободная печать необходима современному обществу.

Было бы, конечно, прекрасно, если бы все эти мероприятия были проведены по инициативе сверху. Однако на такую инициа­тиву трудно рассчитывать. Необходимо давление снизу, то есть осуществление своих гражданских прав "в явочном порядке". Помогает этому то, что на словах наше государство считается свободным и демократическим, поэтому борьба за права челове­ка является в значительной степени приведением дел в соответ­ствие со словами.

Свобода ассоциаций, независимых от партийно‑государствен­ных органов, даже с учетом того ограничения, которое фигури­рует в моей программе, остается наиболее трудным пунктом. Но без свободы ассоциаций смешно говорить о правах человека и о демократии. В рамках Движения за Права Человека возникло несколько независимых ассоциаций, последней из которых яв­ляется Группа содействия выполнению Хельсинкских соглаше­ний, образованная в мае 1976 г. В перспективе, мне кажется, на­до надеяться на возникновение все большего числа независимых ассоциаций, связанных с самыми различными аспектами жизни. Возьмем, например, такой волнующий всех вопрос, как загряз­нение окружающей среды. Государственные органы вряд ли мо­гут обеспечить детальный и объективный контроль, ибо борьба за чистоту среды -- именно в интересах частных граждан, а не го­сударственных организаций. Необходимы независимые общест­венные организации. Другой вопрос такого рода -- безопасность на транспорте. Об авиационных катастрофах у нас в газетах сооб­щается только в том случае, если в числе погибших есть иност­ранцы. Статистика катастроф не публикуется. Очевидно, госу­дарство опасается, что, познакомившись с этой статистикой, граждане откажутся летать на самолетах. Один инженер сказал:

"Я никогда не летаю на самолете. И знаете почему? Я работаю на заводе, где делают самолеты, и знаю, как их делают". Секрет­ность в вопросе о катастрофах несомненно увеличивает число ка­тастроф, ибо она позволяет затушевать глубокие причины, вызы­вающие их. Секретность позволяет пренебрегать интересами ря­довых граждан ради удобства каких‑то лиц в иерархии государ­ственного управления. Это то самое, что в отношении капитали­стических стран на нашем пропагандистском языке называется "приносить интересы трудящихся в жертву интересам монопо­лий". Впрочем, когда речь идет о человеческих жизнях, слово "интересы" становится слишком слабым. Фактически, отказ го­сударства публиковать статистику катастроф и обстоятельства вокруг катастроф -- это завуалированное убийство им своих граждан. И здесь выходом из положения является только созда­ние независимых общественных организаций для контроля, так как государство -- слишком заинтересованная сторона. Наконец, на каждом предприятии, в каждом коммунальном хозяйстве есть множество вопросов, которые могут быть решены только при наличии независимых контролирующих органов, а независи­мые ассоциации ученых, художников и т.~д. могли бы сыграть огромную роль в плане чисто профессиональной деятельности.

Борьба за права человека во многих отношениях более труд­ное дело, чем революционная борьба за власть. Зато она приво­дит к устойчивым благотворным переменам в образе мышления и образе жизни общества. Я бы назвал ее скорее не борьбой, а ра­ботой. Это трудная, долгая и в наших условиях опасная работа. Однако это единственный путь к человеческому существованию. Это демократия в действии, демократия в массах. На этот путь встали и продолжают по нему упорно идти вперед народы веду­щих стран Запада. И другого пути нет.

\section{Восток и Запад}

\epigraph{Ныне во всех континентах, во всех районах Земного шара организованы и функционируют "кружки по изучению великих идей чучхе товарища Ким Ир Сена", "исследовательские институты по трудам товарища Ким Ир Сена", "общества читателей трудов товарища Ким Ир Сена", движение за изучение идей чучхе уже вышло из рамок кружковой деятельности и расширяется в порядке зонального и континентального общего движения\ldots

С каждым днем увеличивается число читателей бессмертных классических трудов великого вождя товарища Ким Ир Сена.

Журнал "Корея", №10 (229), 1975 г.

Доклад Генерального секретаря ЦК КПСС товарища Леонида Ильича Брежнева еще не был окончен, а огромный, всеохватывающий к нему интерес в Англии рос от часа к часу\ldots

Надо ли говорить, что сегодня в Англии нет ни одной столичной газеты, которая бы не отвела место на первой полосе репортажам, посвященным работе форума советских коммунистов, как правило, с портретами Генерального секретаря ЦК КПСС на трибуне съезда.}{"Комсомольская Правда", 26 февраля 1976 г.}



Если бы Советский Союз был единственной страной на Земном шаре, то, возможно, что уже сделанные им шаги на пути к стационарному тоталитаризму оказались бы необратимыми.

Однако действительное положение в мире не так мрачно. Страны свободного мира все еще занимают ведущее положение в мировом хозяйстве и культуре, и пока они сохраняют это поло­жение, победу тоталитаризма ни в какой "отдельно взятой" стране нельзя считать окончательной. Тоталитаризм не стал еще общемировой нормой, тоталитарным странам приходится считаться с мировым общественным мнением.

Эволюционирующее мировое общество предъявляет ко всем странам требования, которые не могут быть удовлетво­рены на пути застойного тоталитаризма. Поэтому, если тотали­таризм не захлестнет весь мир, то, надо полагать, свежий воз­дух рано или поздно проникнет и в Советский Союз, и в Ки­тай. Противное возможно только в случае жестокой изоляци­онистской политики со стороны тоталитарных стран. Реальна ли эта возможность? Сколько бы ни пытались китайцы или ко­рейцы убедить себя и других, что свет льется на современный мир с Востока, они прекрасно понимают, как понимаем и мы, что исторический процесс, охвативший сейчас весь мир, есть вестернизация,  усвоение и распространение культуры, возник­шей в Западной Европе. Распространяясь, западная культура впитывает в себя некоторые элементы восточных культур. Это, конечно, очень важно для будущей глобальной цивили­зации, но не меняет того факта, что в формирующейся культу­ре активной стороной является культура Запада, а не Востока. Тоталитарные страны -- это фронтовые зоны формирующейся культуры, отсюда вытекают основные черты отношения чело­века в этих странах к человеку "тыла", центральной зоны. Это смесь болезненного ощущения своей неполноценности со здо­ровым ощущением своей энергии, своего желания "догнать и перегнать". Длительный, систематический изоляционизм в по­добной ситуации немыслим.

Для тоталитарных стран, как и для стран Третьего мира, Западный мир является единственным и естественным масшта­бом, мерой всех вещей. Самовосхваление советской пропаган­ды, ее попытки внушить, что "социалистические" страны ста­ли центром мировой культуры, -- не больше, чем один из при­емов выполнения ею основной задачи ‑ дезинформации насе­ления. До какой‑то степени она достигает цели, и от "простого человека" можно иногда услышать словесные формулировки в этом смысле. Но в глубине души все без исключения знают, где масштаб и где точка отсчета. Знают это и вожди. Для вож­дей тоталитарных стран очень важно, чтобы изучение их "вели­ких идей" расширялось "в порядке зонального и континенталь­ного общего движения" и чтобы их еще не оконченные докла­ды вызывали в западных странах "огромный, всеохватываю­щий интерес", который бы "рос от часа к часу". А также, чтобы ни одна столичная газета не упустила случая напечатать их порт­рет.

Два обстоятельства играют роль в возросшей ‑ и возрас­тающей -- чувствительности советских руководителей к общест­венному мнению на Западе.

Во‑первых, произошел переход от революционной фазы советского общества к стационарной фазе. В революционной фазе многое прощается как временное, переходное. Накал страстей так велик, что на его фоне внешние воздействия теря­ются. Определенная доля изоляционизма в такие периоды неизбежна, общество замыкается само на себя. Стационарная фаза предъявляет новые требования к взаимодействию с внеш­ним миром, его роль увеличивается.

Во‑вторых, изменились настроения в важных для Советско­го Союза кругах западного общества. Советское государство никогда не было равнодушно к своему образу в западном ми­ре, оно и возникло как "прорыв цепи мирового капитализма", лозунг мировой революции не сходил со сцены в течение многих лет после октября 1917 года. Неверно думать, будто Сталин не заботился о престиже СССР на Западе: просто в тех кругах, на которые он опирался, то есть среди рабочих и левой интел­лигенции, этот престиж был неизменно высок, несмотря на все совершаемые режимом Сталина зверства. Желание верить в то, что первое в мире социалистическое -- без кавычек -- го­сударство в самом деле существует, было так велико, что люди отказывались признавать очевидные факты и восторженно аплодировали палачам. Пакт 1939 года с Гитлером был силь­ным ударом по престижу Советского Союза, но разгром фа­шистской Германии, героизм, проявленный народом в войне, и его тяжелые жертвы реабилитировали режим -- хотя это и нелогично -- в глазах многих людей на Западе. Возросло вли­яние иностранных компартий, которые тогда все без исключения рассматривали КПСС как единственно возможный образец для подражания и славословили Сталина. В 1945 году Теодор Драйзер обратился к председателю коммунистической партии США Фостеру с просьбой принять его в партию. Он писал:

"Вера в величие и достоинство человека всегда была руководящим принципом моей жизни. Логика моей жизни и моей работы привела меня в коммунистическую партию".3 И это в то время, как разработанная сталинцами массовая технология растаптывания человека, унижения его достоинства стали основой стабильности нового общества!

Только после XX съезда КПСС в 1956 году (бессмертное деяние Хрущева) стала пелена постепенно спадать с глаз. Эпо­ха гарантированных аплодисментов кончилась, и для поддер­жания престижа руководство оказалось вынужденным идти на уступки общественному мнению.

Политическое и культурное взаимодействие с внешним ми­ром делает необходимым и экономическое взаимодействие. Стараясь не слишком отстать от западного уровня жизни, мы ввозим пшеницу; стараясь угнаться за научно‑техническим про­грессом, мы ввозим компьютеры; стараясь не выглядеть дика­рями, мы следим за западной модой и снабжаем верхний слой общества заграничной одеждой, обувью, косметикой, предме­тами домашнего обихода и т.~п. В результате совокупного действия всех этих факторов западный мир продолжает оста­ваться фокусом, центром притяжения в психологии советско­го человека. Поездки за границу и особенно в страны Запада ‑одна из главных приманок, с помощью которых власти со­здают послушную им прослойку интеллигенции и постоянный рычаг воздействия на нее.

Трудно представить себе, чтобы кто‑то из главных начальни­ков в СССР взялся систематически свертывать и обрезать кон­такты с Западом. Он не найдет в этом деле поддержки. В борь­бе за власть изоляционистские лозунги возможны, они прак­тически не отделимы от призыва к завинчиванию гаек. Дескать, "распустили народ", и все из‑за этих контактов с Западом. Можно допустить, что такого сорта демагогия будет использована, чтобы кого‑то столкнуть сверху, а кого‑то поставить вместо него. Но будет наивен тот, кто примет эти слова все­рьез: они лишь прием в борьбе, тактический ход. Как только новые люди окажутся у власти, логика вещей заставит их нала­живать и углублять контакты с Западом. (Я не обсуждаю здесь конкретных политических вопросов, таких как отношения с Китаем, а ведь они играют далеко не последнюю роль.) Так бы­ло с Хрущевым, так обстоит дело с Брежневым, так будет при его преемниках в предвидимом будущем.


\section{Победа "реализма"}

Запад далеко не полностью использует свои возможности влияния на Советский Союз. Говоря о влиянии, я имею в виду, конечно, влияние в области основных прав человека, осуще­ствляемое через сферу культуры и торговлю. Находясь в здра­вом уме, никто не станет сейчас призывать к применению силы или к угрозам применения силы, каково бы ни было соотно­шение военных потенциалов сверхдержав. Межгосударствен­ные договоры, имеющие целью уменьшить опасность воору­женного конфликта, -- большое достижение политического ра­зума и здравого смысла; все это так очевидно, что нет необ­ходимости и говорить об этом. Но распространение своих идей через посредство мирных дружественных связей -- законное право каждого человеческого сообщества, а если оно по‑насто­ящему верит в справедливость своих идей, то не только право, но и обязанность. Идея свободы личности -- великая идея, ле­жащая в основе западной цивилизации. Пассивность и нереши­тельность Запада в распространении этой идеи -- свидетельство кризиса западной цивилизации, кризиса веры.

Джинсы и поп‑музыка легко пересекают государственные границы. Наверное, потому, что они всегда на виду, их воздей­ствие непрерывно. Но когда западный человек вступает в об­щение с советской системой, он свои идеи вежливо прячет в карман. Считается признаком хорошего тона при професси­ональных контактах между людьми из разных "лагерей" время от времени клясться друг другу в отсутствии намерения втя­нуть собеседника в "политический" спор. <<Ну, это уже политика>> --говорит советский ученый за границей, скажем, в Аме­рике, почуяв, что разговор поворачивается в опасном направ­лении. "О да, конечно, оставим это, ~‑ говорит американец с готовностью, и даже как будто извиняясь. -- Не будем зани­маться политикой: у вас свои взгляды, у нас -- свои. Наука нейтральна. Давайте выпьем за расширение научных связей, за дружбу между нашими народами!" Взаимное удовлетворе­ние и полное согласие. Дружба между народами. Разрядка меж­дународной напряженности. Детант. Меня только интересует вот что: понимает ли американец, что если наука и нейтральна, то он, занимая эту позицию, вовсе не нейтрален, а служит опо­рой и поддержкой тоталитаризма? Понимает ли он, что, отделяя себя от того, что он вслед за своим советским коллегой назы­вает "политикой", он в действительности отделяет себя от простой порядочности, от фундаментальных этических прин­ципов? Ведь для советского человека все, что выходит за рамки физиологических отправлений и указаний начальника по рабо­те, -- политика. Нравственность? Политика! Гуманность? По­литика!! Совесть? Политика!!! Простая искренность, отвлечен­ная от политических соображений (а если не отвлеченная, то какая же это искренность?), объявляется идейно порочной "абстрактной" искренностью. Так была заклеймена   статья журналиста Померанцева "Об искренности в литературе", по­явившаяся в первый период оттепели после смерти Сталина. "Литературная газета" писала: "Советской общественностью уже справедливо оценена статья В. Померанцева "Об искрен­ности в литературе" как идейно порочная, написанная с иде­алистических позиций, противопоставляющая принципам идей­ности, партийности литературы, критерию правдивого отобра­жения действительности абстрактную искренность".4

Граждане СССР делятся на две категории: граждане перво­го сорта, которым разрешают ездить за границу, и граждане второго сорта, которых за границу не пускают. Граждане пер­вого сорта ‑ это особо проверенные, надежные люди, относи­тельно которых начальство уверено, что они будут в точности следовать всем данным им указаниям и никогда не позволят себе проявить свое собственное лицо (ежели таковое существует). Причем, это относится к поведению не только за границей, но и дома, ибо, как я уже отмечал, поездка за границу является наградой, которую еще надо заслужить. <<Подпишешь письмо против Сахарова ‑ поедешь, не подпишешь ‑ не пое­дешь>> --это не фигуральное выражение, а буквальные слова, сказанные одному сотруднику института, где Сахаров рабо­тает. Таким вот образом и осуществляется выбор тех, кому разрешено выходить на международную сцену. Мировое сооб­щество санкционирует этот выбор. Рукопожатия, тосты, апло­дисменты.

Я далек от мысли предлагать бойкот культурных связей. И я отнюдь не против рукопожатий с лицами, облеченными доверием властей. Санкция -- не в том, что западный мир прини­мает тех, кто отобран властями. Пусть себе едут. И чем боль­ше, тем лучше. Санкция -- в том, что принимают только их, что отбор, производимый властями, стал нормой, против кото­рой никто всерьез не протестует. А что стоило бы, например, принять такой принцип: в каждом мероприятии по культур­ному обмену (научная конференция, обмен студентами, га­строльные поездки и т.~п.) небольшая часть участников, напри­мер, одна пятая или десятая, определяется по выбору проти­воположной стороны; если же указанным людям не дают раз­решения на поездку по политическим причинам, то все меро­приятие отменяется. (Этот принцип, разумеется, должен был бы проводиться исключительно на уровне общественного мне­ния в заинтересованных организациях, без участия правитель­ственных органов.) Но нет, западные партнеры предпочитают без звука принимать тоталитарные правила игры.

Волна протестов на Западе против преследований за убеж­дения в СССР в последние годы нарастает, и она приносит свои плоды. Среди ученых наибольшую активность проявляют мате­матики. Благодаря их усилиям, в частности, был в конце кон­цов освобожден из психиатрической больницы Л. Плющ. Во время подготовки 4‑ой Международной конференции по искус­ственному интеллекту (Тбилиси, сентябрь 1975 г.) группа американских членов оргкомитета пригрозила советским уст­роителям конференции бойкотом, в случае если проф. Лер‑неру (уволенному с работы после подачи заявления на выезд в Израиль) не будет предоставлена возможность выступить с докладом. Но все действия подобного рода в некотором смыс­ле "нетипичны": они инициируются отдельными поборниками прав человека скорее вопреки желанию основной массы, чем в соответствии с ним. Не они определяют атмосферу культур­ных связей. Атмосфера все еще такова, что преследование инакомыслящих и удушение свободы творчества -- это "внут­реннее дело" тоталитарных стран, в которое человек Запада не должен вмешиваться. Пока сохраняется эта атмосфера, культурный обмен работает на укрепление тоталитаризма: в назидание молодежи советские власти выводят на международ­ную арену послушных и заживо погребают непослушных.

Стационарный тоталитаризм нуждается в международном признании. И это ему удается. Тоталитаризм стал одним из спо­собов существования общества, столь же законным, как лю­бой другой. Представители тоталитарной культуры -- это коле­сики машины, перемалывающей человеческое сознание , -- разъез­жают по свету, не испытывая ни малейшего неудобства, а на­против, встречая всюду почет и уважение. Тоталитаризм стано­вится нормой. Пока -- одной из норм. Но не станет ли он завт­ра единственной  ("единственно‑научной", "единственно про­грессивной" и т.~д.) нормой?

Признание тоталитаризма нормой со стороны западной об­щественности приветствуется советской и просоветской про­пагандой как "реалистический подход", "политический реа­лизм", "победа реализма" и т.~п. Слово "реализм" эксплуати­руется со вкусом, с удовольствием. Понимать его надо, конеч­но, как примирение с реальностью тоталитаризма, как сдачу позиций. Этот же термин эксплуатируется и капитулянтами с западной стороны. Поскольку он несет положительную эмо­циональную нагрузку (ассоциируясь с трезвостью, с мудро­стью),сдача позиций представляется чуть ли не победой.

И еще есть одно понятие, которое служит постоянным пово­дом для спекуляций. Это "беспристрастность".

Что такое беспристрастность?

В августе 1973 г. голландская секция "Международной Ам­нистии" направила Международному исполнительному комитету в Лондоне документ под названием "Является ли Амнистия достаточно беспристрастной?". Соображения по этому поводу были наиболее ясно сформулированы в указанном документе г‑ном X. Левенбергом.5 Вкратце они сводятся к следующему.

Люди, поддерживающие "Международную Амнистию", сосре­доточены главным образом в странах Западной Европы и Се­верной Америки. В других же странах, причем не только тота­литарно‑социалистических, но и в странах Третьего мира, "Международная Амнистия" представлена лишь опекаемыми узника­ми совести. Это наносит серьезный ущерб представлению о "Международной Амнистии" как о нейтральной организации, которое всегда считалось ее козырем. Одну из причин непри­ятия "Международной Амнистии" автор усматривает в том, что в своей деятельности эта организация делает упор на граждан­ские и политические права человека, а социальным, экономи­ческим и культурным правам отводит подчиненное место.

Различие между этими двумя группами прав состоит в сле­дующем. Гражданские права необходимы для "свободы от страха"; государству предъявляется требование, чтобы оно не препятствовало свободному выражению идей и обмену информацией между гражданами. Необходимость соблюде­ния гражданских прав -- основа западноевропейского либера­лизма, ведущего начало от эпохи Возрождения. Социально‑экономические права имеют своей целью "свободу от нуж­ды". От государства требуется, чтобы оно тем или иным пу­тем эту  "свободу от нужды" обеспечило. Представление о социально‑экономических правах личности -- гораздо более позднего происхождения, это одна из основ социализма. В 20‑м веке под влиянием рабочего движения социально‑экономиче­ские права были включены в конституции многих капиталисти­ческих стран, однако в социалистической идеологии они зани­мают гораздо более видное место. По учению Маркса, произ­водительные силы и производственные отношения, которые отражаются в социально‑экономических правах, являются определяющим элементом, "базисом", а гражданские и поли­тические права есть нечто вторичное, "надстройка".

Таким образом, упор на ту или другую группу прав чело­века имеет отчетливо видимую идеологическую компоненту, что нашло отражение в истории соответствующих международ­ных документов. Всеобщая декларация прав человека ООН была результатом инициативы западных держав (1948 год). В 1966 году Международная конвенция по гражданским и по­литическим правам и Международная конвенция по экономи­ческим, социальным и культурным правам были приняты Ге­неральной Ассамблеей ООН после 18 лет дискуссий и ком­промиссов, которые показали, что первую группу прав про­двигали западные страны, а вторую -- социалистические и боль­шинство развивающихся стран.

Статьи 5, 8, 18 и 19 Всеобщей декларации прав человека, на которых основывает свою деятельность "Международная Ам­нистия" целиком покрываются Конвенцией по гражданским и политическим правам. "Вывод самоочевиден, -- пишет X. Левенберг, -- до тех пор, пока "Международная Амнистия" ограни­чивается категорией гражданских и политических прав, ее нейтральность, с точки зрения социалистических и развива­ющихся стран, воображаемая. Это ограничение означает от­каз признать социалистическую концепцию прав человека, и следовательно, отдает предпочтение капиталистической идео­логии\ldots Подводя итог, можно сказать, что из‑за своего молча­ливого предпочтения капиталистической концепции свободы (которое логически следует из одностороннего продвижения гражданских и политических прав) "Международная Амнистия" не смогла доказать правительству социалистических и развива­ющихся стран провозглашаемой ею нейтральности".

X.~Левенберг заканчивает свои соображения рекоменда­цией изучить, какую роль в отрицательном отношении этих стран к "Международной Амнистии" играют экономические и со­циальные права человека, выраженные в статьях 20 и 23 Всеоб­щей декларации прав человека, с тем чтобы предпринять, ес­ли это окажется необходимым, пересмотр целей организации.

В этом подходе к проблеме меня удивляет диспропорция между соображениями о нейтральности и беспристрастности "Международной Амнистии", с одной стороны, и целями и прин­ципами этой организации, с другой стороны. Как будто не беспристрастность должна вытекать из целей и принципов "Международной Амнистии" а ее цели должны приспосабливаться к чьему‑то представлению о беспристрастности, причем далеко не беспристрастному представлению!

Г‑н~Левенберг совершенно справедливо противопоставляет друг другу две концепции прав человека, а по существу, две концепции общества. Западная концепция провозглашает граж­данские и политические права личности основой основ чело­веческого общежития, непреложным принципом, который на­до соблюдать здесь и сейчас. Экономическая и социальная спра­ведливость рассматривается скорее как цель, как идеал, об осуществлении которого можно говорить лишь в смысле отно­сительном. Тоталитарно‑социалистическая концепция общества полностью отрицает гражданские и политические права лично­сти, но частично признает ее социально‑экономические права:

в той степени, в которой это необходимо для функционирова­ния государственной машины. Однако, поскольку отрицать права человека сейчас не модно, идеологи тоталитаризма не­сколько приукрашивают свою концепцию для внешнего упот­ребления. В результате на словах признаются все права челове­ка, но на первое место выдвигаются социально‑экономиче­ские права. Фактически, это лишь уловка, имеющая целью от­влечь внимание от проблемы гражданских прав. Вряд ли у ко­го‑нибудь из серьезных людей могут быть сомнения на этот счет. Сам г‑н Левенберг пишет об этой концепции, как мне ка­жется, с явной иронией: "Когда все формы экономической эксплуатации будут искоренены и будет построено бесклассо­вое общество, иными словами, когда экономические, социаль­ные и культурные права будут гарантированы политически, законодательство автоматически приспособится к новому экономическому базису. И только тогда будут созданы условия для осуществления гражданских и политических прав".
Итак, перед нами две концепции, две идейные платформы. Что же должна означать беспристрастность в этом контексте? Что 

"Международной Амнистии" следует отказаться от выбора между этими двумя концепциями и признать их равноправны­ми? Но это было бы безыдейностью и беспринципностью, а не беспристрастностью. Беспристрастность международной органи­зации означает лишь равное применение ее идей и принципов ко всем странам, общинам, людям.

Акцент на гражданских и политических правах человека -- это идейная основа" Международной Амнистии," на которой она возникла и 
приобрела влияние в международной жизни. Социально‑экономические права -- совсем другая проблема; пусть важная, но другая. 
Существует огромное операционалистское различие между гражданскими правами и социально‑экономическими правами. Первые 
сравнительно легко опреде­лимы и проверяемы. Требование гражданских свобод состоит в том, что государство не должно преследовать 
людей за выра­жение убеждений и обмен информацией. Поэтому совершенно ясно, что должно сделать государство, чтобы удовлетворить 
это требование: просто прекратить преследования. Социально‑экономические права допускают множество  неоднозначно­стей в трактовке 
и проверке. Но самое главное ‑ их осуще­ствление требует сложного комплекса экономических, полити­ческих, социальных и культурных 
мероприятий и не известно­го заранее количества времени, причем относительно того, ка­кие именно нужны мероприятия, может быть 
множество раз­личных теорий и мнений, порождающих различные полити­ческие течения. Универсального рецепта, как достичь изобилия и 
социальной справедливости, увы, не существует. Критика социально‑экономических условий в отдельных странах неиз­бежно привела бы 
к распаду "Международной Амнистии" на враждующие политические фракции. И тогда она не смогла бы выполнить свою основную функцию, 
ради которой она была создана -- стоять на страже гражданских и политических прав. Именно эту цель ‑ политизации и распада 
<<Международной Ам­нистии>> -- преследуют тоталитарные страны, стараясь привлечь внимание к социально‑экономическим правам в ущерб граж­данским правам.

Другой важный принцип, входящий в идейную основу " Меж­дународной Амнистии" это ограничение сферы защиты лишь теми лицами, которые не применяют и не пропагандируют на­силия. Этот принцип тоже подвергается кое‑кем критике. Ар­гумент таков. Бывают такие режимы, которые, с одной сторо­ны, бесчеловечны и отрицают основные права личности, а с другой стороны, могут быть изменены только насильственным путем; в этих случаях насилие надо признать оправданным.

Эти доводы совершенно неосновательны. Допущение насилия хотя бы в одном случае будет означать, что "Международная Амнистия" взяла на себя функцию общей оценки политических режимов и даже выбор средств для их свержения. Это ради­кально изменило бы идейную основу организации. От полити­ческой нейтральности не осталось бы и следа." Международная Амнистия" сильна тем, что она борется именно за гражданские права,  а не против режимов,  которые эти права нарушают. В этом различии и проявляется нейтральность "Международной Амнистии"

Беспристрастность не есть безыдейность. Да, акцент на граж­данских правах ‑ западная идея. Значит ли это, что междуна­родная организация глобального характера должна от нее отка­заться? Но ведь это -- великая идея, необходимая всему миру и незападным странам в первую очередь.

В истории западной цивилизации социально‑экономические права человека были завоеваны (в той степени, в которой они завоеваны) после и в результате признания гражданских прав. Взрывоподобное развитие промышленности и науки, произо­шедшее в новое время, неотделимо от западной идеи о сво­боде личности. Однако для стран, проходящих стадию инду­стриализации в 20‑м веке, ситуация складывается иначе. Исполь­зуя западную технологию и экономическую помощь великих держав (то есть, в конечном счете, ту же технологию), прави­тельства этих стран получают возможность удовлетворить первичные потребности своих граждан, отказав им в то же время в элементарных гражданских правах. Опять‑таки, благо­даря   западной технологии в виде современного оружия и средств массовой коммуникации правительства могут кон­тролировать интеллектуальную и эмоциональную жизнь граж­дан с немыслимой ранее эффективностью.

Этот тоталитарный подход к проблеме модернизации -- со­блазн для руководителей страны, сулящий власть и богатство, для нации же в целом ‑ это самообман, приводящий к види­мости успеха лишь в первое время. Это опьянение, "алкого­лем" которого является технология, созданная Западом. Поэто­му и долг Запада -- противодействовать тоталитарному опья­нению. Нет ничего удивительного, что "Международная Амнития" черпает свою силу в западных странах. В свое время евро­пейские колонисты и торговцы, продавая спиртные напитки туземцам, доводили до вырождения целые селения и народы. Останется ли западная общественность равнодушной к тому, 1-го это повторяется теперь в новом качестве и в новом мас­штабе, грозя затопить весь мир, в том числе и Запад? От потен­циальных потребителей "алкоголя" не приходится ожидать сопротивления -- ведь они еще не знают, что это такое.
То, что в странах Запада так много людей поддерживают "Международную Амнистию" дает основания для надежды. В маленькой Голландии -- 15 тысяч членов "Международной Амнистии". Это замечательно.


\section{Беспристрастность в поляризованном мире}

Тот факт, что "Международная Амнистия" построена на за­падных идеях, не мешает ей быть организацией глобальной и политически беспристрастной. К сожалению, западное обще­ственное мнение часто бывает склонно в стремлении к ложно понятой "беспристрастности" и "нейтральности" добровольно сдавать позиции в идейной борьбе с тоталитаризмом, прино­сить в жертву жизненно важные идеи. Это проявилось недавно в реакции некоторых кругов на присуждение А.~Д.~Сахарову Нобелевской премии мира -- крупное событие в плане борьбы за права человека и взаимоотношений между Востоком и За­падом.

Один из старейших и известнейших научных журналов ми­ра "Нейчур" пишет в редакционной статье:

"Присуждение Нобелевской премии мира академику А.Д. Сахарову удивило почти всех. Несомненно, одно время его можно было рассматривать как кандидата на Нобелевскую премию по физике, но мало кто думал о нем в контексте премии мира. Так было ли решение комитета в Осло вдохновенным жестом, направленным на расширение сферы "мира" путем включения в нее прав человека как <<единственного прочного основания для подлинной и долговечной системы международного сотрудничества>> --как гласит формулировка премии, или это был просто политический акт, который мог быть совершен из относительной безопасности Скандинавии и был рассчитан на причинение Советскому Союзу некоторых неприятностей?

Прежде всего необходимо сказать, что деятельность Сахарова -- вначале как физика, затем как пропагандиста ограничения вооружений, затем как социал‑демократа, борющегося за гражданские права, -- достойна восхищения. Ес­ли присуждение премии этого года и оспаривается, то спор ни­коим образом не идет вокруг личных качеств академика Са­харова. Далее, не подлежит сомнению, что присуждение пре­мии поддержит Сахарова, если поддержка необходима, в его работе.

Все это хорошо. Но даже если Запад может в целом рассмат­ривать эту премию как присужденную за работу в области прав человека, Советский Союз несомненно рассматривает ее как образчик политического цинизма, имеющий целью поддержку смутьянов. Если бы Нобелевский фонд был политической ор­ганизацией, созданной для поддержки западных идеалов, это не имело бы большого значения -- но тогда и Нобелевские премии имели бы только значение Сталинских премий. Пре­тендуя на глобальное значение, Нобелевские премии должны быть свободны от той двусмысленности и поляризации, кото­рые были порождены премией этого года; и не только этого года -- недавние награждения Вилли Брандта, Ле Дык Тхо и Генри Киссинджера, все были в своем роде политическими и спорными".6
Я был чрезвычайно обрадован решением Нобелевского комитета. Потому что не было в 1975 году человека, в большей степени заслужившего Нобелевскую премию мира, чем акаде­мик Сахаров. Формулировка премии верна: соблюдение основ­ных прав человека в глобальном масштабе -- единственная на­дежная основа мира. С закрытым обществом подлинного мира быть не может. Вклад Сахарова в дело прав человека -- вклад в дело мира, и он значительнее, чем торжественно провозгла­шаемые обязательства, которые нарушаются на следующий день после подписания. Далее, при присуждении премии мира играет роль не отдельный продукт личности -- научное открытие, художественное произведение -- а личность в целом, ее влияние на современников. Мы видим в Сахарове одну из немногих титанических личностей нашего времени.

Все эти соображения, для меня совершенно несомненные, отнюдь не были очевидными, как мне было известно, для широ­кой публики на Западе. Решение Нобелевского комитета было нетривиальным, оно требовало известного мужества и вызвало приятное удивление.
Статья в "Нейчур", напротив, вызывает неприятное удив­ление. Дело не в том, что кто‑то не согласен с решением Нобе­левского комитета, дело в аргументах, выдвигаемых журналом. Я готов был бы выслушивать и обсуждать любые доводы отно­сительно вклада Сахарова в дело мира. Но вклад Сахарова в дело мира в статье не обсуждается, а его личности дается высо­кая оценка. Единственная причина, по которой "Нейчур" кри­тикует решение Нобелевского комитета, -- что оно вызовет недовольство Советского Союза. Итак, неважно, какова роль Сахарова в международной жизни, неважно, заслужил он пре­мию мира или нет, -- но раз советские власти рассматривают Сахарова как "смутьяна", от присуждения премии надо было воздержаться.

И это называется нейтральностью? Беспристрастностью? Тогда что же такое пристрастность? И что такое подобостра­стность --  худший вид пристрастности, проистекающей от стра­ха перед силой?

Когда "Нейчур" сопоставляет Нобелевские и Сталинские премии, аргументация становится на первый взгляд убедитель­ной. В самом деле, очень не хочется, чтобы Нобелевские пре­мии были как Сталинские. Но при более внимательном анали­зе убедительность этого аргумента рассыпается. Для такого анализа нам снова придется вернуться к понятию бесприст­растности.

Мы находим в современном мире две противостоящие иде­ологии: западную (либерально‑демократическую) и восточ­ную (тоталитарно‑социалистическую). Им соответствуют проти­востоящие политические блоки, обладающие противостоящи­ми политическими целями: каждая сторона хотела бы, чтобы человечество приняло ее идеологию и устроило жизнь по ее модели. Что же такое политическая беспристрастность в этой ситуации? И существует ли она вообще?

Западная и восточная идеологии дают на этот вопрос  проти­воположные ответы.

Западная идеология утверждает, что беспристрастность су­ществует и в определенных ситуациях необходима. Человече­ский мозг обладает способностью смотреть на себя как бы со стороны: свою систему идей, оценок, целей рассматривать с точки зрения более обширной метасистемы. В частности, человек может отвлечься от своих целей, как бы страстно он к ним ни стремился в действительности, и анализировать вещи так, как если бы этих целей у него не было. Это и есть беспристрастность. Без нее не было бы науки и была бы невозможна разумная коррекция целей.

Восточная идеология утверждает, что никакой беспристрастности -- во всяком случае, в вопросах, связанных с общественной жизнью, -- нет и быть не может. Каждая мысль и каждый поступок индивидуума непосредственно служит целям того лагеря, к которому он принадлежит. Беспристрастность -- либо обман, либо самообман. Обычно о беспристрастности говорит тот, кто тайно перешел на сторону врага.

Различие между Сталинскими и Нобелевскими премиями состоит в том, что Сталинские премии должны быть политически пристрастны и по замыслу, и по осуществлению, а Нобелевские премии по своему замыслу должны и могут быть беспристрастными. Конечно, человек -- существо несовершенное, и в осуществлении этого замысла неизбежны какие‑то проявления пристрастия. Однако сам замысел и наличие в арсенале западной цивилизации определенной традиции беспристрастного суждения, умения быть беспристрастным заставляет мыслящих людей во всем мире, в том числе и в Советском Союзе (а также и советских руководителей!), относиться к Нобелевским премиям совсем не так, как к Сталинским. Что ка­сается данного случая, то именно отказ от присуждения Нобе­левской премии Сахарову по политическим соображениям, как этого хотелось бы журналу "Нейчур", был бы отклонением от замысла. Люди ожидают, что Нобелевский комитет будет руководствоваться исключительно своим видением вклада кандидатов в дело мира, отвлекаясь от политических целей как своей, так и противоположной стороны.

В выступлении "Нейчур" мы наблюдаем странное явление: люди, явно принадлежащие к западной стороне, стремясь как будто к беспристрастности, призывают на деле к политиче­ской дискриминации в пользу восточной стороны. Причина этого явления кроется в том фундаментальном факте, на кото­рый я указал выше: подлинное, без кавычек, понятие бесприст­растности -- понятие целиком западное. Мнимая беспристраст­ность, являющаяся на деле безыдейностью, призывает к отказу от этого понятия как чуждого и неприемлемого для восточной стороны. Беспристрастность заменяется на политический праг­матизм, проще говоря -- на постоянную заботу о том, чтобы не разгневать Москву. Во многих случаях (и присуждение Но­белевской премии мира академику Сахарову ~ один из них) беспристрастный подход приводит к неблагоприятным для тоталитарного лагеря результатам. Тогда мнимая "беспристраст­ность" выступает в качестве защитника теории и практики то­талитаризма.

"Нейчур" заканчивает свою статью словами: "Мы отмечали и раньше, что Нобелевские премии часто оставляют за собой разрушительный след. Это особенно так по отношению к Пре­мии мира: Нобелевскому фонду пора серьезно заняться поис­ком других способов поддержки своих достойных идеалов". Из приведенной ранее части статьи мы видим, что эти другие способы не должны были бы, по мысли журнала, порождать "поляризации". Но как можно избежать поляризации в нашем поляризованном мире, когда даже понятие беспристрастности отвергается одной из сторон? Очевидно, лишь отказавшись от беспристрастности!

Современному миру жизненно необходимы внеполитические беспристрастные организации глобального охвата, такие, как Нобелевский фонд и Международная Амнистия. И в идей­ном, и в организационном отношении они могут базироваться только на Западе. На Востоке ничего подобного не может быть по определению. Смешанная восточно‑западная организация мо­жет оказаться способной на компромисс, но никак не на бес­пристрастность. Однако и в западных условиях сохранение бес­пристрастности ‑ дело нелегкое. Критика таких организаций, основанная на рассмотрении концепций и решений по существу, необходима и конструктивна. Критика, порожденная политическим конформизмом, -- деструктивна.



\section{Слабость Запада}

\epigraph{И все же, пока я сам не попал на За­пад и не осмотрелся здесь два года, я не мог бы представить, до какой крайней степени Запад желает быть слепым к мировой ситуации, до какой крайней степени Запад уже обратился в мир потерянной воли, цепенеющей перед опасностью и более всего угнетенный необходимостью защищать свою свободу.}{А.И. Солженицын [7]}



Мало кто на Западе питает сейчас иллюзии насчет "страны победившего социализма", но масштаб угрозы, которую пред­ставляет тоталитаризм как мировое явление, далеко еще не осознан. Россия первой попала в эту волчью яму, а потом затянула туда еще несколько стран. Очевидно, только тот, кто побывал в этой яме -- не туристом, не заглядывая сверху, а живя внизу, -- понимает, что это такое и как трудно оттуда выбраться. Когда он рассказывает об этом человеку Запада, тот недоверчиво пожимает плечами. Он не хочет замечать яму, которая залегла буквально у его ног.

Сахаров и Солженицын не раз выступали с предупреждениями о серьезности тоталитарной угрозы. <<Я -- не критик Запада. Я -- критик слабости Запада>> -- сказал Солженицын. Больше всего от него досталось англичанам, и это их, кажется, сильно задело. Выступления Солженицына в марте 1976 г. по Би~Би~Си обсуждались во всей стране. В передаче из серии "Панорама", организованной Би~Би~Си для обсуждения выступлений Солженицына, участвовали весьма видные лица: бывший премьер‑министр Э. Хит, американский сенатор X. Хамфри, бывший министр обороны США Дж. Шлезинджер и генераль­ный секретарь НАТО д‑р~Дж.~Луне. Критике Солженицына они противопоставили ряд аргументов в пользу того, что Запад не так уж слаб. Какие же это были аргументы? Военные, политические, экономические. Но соображения Солженицына касательно этих сфер -- дело второстепенное. Сущность его кри­тики в другом. Солженицын -- писатель, его дело -- смотреть, чем люди живы. Эти‑то наблюдения и привели его к выводу о слабости Запада. Он не увидел активной веры в высшие цен­ности, без которой не может быть мужества и единства. Он уви­дел оппортунизм и разобщенность. И он ясно сказал, что имен­но в этом он прежде всего видит слабость Запада. Сопоставь­те эту картину с механическим, железным единством тотали­тарной машины и вы поймете, откуда берутся апокалиптические ноты в выступлениях Солженицына. Дефект тоталитариз­ма--в неспособности к творчеству, к созданию чего‑либо ради­кально нового. Но эта слабость компенсируется потоком науч­но‑технической информации из свободного мира. А в сфере собственно военной или военно‑политической он обладает преимуществами единства, концентрации и дисциплины. То­талитаризм не изобретает пороха, но сумеет им лучше восполь­зоваться. Когда думаешь о надвигающейся биологической ре­волюции, в голову приходят самые мрачные мысли.

Западные обозреватели нередко описывают ситуацию в столь же апокалиптических выражениях, как и Солженицын. У.~Лакер и Л.~Лабедз в статье, озаглавленной "Вопрос жизни и смерти", пишут:

"Америка и другие демократии Запада стоят перед лицом наиболее жестокого кризиса в своей истории -- тяжелой эконо­мической депрессии в сочетании с быстрым уменьшением их влияния в международных делах и параличом и беспорядками на внутреннем фронте\ldots Есть все основания утверждать, что нынешний кризис имеет совершенно беспрецедентный харак­тер. Это кризис обществ, их норм и ценностей, кризис исчез­новения тех общепринятых принципов, которые в прошлом придавали обществу единство. Поэтому сравнение с ранее слу­чавшимися кризисами мало что дает, и надежда, что положе­ние нормализуется с подъемом деловой активности к концу года ‑ или в крайнем случае в будущем году, -- мало обосно­вана". [8]

Преобладающее настроение на Западе эти авторы характе­ризуют как "пессимистический детерминизм". Пророчества обреченности производят наркотическое действие, парализуют политическую волю, представляя поражение Запада немину­емым. Во внешней политике европейских стран преобладает узкий национализм. Параллельно этому на внутреннем фрон­те проявляется растущее стремление различных групп населе­ния преследовать свои узкогрупповые интересы, невзирая на трудности, которые это доставляет другим группам и стра­не в целом. Лидирует в этом отношении Великобритания. "По мере того как разрушается общественное взаимопонимание, анархия в Великобритании все возрастает, а остальные евро­пейские страны следуют в том же направлении, отставая лишь ненамного. Вряд ли можно применить термин классовая борьба к  той борьбе, которая сейчас происходит во многих евро­пейских странах; классовая борьба неизбежна в обществе, раз­деленном на классы, это неизбежный спутник демократиче­ского процесса. Но нынешняя ситуация все в большей степени характеризуется новым явлением -- некоторые професси­ональные группы беззастенчиво добиваются увеличения зара­ботной платы, игнорируя не только общее экономическое положение, но и интересы хуже оплачиваемых групп, которые не имеют возможности действовать подобным же образом. Солидарность рабочего класса уступает место закону джунг­лей, когда небольшие группы специалистов, техников или ра­бочих фактически парализуют целые отрасли промышленно­сти, вопреки желанию большинства. Это возврат к практике средневековых гильдий в сочетании с идеологий (если это идеология) социального дарвинизма и принципа невмешатель­ства. Но в то время, как в Средние века существовала власть -- будь то папы или императора, -- которая могла призвать к поряд­ку, власть в демократических обществах Запада непрерывно слабеет, а в некоторых странах разрушилась совершенно. Со­временное общество, в отличие от джунглей, не может функ­ционировать без некоторого минимума порядка. Поэтому альтернативой анархии является возникновение авторитарных режимов, если только разум, ответственность и понимание долгосрочных интересов не утвердят себя вовремя, чтобы предотвратить эффект бумеранга". [9]

\section{Маркузе}

Вспышка ярости в левом движении 60‑ых годов свидетель­ствует, во‑первых, о наличии недовольства среди молодых лю­дей, приходящих в современное западное общество, о жела­нии каких‑то перемен, а во‑вторых, о полном непонимании то­го, какие же именно перемены нужны и как их осуществить. Сочетание этих двух черт, прежде чем оно воплотилось в социальных акциях "новых левых", было воплощено в философских работах их идейных вождей и прежде всего -- Герберта Маркузе. Я читал только одну книгу Маркузе, "Одномерный человек", но и ее вполне достаточно, чтобы составить представление об этом поразительном явлении, этом сочетании страстной проповеди перемен с полным отсутствием конструктивных идей о переменах. <<Одномерный человек>> --это книга посвященная систематической критике современного западного общества. От такой книги не обязательно требовать какой‑то конкретной социально‑политической программы. Но идейная позиция автора, его философия, проявляется в критике с полной определенностью.
Что поражает в книге Маркузе -- это отсутствие понятия об эволюции.  При всей страсти к переменам философия Маркузе антиэволюционна. Ибо эволюция это не просто последовательность перемен, а последовательность перемен, подчиненная определенному общему плану. Говорить об эволюции -- значит говорить об этом плане. Ничего подобного у Маркузе мы не находим. Напротив, мы обнаруживаем у него полное непонимание сущности и механизма эволюции. Без эволюции нет и революции, ибо последняя есть лишь форма первой. Позиция "перемены ради перемен" может породить лишь бессмысленное буйство, что и произошло во время студенческих беспо­рядков 1968 года.

В предисловии к своей книге Маркузе пишет:

"\ldots Развитое индустриальное общество встречает критику ситуацией, которая, по‑видимому, лишает ее самой ее основы. Технический прогресс, распространенный на всю систему доминирования и координации, создает такие формы жизни (и власти), которые примиряют силы, противостоящие системе, и обрекают всякий протест на поражение во имя исторических перспектив свободы от эксплуатации и доминирования. Современное общество способно, по‑видимому, сдерживать социальные перемены -- качественные перемены, которые привели бы к созданию существенно новых учреждений, к новому направлению процесса производства, новым способам человеческого существования. Это сдерживание социальных перемен являет­ся, пожалуй, наиболее выдающимся достижением развитого индустриального общества; общее принятие Национальной Цели, двухпартийная политика, упадок плюрализма, тайный сговор Бизнеса и Труда в рамках сильного Государства сви­детельствуют об интеграции противоположностей, которая яв­ляется результатом, так же как и предпосылкой этого дости­жения". [10]

Начало этого отрывка располагает нас к автору. Если обще­ство утратило способность меняться, то это явно нехорошо. В нашем воображении возникает представление о конце пути, о загнивании и неизбежном распаде. Мы склонны сочувство­вать критику, который обвиняет общество в этом. Но читая дальше, мы видим, что причиной и следствием этой остановки автор считает интеграцию противоположностей, он отождест­вляет интеграцию противоположностей с остановкой. С этим никак нельзя согласиться. Интеграция противоположностей -- один из аспектов метасистемного перехода, необходимый эле­мент развития, так что наличие этого элемента само по себе ни в коей мере не свидетельствует о прекращении развития, об остановке. Интеграция противоположностей в процессе раз­вития происходит благодаря созданию (и в процессе создания) нового уровня иерархии по управлению, который осуществля­ет координацию противоположных элементов в рамках целого. Возьмем простейший пример. В процессе развития двигатель­ного аппарата появляются мышечные волокна, сокращение ко­торых имеет взаимно противоположные последствия: скажем, мышцы, которые сгибают и разгибают конечность в одном и том же суставе. Одновременное сокращение этих мышц было бы бессмысленным. "Борьба" этих "противоположностей" (в марксо‑гегелевских терминах) или "победа" одной из них не приведет к конструктивной эволюции. Для осуществления эволюционного сдвига (он же революционный сдвиг) необхо­димо создание зачатков нервной системы, которая управляла бы сокращениями противоположных мышц и обеспечила бы возможность ритмических движений. Это -- метасистемный переход, включающий в себя интеграцию противоположностей. Он происходит в контексте  борьбы противоположностей, но отнюдь не сводится  к ней и не является ее прямым следствием. Он требует некоторого творческого усилия, он требует встать над  борьбой противоположностей.

Означает ли интеграция противоположностей конец разви­тия? Никоим образом. Просто действие переносится на следу­ющий уровень; точка приложения творческого усилия переме­щается на одну ступеньку вверх. Теперь будет совершенство­ваться нервная система. Простые механизмы ритмического движения будут интегрироваться в более сложные планы по­ведения. И здесь тоже процесс развития будет происходить пу­тем количественного накопления и качественных скачков -- метасистемных переходов.

Нужно совершенно не понимать принципов эволюции, что­бы обрушиваться на "сговор Бизнеса и Труда в рамках силь­ного Государства" как на препятствие для дальнейшего дви­жения вперед к "новым способам человеческого существова­ния". В действительности это является необходимым этапом, через который нельзя не пройти, шагом, который так или ина­че надо сделать. Этот шаг относится только к системе матери­ального производства. Сделав его, надо переносить точку при­ложения усилий на следующий, более высокий уровень -- в об­ласть человеческих взаимоотношений, культуры, высших целей.

Если общество неспособно на этот перенос, на этот следу­ющий метасистемный переход, то оно действительно обречено на остановку в развитии. Но не интеграция противоположно­стей на уровне материального производства тому виной, а гораз­до более глубокие причины в сфере культуры. Их‑то и надо вытаскивать на свет божий. А оживление противоречий в систе­ме производства, как и борьба против других форм интегра­ции противоположностей на низших уровнях, может привести лишь к регрессу, к движению назад. Поясним это снова на при­мере мышц. Когда выработан механизм простого ритмического движения, следующий этап -- приспособление этих движений к текущей ситуации. Для этого нужны датчики информации о внешнем мире -- органы чувств; нужна также система коор­динации движений, которая ставила бы движение в зависи­мость от этой информации. Поможет ли решению этой задачи "борьба противоположностей" между мышцами‑сгибателями и мышцами‑разгибателями? Одновременное сокращение проти­воположных мышц -- это судорога, которая отнюдь не улучшает координации движений.

Вся философия Маркузе -- это тоска по судороге и попытка вызвать судорогу, попытка не вполне безуспешная.

Метафизический стиль мышления, отсутствие четкости в представлениях о развитии приводят к ложному образу, со­гласно которому развитие как бы порождается борьбой про­тивоположностей -- нечто вроде выработки нового вещества в процессе взаимодействия "противоположных" веществ, как, скажем, кислота и щелочь. Этот образ наводит на мысль, что, когда запас "противоположностей" исчерпывается, развитие прекращается. Маркузе очень сокрушается, сравнивая поло­жение, в котором находится критика капиталистического об­щества сейчас, с положением, в котором она находилась при своем возникновении в первой половине 19‑го века. Тогда она "приобретала конкретность", опираясь на борьбу противо­стоящих друг другу классов ‑ буржуазии и пролетариата. "В капиталистическом мире эти классы все еще являются основ­ными классами. Однако капиталистическое развитие измени­ло структуру и функцию этих двух классов таким образом, что они больше не являются агентами исторических преобра­зований. Заинтересованность в сохранении и улучшении инсти­туционального статус кво, которая стала решающим фактором, объединяет прежних антагонистов в наиболее развитых рай­онах современного общества". [11] Казалось бы, из этого фак­та следует сделать вывод о том, что надо менять точку прило­жения критики, переносить ее в новые сферы, на другой уро­вень. Это было бы естественной реакцией здорового творче­ского разума, который видит в критике активное начало, пре­образующее мир. Но не такой вывод делает Маркузе. Он мыслит в терминах исторического материализма, для которого всякая мысль, в том числе и критическая, есть отражение социальной реальности. Критическая мысль -- не фактор, вносящий в жизнь новое, а лишь форма проявления объективных социальных противоречий. Маркузе пишет: "В отсутствие обнаружимых агентов социальных перемен критика отбрасывается на высо­кий уровень абстрактности. Нет почвы, на которой могли бы встретиться теория и практика, мысль и действие".[12] Но ведь обнаружимость агентов перемен зависит от того, с каких пози­ций ведется критика. Если постулировать, что развитие может быть только следствием борьбы противоположных интересов буржуазии и пролетариата, а потом констатировать интегра­цию противоположностей, то мы тривиальным образом при­ходим к выводу о невозможности развития и закрываем себе путь для дальнейших поисков. Самое большее, на что может рассчитывать Маркузе, -- это взрыв, разрушение общества. Словами предисловия: "Одномерный человек  будет все время вращаться в кругу двух противоположных гипотез: (1) что развитое индустриальное общество способно сдерживать ка­чественные перемены в течение предвидимого будущего; (2) что существуют силы и тенденции, которые могут сломать сдерживающее начало и взорвать общество".[13] Возможность конструктивной эволюции отвергается даже как гипотеза, по­нятия такого нет. Либо статус кво, либо судорожное напряже­ние противоположных сил, ломающее уже созданные регуля­торы и отбрасывающее общество назад, к золотому веку марк­систской критики -- первой половине 19‑го века.

Если, с одной стороны исторический материализм Маркузе приводит его к зачеркиванию творческой роли мысли, то с дру­гой стороны, он приводит к фетишизации техники и техниче­ского прогресса. Это та же фетишизация техники, которую мы видим в советском официальном марксизме, но только со знаком минус. Маркузе видит источник всех бед -- точнее, источник застоя -- в современной промышленной технологии. "В этом обществе, -- пишет он, имея в виду индустриальное общество, -- аппарат производства стремится стать тоталитар­ным в той степени, в которой он определяет не только соци­ально необходимые виды деятельности, навыки и предрасполо­жения, но также индивидуальные потребности и стремления". [14]

Еще цитата: "Современное индустриальное общество посред­ством способа, которым оно организовало свою технологиче­скую базу, стремится быть тоталитарным. Ибо "тоталитарной" является не только террористическая политическая коорди­нация общества, но также и нетеррористическая технико‑эко­номическая координация, осуществляемая корпорациями, кото­рые манипулируют потребностями людей". [15]

Так писать и так понимать тоталитаризм можно только, ес­ли полностью отказать человеку в способности возвышаться над средой или во всяком случае отрицать значение этой спо­собности для исторического развития. Ибо технология как та­ковая меняет только среду обитания  для каждого индивиду­ума и больше ничего. Эта новая среда является, с точки зрения биологической, не менее, а более благоприятной для развития жизни и ее высших форм: она обеспечивает удовлетворение потребностей низшего уровня, увеличивает продолжительность жизни, оставляет больше свободного времени. Если в этих ус­ловиях те формы жизни, которые мы наблюдаем, обнаружи­вают тенденцию не к развитию, а к застою, то причины надо ис­кать не на материально‑техническом уровне, а на уровне идей­но‑политическом. В частности, тоталитарный тупик -- явление политическое и идеологическое. Тоталитаризм останавливает развитие путем массового физического подавления свободной мысли и творчества. Индивидууму противостоит здесь не пассив­ная среда, а активная человеческая сила. Пока нет этого проти­востояния, активного подавления свободы, нельзя говорить о тоталитаризме, это значило бы изменить сущность понятия. (Не знает, не знает товарищ Маркузе, что такое настоящий то­талитаризм!) Есть различие между человеком со связанными руками и ногами и человеком, заблудившимся в лесу. Отсутст­вие крупных общественных идей может быть просто следстви­ем того тривиального факта, что на выработку идей необхо­димо время. 

При всем том, что движение новых левых оказалось неспо­собным внести конструктивный вклад в эволюцию общества, оно является, так сказать, эволюционным по происхождению. Оно демонстрирует наличие в обществе, и в первую очередь, конечно, в среде молодежи, эволюционного потенциала, потребности выйти за пределы данного, совершить метасистемный переход. Маркузе ведет свою критику, опираясь на кон­цептуальный аппарат исторического материализма, чем и обре­кает ее на бесплодность. Но его целевая и эмоциональная пози­ции вызывают у меня полное понимание. В сущности, Маркузе критикует западное общество за то, что из всех аспектов эво­люции оно сохранило лишь один аспект -- научно‑технический прогресс, а в остальных отношениях перестало эволюциони­ровать, утратило творческий потенциал. Именно это, очевид­но, и привлекло к его философии студенческую молодежь.

Центральной мишенью его критики является отсутствие у индивидуума внутренней свободы от общества, отсутствие лич­ного "внутреннего пространства", которое необходимо для творчества. Из‑за отсутствия этого пространства человек и об­щество становятся "одномерными", теряют измерение "пер­пендикулярное" к существующему порядку вещей.

"Сегодня, -- пишет Маркузе, -- в это личное пространство вторгается и сводит его на нет технологическая реальность. Массовое производство и массовое распределение предъявля­ют притязания на всего  индивидуума, а индустриальная психо­логия давно перестала ограничиваться территорией завода. Множественные процессы интроекции окостеневают, порож­дая почти механические реакции. Результатом является не при­способление, а подражание:  непосредственная идентификация индивидуума с его  обществом и таким образом с обществом в целом.

Эта непосредственная автоматическая идентификация (ко­торая, возможно, была характерной для первобытных форм ассоциации) появляется снова в высокоразвитой промышлен­ной цивилизации; ее новая "непосредственность", однако, есть продукт изощренного научного управления и организа­ции. В этом процессе "внутреннее" измерение духа, в котором берет начало оппозиция к статус кво, сходит на нет. Потеря этого измерения, которое служит вместилищем отрицатель­ного мышления -- критической силы Разума -- является иде­ологическим аналогом тех самых материальных процессов, с помощью которых развитое индустриальное общество заглу­шает и смиряет оппозицию". [16]

Если оставить в стороне сатанинскую роль, которую Маркузе приписывает промышленной технологии и научно‑техниче­скому прогрессу, то в остальном его обвинение современного западного общества в "одномерности", вероятно, имеет" осно­вания. Не имея опыта жизни на Западе, не берусь высказывать на этот счет собственных суждений, но направленность критики здесь вперед, а не назад. В заключительной части "Одномерно­го человека", где автор набрасывает возможные альтернативы, общая направленность его мысли также вперед. Он представ­ляет себе альтернативу как "трансцендентный проект" (почти "метасистемный переход"), то есть такую программу, кото­рая полностью выходит за границы принятого и признанного в сфере общественной жизни. Приятно также читать те строки, где Маркузе отдает должное воображению как фактору исто­рического развития. Лозунг новых левых "воображение у власти", серьезное отношение к утопическому мышлению, призывы к <<переопределению потребностей>> --все это всегда вызывало у меня сочувствие. Мне кажется неоспоримым, что решение проблем, стоящих перед современным обществом, требует радикальных перемен, разрыва со многими представ­лениями и принципами, которые общественное мнение сейчас рассматривает как абсолютные и не подлежащие пересмотру.

Но одного признания необходимости "трансцендентного проекта" недостаточно. Надо придать этому проекту хотя бы приблизительные черты, а это предполагает определенную кон­цепцию относительно причин нынешнего положения. Концеп­ция Маркузе -- несмотря на то, что он вносит поправки в те­орию стоимости Маркса и в понятие отчуждения, -- остается марксистской. Он ищет (и полагает, что находит) причины снижения творческого потенциала общества и расцвета потре­бительской психологии в способе производства: промышлен­ной технологии, отношениях между классами и т.~п. Я нахожу это нелепым. Роль среды, фона, которую играет способ произ­водства, -- важная, незаменимая роль, но это не та роль, от ко­торой зависит ход действия. Среда, например, может быть в той или иной степени благоприятной для развития болезнетвор­ных бактерий, но характер и ход течения болезни определяется видом бактерий, и пока мы не обнаружим этого факта и не изучим бактерии, мы не продвинемся вперед в лечении болезни. В этом примере, как и всюду, ми видим один и тот же закон: нижние уровни организации создают условия, верхние уровни -- определяют развитие событий.

Ситуация, вызывающая критику новых левых, может быть описана, мне кажется, более непосредственно, в прямых, лобо­вых терминах, как отсутствие в культуре общества высшей це­ли -- по крайней мере, такой высшей цели, которая удовлетво­рила бы молодых людей и была бы ими принята в качестве та­ковой. В современном западном обществе наблюдается эрозия того слоя культуры (религиозного  слоя в широком смысле слова), который дает ответ на вопрос о смысле жизни. Поэто­му слой культуры, связанный с материальным производством и потреблением, занимает все более доминирующее положение, становится высшим уровнем иерархии целей и планов поведе­ния. Именно в этом, в исторически обусловленном распаде ре­лигиозного слоя культуры, а вовсе не в каких‑то чудовищных свойствах современного способа производства, лежат причины того явления, которое получило название "общества потребле­ния". Размерность, "перпендикулярная" к реальности, об исчез­новении которой говорит Маркузе, это размерность высших целей.

В обществе потребления -- и в этом я целиком солидарен с критикой новых левых -- высшие цели устанавливаются, по су­ществу, не человеком, а аппаратом производства. Человек пе­рестает быть творцом и повелителем, он отдает себя во власть созданной им самим среде. Общество становится неспособным к конструктивной эволюции.

Из такого понимания ситуации следует, что конструктивная критика современного индустриального общества должна на­чинаться с обсуждения вопроса о Высшей Цели. Этот вопрос -- центральный, это узкое место; мера, в которой удастся продви­нуться в обсуждении этого вопроса, есть мера продвижения в решении всех остальных проблем.

В книге Маркузе вопрос этот, по существу, обходится; ав­тор отделывается довольно туманными замечаниями о "более человечном" способе жизни, вместе с Уайтхедом говорит о "со­вершенствовании искусства жизни". Так же поступают и другие авторы. Все как будто исходят из предположения, что во­прос о смысле жизни решается сам собой, ответ на него всем более или менее ясен, и поэтому такого вопроса нет. Однако он есть. И в нашу эпоху, когда мы научились производить жизнеобеспечение в масштабах, не снившихся нашим предкам, этот вопрос становится особенно жгучим, он впервые в исто­рии человечества становится вопросом для масс,  а не для тонкой привилегированной прослойки.

Перед нами есть выбор. Мы устроены таким образом, что наши цели, наши потребности, наши желания ‑ не есть нечто фиксированное, данное нам свыше. Они в значительной степе­ни зависят от Высшей Цели, которую мы сами себе поставим. Так какую же мы выберем Высшую Цель?

От ответа на этот вопрос зависит, какое мы получим общество.


\section{Тоффлер}

Я переворачиваю последнюю страницу книги Альвина Тоффлера "Футурошок" [17]. Очень интересная книга, содержащая много наблюдений и мыслей, обильно снабженная фактическим материалом. Тоффлер рисует превращение индустриального общества в "постиндустриальное" (имея в виду главным образом США). Динамика этого превращения, грандиозный размах и стремительный темп перемен захватывают воображение.

-- Все больше вещей изготовляется для однократного пользования и выбрасывания. Бумажные салфетки. Картонные подносы. Носовые платки из бумажной ткани. Зубная щетка с уже нанесенной на нее пастой, которую вы выбрасываете, почистив один раз зубы. Тысячи новых разновидностей продуктов и предметов массового потребления появляются ежегодно на рынке.

-- Все возрастает подвижность населения -- как в смысле путешествий, так и в смысле смены места жительства. В 1967~г. 108 миллионов американцев совершили 360 миллионов путешествий. Средний владелец машины в США проезжает в год 10 тысяч миль. Четыре миллиона американцев ежегодно отдыхают за морем. Полтора миллиона немцев едут на каникулы в одну только Испанию. За один год 37 миллионов американцев меняют место жительства -- чаще всего из‑за перемены работы. Телефонная книга в Вашингтоне меняется за год больше чем наполовину.

-- Возникают новые организации, перестраиваются старые. В период с 1967 по 1969~г. "Квестор Корпорейшн" купила восемь компаний и продала две. В промышленности, науке, общественной жизни применяются все более гибкие и динамичные формы организации. Традиционная стабильная бюрократия уступает место временным образованьям, создаваемым ад хок для решения конкретной задачи, а затем умирающим естественной смертью. Когда "Локхид Корпорейшн" получила заказ на строительство 58 гигантских военно‑транспортных самолетов, была создана специальная организация из 11 тысяч человек, координировавших производство 120 тысяч частей для этих самолетов, производимых на предприятиях шести тысяч компаний.

-- Достижения науки входят в жизнь каждого человека, "задевают его за живое" в прямом смысле этого слова. Уже бо­лее 13 тысяч американцев с больным сердцем носят вшитый в грудную клетку крошечный приборчик -- ритмизатор, который посылает сердцу периодические электрические импульсы для его стимулирования. Еще 10 тысяч человек "оборудованы" вживленными сердечными клапанами из искусственного материала. Разрабатываются вживляемые слуховые аппараты, искусственные артерии, бедренные суставы, почки, легкие. Назревает биологическая революция. На повестке дня: генная инже­нерия -- получение животных, в том числе и человека, с заданными наследственными признаками; воспроизведение по ядру одной соматической (не половой) клетки целого организма, генетически тождественного оригиналу; развитие человеческого эмбриона вне тела матери; киборги -- человеко‑машинные гибриды. "Сейчас мы стараемся делать сердечные клапаны или искусственные артерии, которые лишь имитируют естественные. Но решив эту проблему, мы не ограничимся тем, что будем просто вставлять пластмассовую аорту, когда натуральная аорта выходит из строя. Мы будем вживлять специально сконструированные органы, которые будут лучше,  чем природные, а затем и такие, которые дадут их владельцу новые возможности и способности, отсутствовавшие первоначально. Подобно тому как генная инженерия обещает производить "сверхчеловеков", так и технология искусственных органов сулит появление спортсменов со сверхемкими легкими и сверхмощными серд­цами; скульпторов с нейронными приспособлениями, повы­шающими восприимчивость к текстуре; любовников с ней­ронной машинерией для интенсификации секса. Короче говоря, мы будем вживлять органы не только для того, чтобы спасти жизнь, но и чтобы сделать ее лучше -- сделать возможным дости­жение таких ощущений, состояний, экстазов, которые в насто­ящее время нам не доступны ".[18]

-- По мере возрастания производительности труда уменьша­ется процент людей, занятых производством жизнеобеспечения. Несколько процентов населения обеспечивают Соединен­ные Штаты необходимыми продуктами питания. Доля людей, занятых в промышленности, падает, а в сфере обслуживания -- возрастает. Но сфера обслуживания, в том виде, как она су­ществует сейчас, тоже не будет долго расширяться: автомати­зация и здесь приведет (и уже приводит) к огромной экономии труда. Точка приложения усилий смещается в сторону эстетики и психологии. Вскоре мы будем свидетелями "революционного расширения некоторых видов промышленности, продукцией которых будут не товары и не обычные услуги, а запрограммированные "переживания". Промышленность переживаний, возможно, станет одним из столпов супериндустриализма самой основой постсервисной экономики. По мере того как подъем благосостояния и приходящесть вещей безжалостно подрезают древнюю тягу к обладанию, потребители начинают копить переживания так же сознательно и страстно, как они некогда копили вещи. Сегодня, как показывает пример авиаци­онных компаний, переживания продаются как довесок к каким‑либо более традиционным услугам. Переживания, так сказать, служат глазировкой на пирожном. Но с движением в будущее все больше и больше переживаний будет продаваться исклю­чительно как переживания, как если бы это были вещи". [19]

-- И т.~д. и т.~п.
И все же, когда читаешь книгу Тоффлера, то очень скоро начинаешь испытывать чувство тоски. Ибо во всем этом движе­нии, в этой спешке, в мелькании предметов и лиц, в погоне за "переживаниями" нет чего‑то очень нужного и важного. Чего же? Смысла. Цели.
Лишь на последних сорока страницах из общего объема книжки в пятьсот страниц автор касается вопроса о целях деятельности. Да и как касается? Он фактически обсуждает не цели, а  процедурный вопрос: как вырабатывать цели "демократическим" путем (я вернусь к этому ниже). В основной же части книги, там, где описываются процессы, протекающие в совре­менном обществе, и ближайшее будущее, понятие о цели просто отсутствует, как будто такого понятия и не существует, как будто читателю не должен и не может прийти в голову этот фундаментальнейший для разумного существа вопрос: зачем?

Мы видим перед собой чудовищное ускорение кругооборо­та вещей, но не видим обсуждения, что здесь хорошо и что пло­хо. Мы видим огромное возрастание подвижности людей, все ускоряющуюся смену мест, мы видим растущую эфемерность отношений между людьми (если верить автору на сто процен­тов), но нам остается непонятным, как относятся к этому са­ми американцы. Мы знакомимся с проектами изощренных чувственных удовольствий и не знаем, исчерпывается ли этим то, что постиндустриальное общество может предложить чело­веку, или же все‑таки есть что‑то еще. Духовная культура по­просту игнорируется. О литературе, искусстве, философии, ре­лигии, об их влиянии на общество в настоящем и ближайшем бу­дущем не говорится ни слова. Стороннему читателю остается не­понятным, то ли духовная культура и в самом деле не играет решительно никакой роли в жизни американского общества, то ли это особенность персонального взгляда на вещи автора книги.

В начале книги Тоффлер так характеризует ее предмет:

"Эта книга ‑ о переменах и о том, как мы приспосабливаем­ся к ним. Она о тех, кто преуспевает в обстановке перемен, кто способен оседлать гребень бегущей волны, а также и о великом множестве других, кто сопротивляется переменам или бежит от них. Эта книга -- о нашей способности приспособляться. Книга о будущем и о шоке, который вызывает его наступле­ние" [20].

В картине, которую рисует Тоффлер, перемены предстают как нечто внешнее по отношению к человеку -- к отдельному лицу и к обществу в целом. Не человек создает перемены, стре­мясь к каким‑то целям, а напротив, перемены (подчиняющиеся, надо думать, "объективным закономерностям движения мате­рии", хотя этого марксистского термина в книге и нет) созда­ют человека, велят ему приспособиться к ним. А кто не может приспособиться -- заболевает "футурошоком" ‑ шоком от на­ступления будущего. "Футурошок -- это головокружение и де­зориентация, вызываемая преждевременным наступлением буду­щего. Вполне возможно, что это наиболее важная болезнь завт­рашнего дня".[21]

(Вот как интересно: живя в одном и том же обществе, в од­но и тоже время люди видят в нем прямо противоположное. Маркузе видит застой, отсутствие перемен и призывает произ­вести перемены любой ценой; Тоффлер видит избыток пере­мен и строит планы, как помочь людям к ним приспособлять­ся. Общее между ними то, что ни у того, ни у другого нет поня­тия о конструктивных  переменах, об эволюции.)

Итак, по Тоффлеру, только приспособленец, конформист, имеет хорошие шансы на психическое здоровье в этом "бра­вом новом мире". Тоффлер, конечно, не употребляет этих вы­ражений, он говорит о будущем с воодушевлением. Это остав­ляет меня в недоумении относительно его образа мышления: является ли позиция автора литературным приемом, как это делают авторы антиутопий, или же зрелище энергичной суеты должно, по его мысли, вызывать положительные эмоции?

Что бы ни думал автор "Футурошока", я читал эту книгу как самую мрачную антиутопию. Стандартом антиутопии являет­ся изображение стабильного, навеки застывшего общества. Тоффлер же проделывает следующий поучительный экспери­мент. Он не переносится в будущее, он описывает настоящее и сохраняет всю интенсивность перемен, ему свойственную. И он делает некоторые проекции этих перемен в ближайшее буду­щее -- делает вполне квалифицированно и правдоподобно. Но он отнимает у перемен их цель и получающуюся в результате этой операции картину представляет на суд читателю.

Первым делом исчезает понятие о творчестве. Забавно: да­же в предметном указателе слово "творчество" не сочтено са­мостоятельным понятием; соответствующая строка гласит:

"Творчество,  см. Воображение".  Но воображение отнюдь не равнозначно творчеству. Творчество предполагает выход за пределы личности, из сферы субъективного в сферу объективного. Воображение может быть инструментом творчества, но может и не быть. Воображение может работать и вхолостую, порож­дая <<переживания>> --и больше ничего. Онанизм -- пример та­кого использования воображения.

Жизнь без творчества являет собой тягостную картину. В сущ­ности, это то же оцепенение классических антиутопий Хаксли и Орвелла. Хорошо известен один из первых опытов, в котором было доказано существование в мозгу животного центров удо­вольствия. В мозг крысы был вживлен электрод, с помощью которого на нужную точку могло быть подано электрическое напряжение. Достаточно было крысе нажать лапой на педаль, чтобы цепь замкнулась и крыса получила удовольствие. Пос­ле того как крыса делала это открытие, она начинала безоста­новочно нажимать на педаль. Неужели человечество ожидает судьба этой крысы? Становится не по себе, когда читаешь такие, например, пророчества Тоффлера:

"\ldots Будущие конструкторы переживаний создадут, к приме­ру, игорные дома, в которых посетители будут играть не на деньги, а на переживания -- свидание с симпатичной и благо­расположенной дамой, если посетитель выигрывает, и, скажем, сутки одиночного заключения, если он проигрывает. По мере роста ставок будут изобретаться более изощренные удовольст­вия и наказания.

Проигравший, возможно, вынужден будет в течение несколь­ких дней служить (по добровольному предварительному согла­шению) победителю в качестве "раба". Победитель может быть вознагражден десятью минутами электрического стимулирова­ния мозгового центра удовольствия. Игрок, возможно, будет рисковать поркой или ее психологическим эквивалентом -- уча­стием в однодневной сессии, во время которой выигравшим разрешается разряжать свои агрессивные побуждения на проиг­равших, а именно : кричать на них, насмехаться, оскорблять их и т.~п.

Любители крупных ставок смогут играть на бесплатный трансплантат сердца или легкого, в случае если таковой пона­добится. Проигравшему, возможно, придется расстаться с поч­кой. Такие выигрыши и проигрыши могут бесконечно варьироваться и нарастать по интенсивности. Конструкторы пережи­ваний обратятся за идеями к страницам Крафта ‑ Эбинга и мар­киза де Сада. Только воображение, технические возможности и стеснения со стороны в общем смягченной нравственности ограничивают разнообразие вариантов". [22]

"Разнообразие новых переживаний, выстроенных перед по­требителем, будет результатом работы конструкторов пережи­ваний, которые будут набираться из рядов наиболее творче­ских членов общества. Девизом этой профессии будет : "Если ты не можешь подать это в реальности, подай в виде заменителя. Бели ты мастер своего дела, клиент не заметит разницы!" Содер­жащееся здесь размытие границы между реальным и нереаль­ным поставит общество перед серьезными проблемами, но не предотвратит и даже не замедлит появление "промышленности психообслуживания" и "психокорпусов". Гигантские, охваты­вающие весь Земной шар синдикаты создадут супердиснейленды, разнообразие, размах и эмоциональную мощь которых нам трудно даже вообразить". [23]


\section{Антирелигия}

Обратимся к последним сорока страницам "Футурошока". Это последняя глава книги, которая называется "Стратегия со­циального футуризма". Она начинается словами:

"Можно ли жить в обществе, которое вышло из‑под конт­роля? Вот вопрос, который ставится перед нами понятием футурошока. Ибо именно такова ситуация, в которой мы находим­ся. Если бы только одна технология потеряла управляемость, наши проблемы были бы уже достаточно серьезны. Но весь ужас в том, что множество других социальных процессов также вышли из‑под контроля, бешено осциллируя, сопротивляясь всем нашим усилиям управлять ими". [24]

Основную причину неуправляемости Тоффлер видит в непра­вильном планировании, основанном на принципах "технократии". Такой подход ставит во главу угла экономику, произ­водство материальных благ, порождает стремление максимизи­ровать производство и игнорировать все остальные цели и ас­пекты жизни. Но в наше время, говорит автор, именно эти остальные цели и аспекты приобретают решающее значение, опре­деляют социальное поведение людей. Технократическое плани­рование "эконоцентрично" и поэтому неспособно правильно прогнозировать социальные результаты принятия различных ре­шений, в том числе и чисто экономических.

Критика Тоффлером технократического планирования спра­ведлива и обоснована. Но только я очень сомневаюсь, что на не­го можно возложить вину за неуправляемость общества. Не вер­нее ли сказать, что причина неуправляемости -- отсутствие еди­нодушия относительно какой‑либо высшей цели (что надо бы иметь в свободном обществе) при отсутствии жесткой систе­мы принуждения (что имеет место в тоталитарном обществе). Тоффлер пишет: "Сегодня накапливающиеся свидетельства того, что общество вышло из‑под контроля, питают разочаро­вание в науке. Вследствие этого мы видим явное оживление мистицизма. Астрология вдруг входит в моду. Дзен‑буддизм, йога, спиритические сеансы и магия становятся популярным времяпрепровождением".25 Сомневаюсь, очень сомневаюсь. Не вернее ли сказать, что неудовлетворенность системой ценно­стей и целей и поиски высших духовных ценностей питают эти увлечения?

Я совершенно согласен с Тоффлером, когда он говорит о необходимости новой стратегии при решении социальных проб­лем. Но какой стратегии? Какова должна быть система целей?

"В свое время и в своем месте одержимость индустриально­го общества идеей материального прогресса сослужила челове­честву хорошую службу. Но теперь, когда мы несемся к супер­индустриализму, возникает новый этос, в котором другие це­ли начинают приобретать такую же роль и даже вытеснять це­ли экономического благосостояния. В личном плане самоосу­ществление, общественная ответственность, эстетические дости­жения, гедонистический индивидуализм и полчище других це­лей соперничают с голым стремлением к материальному успе­ху и часто перевешивают его. Изобилие служит основой, оттал­киваясь от которой люди начинают стремиться к разнообраз­ным постэкономическим целям". [26]

Насчет вытеснения целей экономического благосостояния "полчищем" других целей все правильно. Но только если это полчище не образует системы, иерархии,  и не имеет, следовательно, главнокомандующего, то человек будет жить в суете и тоске, а общество -- если оно вообще возможно -- будет не­управляемым.

Много целей -- значит ни одной цели. Чувственное удовольст­вие -- чудесная, незаменимая вещь. Но кибернетическое устрой­ство, называемое человеком, устроено таким образом, что кро­ме чувственного удовольствия и разнообразных переживаний ему еще необходима некая Высшая Цель. Без этого человече­ская кибернетика разлаживается, выходит из строя. Человек ску­чает. Болеет. Стареет раньше времени. Становится невротиком. Сходит с ума. Кончает самоубийством.

Популярный американский журнал констатирует:

"Несмотря на чрезвычайное разнообразие и богатство Аме­рики, несмотря на ее любовь к зрелищам и лихорадочную пого­ню за развлечениями, Америке скучно. Массированное насту­пление на скуку, предпринятое в Соединенных Штатах, потер­пело поражение, и скука стала болезнью нашего времени. Авто­ритетные люди затрудняются указать, сколько людей страдает от скуки, но их миллионы, и их число возрастает.

Молодые люди в особенности подвержены скуке. Доктор М. Роберт Уилсон, главный психиатр Констанс‑Балтмен‑Уил‑сон‑Центра в Фэрибо штат Миннесота, специализирующийся по проблемам молодежи, оценивает в $20\%$ число юношей и де­вушек в Америке, которым скука и депрессия доставляет се­рьезные трудности. Эти трудности часто ведут к потере самоува­жения, а в крайних случаях ‑  к самоубийству. И в самом деле, число самоубийств среди молодежи в Америке резко возрос­ло: с 1960 года частота самоубийств в возрасте от 15 до 25 лет удвоилась, и самоубийство сейчас вторая по распространенно­сти причина смерти в этой возрастной категории.

По мнению некоторых экспертов, одно из самых ужасных массовых убийств в истории Америки последнего времени, совершенное "семьей" Чарльза Мэнсона, было совершенно от скуки". [27]

Книга Тоффлера написана с позиций доктрины, которую мож­но назвать антирелигией.  Она является перенесением либераль­но‑демократической доктрины из сферы политики в сферу культуры. Символ веры антирелигии таков. Цели, которые ставит перед собой человек, -- его личное дело, вторгаться в которое -- неэтично. Это почти то же самое, что физическое принуждение. Установление обществом иерархии целей -- недопустимо. Тем более недопустимо социально определенное понятие о Высшей Цели. Это почти то же самое, что диктатура бюрократии и Вер­ховного Владыки.

Антирелигия возникает в либерально‑демократическом об­ществе вследствие распада религии. Из средства достижения Высшей Цели, органически связанной с определенным пред­ставлением о бытии и о человеке, которое возникло в рамках христианской культуры, либеральная демократия превращает­ся в самоцель. Политика наступает на культуру. Ведь какая‑то форма согласия относительно целей необходима. Согласие об отсутствии Высшей Цели заменяет собой согласие о наличии Высшей Цели, которое имеет место в религиозном обществе. Но симметрия эта весьма относительна. Наличие Высшей Цели обеспечивает социальную интеграцию, ее отсутствие, возведен­ное в принцип, ведет к дезинтеграции общества.

"Акселерация, -- читаем мы у Тоффлера, -- производит все более быстрый кругооборот целей. Захваченные этой бурлящей, кишащей целями средой, мы испытываем чувство ошеломле­ния, футурошок, мы движемся от кризиса к кризису, пресле­дуя множество противоречащих друг другу и каждая самой себе целей.

Нигде это не проявляется с такой очевидностью, как в наших безуспешных попытках управлять городами. Жители Нью‑Йорка за короткий промежуток времени пережили кошмарную цепь почти катастрофических событий: нехватку воды, забастовку работников метро, расовые беспорядки в школах, восстание студентов в Колумбийском университете, забастовку мусорщиков, нехватку жилой площади, топливную забастовку, нарушение телефонного обслуживания, забастовку учителей, отключение электричества -- и это еще не все\ldots
Нельзя сказать, что никто не занимается планированием. Напротив, в этом кипящем социальном котле технократические планы, субпланы и контрпланы льются рекой. Они требуют новых дорог, новых электростанций, новых школ. Они обещают улучшение положения с больницами, жильем, социальным обеспечением. Но планы отменяют, противоречат или подкрепляют один другой по закону случая. Редко бывает, чтобы они были логически связаны друг с другом, и никогда -- с какой‑либо общей картиной города будущего. Никакая мечта -- утопиче­ская или не утопическая -- не заряжает нас энергией. И нет ника­ких рационально интегрированных целей, которые бы внесли порядок в этот хаос. На национальном и международном уров­нях отсутствие связной, последовательной политики в равной степени очевидно и вдвойне опасно". [28] 

Но почему же нет интегрированной системы целей, почему нет "мечты"? Тоффлер винит в этом опять‑таки технократов. Он рассказывает, что три последовательных президента США -- Эйзенхауэр, Джонсон и Никсон -- учреждали специальные орга­низации для выработки системы целей и основ планирования. Но все эти попытки, считает он, были безуспешны по той при­чине, что носили на себе "безошибочно распознаваемый отпе­чаток технократической психологии".29 Ибо они оставляли в стороне основной, как считает Тоффлер, вопрос: кто будет устанавливать цели для будущего? "Революционно новый" под­ход Тоффлер видит в демократизации процесса выработки це­лей:

"Пришло время для драматической переоценки направлений движения, переоценки, делаемой не политиками или социоло­гами, или духовенством, или элитарными революционерами, а также не техническими специалистами или президентами кол­леджей, а самим народом. Мы должны в буквальном смысле слова "идти в народ" с вопросом, с которым к нему почти ни­когда не обращались: "Какого мира вы хотите через 10, 20 или 30 лет?" Короче, нам нужно организовать непрерывный плебис­цит о будущем".30

В каждой стране, в каждом городе, в каждой окрестности надо на демократических началах созвать ассамблеи, задачей которых будет определение конкретных социальных целей на период времени до конца столетия. Такие ассамблеи могут представлять не только географические, но и социальные единицы -- промышленность, интеллектуальное сообщество, искусство, церкви, женщины, этнические группы и т.~п. Все должны иметь право в равной степени определять будущее: и видные интеллектуалы, и те, кто не умеет ясно выразить своих мыслей и чувств. Этот проект носит название "демократии ожиданий" по аналогии с "демократией участия" (английские термины звучат похоже: anticipatory democracy и participatory democracy).

"Некоторым людям этот призыв к своего рода неонародни­честву покажется, без сомнения, наивным", -- пишет Тоф­флер 31 . Должен признаться, что я отношусь к их числу. Наив­ность проекта граничит с нелепостью, он напоминает попытку установить правильность или ошибочность математической те­оремы путем всеобщего равного голосования. Проблема раци­ональной интеграции целей, проблема "мечты" о будущем от­носится к сфере культуры, а не к сфере политики. Она решает­ся не путем голосования, не путем компромисса между инте­ресами, а в результате упорного, длительного труда, творче­ской деятельности. Ибо рационально интегрировать цели можно только через посредство введения высших целей и ценностей и построения иерархии целей. В отличие от целей низшего уров­ня ("хочу, чтобы в будущем было много меду; и блинов с крас­ной икрой; и чтобы окна с видом на Эльбрус"), высшие цели не могут быть выражены в конкретных понятиях. Они требуют абстрактных понятий и обобщенных образов, которые должны быть не только созданы, но и получить распространение, войти в жизнь масс. Этим и занимаются писатели, философы, пророки, ученые, кинорежиссеры и многие, многие другие. А что будут делать ассамблеи Тоффлера? Согласовывать заявки на блины с икрой и виды из окон? Участие широких масс в культуре, а не только в плебисцитах -- вот признак подлинной демокра­тии,

Интеграция целей в конечном счете требует одной Высшей Цели. Нет спора, привлечение широких масс к проблеме целей и будущего, содержащееся в проекте Тоффлера, -- здоровая, жизненно важная в наше время идея. Но центральным момен­том здесь должно быть понятие о Высшей Цели, что означает, по существу, создание религиозного движения.

Цели и планы обладают свойством образовывать иерархию и определяться сверху вниз: цели и планы низшего уровня определяются целями и планами высшего уровня, но никак не наоборот. Это, конечно, очень "не демократично", но что поде­лаешь, если такова их природа! Тоффлер, однако, не хочет с этим смириться. Одно из его самых тяжелых обвинений против технократов (о, несчастные технократы, они служат козлом отпущения всех грехов!) ‑ что они планируют сверху вниз, а не снизу вверх: "Продолжение технократического определения целей сверху вниз приведет ко все большей и большей социаль­ной нестабильности, все меньшему и меньшему контролю над си­лами перемен и все большей опасности человекоубийственного катаклизма".32 Это и есть источник нравственного пафоса тоффлеровского проекта, который он объявляет "захватыва­ющим дух утверждением народной демократии"33 :цели здесь определяются снизу вверх, что только и может быть приемлемо для настоящих демократов, борцов против технократии, бю­рократии, иерархии и прочих пережитков проклятого прошлого. Перед нами пример насилия политической идеологии над реаль­ностью, воли над знанием. Советские люди знакомы со многи­ми примерами этого явления. В Советском Союзе конформист выслуживается перед властью, на Западе -- перед человеком мас­сы: читателем, покупателем, избирателем. У нас он объявляет ключом к решению всех проблем марксизм‑ленинизм, там ‑ демократию.

"Демократия ожиданий" Тоффлера логически завершает его антиутопию. Если на протяжении девяти десятых книги мы только видели перед собой коловращение мира, лишенного Высшей Цели, то теперь мы получаем нечто вроде теоретиче­ского обоснования того, что Высшей Цели и не нужно, а нужно только собирать информацию о том, какие у людей есть  цели. <<Демократия ожиданий>> --это равноправие целей, это анти­религия. Антиутопия Тоффлера -- продукт его антирелигии. Одного только обстоятельства не учитывает автор "Футурошока": осуществимость этой утопии серьезнейшим образом за­висит от соседей "постиндустриального общества". Ибо общест­во, в котором наиболее творческие умы заняты размытием гра­ницы между реальностью и иллюзией и обращаются за идеями к страницам маркиза де Сада, вряд ли сможет оказать сопротив­ление тоталитаризму.

Интересно отметить, что концепция Тоффлера, как и всякая концепция, лишенная понятия творчества, детерминистична. Он настойчиво говорит о необходимости изучения  будущего -- фразеология, которую я рассматриваю как совершенно непри­емлемую. В одном месте он высказывается весьма решитель­но: "\ldots пора стереть с лица Земли популярный миф, что буду­щее "непознаваемо"\ldots."34 . Я же, напротив, выдвигаю лозунг: "Будущее непредсказуемо". Это расхождение, разумеется, от­носительно. Обе стороны в споре понимают, что предсказания будущего, во‑первых, возможны, а во‑вторых, всегда частичны. Различие состоит в том, какой из этих двух аспектов выд­вигается на передний план. Тоффлер видит общество в свете индивидуализма. Каждый член общества руководствуется своими личными целями, находясь в определенных, общих для многих людей социальных условиях. По закону больших чисел, результат их совокупных действий может быть предсказан с большой точностью. Я же вижу общество как единое сверх­существо, способное к творческому акту, -- метасистемному пе­реходу. Результаты этого акта непредсказуемы, потому что метасистемный переход, в частности, включает в себя самоописа­ние  и самопознание;  это значит, что все сделанные нами пред­сказания являются информацией, которую мы можем принять во внимание. (В математике это соответствует теореме Геделя и другим "отрицательным" результатам, о которых я упоминал в части 2.) Мы можем предсказывать будущее только на отрез­ках от одного метасистемного перехода до другого. Но именно в этих разрывах предсказуемости, в этих творческих актах я и вижу высшую прелесть жизни.
Советского читателя "Футурошока" не может не интересо­вать вопрос: является ли основная черта книги -- удивительное, я бы сказал, уникальное  отсутствие понятия о высших целях и ценностях ‑ типичной для современного американца или же это специфическая черта автора? Тоффлер печатался во многих широко читаемых журналах. Можно предположить, что он зна­ком с образом мышления американской аудитории и в какой‑то степени выражает его. Печально, если это так.


\section{Деполитизация социализма}

Я думаю, что западный мир может совладать с порожденным им же самим тоталитаризмом, только лишив его монополии на сознательную социальную интеграцию. Иначе говоря, тота­литаризм может быть побежден в конечном счете только со­циализмом. Я не имею в виду, разумеется, тех политических партий, которые в данный момент называют себя социалисти­ческими или коммунистическими; я имею в виду социализм как явление культуры в том смысле, как я определил его во второй части книги.

Современный социализм как политическое явление не рас­полагает к оптимизму. На правом фланге мы видим социал‑демократов, для которых идея социальной интеграции ограни­чивается сферой финансово‑экономической. Более привлека­тельного идеала, чем общество потребления, они предложить не могут. Радикальные социалисты на левом фланге сплошь мыс­лят в марксистских терминах, что делает их потенциальными или действительными разносчиками тоталитаризма, а не борца­ми с ним. Негативный элемент, по‑видимому, преобладает над позитивным, борьба за власть преобладает над борьбой за идеи.
Интересно отметить, что коммунисты, в отличие от более умеренных социалистов и социал‑демократов, в середине двад­цатого века вдруг обнаружили сильную тенденцию к национа­лизму. Коммунизм стал национальным -- чтобы не сказать на­ционалистическим. Как это случилось? Ведь коммунисты счита­ют себя самыми верными последователями Маркса, а Маркса можно обвинить в чем угодно, но только не в национализме.

Дело в том, что реальная сила, которая делала и делает ком­мунистов сильными, это интеграционизм, с вытекающей отсю­да системой ценностей и акцентом на организации. Маркс и Эн­гельс в начале своей деятельности исходили из того, что проле­тариат образует своеобразный всемирный народ, этнос. В этом смысле марксизм всегда был идеолдеолдеолнического единства. Однако двадцатый век показал со всей очевидностью, что чело­век гораздо охотнее думает о себе как о члене нации, чем как о члене класса. Деление на классы размылось, да оно и не было никогда столь четким, как деление на этнические группы. Двадцатый век стал веком национализма: после упадка великих религий нация оказалась единственной или во всяком случае, наиболее популярной основой интеграции. С некоторым за­позданием это поняли и коммунисты. Этническая интеграция опирается на духовную культуру, а классовая -- на материаль­ный интерес. Этнос оказался сильнее класса, это еще раз подт­верждает примат духовного начала в человеке. Не желая упу­стить своей питательной среды, коммунисты и другие радикаль­ные социалисты стали соскальзывать в национализм.

Левые марксисты и сейчас, по‑видимому, мало отличаются от большевиков‑ленинцев, не зря они клянутся именем Ленина, а иные и Троцкого. Это означает, что от их прихода к власти надо ожидать того же, что мы видели в России от большевиков. Лешек Колаковский рассказывает о своем разговоре с одним латиноамериканским революционером. Речь зашла о пытках в Бразилии, "Я спросил, ‑ пишет Колаковский, ‑ а чем плохи пытки?" Он удивился: 'То есть как? Вы хотите сказать, что это нормально? Вы оправдываете пытки?" Я сказал: "Напротив, я просто спрашиваю, думаете ли вы, что пытки -- чудовищное зло­деяние, недопустимое морально?" "Конечно", -- ответил он. "А пытки на Кубе?" "Ну, -- ответил он, ‑ это другое дело. Ку­ба -- маленькая страна, находящаяся под постоянной угрозой со стороны американского империализма. Они вынуждены использовать все средства самозащиты, как это ни прискорб­но"\ldots." [35]

Я вспомнил об этом эпизоде недавно, когда разговаривал с молодым англичанином, левым, но не коммунистом. Мы го­ворили об идее "исторического компромисса", только что вы­сказанного Берлингуэром. "Как вы относитесь к коалиции меж­ду коммунистами и христианскими демократами?", -- спросил я. Он задумался, потом ответил: "Такую коалицию можно рас­сматривать как положительное явление, если она позволит в дальнейшем сформировать чисто левое правительство, без христианских демократов".

Н‑да, подумал я, не хотел бы я очутиться в одной коалиции с этими ребятами. Англичанин, в сущности, мыслит точно так же, как тот латиноамериканец, с которым беседовал Колаков­ский. Подход один и тот же. Ленинский подход.

Возможно, что Берлингуэр думает не так. Но итальянская компартия не состоит из одного Берлингуэра. В каких терми­нах мыслит основная масса коммунистов? Итальянская ком­партия постоянно подчеркивает свою преданность идее прав че­ловека. В последнее время все больше говорят об этом и фран­цузские коммунисты. И те, и другие открыто осуждают обра­щение с инакомыслящими в Советском Союзе. Само по себе это замечательно. Это большой сдвиг в международном ком­мунистическом движении, который порождает надежду, что этому движению, возможно, предстоит в конце концов сы­грать положительную роль в истории. Но для этого необходимо интеллектуальное и нравственное возрождение радикального социализма. Прежде всего, он должен отказаться от нигилисти­ческих и разрушительных аспектов марксизма и от ленинской политизированности. Но этого мало. Необходимо глубокое обновление теоретического багажа, необходимы новые живые идеи. Без этой перестройки базиса самый интеграционизм лево‑социалистического движения -- его отличительная черта и при­влекательное качество‑ будет неизбежно порождать механиче­скую интеграцию, тоталитаризм.

Нынешнюю ситуацию в радикальном социализме я не могу характеризовать иначе, как духовное и интеллектуальное оску­дение. Когда мне попадает в руки зарубежный марксистский журнал или книга, я не перестаю удивляться, насколько они по­хожи на советские издания. Та же убогость мысли, те же заучен­ные мертвые формулировки, та же вульгарная, почти карика­турная политизация, та же поучающая самоуверенность. При­веду еще некоторые наблюдения Л. Колаковского:

"Когда я покидал Польшу в конце 1968 года (я не был на Западе по крайней мере шесть лет), я имел довольно смутное представление о радикальном студенческом движении в различ­ных лефтистских группировках и партиях. То, что я увидел и о чем прочитал, я нашел жалким и отвратительным почти во всех (все же: не во всех) случаях. Я не проливаю слез по пово­ду нескольких окон, разбитых в демонстрациях; эта старая сука, потребительский капитализм, переживет такое несчастье. Я также не нахожу скандальным довольно‑таки естественное невежество молодежи. Что потрясло меня, так это духовная деградация такого рода, какой я ранее никогда не видел ни в одном левом движении.

Я увидел молодых людей, пытающихся "реорганизовать" университеты и освободить их от ужасающего, дикого, чудо­вищного, фашистского угнетения. Список требований был, с некоторыми вариациями, очень похожим во всех университе­тах. Эти фашистские свиньи из истеблишмента хотят, чтобы мы сдавали экзамены, в то время как мы делаем Революцию; пусть поставят нам высшие оценки без экзаменов. Очень час­то были требования вообще отменить некоторые предметы как ненужные ‑ например, иностранные языки\ldots

В одном месте революционные философы объявили заба­стовку, потому что в список книг для чтения были включены Платон, Декарт и другие буржуазные идиоты, вместо действи­тельно необходимых великих философов, как Че Гевара и Мао\ldots В другом месте благородные мученики мировой революции по­требовали, чтобы их экзаменовали только другие студенты, вы­бранные ими самими, а не эти старые реакционные псевдо­ученые. Профессора должны назначаться (студентами, конеч­но) в соответствии с их политическими взглядами. В несколь­ких случаях в США авангард угнетенных трудящихся масс поджигал университетские библиотеки (никому не нужное псевдознание истеблишмента)\ldots И все это, конечно, были марк­систы, что означало, что они знают три‑четыре изречения Маркса или Ленина, в частности, изречение: "До сих пор философы только объясняли мир различными способами, задача же заклю­чается в том, чтобы изменить его". (Им было очевидно, что этим изречением Маркс хотел сказать, что нет никакого смысла учиться.)"36

Идейные рамки социализма в настоящее время ограничены сферой экономики и политики. Среди причин плачевного состояния социалистической мысли это обстоятельство играет почетную роль. Радикальный социализм, с одной стороны, притягивает людей с сильно развитым религиозным чувством, а с другой стороны, не может его удовлетворить. Религиозный элемент существует в нем, так сказать, подпольно, в организационной и политической практике.

Только понимая социализм как явление духовной культуры, только включив в него религиозную концепцию мира и человека, можем мы выйти за пределы "общества потребления". На это не способно ни безрелигиозное либерально‑демократическое или социал‑демократическое общество, ни советский тоталитаризм. Интересно, что официально провозглашенной Высшей Целью советского общества является "максимальное удовлетворение постоянно растущих материальных и духовных потребностей человека". Это предельно точная формулировка того, что такое общество потребления.

Марксизм‑ленинизм проповедует в теории и проводит на практике предельную политизацию всех аспектов обществен­ной жизни. Он всюду видит (а где не видит, там насаждает!) борьбу экономических интересов и борьбу за политическую власть. Мир нуждается сейчас в противоположной установке -- в установке на деполитизацию  важнейших аспектов жизни. Успех "Международной Амнистии", возрастание ее численно­сти и влияния показывает, что все больше людей на планете начи­нает понимать это. Основная идея <<Международной Амнистии>> -- деполитизировать представление о гражданских и поли­тических правах личности, о недопустимости пыток и других форм негуманного обращения с людьми. Испокон веков эти вопросы относились к сфере политики. "Международная Амни­стия" переводит их в сферу общечеловеческой нравственности, в сферу духовной культуры. Парадокс: как можно деполитизи­ровать понятие о политических  правах? А "Международная Амнистия" делает это. Делает путем полного отстранения от борьбы за власть  и всего, что с ней связано. Она полностью, подчеркнуто отстраняется от предпочтения одних политиче­ских режимов или течений другим. И эта стратегия работает. Отделение проблемы прав личности от политической борьбы превращает эту проблему в проблему культуры. Я думаю, это одно из самых многообещающих явлений нашего времени.

Создание подлинно социалистической культуры требует деполитизации социализма. Вероятно, необходимы социалисти­ческие в неполитические  организации, а в конечном счете еди­ная всемирная организация, которая бы не только не ставила своей целью борьбу за власть, но, подобно "Международной Амнистии", пунктуально устранялась бы даже от высказывания любых политических предпочтений. Это не значит, конеч­но, что член такой организации, как и член "Международной Амнистии", не должен иметь  политических предпочтений. Каж­дому ясно, что люди, работающие в "Международной Амни­стии", сочувствуют политическим режимам, которые более или менее соблюдают права человека, и возмущаются режимами, где эти права грубо попираются. Но в своей деятельности в рам­ках этой организации они эти симпатии и антипатии игнорируют. Таков, по‑видимому, единственный путь к интеграции челове­чества.

Процессы в культуре ‑ дело долгое, а тут угроза тоталита­ризма. Это и есть основная проблема нашего времени. Способ­ность современной, переходной, культуры Запада устоять против тоталитаризма сомнительна. Что раньше: сложится обновлен­ная   культурно‑социально‑экономическая система, которая даст свободному миру необходимый минимум единства и ус­тойчивости, или социально‑политические конфликты, отсутст­вие общей воли и эгоизм отдельных групп приведут к глобаль­ному тоталитаризму?

В гонке этих двух процессов ключевую роль играет разви­тие событий в Советском Союзе. Россия первой из развивающих­ся стран вступила в тоталитарный период и стала одной их двух сверхдержав мира. Ее пример, не говоря уж о прямом полити­ко‑экономическом влиянии, будет в значительной степени определять судьбу других стран. Если Россия сумеет встать на путь гуманизации и либерализации социализма и выйти из тоталитарного тупика, это будет почти решением проблемы в мировом масштабе. Напротив, стабилизация тоталитаризма в России даст сигнал к усилению тоталитарных тенденций во всех странах. Поэтому проблема прав человека в Советском Союзе далеко выходит за пределы наших национальных проблем, это проблема общемировая. Защищая права человека в Советском Союзе, американец, англичанин или француз заботятся о будущем своих потомков.



\section{Инерция страха и мировоззрение}

\epigraph{Таким образом, возрождение нашей эпохи должно начаться с возрождения мировоззрения. Кажущееся далеким и абстрактным так насущно необходимо, как, вероятно, ничто другое\ldots
Призвание каждого человеческого существа состоит в том, чтобы, выработав собственное, мыслящее мировоззрение, стать подлинной личностью.}{Альберт Швейцер [37]}

Но никто не может нам помочь, если мы не позаботимся о себе сами.

Когда умер Сталин, я принадлежал к тому меньшинству мо­лодых людей, которые радовались его смерти, хотя ни их лично, ни их родителей почти не коснулись репрессии. (Я говорю "почти не коснулись", потому что родителей не арестовывали, а по тогдашним масштабам, все остальное выглядело терпимо. Но моя мать ‑ из раскулаченной середняцкой семьи. В молодости ей приходилось скрывать свое "происхождение", живя из‑за этого в постоянном страхе перед разоблачением. Моя бабка, мать матери, не только лишилась хозяйства во время коллекти­визации, но и была свидетельницей страшного голода на Укра­ине, когда целые села вымирали просто потому, что "народ­ная" власть отобрала у них весь хлеб до зернышка. Я помню, как незадолго перед смертью в 1945 году она сказала мне о Сталине: "Если бы я могла, я задушила бы его этими руками". Руки у нее, хотя она и прожила последние пятнадцать лет в го­роде, остались крестьянские, описания которых я читал в русской и западной литературе: темные от солнца и ветра, со вздутыми жилами и суставами. Мой отец, профессор агрохимии, был человеком подчеркнуто беспартийным. Он ясно понимал смысл происходящего и передал это понимание детям.)

С детства я был воспитан в твердом принципе: за пределами семейного круга -- никаких разговоров о политике, никаких упоминаний о репрессиях, никаких анекдотов. После смерти Сталина, а особенно после 1956 года, о политике стали говорить все. Появилась реальная возможность без большого риска как‑то воздействовать на общественную жизнь вокруг себя, хотя бы на уровне профсоюзного собрания, стенгазеты, фило­софского семинара. Я стал пытаться использовать эти возмож­ности, и мне казалось, что каждый нормальный человек должен вести себя и будет вести себя так же. Мне казалось, что борьба за прекращение всеобщего холуйства перед начальством, за гласность общественной жизни, за право самовыражения не только необходимо следует из общих этических принципов, но и должна доставлять душевное удовлетворение, которое оправдает какой‑то необходимый уровень риска. И я был очень удивлен, когда обнаружил, что у подавляющего большинства окружающих меня людей этот уровень просто‑напросто равен нулю. Максимум, на что можно было рассчитывать, -- что они будут рады, если необходимые перемены произойдут сами собой.

Пытаясь понять это явление, я вступал в длительные обсуж­дения и споры. Сначала они вращались исключительно вокруг вопросов политических и экономических. Но постепенно я стал понимать, что не здесь лежат первопричины, определяющие поведение людей. Все цепочки аргументов, начинающиеся с конкретных, локальных проблем, проходили через общие социально‑экономические и политические проблемы, а затем настойчиво и зримо упирались в основную проблему: зачем мы живем? Каков смысл жизни? Раньше я думал, что частные полурешения этой проблемы, которые так или иначе принимает каждый человек (я, например, уже давно решил посвятить свою жизнь науке), вполне достаточны и для него самого, и для общества в целом. Теперь я увидел, что их отнюдь не достаточно. Я увидел, что ни у меня, ни у моих друзей нет необходимых общих критериев для принятия решений. Я не имею в виду, конечно, формальных  критериев, которые позволяли бы вычислить значение некоего икса, и если он окажется меньше единицы, то выбирать да,  а если меньше единицы, то нет.  Таких критериев нет и никогда не будет. Человек, стоящий перед проблемой нравственного выбора, в конечном счете всегда решает эту проблему интуитивно, в соответствии со своей совестью. Но интуиция не висит в безвоздушном пространстве, работа интуиции определяется фундаментальной системой идей и цен­ностей, мировоззрением человека. Вопрос о смысле жизни не может быть исчерпан одной фразой или несколькими правилами. Ответом на него является мировоззрение.

Но никакого мировоззрения не было. Какие‑то остатки старого, какие‑то зародыши нового, а в общем -- ничего. Пустота. Вакуум. И единственной различимой силой в этом вакууме была инерция страха, инерция, не встречавшая сопротивления и поэтому имевшая все шансы продолжаться до бесконечности.

Я стал все реже и реже вступать в спор на политические те­мы, а потом и вовсе прекратил это занятие. Я убедился, что да­же небольшие различия в остатках мировоззрения в большей степени влияют на поведение человека в обществе, чем все со­циально‑политические дискуссии. Логика не поможет убедить человека сделать нравственное усилие, если он не знает, зачем он живет, или знает, что живет не за этим. За критическими вы­сказываниями таких людей я различал знакомые до тошноты очертания кукиша в кармане, а в каждом рассуждении невоз­можно было не видеть, что оно построено по столь же знакомым законам логики самооправдания. Я понял, что мне необходим синтез моих общественных воззрений с представлением о смыс­ле жизни, которое раньше казалось мне сугубо личным делом. Без такого синтеза я не мог найти основы для принятия кон­кретных решений. Без синтеза была пустота, а в пустоте тела движутся по инерции безостановочно. Мне стало ясно, что книга об инерции страха, а вернее, против  страха, которую я задумал, должна быть, по существу, книгой о мировоззрении.

Наверное, ни один политический переворот в истории не про­изводил такого разрыва в культурной традиции, как револю­ция 1917 года. Строя свое личное мировоззрение, каждый из нас не может просто продолжить какую‑либо из линий дорево­люционной русской культуры. Мы можем и должны их исполь­зовать, но не в нашей власти срастить омертвевшие ткани. Мы должны начинать с начала -- живя в промышленно развитой стране в конце двадцатого века. Это эксперимент, имеющий ми­ровое значение.

Я надеюсь, что тот, кто ощущает необходимость целостной системы взглядов, прочтет мою книгу с интересом. Мировоз­зрение, к которому он идет или пришел, будет в чем‑то совпа­дать с моим, в чем‑то от него отличаться. Но в одном я уверен. Любое целостное мировоззрение, способное увлечь человека, определить для него смысл жизни, будет антитоталитарным, оно будет побуждать человека к активному вмешательству в общественную жизнь. Ибо наша общественная жизнь унизитель­на и абсурдна. Она отнимает смысл у всего, к чему прикасается, а прикасается она ко всему. Какой бы род деятельности вы ни избрали -- науку ли, искусство, производство товаров или вос­питание детей или любой другой -- вы постоянно убеждаетесь, что в попытках сделать нечто значительное тоталитаризм снова и снова преграждает вам дорогу, и если вы в самом деле хоти­те что‑то сделать, то вынуждены будете вылезть из скорлупы своей узкой профессии. (Мальчики и девочки! Не слушайте сво­их родителей, когда они поучают вас расхожей мудрости тота­литарного человека. Они хорошие люди, но искалечены годами страха и унижений. Они хотят вам добра, но учат, как лишать жизнь смысла. Жизнь интереснее, а вклад в нее каждого инди­видуума ‑ значительнее, чем они привыкли думать.)

Сейчас реальный вклад в оздоровление нашей общественной жизни -- это в той или иной форме борьба за основные права личности. В конечном счете эта борьба невозможна без откры­того противостояния тоталитаризму, открытого требования соблюдения прав человека. Никакими софизмами не обойти этой истины.

Каждое выступление в защиту прав человека в некоторой степени меняет ситуацию в стране. Ни одно нравственное дейст­вие не остается без последствий. Движение за Права Человека -- это уже заметная брешь в глухой стене тоталитаризма. За год, прошедший с июня 1975 г., когда я заканчивал первую часть книги (я отдавал книгу в самиздат по частям), произошло два важных события, укрепивших Движение. Первое -- присужде­ние Нобелевской премии мира академику Сахарову; второе ~ рост возмущения на Западе преследованиями инакомыслящих в Советском Союзе, приведший, в частности, к изменению по­зиции французской компартии. Тот факт, что Жорж Марше не приехал на XXV съезд КПСС в феврале 1976 г. "из‑за расхожде­ний по вопросу обращения с инакомыслящими", как он заявил корреспондентам, свидетельствует, что диссиденты в СССР стали явлением, с которым уже нельзя не считаться.

Значение диссидентов для общественной жизни внутри страны в том, что они создают новую модель поведения, которая одним своим присутствием влияет на каждого члена общества. Диссидентство -- это вынужденный выход из системы. Но судьба системы зависит в конечном счете от тех, кто остается в ней. Малые изменения в мышлении и поведении многих людей -- вот что сейчас нужно в первую очередь. И конечно, для демокра­тизации необходимо, чтобы какая‑то ощутимая, не исчезающе малая часть общества ее активно добивалась. <<А что я могу сделать?>> --этот вопрос часто приходится слышать, когда заходит речь о демократизации и о правах личности. В подавляющем большинстве случаев этот вопрос неискренен. Важно всерьез хотеть. 

Будущее не предопределено. Ни для отдельного человека, ни для народа в целом нет непреодолимой силы, которая фатально влекла бы его в бездну. Наша судьба зависит от нас самих, от нашего коллективного разума и воли. Осмысленность нашей жизни -- в наших руках.



\begin{flushleft}
Москва, июнь 1976 г.  
\end{flushleft}


% Last pages for ToC
%-------------------------------------------------------------------------------
\newpage
% Include dots between chapter name and page number
\renewcommand{\cftchapdotsep}{\cftdotsep}
%Finally, include the ToC
\tableofcontents




\end{document}

